\documentclass[12pt]{article}

\usepackage{ishn}

\makeindex[intoc]

\begin{document}

\hypersetup{pageanchor=false}
\begin{titlepage}
	\begin{center}
		\vspace*{1em}
		\Huge
		\textbf{II Classical Dynamics}

		\vspace{1em}
		\large
		Ishan Nath, Michaelmas 2023

		\vspace{1.5em}

		\Large

		Based on Lectures by Prof. David Skinner

		\vspace{1em}

		\large
		\today
	\end{center}
	
\end{titlepage}
\hypersetup{pageanchor=true}

\tableofcontents

\newpage

\section{Introduction}
\label{sec:intro}

The classic version of Newton's laws is
\[
\mathbf{F} = \mathbf{\dot p}, \qquad \mathbf{p} = m \mathbf{\dot x}.
\]

\begin{itemize}
	\item This is a good description of classic physics, with $\hbar = 0$ and $c = \infty$, however...
	\item it is difficult to describe $n \gg 1$ particles, or systems with constraints; e.g. beads stuck to wire frames.
	\item Newton's laws work best in inertial frames, where we use `fictitious forces'. It would be nice to find a formulation that is the same in all frames.
\end{itemize}

These problems were overcome by the Lagrangian and Hamiltonian reformulations of classical dynamics. These formulations:
\begin{itemize}
	\item make the role of symmetry much more central, e.g. invariance under spatial translations is equivalent to conservation of momentum,
	\item make the connection to quantum mechanics much clearer.
\end{itemize}

The Hamiltonian formulation is closely related to the Schr\"{o}dinger picture of QM, whereas the Lagrangian is related to the Feynman path integral.

The space of solutions to Newton's equations itself has a rich geometric structure, which leads to the study of symplectic geometry. Here the metric assigned to the geometry is antisymmetric.

\newpage

\section{Galilean Space-Time}
\label{sec:galilean_st}

An $n$-dimensional affine space $A^{n}$\index{affine space} is a set together with an operation $-$ such that $a - b \in \mathbb{R}^{n}$, for all $a, b \in A^{n}$

Heuristically, $A^{n}$ is like $\mathbb{R}^{n}$, but without a preferred origin.

\emph{Galilean space-time}\index{Galilean space-time} is an affine space $A^4$ with a preferred linear map $t : \mathbb{R}^{4} \to \mathbb{R}$ called \emph{time}. The points $a, b \in A^4$ are called \emph{events}\index{events}.

In particular, $t(b-a)$ is the time between events $b$ and $a$. If $b - a \in \ker(t)$, then the events are called \emph{simultaneous}\index{simultaneous events}. By rank-nullity, $\ker(t) \cong \mathbb{R}^3$.

If $b, a \in A^{4}$ are simultaneous, then the \emph{distance}\index{distance} between them is $d(b, a) = \sqrt{(b-a,b-a)}$, where $(\cdot, \cdot)$ is the Euclidean inner product in $\mathbb{R}^3$.

\subsection{Inertial Frames}
\label{sub:in_f}

An \emph{inertial frame}\index{inertial frame} is a vector space $\mathbb{R} \times \mathbb{R}^3$ with the usual inner product on $\mathbb{R}^3$. These are naturally Galilean spaces with a choice of an origin. However, they are not canonically Galilean.

Consider the transformations $\mathbb{R} \times \mathbb{R}^3 \to \mathbb{R} \times \mathbb{R}^3$ given by
\[
g_1 : (t, \mathbf{x}) \mapsto (t+s, \mathbf{x} + \mathbf{c}),
\]
\[
	g_2 : (t, \mathbf{x}) \mapsto (t, R\mathbf{x}) \text{ for }R \in O(3),
\]
\[
g_3 : (t, x) \mapsto (t, \mathbf{x} + \mathbf{v}t).
\]
Here $g_1$ corresponds to translations, $g_2$ to rotations and reflections, and $g_3$ to Galilean boosts\index{Galilean boost}.

These transformations generate the Galilean group\index{Galilean group} $G$, which is specified by 10 parameters. Any two inertial frames are related by some element of the Galilean group.

A \emph{motion}\index{motion} of a particle is a smooth map $\mathbf{x} : T \to \mathbb{R}^3$ where $I \subset \mathbb{R}$ is an interval. The image of $\mathbf{x}$ is called a \emph{trajectory}\index{trajectory}.

The graph $(t, \mathbf{x}(t)) \in \mathbb{R} \times \mathbb{R}^3$ of the trajectory is the particle's \emph{worldline}\index{worldline}.

The \emph{velocity}\index{velocity} of a particle is $\mathbf{\dot x} = \diff \mathbf{x} / \diff t$, and the \emph{acceleration}\index{acceleration} id $\mathbf{\ddot x} = \diff^2 \mathbf{x}/\diff t^2$.

For a system of $n$ particles, we must specify $n$ motions $\mathbf{X}(t) = (\mathbf{x}_1(t), \mathbf{x}_2(t), \ldots, \mathbf{x}_n(t))$. $\mathbf{X}(t)$ is naturally a point in $\mathbb{R}^{3n}$, and here $\mathbb{R}^{3n}$ is the \emph{configuration space}\index{configuration space} of our $n$-particle system.

%lecture 2

\subsection{Dynamics}
\label{sub:dynamix}

In classical mechanics, a systems dynamics are determined by
\[
	\mathbf{\dot p}_i = \mathbf{F}_i (\mathbf{x}_1, \ldots, \mathbf{x}_n, \mathbf{\dot x}_1, \ldots, \mathbf{\dot x}_n, t), \tag{N2}\label{eq:n2}
\]
and $\mathbf{p}_i = m_i \mathbf{\dot x}_i$. Here $m_i$ is the mass of the $i$'th particle.

The \emph{principle of Galilean invariance}\index{principle of Galilean invariance} says that if $\{\mathbf{x}_i (t)\}$ is a solution of (\ref{eq:n2}), so too is $\{g \circ \mathbf{x}_i(t)\}$ for every element $g \in \Gal$.

For example, if $\{\mathbf{x}_i(t)\}$ solves (\ref{eq:n2}), then so too does $\{\mathbf{x}_i(t+s)\}$, for all $s \in \mathbb{R}$. Hence
\[
	m_i \mathbf{\ddot x}_i(t+s) = \mathbf{F}_i{\mathbf{x}_j(t+s), \mathbf{\dot x}_j(t+s}, t),
\]
or equivalently,
\[
	m_i \mathbf{\ddot x_i}(t') = \mathbf{F}_i(\{\mathbf{x}_j(t'), \mathbf{\dot x}_j(t')\}, t' - s),
\]
where $t' = t + s$. However, as $t'$ is free, and so is $s$, we see that $\mathbf{F}_i$ has no explicit time dependence.

As another example, if $\{\mathbf{x}_i(t)\}$ obey (\ref{eq:n2}), then so too does $\{\mathbf{x}_i(t) + \mathbf{c}\}$, for all $\mathbf{c} \in \mathbb{R}^3$. Hence the force can only depend on the relative locations,
\[
	\mathbf{F}_i(\{\mathbf{x}_j - \mathbf{x}_k\}, \{\mathbf{\dot x}_j\}).
\]
Again if $\{\mathbf{x}_i(t)\}$ satisfies (\ref{eq:n2}), then so too do $\{\mathbf{x}_i(t) + \mathbf{v} t\}$, which gives
\[
	\mathbf{F}_i = \mathbf{F}_i(\{\mathbf{x}_j - \mathbf{x}_k, \mathbf{\dot x}_j - \mathbf{\dot x}_k\}).
\]
And if $\{\mathbf{x}_i(t)\}$ satisfies (\ref{eq:n2}), so too do $\{R \mathbf{x}_i(t)\}$. Therefore
\[
	\mathbf{F}_i(\{R(\mathbf{x}_j - \mathbf{x}_k), R(\mathbf{\dot x}_j - \mathbf{\dot x}_k)\}) = R \mathbf{F}_i(\{\mathbf{x}_j - \mathbf{x}_k, \mathbf{\dot x}_j - \mathbf{\dot x}_k\}).
\]

Let's consider a Galilean invariant system that consists of just one particle. Then Galilean invariance implies that $\mathbf{F} = 0$. Hence $\mathbf{\dot p}_1 = 0$, so $\mathbf{x}_1(t) = \mathbf{x}_0 + \mathbf{u}t$.

Now consider a Galilean invariant system of two particles:
\begin{align*}
	m_1 \mathbf{\ddot x}_1 &= \mathbf{F}_1 (\mathbf{x}_1 - \mathbf{x}_2, \mathbf{\dot x}_1 - \mathbf{\dot x}_2),\\
	m_2 \mathbf{\ddot x}_2 &= \mathbf{F}_2(\mathbf{x}_2 - \mathbf{x}_1, \mathbf{\dot x}_2 - \mathbf{\dot x}_1).
\end{align*}

By rotational invariance,
\begin{align*}
	\mathbf{F}_1(\mathbf{x}_1 - \mathbf{x}_2, \mathbf{\dot x}_1 - \mathbf{\dot x}_2) = (&\mathbf{x}_1 - \mathbf{x}_2) f(|\mathbf{x}_1 - \mathbf{x}_2|, |\mathbf{\dot x}_1 - \mathbf{\dot x}_2|) \\
	+ (&\mathbf{\dot x}_1 - \mathbf{\dot x}_2)g(|\mathbf{x}_1 - \mathbf{x}_2|, |\mathbf{\dot x}_1 - \mathbf{\dot x}_2|)
\end{align*}
for scalar functions $f, g$. In particular,
\[
m_1 \mathbf{\ddot x}_1 \cdot ((\mathbf{x}_1 - \mathbf{x}_2) \times (\mathbf{\dot x}_1 - \mathbf{\dot x}_2)) = 0.
\]
Hence motion is confined to a plane.

A \emph{closed system}\index{closed system} is one in which the forces are forces of interaction, i.e.
\[
\mathbf{F}_i = \sum_{j \neq i} \mathbf{F}_{ij},
\]
where $\mathbf{F}_{ij}$ is the force on the $i$'th particle due to the $j$'th particle. Recall Newton's third law:
\[
	\mathbf{F}_{ij} = - \mathbf{F}_{ji}. \tag{N3}\label{eq:n3}
\]

We also have the extended Newton's third law:
\[
	\mathbf{F}_{ij} = -\mathbf{F}_{ji} = (\mathbf{x}_i - \mathbf{x}_j) f(|\mathbf{x}_i - \mathbf{x}_j|, |\mathbf{\dot x}_i - \mathbf{\dot x}_j|). \tag{N3'}\label{eq:n3'}
\]

Define $M = \sum m_i$, the total mass\index{total mass}, and
\[
	\mathbf{x}_{\mathrm{com}} = \sum_i \frac{m_i \mathbf{x}_i}{M},
\]
the centre of mass\index{centre of mass}. Then
\[
M \mathbf{\ddot x}_{\mathrm{com}} = \sum_{i} m_i \mathbf{\ddot x}_i = \sum_i \Biggl( \sum_{j \neq i} \mathbf{F}_{ij} \Biggr) = \sum_{i \neq j} (\mathbf{F}_{ij} + \mathbf{F}_{ji}) = \mathbf{0},
\]
hence $\mathbf{x}_{\mathrm{com}}(t) = \mathbf{x}_0 + \mathbf{U}(t)$. It is often convenient to do a boost and translation such that $\mathbf{x}_{\mathrm{com}}(t) = \mathbf{0}$.

Define the total angular momentum\index{total angular momentum}
\[
\mathbf{L} = \sum_i \mathbf{x}_i \times \mathbf{p}_i = \sum_i \mathbf{x}_i \times (m_i \mathbf{\dot x}_i),
\]
then the change in angular momentum is
\[
\mathbf{\dot L} = \sum_i \mathbf{x}_i times \mathbf{F}_i = \sum _i \mathbf{x}_i \times \Biggl( \sum_{j \neq i} \mathbf{F}_{ij} \Biggr).
\]
We have
\[
\mathbf{x}_1 \times \mathbf{F}_{12} + \mathbf{x}_2 \times \mathbf{F}_{21} = (\mathbf{x}_1 - \mathbf{x}_2) \times \mathbf{F}_{12} = \mathbf{0},
\]
by (\ref{eq:n3'}), so $\mathbf{L}$ is conserved if (\ref{eq:n3'}) holds.

\subsection{Non-closed systems}
\label{sub:non_closed_systems}

There often arise from considering only the relative motion, or neglecting back-reaction.

For a closed system of two particles,
\[
m_1 \mathbf{\ddot x}_1 = \mathbf{F}_{12}, \qquad m_2 \mathbf{\ddot x}_2 = - \mathbf{F}_{12}.
\]
Define
\[
\frac{1}{m} = \frac{1}{m_1} + \frac{1}{m_2}, \qquad m = \frac{m_1m_2}{m_1 + m_2},
\]
the relative mass\index{relative mass}, and $\mathbf{x}_r = \mathbf{x}_1 - \mathbf{x}_2$, the relative location. Then,
\[
\mathbf{\ddot x}_r = \mathbf{\ddot x}_1 - \mathbf{\ddot x}_2 = \biggl( \frac{1}{m_1} + \frac{1}{m_2} \biggr) \mathbf{F}_{12} \implies m \mathbf{\ddot x}_r = \mathbf{F}(\mathbf{x}_r, \mathbf{\dot x}_r),
\]
so we get a closed system for the relative motion. Note that this relative motion does not need to obey Galilean invariance.

In particular, if $m_2 \gg m_1$, then $m \approx m_1$ and we can approximate the motion $\mathbf{x}_1(t) \approx \mathbf{x}_r(t)$ in the frame in which $\mathbf{x}_{\mathrm{com}}(t) = 0$.

%lecture 3

\subsection{Energy}
\label{sub:energy}

Define the \emph{work done}\index{work} along a path from $\mathbf{x}_0$ at $t_0$, to $\mathbf{x}_1$ at $t_1$, by
\begin{align*}
	W &= \int_{\mathbf{x}_0}^{\mathbf{x}_1} \mathbf{F} \cdot \diff \mathbf{x} = \int_{t_0}^{t_1} \mathbf{F} \cdot \frac{\diff \mathbf{x}}{\diff t} \diff t \\
	  &= \int_{t_0}^{t_1} m \mathbf{\ddot x} \cdot \mathbf{\dot x} \diff t = \frac{1}{2} \int_{t_0}^{t_1} \frac{\diff}{\diff t} (m \mathbf{\dot x}^2) \diff t \\
	  &= T(t_1) - T(t_0),
\end{align*}
where we define the \emph{kinetic energy}\index{kinetic energy} $T(t) = \frac{1}{2} m \mathbf{\dot x}^2 (t)$.

A force is \emph{conservative}\index{conservative} if $W$ is independent of the path taken. Hence $\mathbf{F} = - \nabla V$ for a function $V : \mathbb{R}^3 \to \mathbb{R}$ called the \emph{potential}\index{potential}.
\[
W = - \int_{\mathbf{x}_0}^{\mathbf{x}_1} \nabla V \cdot \diff \mathbf{x} = V(\mathbf{x}_0) - V(\mathbf{x}_1).
\]
Comparing these equations, we see that
\[
T(t_1) + V(\mathbf{x}(t_1)) = T(t_0) + V(\mathbf{x}(t_0)).
\]
Hence the \emph{total energy}\index{total energy} $E = T + V$ is conserved.

\newpage

\section{The Lagrangian Approach}
\label{sec:lagrange}

\subsection{Principle of Least Action}
\label{sub:least_action}

Consider a (smooth) motion/curve $\mathbf{x}(t)$ with $\mathbf{x}(t_0) = \mathbf{x}_0$ and $\mathbf{x}(t_1)= \mathbf{x}_1$ fixed.

We define the \emph{action}\index{action} of such a curve to be
\[
	S[\mathbf{x}] = \int_{t_0}^{t_1} L(\mathbf{x}(t), \mathbf{\dot x}(t), t) \diff t,
\]
where the function $L : \mathbb{R}^3 \times \mathbb{R}^3 \times \mathbb{R} : \mathbb{R}$ is called the \emph{Lagrangian}. We will usually choose $L = T - V$. Here $S$ is a function on the infinite dimensional space of curves, sometimes called a \emph{functional}.

Let's consider critical points of this action. A curve $\mathbf{x}(t)$ will  be an extremum of $S[\mathbf{x}]$ if $S[\mathbf{x}+\delta \mathbf{x}] - S[\mathbf{x}] = \mathcal{O}(\delta \mathbf{x}^2)$ for all variations $\delta \mathbf{x}$. Indeed, expanding,
\begin{align*}
	S[\mathbf{x} + \delta \mathbf{x}] &= \int_{t_0}^{t_1} L(\mathbf{x} + \delta \mathbf{x}, \mathbf{\dot x} + \delta \mathbf{\dot x}, t) \diff t \\
					  &= S[\mathbf{x}] + \int_{t_0}^{t_1} \biggl( \delta \mathbf{x} \cdot \frac{\partial L}{\partial \mathbf{x}} + \delta \mathbf{\dot x} \cdot \frac{\partial L}{\partial \mathbf{\dot x}} \biggr) \diff t + \mathcal{O}(\delta \mathbf{x}^2).
\end{align*}
Then the variation is
\[
\delta S = \int_{t_0}^{t_1} \delta \mathbf{x} \cdot \biggl( \frac{\partial L}{\partial \mathbf{x}} - \frac{\diff}{\diff t} \frac{\partial L}{\partial \mathbf{\dot x}} \biggr) + \delta \mathbf{x} \cdot \frac{\partial L}{\partial \mathbf{\dot x}} \biggr|_{t_0}^{t_1},
\]
where the latter term is 0 as the variation must fix the endpoints. Hence as $\delta S = 0$ for every variation, we get
\[
\frac{\diff}{\diff t} \biggl( \frac{\partial L }{\partial \mathbf{\dot x}} \biggr) - \frac{\partial L}{\partial \mathbf{x}} = 0.
\]
These are called the \emph{Euler-Lagrange equations}\index{Euler-Lagrange equations}.

\begin{exbox}
	The length $\ell$ of a curve from $\mathbf{x}_0$ to $\mathbf{x}_1$ in $\mathbb{R}^2$ is
	\[
		\ell = \int_{t_0}^{t_1} \sqrt{\dot x^2 + \dot y^2} \diff t = \int_{x_0}^{x_1} \sqrt{1 + \biggl( \frac{\diff y}{\diff x} \biggr)^2} \diff x,
	\]
where we parametrise by the $x$ coordinate. The shortest curve extremizes this length, so must obey the Euler-Lagrange equations for the Lagrangian $\sqrt{1+(y')^2}$. Hence
\[
	\frac{\diff}{\diff x} \biggl( \frac{y'}{\sqrt{1 + (y')^2}} \biggr) = 0 \implies \frac{y'}{\sqrt{1 + (y')^2}} = c \implies y = C,
\]
so $y(x) = Cx + D$, a straight line.
\end{exbox}

Hamilton's principle of least action says that the motion of a particle extremizes
\[
	S[\mathbf{x}] = \int_{t_0}^{t_1} L(\mathbf{x}, \mathbf{\dot x}, t) \diff t
\]
for $L = T - V = \frac{1}{2} m \mathbf{\dot x}^2 - V(\mathbf{x}, t)$.

For this $L$, the Euler-Lagrange equations become
\[
\frac{\diff}{\diff t}(m \mathbf{\dot x}) + \nabla V = 0,
\]
This is equivalent to (\ref{eq:n2}), but gives us a different perspective:
\begin{itemize}
	\item The focus is on curves/trajectories, rather than forces. This is a more global picture.
	\item In quantum mechanics, all paths make a contribution:
		\[
			\mathbb{P}(x_1, t_1 \mid x_0, t_0) = | \braket{x_1(t_1)|x_0(t_0)}|^2,
		\]
		where
		\[
			\braket{x_1(t_1)|x_0(t_0)} = \int e^{iS[x]/\hbar} \diff x,
		\]
		an integral over all paths starting at $x_0$ and ending at $x_1$. This is Feynman's path integral.
	\item We can easily generalize this to field theory rather than mechancics, where
		\[
			S = \int \biggl[ R + \frac{1}{2} F^2 + \bar \psi D \psi \biggr] \sqrt{-g} \Diff4 x.
		\]
		Don't ask what anything in the above equation means.
\end{itemize}

%lecture 4

\subsection{Generalized Coordinates}
\label{sub:gen_coords}

Suppose we introduce a new coordinate system $q^{a}$, where $q^{a} = q^{a}(\mathbf{x}, t)$ are smooth functions of $(\mathbf{x}, t)$. For this to be a good system of coordinates, we need
\[
\det \biggl( \frac{\partial q^{a}}{\partial \mathbf{x}} \biggr) \neq 0,
\]
so it must be invertible with $\mathbf{x} = \mathbf{x}(q^a, t)$. Then,
\[
\mathbf{\dot x} = \frac{\diff \mathbf{x}}{\diff t} = \frac{\partial \mathbf{x}}{\partial q^b} \dot q^b + \frac{\partial \mathbf{x}}{\partial t}.
\]
Let's now consider our Lagrangian,
\[
L = L(\mathbf{x}(q^a, t), \mathbf{\dot x}(q^a, \dot q^a, t), t).
\]
We have
\begin{align*}
	\frac{\partial L}{\partial q^{a}} &= \frac{\partial L}{\partial \mathbf{x}} \cdot \frac{\partial \mathbf{x}}{\partial q^{a}} + \frac{\partial L}{\partial \mathbf{\dot x}} \cdot \frac{\partial \mathbf{\dot x}}{\partial q^{a}} \\
					  &= \frac{\partial L}{\partial \mathbf{x}} \cdot \frac{\partial \mathbf{x}}{\partial q^a} + \frac{\partial L}{\partial \mathbf{\dot x}} \cdot \biggl( \frac{\partial^2 \mathbf{x}}{\partial q^{a} \partial q^b} \dot q^b + \frac{\partial^2 \mathbf{x}}{\partial q^a \partial t} \biggr), \\
	\frac{\diff}{\diff t} \biggl( \frac{\partial L}{\partial \dot q^a} \biggr) &= \frac{\diff}{\diff t} \biggl( \frac{\partial L}{\partial \mathbf{\dot x}} \cdot \frac{\partial \mathbf{\dot x}}{\partial \dot q^a} \biggr) = \frac{\diff}{\diff t} \biggl( \frac{\partial L}{\partial \mathbf{\dot x}} \cdot \frac{\partial \mathbf{x}}{\partial q^a} \biggr) \\
										   &= \frac{\diff}{\diff t} \biggl( \frac{\partial L}{\partial \mathbf{\dot x}} \biggr) \cdot \frac{\partial \mathbf{x}}{\partial q^a} + \frac{\partial L}{\partial \mathbf{\dot x}} \cdot \biggl( \frac{\partial^2 \mathbf{x}}{\partial q^a \partial q^b} \dot q^b + \frac{\partial^2 \mathbf{x}}{\partial t \partial q^a} \biggr).
\end{align*}
Comparing these, we see that
\[
	\frac{\diff}{\diff t} \biggl( \frac{\partial L}{\partial \dot q^a} \biggr) - \frac{\partial L}{\partial q^a} = \biggl[ \frac{\diff}{\diff t} \biggl( \frac{\partial L}{\partial \mathbf{\dot x}} \biggr) - \frac{\partial L}{\partial \mathbf{x}} \biggr] \cdot \frac{\partial \mathbf{x}}{\partial q^a}.
\]
Since the matrix $\frac{\partial \mathbf{x}}{\partial q^a}$ is non-degenerate, we see that
\[
\frac{\diff}{\diff t} \biggl( \frac{\partial L}{\partial \dot q^a} \biggr) - \frac{\partial L}{\partial q^a} = 0 \iff \frac{\diff}{\diff t} \biggl( \frac{\partial L}{\partial \mathbf{\dot x}} \biggr) - \frac{\partial L}{\partial \mathbf{x}} = 0,
\]
so the Euler-Lagrange equations in the $q$-coordinates hold if and only if they hold in the $\mathbf{x}$-coordinates.

This is a huge advantage. We only ever need to write out the form of one function $L(q^a, \dot q^a, t)$ to obtain the motion in our generalized coordinate system.

\subsection{Rotating Frame}
\label{sub:rot}

Consider a new coordinate system $(x', y', z')$, which is $(x, y, z)$ but rotating about the $z$-axis with angular velocity $\omega$. Then
\[
x' = x \cos \omega t + y \sin \omega t, \qquad y' = y \cos \omega t - x \sin \omega t, \qquad z' = z.
\]
The free particle in the inertial frame $(x, y, z)$ has
\[
L = \frac{1}{2} m \mathbf{\dot x}^2 = \frac{1}{2} m (\mathbf{\dot x}' + \bm{\omega} \times \mathbf{x}' )^2.
\]
Hence in the rotating frame, the Euler-Lagrange equations become
\[
	0 = \frac{\diff}{\diff t} \biggl( \frac{\partial L}{\partial \mathbf{\dot x}'} \biggr) - \frac{\partial L}{\partial \mathbf{x}'} = m \frac{\diff}{\diff t} (\mathbf{\dot x}' + \bm{\omega} \times \mathbf{x}') - m(\mathbf{\dot x}' \times \bm{\omega} - \bm{\omega} \times (\bm{\omega} \times \mathbf{x}'))
\]
\[
	\implies \mathbf{\ddot x}' + 2 \bm{\omega} \times \mathbf{\dot x}' + \bm{\omega} \times(\bm{\omega} \times \mathbf{x}') = 0.
\]
This recovers the fictitious forces that we worked hard to derive in earlier courses.

\subsection{Constraints}
\label{sub:constraints}

Sometimes we want to force our system to lie on some surface $\Sigma \subset \mathbb{R}^{d}$. For example, perhaps the particle is forced to stay on a wire hoop.

The constraints are called \emph{holonomic}\index{holonomic} if we can write
\[
	\Sigma = \{f_n(\mathbf{x}, t) = 0 \mid r = 1, \ldots, d-s \} \subset \mathbb{R}^{d}
\]
for some functions $f$ that are independent of $\mathbf{\dot x}$. For example, given motion on $S_a^2 \subset \mathbb{R}^3$, we can take $f = x^2 + y^2 + z^2 - a^2$.

For each $r \in \{1, \ldots, d-s\}$, the $d$-component vector $\frac{\partial f_r}{\partial \mathbf{x}}$ is orthogonal to $\Sigma$. If the $(d-s) \times d$ dimensional matrix
\[
\frac{\partial f_r}{\partial \mathbf{x}} =
\begin{pmatrix}
	\partial f_1/\partial \mathbf{x} \\
	\partial f_2 / \partial \mathbf{x} \\
	\vdots \\
	\partial f_{d-s} / \partial \mathbf{x}
\end{pmatrix}
\]
has maximal rank, then the rows form a basis of the space orthogonal to $\Sigma$ at each point on $\Sigma$.

For example if $f = x^2 + y^2 + z^2 - a^2$, then $\frac{\partial f}{\partial \mathbf{x}} = (x, y, z)$, which points radially:
\[
\frac{\partial f}{\partial \mathbf{x}} \biggr|_{\Sigma} = a \mathbf{\hat x}.
\]

Then, by inverse function theorem, we can find generalized coordinates $q^a$ that parametrize the constraint function $\Sigma$. In the case above, we can take $q^a = (q^1, q^2) = (\theta, \phi)$, where
\[
x = a \sin \theta \cos \phi,\qquad y = a \sin \theta \sin \phi, \qquad z = a \cos \theta.
\]
The constrained motion will then be $\mathbf{x}(q^a, t)$. To see this, consider a new Lagrangian
\[
L(\mathbf{x}, \mathbf{\dot x}, \lambda_r, t) = L(\mathbf{x}, \mathbf{\dot x}, t) + \sum_{r = 1}^{d-s} \lambda_r f_r(\mathbf{x}, t).
\]
Here $\lambda_r$ are Lagrange multipliers. They appear as non-dynamical variables, and $\frac{\partial L'}{\partial \lambda_r} = 0$.

The Euler-Lagrange equations for $L'$ give
\[
\frac{\partial L'}{\partial \lambda_r} = 0 \implies f_r(\mathbf{x}, t) = 0,
\]
i.e. motion is constrained, and also
\[
\frac{\diff}{\diff t} \biggl( \frac{\partial L'}{\partial \mathbf{\dot x}} \biggr) - \frac{\partial L'}{\partial \mathbf{x}} = 0 \implies \frac{\diff}{\diff t} \biggl( \frac{\partial L}{\partial \mathbf{\dot x}} \biggr) - \frac{\partial L}{\partial \mathbf{x}} - \sum_{r = 1}^{d-s} \lambda_r \frac{\partial f_r}{\partial \mathbf{x}} = 0.
\]
The new term $\mathbf{R} = \sum \lambda_r \frac{\partial f_r}{\partial \mathbf{x}}$ are the \emph{constraint forces}\index{constraint forces}. They do no work because motion obeys $\diff \mathbf{x} \cdot \mathbf{R} = 0$.

Let's instead use generalized coordinates $(q^a, f_r)$. The Euler-Lagrange equations says $f_r = 0$, so
\[
\frac{\diff}{\diff t} \biggl( \frac{\partial L}{\partial \dot q^a} \biggr) - \frac{\partial L}{\partial q^a} = \sum_r \lambda_r \frac{\partial f_r}{\partial q^a} = 0.
\]
So the motion in $q$-coordinates does not care about the constraints.

\subsection{Examples}
\label{sub:lagrange_ex}

Consider a simple pendulum, with mass $m$, length $l$ from the origin, and angle $\theta$ to the vertical. We can of course derive the equations of motion from Newtonian mechanics, but we can also describe this system using the Lagrangian: if the pendulum is at point $(x, y)$, then
\[
L' = \frac{1}{2}m(\dot x^2 + \dot y^2) - mgy + \lambda(x^2 + y^2 - l^2)
\]
The Euler-Lagrange equations give
\begin{align*}
	x^2 + y^2 &= l^2, \\
	m\ddot x &= 2\lambda x, \\
	m\ddot y &= -mg + 2\lambda y.
\end{align*}
We can solve the constraint $x^2 + y^2 = l^2$ by setting $x = l \sin \theta$ and $y = -l \cos \theta$, where upon the remaining equations become
\begin{align*}
	ml(\ddot \theta \cos \theta - \dot \theta^2 \sin \theta) &= 2 \lambda l \sin \theta, \tag{$\ast$}\\
	ml(\ddot \theta \sin \theta + \dot \theta^2 \cos \theta) &= - mg + 1\lambda l \cos \theta. \tag{$\dagger$}
\end{align*}
Then $\cos \theta \times (\ast) + \sin \theta \times (\dagger)$ gives
\[
\ddot \theta = - \frac{g \sin \theta}{l},
\]
which is the equations of motions along the constraint surface. We also see that the reaction force
\[
\lambda \frac{\partial f}{\partial \mathbf{x}} = 2 \lambda
\begin{pmatrix}
	x \\ y
\end{pmatrix}
 = 2 \lambda l
 \begin{pmatrix}
 	\sin \theta \\ - \cos \theta
 \end{pmatrix}
  = \bm{\tau},
\]
the tension in the rod.

This is a roundabout way of doing this problem. It is much better to use generalized coordinates adapted to the constraints. Here, these are
\[
	(q^a, f_r) \to (\theta, f).
\]
In these coordinates, our Lagrangian becomes
\[
L'(\theta, f) = \frac{1}{2} m l^2 \dot \theta^2 + mgl \cos \theta + \lambda f.
\]
The Euler-Lagrange equation for $\lambda$ gives $f = 0$, and the remaining equation for $\theta$ gives
\[
m l^2 \dot \theta \ddot \theta - mgl \sin \theta \dot \theta \implies \ddot \theta = - \frac{g \sin \theta}{l}.
\]

Now lets think about a bead of mass $m$ on a hoop of radius $a$, rotating with angular velocity $\omega$ about the $z$-axis. We can solve the constraints by letting $\psi$ be the angle of the bead from the vertical. Then setting coordinates
\[
x = a \sin \psi \cos \omega t, \qquad y = a \sin \psi \sin \omega t, \qquad z = a - a \cos \psi,
\]
we get the Lagrangian involving
\begin{align*}
	T &= \frac{1}{2}(\dot x^2 + \dot y^2 + \dot z^2) = \frac{1}{2} ma^2 (\dot \psi^2 + \omega^2 \sin^2 \psi),\\
	V &= mgz = mga(1 - \cos \psi).
\end{align*}
Hence the Euler-Lagrange equations for the generalized coordinate $\psi$ become
\[
\frac{\diff}{\diff t} \biggl( \frac{\partial L}{\partial \dot \psi} \biggr) = \frac{\partial L}{\partial \psi} \implies \ddot \psi = -\frac{\partial V_{\mathrm{eff}}}{\partial \psi},
\]
where the effective potential is
\[
V_{\mathrm{eff}} = \frac{1}{ma^2} \biggl(-mga \cos \psi - \frac{1}{2} ma^2 \omega^2 \sin^2 \psi\biggr).
\]
In equilibrium, the bead must sit at an extremum of $V_{\mathrm{eff}}$. As expected, for $\omega = 0$ this boils down to the previous pendulum problem, and the bead will sit at the bottom.

For small $\omega$, the bead will still prefer to stay at the bottom. However as we increase $\omega$, and in particular if $\omega^2 > \frac{g}{a}$, the stable equilibrium will be away from $0$, and the bead will prefer to hover.

\subsection{Noether's Theorem}
\label{sub:noe_thm}

There is a very beautiful relation between symmetries of a Lagrangian, and conserved quantities. We define the generalized momentum\index{generalized momentum} $p_a$ associated to a generalized coordinate $q_a$ by
\[
p_a = \frac{\partial L}{\partial \dot q^a}.
\]
Compare this to regular momentum in Cartesian coordinates. We also say that a quantity $f(q^a, \dot q^a, t)$ is conserved along the motion\index{conserved quantity} if
\[
0 = \frac{\diff f}{\diff t} = \frac{\partial f}{\partial q^a} \dot q^a + \frac{\partial f}{\partial \dot q^b} \ddot q^b + \frac{\partial f}{\partial t},
\]
for the motion $q^a(t)$ that solves the Euler-Lagrange equation.

Consider a 1-parameter family of motions $Q^a(s, t)$ with $Q^a(0, t) = q^a(t)$. This transformation is said to be a \emph{symmetry}\index{symmetry} of $L$ if
\[
\frac{\partial}{\partial s} L (Q^a(s, t), \dot Q^a(s, t), t) = 0.
\]
\emph{Noether's theorem}\index{Noether's theorem} says there exists a conserved quantity for each such symmetry. Indeed,
\begin{align*}
	0 &= \frac{\partial L}{\partial s} \biggr|_{s=0} = \frac{\partial L}{\partial q^a} \frac{\partial Q^a}{\partial s} \biggr|_{s = 0} + \frac{\partial L}{\partial \dot q^a}\frac{\partial \dot Q^a}{\partial s} \biggr|_{s = 0} \\
	  &= \frac{\diff}{\diff t} \biggl( \frac{\partial L}{\partial \dot q^a} \biggr) \frac{\partial Q^a}{\partial s} \biggr|_{s = 0} + \frac{\partial L}{\partial \dot q^a} \frac{\diff}{\diff t} \biggl(\frac{\partial Q^a}{\partial s} \biggr) \biggr|_{s = 0} \\
	  &= \frac{\diff}{\diff t} \biggl( \frac{\partial L}{\partial \dot q^a} \frac{\partial Q^a}{\partial s} \biggr|_{s = 0} \biggr) = 0.
\end{align*}
Hence
\[
\frac{\partial L}{\partial \dot q^a} \frac{\partial Q^a}{\partial s} \biggr|_{s = 0}
\]
is conserved if $Q^a(0, t) = q^a(t)$ obeys the Euler-Lagrange equations.

%lecture 6

For example, homogeneity of space implies our $L$ should be invariant under constant translations: $\mathbf{x}_i(t) \mapsto \mathbf{x}_i(t) + \mathbf{c}$. For example, if
\[
L = \sum_{i = 1}^{n} \frac{1}{2} m_1 \mathbf{\dot x}_i^2 - V(\mathbf{x}_i - \mathbf{x}_j),
\]
then setting $Q_i(s, t) = \mathbf{x}_i(t) + s\mathbf{c}$, we have a conserved quantity
\[
\sum_{i} \frac{\partial L}{\partial \mathbf{\dot x}_i} \cdot \mathbf{c} = \Biggl( \sum_{i} \mathbf{p}_i \Biggr) \cdot \mathbf{c}.
\]
As this holds for all $\mathbf{c}$, we have that the sum of the momenta are conserved.

This is a special case of the following: suppose $L(q^a, \dot q^a, t)$ is independent of a particular generalized coordinate $q^1$. Then the Euler-Lagrange equations for $q^1$ say that
\[
\frac{\diff}{\diff t} \biggl( \frac{\partial L}{\partial \dot q^1} \biggr) = \frac{\partial L}{\partial q_1} = 0,
\]
so the corresponding generalized momenta $p_1$ is conserved. Such coordinates are called \emph{ignorable}\index{ignorable}.

Moreover, isotropy of space implies $L$ should be invariant under rotations of all our particles trajectories. An infinitesimal rotation acts as
\[
\mathbf{x}_i(t) \mapsto \mathbf{x}_i(t) + \delta \theta \mathbf{n} \times \mathbf{x}_i(t),
\]
if we rotate by $\delta \theta$ around the $\mathbf{n}$-axis. So, a Lagrangian that is rotationally invariant obeys
\[
\frac{\partial}{\partial \theta} L(\mathbf{x}_i + \theta \mathbf{n} \times \mathbf{x}_i, \mathbf{\dot x}_i + \theta \mathbf{n} \times \mathbf{\dot x}_i, t) \biggr|_{\theta = 0} = 0,
\]
which gives
\begin{align*}
	&\sum_{i = 1}^{n} \frac{\partial L}{\partial \mathbf{x}_i} \cdot (\mathbf{n} \times \mathbf{x}_i) + \frac{\partial L}{\partial \mathbf{\dot x}_i} \cdot (\mathbf{n} \times \mathbf{\dot x}_i) = 0, \\
	\implies 0 &= \mathbf{n} \cdot \sum_i \biggl( \mathbf{x}_i \times \frac{\partial L}{\partial \mathbf{x}_i} + \mathbf{\dot x}_i \times \frac{\partial L}{\partial \mathbf{\dot x}_i} \biggr) \\
		   &= \mathbf{n} \cdot \frac{\diff}{\diff t} \Biggl( \sum_{i} \mathbf{x}_i \times \frac{\partial L}{\partial \mathbf{\dot x}_i} \Biggr) = \mathbf{n} \cdot \frac{\diff}{\diff t} \sum_i \mathbf{x}_i \times \mathbf{p}_i.
\end{align*}
Since this is true for any axis $\mathbf{n}$, the total angular momentum $\mathbf{L} = \sum_i \mathbf{x}_i \times \mathbf{p}_i$ is conserved.

There is a more general form of Noether's theorem. Notice that $L$ and $L + \frac{\diff}{\diff t} f(\mathbf{x}, t)$ lead to the same Euler-Lagrange equations:
\[
S = \int_{t_0}^{t_1} \biggl( L + \frac{\diff f}{\diff t} \biggr) \diff t = \int L \diff t + f \biggr|_{\mathbf{x}_0, t_0}^{\mathbf{x}_1, t_1},
\]
and the boundary term is the same for all trajectories obeying $\delta \mathbf{x}|_{t_0, t_1} = \mathbf{0}$. Consequently, it is enough for our symmetry to preserve $L$ up to a total time derivative.

In fact, Noether's theorem allows more generally $L$ to be preserved up to $\frac{\diff}{\diff t}f(\mathbf{x}, \mathbf{\dot x}, t)$.

\begin{theorem}
	Suppose we have an infinitesimal transformation
	\[
	q^a(t) \mapsto q^a(t) + s u^a(t) + \mathcal{O}(s^2),
	\]
	such that
	\[
	\frac{\partial}{\partial s} L(q^a + s u^a, \dot q^a + s \dot u^a, t) \biggr|_{s=0} = \frac{\diff}{\diff t} f(q^a, \dot q^a, t).
	\]
	Then the quantity
	\[
	\frac{\partial L}{\partial \dot q^a} u^a - f
	\]
	is conserved when the Euler-Lagrange equations holds.
\end{theorem}

\begin{proofbox}
	Let's do some calculations:
	\begin{align*}
		\frac{\diff}{\diff t} \biggl[ \frac{\partial L}{\partial \dot q^a} u^a - f \biggr] &= \frac{\diff}{\diff t} \biggl( \frac{\partial L}{\partial \dot q^a} \biggr) u^a + \frac{\partial L}{\partial \dot q^a} \dot u^a - \frac{\diff f}{\diff t} \\
		&= \frac{\partial L}{\partial q^a} u^a + \frac{\partial L}{\partial \dot q^a} \dot u^a - \frac{\diff f}{\diff t} \\
		&= \frac{\partial}{\partial s} L(q+su, \dot q + s \dot q_u, t) \biggr|_{s=0} - \frac{\diff f}{\diff t} = 0.
	\end{align*}
\end{proofbox}

For example, homogeneity in time implies that $\frac{\partial L}{\partial t} = 0$. For such $L$, consider translation all the trajectories in time, so
\begin{align*}
	q^a(t) &\mapsto q^a(t+s) \approx q^a(t) + s \dot q^a(t) + \mathcal{O}(s^2),\\
	\dot q^a(t) &\mapsto \dot q^a(s+t) \approx \dot q^a(t) + s \ddot q^a(t) + \mathcal{O}(s^2),
\end{align*}
hence
\[
\frac{\partial}{\partial s}L(q^a + s \dot q^a, \dot q^a + s \ddot q^a) \biggr|_{s=0} = \frac{\diff L}{\diff t}.
\]
According to Noether's, the conserved quantity is
\[
H = \frac{\partial L}{\partial \dot q^a} \dot q^a - L.
\]
This is known as the \emph{Hamiltonian}, and represent the energy. For example, if
\[
L = \frac{1}{2} m \mathbf{\dot x}^2 - V(\mathbf{x}),
\]
then the Hamiltonian is
\[
H = m \mathbf{\dot x} \cdot \mathbf{\dot x} - \biggl( \frac{1}{2} m \mathbf{\dot x}^2 - V \biggr) = \frac{1}{2} m \mathbf{\dot x}^2 + V.
\]
If $\mathbf{x} = \mathbf{x}(q)$ with some generalized coordinates $q^a(t)$, then
\[
\mathbf{\dot x} = \frac{\partial \mathbf{x}}{\partial q^a} \dot q^a,
\]
so
\[
L = T_{ab}(q) \dot q^a \dot q^b - V(q),
\]
where
\[
T_{ab} = \frac{1}{2} m \frac{\partial \mathbf{x}}{\partial q^a} \cdot \frac{\partial \mathbf{x}}{\partial q^b} (q).
\]
Then the Hamiltonian is
\[
H = \frac{\partial L}{\partial \dot q^a} \dot q^a - L = T_{ab} \dot q^a \dot q^b + V(q).
\]

\begin{exbox}[Spherical Pendulum]
	A mass $m$ is attached to an inextensible rod of length $l$, but is free to swing in any direction. Then the Lagrangian is
	\[
	L = \frac{1}{2} m \mathbf{\dot x}^2 - mgz + \lambda (x^2 + y^2 + z^2 - l^2).
	\]
	As usual, we first solve the constraint by setting
	\[
	x = l \cos \phi \sin\theta, \qquad y = l \sin \phi \sin \theta, \qquad z = l \cos \theta.
	\]
	In these coordinates,
	\[
	L = \frac{1}{2} m l^2 (\dot q^2 + \sin^2 \theta \dot \phi^2) + mgl \cos \theta.
	\]
	We can see that
	\[
	\frac{\partial L}{\partial t} = 0, \qquad \frac{\partial L}{\partial \phi} = 0,
	\]
	so
	\[
	H = \frac{1}{2} m l^2 (\dot \theta^2 + \sin^2 \theta \dot \phi^2) - mg l \cos \theta,
	\]
	is conserved, and so is
	\[
	J = \frac{ml^2}{2} \sin^2 \theta \dot \phi,
	\]
	the angular momentum in the $\phi$-direction.

	%lecture 7
	We can use this conserved quantity $J$ to simplify
	\begin{align*}
		\ddot \theta &= \frac{ml^2}{ml^2} \frac{J^2}{(ml^2)^2} \frac{\cos \theta}{\sin^3\theta} - \frac{g}{l} \sin \theta \\
			     &= - \frac{\diff V_{\mathrm{eff}}}{\diff \theta},
	\end{align*}
	where
	\[
	V_{\mathrm{eff}}(\theta) = \frac{g}{l}\cos \theta + \frac{J^2}{2(ml^2)^2\sin^2 \theta}.
	\]
	Note that we can only substitute $J$ into the equation of motion, not the original Lagrangian.

	For $J \neq 0$, $V_{\mathrm{eff}}$ grows as $\theta \to 0$. Since the partial $L$ does not depend on time,
	\[
	H = \dot \theta \frac{\partial L}{\partial \dot \theta} + \dot \phi \frac{\partial L}{\partial \dot \phi} - L = ml^2 \biggl(\frac{1}{2} \dot \theta^2 + V_{\mathrm{eff}}(\theta) \biggr) = E,
	\]
	is constant. Hence, for fixed $(E, J)$, the motion in $\theta$ is bounded $\theta \in [\theta_0, \theta_1]$.

	Therefore the motion will oscillate in $\theta$ as it precesses in $\phi$. To go further, we could write
	\[
		\frac{\diff \theta}{\diff t} = \sqrt{\frac{2E}{ml^2} - V_{\mathrm{eff}}(\theta)},
	\]
	hence
	\[
		t_1 - t_0 = \int_{\theta_0}^{\theta_1} \frac{\diff \theta}{\sqrt{\frac{2E}{ml^2} - V_{\mathrm{eff}}(\theta)}}.
	\]
	If we could do this integral, we would know $\theta(t)$, and hence $\phi(t)$. However this turns out to be too hard, except numerically.
\end{exbox}

\subsection{Lagrangian for a Relativistic Particle}
\label{sub:l_relativistic}

In special relativity, a free particle of (rest) mass $m$ hass action
\[
	S[\mathbf{x}] = -mc^2 \int \sqrt{1 - \frac{\mathbf{\dot x} \cdot \mathbf{\dot x}}{c^2}} \diff t.
\]
Note that if $|\mathbf{\dot x}| \ll c$,
\[
L = -mc^2 + \frac{1}{2} m |\mathbf{\dot x}|^2 + \mathcal{O}((\mathbf{\dot x}/c)).
\]
For this $L$, we have
\[
\mathbf{p} = \frac{\partial L}{\partial \mathbf{\dot x}} = \frac{m \mathbf{\dot x}}{\sqrt{1 - |\mathbf{\dot x}|^2/c^2}} = m \gamma \mathbf{\dot x},
\]
\begin{align*}
	H &= \mathbf{\dot x} \cdot \frac{\partial L}{\partial \mathbf{\dot x}} - L = \frac{m \mathbf{\dot x}^2}{\sqrt{1-x^2/c^2}} + mc^2\sqrt{1-\mathbf{\dot x}^2/c^2} \\
	  &= \frac{mc^2}{\sqrt{1-x^2/c^2}} = m\gamma c^2.
\end{align*}
However, this form of $L$ is not obviously Lorentz invariant.

Instead, consider the alternative action
\[
	S[x^{\mu}] = -mc \int \sqrt{\eta_{\mu \nu} \frac{\diff x^{\mu}}{\diff \tau} \frac{\diff x^{\nu}}{\diff \tau}} \diff \tau,
\]
where $x^{\mu} = (ct, \mathbf{x})$, $t$ is an arbitrary parameter, and $\eta_{\mu \nu} = \mathrm{diag}(-1, +1, \ldots, +1)$. This is manifestly Lorentz invariant:
\[
x^{\mu} \mapsto \Lambda^{\mu}_{\nu} x^{\nu}, \qquad \frac{\diff x^{\mu}}{\diff \tau} \mapsto \Lambda^{\mu}_{\nu} \frac{\diff x^{\mu}}{\diff \tau}.
\]
However, we seem to have an extra dimension of freedom $(t(\tau), \mathbf{x}(\tau))$. This is illusory; the particle has to move through time. This is reflected in the redundancy of $S[x^{\mu}]$.

Consider reparametrisations $\tau \mapsto \tilde \tau(\tau)$. Then $\diff \tau \mapsto \frac{\diff \tau}{\diff \tilde \tau} \diff \tilde \tau$, and $\frac{\diff x^{\mu}}{\diff \tau} \mapsto \frac{\diff \tilde \tau}{\diff \tau} \frac{\diff x^{\mu}}{\diff \tilde \tau}$. Hence,
\[
	S[x^{\mu}] = - mc \int \sqrt{-\eta_{\mu \nu} \frac{\diff x^{\mu}}{\diff \tau} \frac{\diff x^{\nu}}{\diff \tau}} \diff \tau = -mc \int \sqrt{-\eta_{\mu\nu} \frac{\diff x^{\mu}}{\diff \tilde \tau} \frac{\diff x^{\nu}}{\diff \tilde \tau}} \diff \tilde \tau.
\]
so the action is reparametrization invariant. We can use this degree of freedom to fix $\frac{\diff t}{\diff \tau} = 1$, so we can always set $t = \tau$.

In this parametrization,
\begin{align*}
	S[x^\mu]_{t = \tau} &= - mc \int \sqrt{c^2 \biggl( \frac{\diff t}{\diff\tau}\biggr)^2 - \frac{\diff x^i}{\diff t} \frac{\diff x^i}{ \diff t}} \diff \tau \\
			    &= -mc^2 \int \sqrt{1 - \frac{|\mathbf{\dot x}|^2}{c^2}} \diff t,
\end{align*}
and our action reduces to the previous one. Also note that, using the covariant action,
\[
	p_\mu = c \frac{\partial L}{\partial (\partial x^\mu/\partial t)} = mc^2 \eta_{\mu \nu} \frac{\diff x^\nu}{\diff t} \frac{1}{\sqrt{-\eta_{\kappa \lambda} \frac{\diff x^{\kappa}}{\diff t} \frac{\diff x^\lambda}{\diff t}}}.
\]
The fact that we had a redundancy appear is because not all $p_\mu$ are independent:
\[
p_\mu p^\mu = m^2c^2.
\]

%lecture 8

\subsection{Charged Particle in Electromagnetic Field}
\label{sub:l_charged}

Recall we can describe an electromagnetic field in terms of a scalar potential $\phi$ and vector potential $\mathbf{A}$:
\[
\mathbf{E} = - \nabla \phi - \frac{\partial A}{\partial t}, \qquad \mathbf{B} = \nabla \times \mathbf{A}.
\]
For a particle of charge $e$ and mass $m$, moving in this background electromagnetic field, we take
\[
L = \frac{1}{2}m \mathbf{\dot x}^2 - e(\phi - \mathbf{\dot x} \cdot \mathbf{A}),
\]
in terms of $\phi$ and $\mathbf{A}$. Notice that $L$ has a term linear in velocities, so now
\[
\mathbf{p} = \frac{\partial L}{\partial \mathbf{\dot x}} = m \mathbf{\dot x} + e \mathbf{A}
\]
is the momentum in the presence of an electromagnetic field. The Euler-Lagrange equations for this Lagrangian gives:
\begin{align*}
	0 &= \frac{\diff}{\diff t} \biggl( \frac{\partial L}{\partial \mathbf{\dot x}^2} \biggr) - \frac{\partial L}{\partial x^i} \\
	  &= \frac{\diff}{\diff t} (m \mathbf{\dot x}^i + e A^i) + e \frac{\partial \phi}{\partial x^i} - e \dot x^j \partial^i A_j \\
	  &= m \ddot x^i + e \frac{\partial A^i}{\partial t} + e \dot x^j (\partial^j A_i - \partial^i A^j) + e\partial^i \phi,
\end{align*}
hence we get
\begin{align*}
	m \ddot x^i &= -e(\partial^i \phi + \partial_t A^i) + e \dot x^j (\partial^i A^j - \partial^j A^i) \\
		    &= e E^i + e(\mathbf{\dot x} \times \mathbf{B})^i,
\end{align*}
since $B^i = \eps^{ijk} \partial^j A^k$, so $\partial i - \partial j - \partial j A^i = \eps^{ijk} B^k$.

This is just the Lorentz force law, justifying our choice of $L$. Notice that $\mathbf{E}, \mathbf{B}$ are unchanged if we change
\[
\mathbf{A} \mapsto \mathbf{A} + \nabla \chi, \qquad \phi \mapsto \phi - \partial_t \chi,
\]
for any function $\chi(\mathbf{x}, t)$. These are \emph{gauge transformations}\index{gauge transformation}.

For a relativistic particle, it is natural to write $A_\mu = (-\phi/c, \mathbf{A})$. Then the coupled terms in $S$ are
\[
\int(-e\phi + e \mathbf{\dot x} \mathbf{A}) \diff t = \int e A_\mu \frac{\diff x^\mu}{\diff t} \diff t.
\]
Again, if we write this in terms of an arbitrary parameter $\tau$ on a worldline as
\[
e \int A_\mu \frac{\diff x^\mu}{\diff \tau} \diff \tau,
\]

For our Lagrangia, this gauge transformation changes
\[
L \mapsto L + e(\partial_t \chi + \mathbf{\dot x} \cdot \nabla \chi) = L + e \frac{\diff \chi}{\diff t},
\]
which is just a boundary term in the action and will not affect the equations of motions.

\subsection{Small Oscillations}
\label{sub:small_oscs}

Consider a generic $L$ of the form
\[
L = \frac{1}{2} T_{ab}(q) \dot q^a \dot q^b - V(q^a),
\]
for some generalized coordinates $q^a$. The Euler-Lagrange equations say
\[
\frac{\diff}{\diff t} (T_{ab}(q) \dot q^b) = - \frac{\partial V}{\partial q^a},
\]
so the critical points of $V$ give points of equilibrium: if we start at some critical point $q^a = q^a_\ast$ with velocity $\dot q^a$, we stay there.

We are often interested in finding out what happens if we perturb around $q^a_\ast$ slightly. We will assume that $T_{ab}(q)$ is real, symmetric and positive definite.

We will also assume our coordinates are chosen such that the critical point of $V$ is at $q^a_\ast = 0$. Then expanding $L$ to quadratic order in $(q^a, \dot q^a)$, gives
\begin{align*}
	L &\approx \frac{1}{2} T_{ab}(0) \dot q^a \dot q^b - V(0) - \frac{1}{2} \frac{\partial^2 V}{\partial q^a \partial q^b} \biggr|_0 q^a q^b + \cdots \\
	  &= \frac{1}{2} (T_{ab}(0) \dot q^a \dot q^b - V_{ab}(0)q^a q^b) + \cdots
\end{align*}
For this quadratic $L$, the Euler-Lagrange equations give
\[
T_{ab}(0) \ddot q^b - V_{ab}(0) q^b,
\]
or $\mathbf{T} \mathbf{\ddot q} = - \mathbf{V} \mathbf{q}$. Since $\mathbf{T}$ is real, symmetric and positive definite, we have
\[
\mathbf{\ddot q} = - (\mathbf{T}^{-1} \mathbf{V})\mathbf{q}.
\]
Hence if we diagonalize $\mathbf{T}^{-1}\mathbf{V}$, the motion will decompose into
\[
	\mathbf{q}(t) = \sum_n \bm{\alpha}_n(t) e^{i \lambda_n(t)},
\]
where $(\mathbf{T}^{-1}\mathbf{V})\bm{\alpha}_n = \lambda_n \bm{\alpha}_n$.

%lecture 9 by Pajer

\subsection{Here Pajer decides to ignore what Skinner did Earlier}

Recall the Lagrangian
\[
L = \frac{1}{2} \mathbf{\dot q}^{T} T \mathbf{\dot q} - \frac{1}{2} \mathbf{q}^{T} V \mathbf{q} + \mathcal{O}(\mathbf{q}^3).
\] 
Here $T$ and $V$ are $m \times m$ matrices. The equation of motion becomes
\[
T \mathbf{\ddot q} = - V \mathbf{q}.
\]
We look for solutions of the form
\[
\mathbf{q}(t) = f(t) \bm{\alpha},
\]
for $\bm{\alpha}$ a constant vector. Plugging these into the equation of motion,
\[
\ddot f(t) T \bm{\alpha} = -f(t) V \bm{\alpha}.
\]
This is only solvable if $\ddot f(t) = - \lambda f(t)$ and $\lambda T \bm{\alpha} = V \bm{\alpha}$, for some $\lambda \in \mathbb{C}$. To find $\lambda$, note
\[
	(\lambda T - V)\bm{\alpha} = 0,
\]
hence $\det(\lambda T - V) = 0$. This characteristic polynomial has $m$ complex solutions, which will be the eigenvalues of $T^{-1}V$. Call the solutions $o\lambda_n$ for $n = 1, \ldots, m$. Then the eigenvectors are
\[
	(\lambda_n T - V) \bm{\alpha}_n = 0.
\]
Each solution has time dependence
\[
\ddot f_n(t) = - \lambda_n f_n(t).
\]

These $\bm{\alpha}_n$ are called \emph{normal modes}\index{normal modes}. Now we claim that $\lambda_n \in \mathbb{R}$. Indeed, since $T$ and $V$ are real and symmetric and furthermore $T$ is positive definite, we get
\[
\bm{\alpha}_n^{\dagger}(\lambda_n T - V) \bm{\alpha}_n = 0.
\]
Since $\bm{\alpha}_n^{\dagger} T \bm{\alpha}_n$ and $\bm{\alpha}_n^{\dagger} V \bm{\alpha}_n$ are real and not equal to $0$, $\lambda_n - \lambda_n^{\ast} = 0$, meaning $\lambda_n \in \mathbb{R}$.

If $\lambda > 0$, then
\[
f(t) = A \cos(\omega t + B),
\]
with $\omega^2 = \lambda$, and two integration constants. The $\omega$ are called \emph{normal frequencies}\index{normal frequencies}.

If $\lambda_n > 0$, then we have stable equilibria. Otherwise there is $\lambda_n < 0$ which is unstable.

\begin{exbox}[3 Springs and 2 Masses]
	Consider a system, which has two masses of weight $m$ joined to a fixed surface by three springs of length $a$ and equal spring constant $k$, one of which joins the two masses together.

	We can model our system by two coordinates $(x, y)$, which describe the displacement of the first and second mass from their resting place.

	The kinetic term is
	\[
	T = \frac{1}{2} m(\dot x^2 + \dot y^2).
	\]
	The potential energy is
	\[
	V = \frac{1}{2}k(x^2 + y^2 + (y-x)^2).
	\]
	Note $x = y = 0$ is an equilibrium, and $L$ is already quadratic. We can see that
	\begin{align*}
		T &= \frac{1}{2}
		\begin{pmatrix}
			\dot x & \dot y
		\end{pmatrix}
		\begin{pmatrix}
			m & 0\\
			 0 & m
		\end{pmatrix}
		\begin{pmatrix}
			\dot x \\ \dot y
		\end{pmatrix},\\
		V &= \frac{1}{2}
		\begin{pmatrix}
			x & y
		\end{pmatrix}
		\begin{pmatrix}
			2k & -k \\
			-k & 2k
		\end{pmatrix}
		\begin{pmatrix}
			x \\ y
		\end{pmatrix}.
	\end{align*}
	The characteristic equation is
	\[
	0 = \det (\lambda T - V) = (m \lambda - k)(m\lambda - 3k).
	\]
	So solutions are
	\[
		\lambda_1 = \frac{k}{m}, \qquad \lambda_2 = \frac{3k}{m},
	\]
	with normal frequencies
	\[
		\omega_1 = \sqrt{\lambda_1} = \sqrt{\frac{k}{m}}, \qquad \omega_2 = \sqrt{\lambda_2} = \sqrt{\frac{3k}{m}},
	\]
	and normal modes
	\[
	\bm{\alpha}_1 =
	\begin{pmatrix}
		1 \\ 1
	\end{pmatrix},
	\qquad
	\bm{\alpha}_2 =
	\begin{pmatrix}
		1 \\ -1.
	\end{pmatrix}
	\]
	These are oscillations in phase and out of phase.
\end{exbox}

\begin{exbox}[Double Pendulum]
	Consider a double pendulum, where both levers have length $l$, and under the force of gravity. Then the Lagrangian is
	\begin{align*}
		L &= \frac{1}{2}m l^2 \dot \theta_1^2 + \frac{1}{2} ml^2 [\dot \theta_1^2 + \dot \theta_2^2 + 2 \cos(\theta_1-\theta_2)\dot \theta_1 \dot \theta_2] \\
		  &\qquad +mgl[\cos \theta_1 + \cos \theta_1 + \cos \theta_2].
	\end{align*}
	We expect an equilibrium point at $\theta_1 = \theta_2 = 0$. Then we expand to order two terms to get
	\begin{align*}
		V &= - mgl\biggl[2 \biggl(1 - \frac{1}{2} \theta_1^2 \biggr) + \biggl(1 - \frac{1}{2} \theta_2^2 \biggr) \biggr] \\
		  &= \frac{1}{2}
		  \begin{pmatrix}
			  \theta_1 & \theta_2
		  \end{pmatrix}
		  \begin{pmatrix}
			  2 mgl & 0 \\
			  0 & 2mgl
		  \end{pmatrix}
		  \begin{pmatrix}
		  	\theta_1 \\ \theta_2
		  \end{pmatrix},
	\end{align*}
	and also
	\[
	T = \frac{1}{2}
	\begin{pmatrix}
		\dot\theta_1 & \dot\theta_2
	\end{pmatrix}
	\begin{pmatrix}
		2ml^2 & ml^2 \\
		ml^2 & ml^2
	\end{pmatrix}
	\begin{pmatrix}
		\dot \theta_1 \\ \dot \theta_2
	\end{pmatrix}.
	\]
	Then the characteristic equation is
	\[
		0 = \det(\lambda T - V) = m^2l^{4}\biggl[2 \biggl(\lambda - \frac{g}{l} \biggr)^2 - \lambda^2 \biggr],
	\]
	which is solved by
	\[
	\lambda_1 = (2 - \sqrt 2)\frac{g}{l}, \qquad \lambda_2 = (2 + \sqrt 2)\frac{g}{l}.
	\]
	Since these are positive, this is a stable equilibrium.

	The normal modes and frequencies are
	\[
		\omega_1 = \sqrt{\lambda_1} = \sqrt{(2-\sqrt 2)\frac{g}{l}}, \qquad \bm{\alpha}_1 =
		\begin{pmatrix}
			1 \\ \sqrt 2
		\end{pmatrix},
	\]
	\[
		\omega_2 = \sqrt{(2+\sqrt 2)\frac{g}{l}}, \qquad \bm{\alpha}_2 =
		\begin{pmatrix}
			1 \\ -\sqrt 2
		\end{pmatrix}.
	\]
	Again the low frequency is in phase, and the high frequency is out of phase.
\end{exbox}

%lecture 10

\newpage

\section{Inertia}
\label{sec:inertia}

\subsection{Motion of Rigid Bodies}
\label{sub:rigid_motion}

A \emph{rigid body}\index{rigid body} is a collection of $n$ particles, whose motion is constrained such that
\[
	|\mathbf{r}_i - \mathbf{r}_j| = c_{ij} = \text{constant}.
\]
We can move the body's centre of mass, and we can rotate the body, but we cannot deform it.

It is useful to introduce two different coordinate systems. Suppose the centre of mass of a rigid body is fixed.

The \emph{space frame}\index{space frame} $\{\mathbf{\tilde e}_a\}$ are our usual fixed Cartesian basis of $\mathbb{R}^3$, and the \emph{body frame}\index{body frame} $\{\mathbf{e}_a(t)\}$ is fixed relative to the body, i.e. $\mathbf{e}_a(t)$ rotates along with the body such that each particle in the body has fixed coordinates with respect to the body frame:
\[
\mathbf{r}(t) = r_a(t) \tilde e_a = r_a \mathbf{e}_a(t).
\]
We will choose both frames to be right-handed orthonormal, so $\mathbf{e}_a(t) = R_{ab}(t) \mathbf{\tilde e}_b$ for some $R_{ab}(t)$. Then notice
\[
\delta_{ab} = \mathbf{e}_a(t) \cdot \mathbf{e}_b(t) = R_{ac} R_{bd} \mathbf{\tilde e}_c \cdot \mathbf{\tilde e}_d = R_{ac}R_{bc} = (RR^{T})_{ab},
\]
hence $R \in \mathsf{O}(3)$, and in fact $R \in \mathsf{SO}(3)$ since both frames are right-handed. We have
\[
\frac{\diff}{\diff t}\mathbf{e}_a(t) = \frac{\diff}{\diff t}(R_{ab})\mathbf{\tilde e}_b = \biggl( \frac{\diff}{\diff t} R_{ab} \biggr) R_{bc}^{T} \mathbf{e}_b(t).
\]
Differentiating the orthogonality relation $RR^{T} = 1$, we have
\[
\dot R R^{T} + R \dot R^{T} = 0,
\]
hence
\[
\Omega = \dot R R^{T} = - R \dot R^{T} = -(\dot R R^{T})^{T} = - \Omega^{T},
\]
and hence $\Omega_{ab} = - \Omega_{ba} = -\eps_{abc} \omega_c$. Therefore,
\[
\frac{\diff}{\diff t}\mathbf{e}_a(t) = \Omega_{ac} \mathbf{e}_c(t) = - \eps_{acd} \omega_d \mathbf{e}_c = (\bm{\omega} \times \mathbf{e}_a),
\]
where $\bm{\omega} = \omega_b \mathbf{e}_b(t)$ is the angular velocity (in the body frame).

For a general point $\mathbf{r}(t)$ in the body, we have
\[
\frac{\diff}{\diff t}\mathbf{r}(t) = \frac{\diff}{\diff t}(r_a \mathbf{e}_a(t)) = r_a \frac{\diff \mathbf{e}_a}{\diff t} = r_a(\bm{\omega} \times \mathbf{e}_a),
\]
hence
\[
\mathbf{\dot r} = \bm{\omega} \times \mathbf{r},
\]
in the body frame.

We can use this to express the kinetic energy of our rotation body (with a fixed centre of mass) as
\begin{align*}
	T &= \frac{1}{2} \sum_{i = 1}^{n} m_i \mathbf{\dot r}_i \cdot \mathbf{\dot r}_i \\
	  &= \frac{1}{2} \sum_{i = 1}^{n} m_i (\bm{\omega} \times \mathbf{r}_i) \cdot (\bm{\omega} \times \mathbf{r}_i) \\
	  &= \frac{1}{2} \sum_{i = 1}^{n} m_i(\bm{\omega}^2 \mathbf{r}_i^2 - (\bm{\omega} \cdot \mathbf{r}_i)^2) \\
	  &= \frac{1}{2} \sum_{i = 1}^{n} m_i \omega_a (\delta_{ab} \mathbf{r}_i^2 - (\mathbf{r}_i)_{a} (\mathbf{r}_i)_{b} ) \omega_b \\
	  &= \frac{1}{2} \omega_a I_{ab} \omega_b,
\end{align*}
where we defined
\[
I_{ab} = \sum_{i = 1}^{n}m_i(\delta_{ab} \mathbf{r}_i^2 - (\mathbf{r}_i)_a (\mathbf{r}_i)_b)
\]
as the \emph{inertia tensor}\index{inertia tensor} of the body. For continuous bodies, we instead use
\[
I_{ab} = \int_{\mathrm{body}}\rho(\mathbf{r})
\begin{pmatrix}
	y^2+z^2&-xy&-xz\\
	-xy&x^2+z^2&-yz\\
	-xz&-yz&x^2+y^2
\end{pmatrix}
\Diff3 r.
\]
We note that $I$ is a real, symmetric matrix, so it can be diagonalized. The eigenvectors of $I$ are called the \emph{principal axes}\index{principal axes} of the body. We often choose to align our body frame with the principal axes.

We also see that
\[
I_{ab} c_a c_b = \sum_i m_i (\mathbf{r}_i^2 \mathbf{c}^2 - (\mathbf{r}_i \cdot \mathbf{c})^2) = \sum_i m_i r_i^2 c^2 \sin^2 \theta_i \geq 0.
\]
In particular, choosing $\mathbf{c}$ to be an eigenvector of $I$,
\[
I_{ab}c_ac_b = \mathbf{c} \cdot (I \mathbf{c}) = \lambda \mathbf{c} \cdot \mathbf{c} \geq 0,
\]
so the eigenvalues $(I_1, I_2, I_3)$ of $I$ are non-negative.

\begin{exbox}
	Consider a uniform disc of radius $r$ with mass $M$. Let $x, y$ be aligned with the disc, and $z$ be going out of the disc perpendicularly.

	Then in this frame, $I_{ab} = \mathrm{diag}(I_1, I_2, I_3)$ and
	\[
	I_1 = I_2 = \int_{\mathrm{disc}} \frac{M}{\pi R^2} (y^2 + z^2) \diff^2 \mathbf{x} = \int_{\mathrm{disc}} \frac{M}{\pi R^2} y^2 \Diff2 \mathbf{x},
	\]
	and also
	\[
	I_3 = \int_{\mathrm{disc}} \frac{M}{\pi R^2} (x^2 + y^2) \Diff2 \mathbf{x} = I_1 + I_2,
	\]
	so $I_3 = I_1 + I_2 = 2I_1$. Now
	\[
	I_3 = \int_{\mathrm{disc}} \frac{M}{\pi R^2} r^2 \Diff2 \mathbf{x} = \frac{2M}{R^2} \int_0^R r^3 \diff r = \frac{MR^2}{2},
	\]
	hence we also get
	\[
	I_1 = I_2 = \frac{MR^2}{4}.
	\]
\end{exbox}

\subsection{Parallel Axis Theorem}
\label{sub:parallel_axis}

If our body is instead fixed at some point $p$ lying a vector $\mathbf{c}$ away from the centre of mass, then the moment of inertia around $p$ is
\[
I_{ab} = \sum_{i=1}^{n} m_i ((\mathbf{r}_i - \mathbf{c})^2 \delta_{ab}  - (\mathbf{r}_i - \mathbf{c})_a (\mathbf{r}_i - \mathbf{c})_b),
\]
where the locations $\mathbf{r}_i$ are measured with respect to the pivot $p$. Hence
\begin{align*}
	I_{ab} &= I_{ab}^{\mathrm{com}} + \sum_{i = 1}^{n} m_i (-2 \mathbf{r}_i \cdot \mathbf{c} \delta_{ab} + (\mathbf{r}_i)_a \mathbf{c}_b + \mathbf{c}_a (\mathbf{r}_i)_b) \\
	       &\qquad + \sum_{i = 1}^{n} m_i(\mathbf{c} \cdot \mathbf{c} \delta_{ab} - (\mathbf{c})_a(\mathbf{c})_b),
\end{align*}
and the middle term disappears as $\sum m_i \mathbf{r}_i = 0$. So if the pivot is displaced by $\mathbf{c}$ from the centre of mass,
\[
I_{ab}^{\mathbf{c}} = I_{ab}^{\mathrm{com}} + M(\mathbf{c} \cdot \mathbf{c} \delta_{ab} - (\mathbf{c})_a(\mathbf{c})_b).
\]

%lecture 12

\subsection{Angular Momentum}
\label{sub:ang_mom}

We can also calculate the angular momentum:

\begin{align*}
	\mathbf{L} &= \sum_i \mathbf{r}_i \times (m_i \mathbf{\dot r}_i) = \sum_i m_i \mathbf{r}_i \times (\bm{\omega} \times \mathbf{r}_i) \\
		   &= \sum_{i} m_i((\mathbf{r}_i)^2 \bm{\omega} - (\bm{\omega} \cdot \mathbf{r}_i) \mathbf{r}_i) = \mathbf{I} \bm{\omega}.
\end{align*}
In particular, if we align the body frame with the principal axes, then
\[
\begin{pmatrix}
	L_1\\L_2\\L_3
\end{pmatrix}
=
\begin{pmatrix}
	I_1\omega_1\\I_2\omega_2\\I_3\omega_3
\end{pmatrix}
.
\]
Notice that in general, $\mathbf{L} \not \propto \bm{\omega}$, so angular momentum is typically not aligned with angular velocity.

\subsection{Euler's Equations}
\label{sub:e_eq}

In the absence of torque, $\mathbf{L} = L_a(t) \mathbf{e}_a(t)$ is conserved. Therefore,
\begin{align*}
	0 &= \frac{\diff \mathbf{L}}{\diff t} = \frac{\diff L_a}{\diff t} \mathbf{e}_a + L_a \frac{\diff \mathbf{e}_a}{\diff t} \\
	  &= \frac{\diff L_a}{\diff t} \mathbf{e}_a + L_a(\bm{\omega} \times \mathbf{e}_a),
\end{align*}
co contracting with $\mathbf{e}_b$ gives
\begin{align*}
	0 &= \frac{\diff L_b}{\diff t} + L_a(\bm{\omega} \times \mathbf{e}_a)\cdot \mathbf{e}_b \\
	  &= \frac{\diff L_b}{\diff t} + L_a \bm{\omega} \cdot (\mathbf{e}_a \times \mathbf{e}_b),
\end{align*}
or in components, using $L_a = I_a \omega_a$,
\begin{align*}
	I_1 \dot \omega_1 + \omega_2 \omega_3(I_3 - I_2) &= 0,\\
	I_2 \dot \omega_2 + \omega_3 \omega_1(I_1 - I_3) &= 0,\\
	I_3 \dot \omega_3 + \omega_1 \omega_2(I_2 - I_1) &= 0.
\end{align*}
These are \emph{Euler's equations}\index{Euler's equations}.

In generally, if our object is spherically symmetric, then $I_{ab} = I \delta_{ab}$ with $I = I_1 = I_2 = I_3$. Then Euler's equations say that $\dot \omega_a = 0$, or $\bm{\dot \omega} = 0$.

In this case, the body keep rotation around the initial axis of rotation.

A \emph{symmetric top}\index{symmetric top} has $I = I_1 = I_2 \neq I_3$. Then Euler's equations become:
\begin{align*}
	I_3 \dot \omega_3 &= 0 \implies \omega_3(t) = \Omega, \\
	I \dot \omega_1 &= \omega_2 \Omega (I - I_3),\\
	I \dot \omega_2 &= \omega_1 \Omega (I_3 - I)
\end{align*}
Define the quantity
\[
\alpha = \frac{\Omega(I - I_3)}{I}.
\]
Then,
\[
\dot \omega_1 = \alpha \omega_2, \qquad \dot \omega_2 = - \alpha w_1,
\]
so $(\omega_1(t), \omega_2(t)) = \omega_0 (\sin \alpha t, \cos \alpha t)$.

Consequently, the body's rotation wobbles (or precesses) around the $\mathbf{e}_3$ axis. Hence it is normal to the plane of symmetry.

The general case $I_1 > I_2 > I_3$ cannot be solved in terms of elementary functions. However, there is a simple construction that allows us to visualise the solution.

Notice that we have two conserved quantities:
\begin{align*}
	L^2 &= L_1^2 + L_2^2 + L_3^2,\\
	2E &= \bm{\omega} \mathbf{I} \bm{\omega} = \frac{L_1^2}{I_1} + \frac{L_2^2}{I_2} + \frac{L_3^2}{I_3}.
\end{align*}

Surfaces of constant energy are ellipsoid body axes in $\mathbf{L}$ space. In this space, surfaces of constant $\mathbf{L}^2$ are sphere, so motion lies on the intersection of these surfaces.

When $L^2 = 2 EI_2$, we have
\begin{align*}
	2E I_2 &= L_1^2 + L_2^2 + L_3^2 \\
	       &= L_1^2 \frac{I_1}{I_2} + L_2^2 + L_3^2 \frac{I_2}{I_3},
\end{align*}
hence
\begin{align*}
	0 &= L_1^2 \biggl( \frac{I_1 - I_2}{I_1} \biggr) + L_3^2 \biggl( \frac{I_3 - I_2}{I_3} \biggr) \\
	\implies  L_3 &= \pm L_1 \biggl( \frac{I_2 - I_1}{I_3 - I_2} \cdot \frac{I_3}{I_1} \biggr)^{1/2},
\end{align*}
which defines two planes. We see that:
\begin{enumerate}[(i)]
	\item If we begin with $\bm{\omega}(0)$ perfectly aligned with any principal axes, it remains constant.
	\item If we perturb $\bm{\omega}$ slightly around either $\mathbf{e}_1$ or $\mathbf{e}_3$, the motion is stable; the orthogonal $\bm{\omega}$ precesses in small circles.
	\item The orbits are always closed, but perturbations around the intermediate axis $\mathbf{e}_2$ are unstable.
\end{enumerate}

%lecture 13

\subsection{Motion of a Spinning Top under Gravity}
\label{sub:top_gravity}

A top is called \emph{heavy}\index{heavy top} if we include the effect of gravity. Consider a spinning top, pivoted at a fixed point $O$.

For an asymmetric top, $E$ is conserved and $\mathbf{\tilde e}_3 \cdot \mathbf{L}$ is conserved, but there are no further conserved quantities.

However if the top is symmetric ($I_1 = I_2 \neq I_3$), there will be a third conserved quantity.

The generic case is unsolved, but the axisymmetric case is solved, and is called a \emph{Lagrangian top}\index{Lagrangian top}.

It is useful to parametrize the motion in terms of \emph{Euler angles}\index{Euler angles}. These describe a succession of three rotations that take us from the space from to the body frame.

First we apply a rotation of $\phi$ around $\mathbf{\tilde e}_3$ to take $\mathbf{\tilde e}_a \overset{R(\phi \mathbf{\tilde e}_3)}{\mapsto} \mathbf{e}_a'$.

Then applying a rotation around $\mathbf{e}_a'$ of angle $\theta$ takes $\mathbf{e}_a' \overset{R(\theta \mathbf{e}_1')}{\mapsto} \mathbf{e}_a''$.

Then $\mathbf{e}_3'' = \mathbf{e}_3$, and we then rotate by a final angle $\psi$ to get to the body frame: $\mathbf{e}_a'' \overset{R(\psi \mathbf{e}_3'')}{\mapsto} \mathbf{e}_a$.

In full, this is
\[
\mathbf{e}_a = R_{ab}(\psi \mathbf{e}_3'')R_{bc}(\theta \mathbf{e}_1') R_{cd}(\phi \mathbf{\tilde e}_3) \mathbf{\tilde e}_d,
\]
which in matrix form is written as
\[
\mathbf{e}_a =
\begin{pmatrix}
	\cos \psi & \sin \psi & 0 \\
	-\sin \psi & \cos \psi & 0 \\
	0 & 0 & 1
\end{pmatrix}
\begin{pmatrix}
	1 & 0 & 0 \\
	0 & \cos \theta & \sin \theta \\
	0 & -\sin \theta & \cos \theta
\end{pmatrix}
\begin{pmatrix}
	\cos \phi & \sin \phi & 0 \\
	-\sin \phi & \cos \phi & 0 \\
	0 & 0 & 1
\end{pmatrix}
\mathbf{\tilde e}_d.
\]
The big matrix in the middle we can write as
\[
	R(\psi, \theta, \phi) =
	\begin{pmatrix}
		\cos \psi \cos \phi - \cos \theta \sin \phi \sin \psi & \sin \phi \cos \psi + \cos \theta \sin \psi \cos \phi & \sin \theta \sin \psi \\
		-\cos \phi \sin \psi - \cos \theta \cos \psi \sin \phi & -\sin \psi \sin \phi + \cos \theta \cos \psi \cos \phi & \sin \theta \cos \psi \\
		\sin \theta \sin \phi & - \sin \theta \cos \phi & \cos \theta
	\end{pmatrix}.
\]
Note that $\mathbf{e}_1' = \cos \psi \mathbf{e}_1 - \sin \psi \mathbf{e}_3$.

In terms of the Euler angles, $\bm{\omega} = \dot R R^T$, but it is much simpler to note $\bm{\omega} = \dot \phi \mathbf{\tilde e}_3 + \dot \theta \mathbf{e}_1' + \dot \psi \mathbf{e}_3$. In the space frame,
\begin{align*}
	\bm{\omega} &= \dot \phi(\sin \theta \sin \psi \mathbf{e}_1 + \sin \theta \cos \psi \mathbf{e}_2 + \cos \theta \mathbf{e}_3) \\
		    & \qquad + \dot \theta (\cos \psi \mathbf{e}_1 - \sin \psi \mathbf{e}_2) + \dot \psi \mathbf{e}_3 \\
		    &= (\dot \phi \sin \theta \sin \psi + \dot \theta \cos \psi) \mathbf{e}_1 + (\dot \phi \sin \theta \cos \psi - \dot \theta \sin \psi) \mathbf{e}_2 + (\dot \psi + \dot \phi \cos \theta)\mathbf{e}_3.
\end{align*}

The Lagrangian of our axisymmetric heavy top is
\begin{align*}
	L &= \frac{1}{2} \omega_a I_{ab} \omega_b - mgl \cos \theta \\
	  &= \frac{1}{2} I_1(\omega_1^2 + \omega_2^2) + \frac{1}{2} I_3 \omega_3^2 - mgl \cos \theta \\
	  &= \frac{1}{2} I(\dot \theta^2 + \dot \phi^2 \sin^2 \theta) + \frac{1}{2} I_3 (\dot \psi + \dot \phi \cos \theta)^2 - mgl \cos \theta.
\end{align*}

Notice as $I_1 = I_2$, $\psi$ is ignorable, while $\phi$ is always ignorable. Hence $(p_\psi, p_\phi, E)$ are conserved for a Lagrange top:
\begin{align*}
	p_\psi &= \frac{\partial L}{\partial \dot \psi} = I_3(\dot \psi - \dot \phi \cos \theta), \\
	p_\phi &= \frac{\partial L}{\partial \dot \phi} = I_1 \dot \phi \sin^2\theta + p_\psi \cos \theta, \\
	E &= \dot q^a \frac{\partial L}{\partial \dot q^a} - L = \frac{1}{2} I(\dot \theta^2 + \sin^2 \theta \dot \phi^2) + \frac{1}{2} I_3(\dot \psi + \dot \phi \cos \theta)^2 + mgl \cos \theta \\
	  &= \frac{1}{2} I \dot \theta^2 + \frac{(p_\phi - p_\psi \cos \theta)^2}{2 I \sin^2 \theta} + \frac{p_\psi^2}{2I_3} + mgl \cos \theta \\
	  &= \frac{1}{2}I_1 \dot \theta^2 + V_{\mathrm{eff}}(\theta).
\end{align*}
We can then rewrite
\[
E' = \frac{1}{2} I \dot \theta^2 + I \frac{(b - a \cos \theta)^2}{2 \sin^2 \theta} + mgl \cos \theta,
\]
where $p_\phi = I b$, $p_\psi = Ia$. To analyse the remaining $\theta$ equation, let $u = \cos \theta$. Then this equation becomes
\[
\dot u^2 = (1 - u^2)(\alpha - \beta u) - (b - au)^2,
\]
where
\begin{align*}
	\dot \phi &= \frac{b - au}{1 - u^2}, \\
	\dot \psi &= \frac{I a}{I_3} - \frac{u(b - au)}{1-u^2}, \\
	\alpha &= \frac{2E'}{I},\\
	\beta &= \frac{2mgl}{I}.
\end{align*}

%lecture 14

Rotation in $\theta$ is called \emph{nutation}\index{nutation}, while motion in $\phi$ is called \emph{precession}\index{precession}. One can solve the remaining $u$ equation in terms of elliptic functions, but we can qualitatively understand the motion by considering
\[
f(u) = (1 - u^2)(\alpha - \beta u) - (b - a u)^2.
\]
Since $f(u) \sim \beta u^2$ for $u \gg 1$, $f$ looks like an upwards cubic.

We are only interested physically in the region $u \in [-1, 1]$ and $f(u) > 0$. Note that
\[
f(\pm 1) = -(b \mp a)^2 \leq 0,
\]
Also, we see that
\[
\dot\phi(x) =
\begin{cases}
	> 0 & u < u_c \approx b/a,\\
	< 0 & u > u_c.
\end{cases}
\]
Let $u_1, u_2$ be the smaller and larger roots of $f(u) = 0$ in $[-1,1]$. If $u_1, u_2 < u_c$, then $\dot \phi > 0$ throughout the motion.

If $u_2 = u_c$, then $\dot \phi = 0$ at $u = u_c$, and the motion has cusps. This appears fine tuned, but is not. Suppose we release the top with $\dot \phi > 0$, but $\dot \theta = \dot \phi = 0$, i.e. we spin the top, but don't push it.

Then initially $\dot \phi = 0$, so $u = u_c$, and $\dot u(0) = 0$. Thereafter, the top falls down under gravity, causing it to decrease. This force $\dot \phi$ to increase, so the top begins to precess.

If $u_1 < u_c$ but $u_2 > u_c$, then $\dot \phi$ changes sign as the top rotates.

Finally, if $f(u)$ has a repeated root $u = u_0 = [-1, 1]$, then we get precession without nutation. This is called \emph{uniform precession}\index{uniform precession}. We must have
\begin{align*}
	f(u_0) &= (1 - u_0^2)(\alpha - \beta u_0) - (b - a u_0)^2 = 0,\\
	f'(u_0) &= - 2u_0 (\alpha - \beta u_0) - \beta(1 - u_0^2) + 2 a(b - a u_0) = 0,
\end{align*}
hence we get
\[
	(\alpha - \beta u_0) = \frac{(b - au_0)^2}{1 - u_0^2} = \frac{a(b - a u_0)}{u_0} - \frac{\beta(1 - u_0)^2}{2 u_0},
\]
hence this corresponds to
\[
\dot \phi^2 = \frac{a}{u_0} \dot \phi - \frac{\beta}{2u_0}.
\]
That is, we need to give the top initial precession $\dot \phi_0$ such that
\[
U u_0 \dot \phi_0^2 - I_3 \omega_3 \dot \phi_0 + mgl = 0.
\]
This quadratic has two real roots for given $u_0$ if
\[
\Delta = I_3^2 \omega_3^2 - 4 I_3 \cos \theta_0 mgl > 0,
\]
or, rearranging,
\[
	\omega_3 > \frac{2}{I} \sqrt{I_3 mgl \cos \theta_0}.
\]
Hence uniform precession is only possible if the top spins quickly.

\subsection{Sleeping Tops}
\label{sub:sleeping_tops}

Snoozy ass tops.

If we start the top at $\theta = 0$, then $a = b$ and $\alpha = \beta$, so we get a double root at $u = 1$. Then at $u_0$,
\[
f(u_0) = (1 - u_0^2)(\alpha (1 + u_0) - a^2) = 0.
\]
The remaining root is at $u_0 = \frac{a^2}{\alpha} -1$.

\begin{itemize}
	\item If $u_0 > 1$, then motion is stable, and there isn't any region near $u = 1$ where $f(u) > 0$.
	\item If $u_0 < 1$, then motion is unstable. Perturbing the top slightly causes it to nutate.
\end{itemize}

\newpage

\section{Hamiltonian Mechanics}
\label{sec:hamilton}

This is a further reformulation of Newtonian mechanics. It allows us to:
\begin{itemize}
	\item Have still more freedom in our choice of coordinates or parametrization of the motion.
	\item Get very close to the formalism of quantum mechanics.
	\item Motivates symplectic geometry.
\end{itemize}

Recall the Euler-Lagrange equations:
\[
\frac{\diff}{\diff t} \biggl( \frac{\partial L}{\partial \dot q^a} \biggr) = \frac{\partial L}{\partial q^a}.
\]
These take the same form under general holonomic coordinate transformations:
\begin{align*}
	(q^a, t) &\mapsto (\tilde q^a(q, t), t) \\
	\dot q^a &\mapsto \dot{\tilde q}^a = \frac{\partial \tilde q^a}{\partial q^b} \dot q^b + \frac{\partial \tilde q^a}{\partial t}.
\end{align*}
However, the Euler-Lagrange equations are not preserved under more general transformations
\[
q^a \mapsto \tilde q^a(q, \dot q, t).
\]
Occasionally, such transformations are useful, and it is useful to put $q^a, \dot q^a$ on a more symmetric footing. We could define
\[
p_a = p_a(q, \dot q, t) = \frac{\partial L}{\partial q^a} \biggr|_{q, t},
\]
and write
\[
\frac{\diff p_a}{\diff t} = \frac{\partial L}{\partial q^a}, \qquad p_a = \frac{\partial L}{\partial \dot q^a}, \qquad \dot q^a = \frac{\diff q^a}{\diff t},
\]
but to reach Hamilton's equations we need to eliminate $\dot q^a$ in terms of $p_a$.

\subsection{Legendre Transforms}
\label{sub:legendre}

Consider a function $f : \mathbb{R} \to \mathbb{R}$ and let
\[
s = \partial_x f.
\]
If $\partial_x s > 0$ in some interval $[a, b] \subset \mathbb{R}$, then it is monotonic on $[a, b]$ so we can invert $s(x)$ to find $x(s)$.

We define the \emph{Legendre transform}\index{Legendre transform} of $f(x)$ to be
\[
g(s) = s x(s) - f(x(s))
\]
viewed as a function of $s$. Note that
\[
\partial_s g = x(s) + s\frac{\diff x}{\diff s} - \frac{\partial f}{\partial x} \frac{\diff x}{\diff x} = x(s).
\]
So we have $\partial_s g = x(s)$, and $\partial_x f = s(x)$, giving a symmetry between $(f, x) \leftrightarrow (g, s)$. This in particular implies that the Legendre transform of $g(s)$ is again $f(x)$.

\begin{exbox}
	If $f(x) = \frac{x^{\alpha}}{\alpha}$ with $\alpha > 1$, then
	\[
	s = f' = x^{\alpha-1},
	\]
	so $x = s^{1/(\alpha - 1)}$. Then our Legendre transform is
	\[
	g(s) = s \cdot s^{1/(\alpha - 1)} - \frac{s^{\alpha/(\alpha-1)}}{\alpha} = s^{\alpha/(\alpha-1)}\biggl(1 - \frac{1}{\alpha} \biggr) = \frac{s^{\beta}}{\beta},
	\]
	where $\frac{1}{\alpha} + \frac{1}{\beta} = 1$ and $\beta > 1$. In this case it is clear that the Legendre transform of $g$ will be $f$.
\end{exbox}

In the case of several variables $f : \mathbb{R}^{n} \to \mathbb{R}$, we define
\[
s_a = \frac{\partial f}{\partial x^a},
\]
or $\mathbf{s} = \nabla f$. The inverse function theorem guarantees that the relations $s_a = s_a(\mathbf{x})$ are invertible to give $x^a = x^a(\mathbf{s})$ in a neighbourhood of $\mathbf{x} \in \mathbb{R}^{n}$ provided that the Jacobian has positive determinant.

In this case, we define the Legendre transform of $(\mathbf{x})$ as
\[
g(\mathbf{s}) = \Biggl( \sum_{a = 1}^{n} s_a x^a(\mathbf{s}) \Biggr) - f(\mathbf{x}).
\]
Again we get that
\[
	\frac{\partial g}{\partial s_b} = x^b(\mathbf{s}) + s_a \frac{\partial x^a}{\partial s_b} - \frac{\partial f}{\partial x^a} \frac{\partial x^a}{\partial s_b} = x^b(\mathbf{s}).
\]

\subsection{Hamilton's Equations}
\label{sub:hams_eqns}

We define the \emph{Hamiltonian}\index{Hamiltonian} $H(\mathbf{q}, \mathbf{p}, t)$ to be the Legendre transform of $L(\mathbf{q}, \mathbf{\dot q}, t)$ with respect to $\mathbf{\dot q} \to \mathbf{p}$ at fixed $(\mathbf{q}, t)$. That is,
\[
H(\mathbf{q}, \mathbf{p}, t) = p_a \dot q^a - L(\mathbf{q}, \mathbf{\dot q}(\mathbf{q}, \mathbf{p}, t), t),
\]
where $p_a = \frac{\partial L}{\partial \dot q^a}$ are the usual generalized momenta.

Treating $(\mathbf{q}, \mathbf{p}, t)$ as independent, we have
\begin{align*}
	\diff H &= \frac{\partial H}{\partial q^a} \diff q^a + \frac{\partial H}{\partial p_a} \diff p_a + \frac{\partial H}{\partial t} \diff t \\
		&= \diff p_a \dot q^a + p_a \diff \dot q^a - \biggr( \frac{\partial L}{\partial q^a} \diff q^a + \frac{\partial L}{\partial \dot q^a} \diff \dot q^a + \frac{\partial L}{\partial t} \diff t \biggr) \\
		&= \dot q^a \diff p_a - \frac{\partial L}{\partial q^a} \diff q^a - \frac{\partial L}{\partial t} \diff t.
\end{align*}
Comparing coefficients, we get
\begin{align*}
	- \frac{\partial H}{\partial q^a} &= \frac{\partial L}{\partial q^a} = \dot p_a, \\
	\frac{\partial H}{\partial p_a} &= \dot q^a, \\
	\frac{\partial H}{\partial t} &= - \frac{\partial L}{\partial t},
\end{align*}
i.e. removing the unnecessary steps,
\[
\dot p_a = - \frac{\partial H}{\partial q^a}, \qquad \dot q^a = \frac{\partial H}{\partial p_a}, \qquad \biggl( \frac{\partial H}{\partial t} = - \frac{\partial L}{\partial t} \biggr).
\]
These are \emph{Hamilton's equations}\index{Hamilton's equations}.

\begin{exbox}
	Take the Lagrangian
	\[
	L = \frac{1}{2} m \dot q^a \dot q^b \delta_{ab} - V(\mathbf{q}).
	\]
	Then we get
	\[
	p_a = \frac{\partial L}{\partial \dot q^a} = m \dot q^b \delta_{ab} \implies \dot q^a = \frac{p^a}{m}.
	\]
	Therefore the Hamiltonian is
	\begin{align*}
		H(\mathbf{q}, \mathbf{p}) &= p_a \dot q^a - L = \frac{p_a p^a}{m} - \frac{p_a p^a}{2m} + V(\mathbf{q}) \\
					  &= \frac{(p_a)^2}{2m} + V(\mathbf{q}),
	\end{align*}
	as is familiar.

	Now suppose that the Lagrangian is
	\[
	L = \frac{1}{2} T_{ab}(\mathbf{q}) \dot q^a \dot q^b + C_a(\mathbf{q}) \dot q^a - V(\mathbf{q}).
	\]
	Then the momenta are
	\[
	p_a = \frac{\partial L}{\partial \dot q^a} = T_{ab} \dot q^b + C_a(\mathbf{q}),
	\]
	so we get that
	\[
	\dot q^a = (T^{-1})^{ab}(p_b - C_b),
	\]
	provided that $T^{-1}$ exists. Then
	\begin{align*}
		H(\mathbf{p}, \mathbf{q}) &= p_a \dot q^a - L \\
					  &= p_a(T^{-1})^{ab}(p_b - C_b) - \frac{1}{2}(p_a - C_a)(T^{-1})^{ab}(p_b - C_b) \\
					  & \qquad \qquad - C_a(T^{-1})^{ab}(p_b - C_b) + V(\mathbf{q}) \\
					  &= \frac{1}{2}(p_a - C_a(\mathbf{q}))(T^{-1}(\mathbf{q}))^{ab}(p_b - C_b(\mathbf{q})) + V(\mathbf{q}).
	\end{align*}
\end{exbox}

%lecture 15

\begin{exbox}
	For a charged particle in an electromagnetic field,
	\[
	L = \frac{1}{2}m \mathbf{\dot x}^2 - e (\phi - \mathbf{\dot x} \cdot \mathbf{A}),
	\]
	so we have momentum
	\[
	\mathbf{p} = \frac{\partial L}{\partial \mathbf{\dot x}} = m \mathbf{\dot x} + e \mathbf{A} \implies \mathbf{\dot x} = \frac{1}{m}(\mathbf{p} - e \mathbf{A}).
	\]
	Hence the Hamiltonian for a charged particle is
	\begin{align*}
		H &= \mathbf{p} \cdot \mathbf{\dot x} - \mathbf{L} \\
		  &= \frac{1}{m} \mathbf{p} \cdot (\mathbf{p} - e \mathbf{A}) - \frac{1}{2m} (\mathbf{p} - e \mathbf{A})^2 + e \phi - \frac{1}{m}(\mathbf{p} - e \mathbf{A})\cdot (e \mathbf{A}) \\
		  &= \frac{1}{2m} (\mathbf{p} - e \mathbf{A}) \cdot (\mathbf{p} - e \mathbf{A}) + e \phi.
	\end{align*}
	Then Hamilton's equations become
	\begin{align*}
		\mathbf{\dot x} &= \frac{1}{m}(\mathbf{p} - e \mathbf{A}), \\
		\dot p_a &= - e \frac{\partial \phi}{\partial x^a} + \frac{e}{m} (\mathbf{p} - e \mathbf{A})_b \frac{\partial A^b}{\partial x^a}.
	\end{align*}
	For example, suppose $\mathbf{E} = 0$ and $\mathbf{B} = B \mathbf{\hat z}$. Then we can choose $\phi = 0$ and $\mathbf{A} = (-By, 0, 0)$ so that $\nabla \times \mathbf{A} = \mathbf{B}$.

	Then the Hamiltonian is
	\[
	H = \frac{1}{2m}\bigl((p_x + e By)^2 + p_y^2 + p_z^2\bigr).
	\]
	Then Hamilton's equations give
	\[
	\dot p_x = 0 = \dot p_z, \qquad \dot p_y = - \frac{eB}{m}(p_x + e By).
	\]
	Then also
	\[
	m \dot x = (p_x + e By), \qquad m \dot y = p_y, \qquad m \dot z = p_z.
	\]
	This is uniform motion in the direction of $\mathbf{\hat B}$, together with
	\begin{align*}
		p_y + eBx &= a \text{ (constant)}, \\
		p_x  = m \dot x - e By &= b \text{ (constant)}.
	\end{align*}
	Thus solving gives
	\begin{align*}
		x &= \frac{a}{eB} + R \sin(\omega(t - t_0)), \\
		y &= - \frac{b}{eB} + R \cos (\omega(t - t_0)).
	\end{align*}
\end{exbox}

\subsection{Phase Space}
\label{sub:phase_space}

The $(q^a, p_a)$ are coordinates on a $2n$-dimensional space, called \emph{phase space}\index{phase space} $P$. For a particle moving in $\mathbb{R}^{n}$,
\[
P \cong \mathbb{R}^{n} \times \mathbb{R}^{n} \cong \mathbb{R}^{2n},
\]
whereas for a simple pendulum, we instead have
\[
P \cong S^1 \times \mathbb{R}.
\]
We can think of Hamilton's equations as defining a trajectory $\gamma : \mathbb{R} \to P$ in phase space by
\[
\gamma : t \mapsto (q^a(t), p_a(t)).
\]
The set $\{\gamma(t)\}$ of all possible trajectories is called the \emph{Hamiltonian flow}\index{Hamiltonian flow}.

\begin{exbox}
	Consider a free particle moving in $\mathbb{R}$. Then $\mathbb{P} \cong \mathbb{R}^2$, and $H = \frac{p^2}{2m}$. Therefore
	\[
	\dot p = 0, \qquad m \dot x = p.
	\]
	Therefore the Hamiltonian flow are all horizontal lines in phase space.

	Skinner draws some interesting diagram which I am not going to \TeX\ up. Basically if $p = 0$, then the trajectories are simply a single point.
\end{exbox}

\begin{exbox}
	Consider the one-dimensional simple harmonic oscillator, with $m = k = 1$. Then
	\[
	H = \frac{1}{2}(p^2 + x^2).
	\]
	In this case 
	\[
	\dot x = p, \qquad \dot p = -x.
	\]
	We can recognize this as circular motion in phase space, hence the Hamiltonian flow are clockwise moving circles around the origin.

	Now if we look at a simple pendulum, the Hamiltonian is
	\[
	H = \frac{p^2}{2m} - mgl \cos \theta.
	\]
	Then the resulting equations are
	\[
	\dot \theta = \frac{p}{m}, \qquad \dot p = -mgl \sin \theta.
	\]
	In this case phase space is a cylinder. Just look at the lecture or Tong's notes for this part.
\end{exbox}

Arbitrary physical observables (such as $H$, $\mathbf{L}$, etc.) can be represented by smooth functions $f : P \to \mathbb{R}$. The particular value $f(\mathbf{q}(t), \mathbf{p}(t))$ of this observable at time $t$ taken by a given particle is just this function evaluated on $\gamma$.

As time evolves,
\begin{align*}
	\frac{\diff f}{\diff t} &= \frac{\partial f}{\partial q^{a}} \frac{\partial q^a}{\partial t} + \frac{\partial f}{\partial p_a} \frac{\partial p_a}{\partial t} + \frac{\partial f}{\partial t} \\
				&= \frac{\partial f}{\partial q^a} \frac{\partial H}{\partial p_a} - \frac{\partial f}{\partial p_a} \frac{\partial H}{\partial q^a} + \frac{\partial f}{\partial t}.
\end{align*}
This motivates us to define \emph{Poisson brackets}\index{Poisson bracket}. For any pair of smooth functions $f, g : P \to \mathbb{R}$, we define
\[
	\{ f, g\} = \frac{\partial f}{\partial q^a} \frac{\partial g}{\partial p_a} - \frac{\partial f}{\partial p_a} \frac{\partial g}{\partial q^a}.
\]
Therefore
\[
	\frac{\diff f}{\diff t} = \{f, H\} + \frac{\partial f}{\partial t}.
\]
This is an analogue of Heisenberg's equation of motion.

The Poisson bracket obeys the following:
\begin{itemize}
	\item $\{f, g\} = -\{g, f\}$ (antisymmetry).
	\item $\{f, \alpha g + \beta h\} = \alpha \{f, g\} + \beta \{f, h\}$ for $\alpha, \beta \in \mathbb{R}$ (linearity).
	\item $\{f, gh\} = \{f, g\} h + g \{f, h\}$ (Leibniz rule).
	\item $\{f, \{g, h\}\} + \{g, \{h, f\}\} + \{h, \{f, g\}\} = 0$ (Jacobi identity).
\end{itemize}

\begin{exbox}
	Computing the Poisson brackets on the standard coordinates,
	\[
		\{q^a, p_b\} = \delta^a_b.
	\]
	This looks a bit like the commutation relation between $X_i$ and $P_j$ in quantum mechanics. This motivates Dirac's quantization trick: taking
	\[
		\{ \cdot, \cdot \} \to \frac{1}{i\hbar} [\cdot, \cdot]
	\]
	as a way to generalise classical dynamical equations to quantum dynamics.

	Looking at the time evolution of the standard coordinates,
	\begin{align*}
		\frac{\diff q^a}{\diff t} &= \{q^a, H\} = \frac{\partial H}{\partial p_a}, \\
		\frac{\diff p_a}{\diff t} &= \{p_a, H\} = - \frac{\partial H}{\partial q^a}.
	\end{align*}
	These align with Hamilton's equations.
\end{exbox}

An important result on Poisson brackets is as follows:
\begin{theorem}[Poisson's Theorem]
	If $f, g$ are both constants of the motion, then so too is $\{f, g\}$.
\end{theorem}
This is easily seen by taking derivatives:
\begin{align*}
	\frac{\diff}{\diff t} \{f, g\} &= \{\{f, g\}, H\} + \frac{\partial}{\partial t} \{f, g\} \\
				       &= \{\{f, H\}, g\} + \{f, \{g, H\}\} + \biggl\{ \frac{\partial f}{\partial t}, g \biggr\} + \biggl\{f, \frac{\partial g}{\partial t} \biggr\} \\
				       &= \biggl\{ \frac{\diff f}{\diff t}, g \biggr\} + \biggl\{f, \frac{\diff g}{\diff t} \biggr\} = 0.
\end{align*}
However $\{f, g\}$ is often not algebraically independent of $f, g$.

%lecture 16

Suppose $H$ has no explicit time dependence. Then,
\[
	-\{ H, \cdot\} = \frac{\partial H}{\partial p_a} \frac{\partial}{\partial q^a} - \frac{\partial H}{\partial q^a} \frac{\partial}{\partial p_a}
\]
is a vector field on $P$ that generates time evolution. We say the Hamiltonian is the \emph{generator} of time evolution.

Suppose we translate our system in configuration space, so
\[
	(q^a, p_b) \mapsto (q^a + c^a, p_b)
\]
for some constant vector $\mathbf{c}$. Then,
\[
f(\mathbf{q}, \mathbf{p}) \mapsto f(\mathbf{q} + \mathbf{c}, \mathbf{p}) = f(\mathbf{q}, \mathbf{p}) + c^a \frac{\partial f}{\partial q^a} + \mathcal{O}(\mathbf{c}^2).
\]
Hence, for infinitesimal translations in direction $\mathbf{c}$, we have
\[
	\delta f = c^a \frac{\partial f}{\partial q^a} = \{f, c^a p_a\} = c^a \{f, p_a\}.
\]
This shows that $-\{p_a, \cdot\}$ generates translations in configuration space.

We call $-\{p_a, \cdot\}$ the \emph{Hamiltonian vector field}\index{Hamiltonian vector field} associated to $p_a$. Again, $p_a$ itself is the generator of these translations.

Under an infinitesimal rotation of our system,
\begin{align*}
	\mathbf{q} &\mapsto \mathbf{q} + \delta \alpha \mathbf{n} \times \mathbf{q} + \mathcal{O}(\delta \alpha^2), \\
	\mathbf{p} &\mapsto \mathbf{p} + \delta \alpha \mathbf{n} \times \mathbf{p} + \mathcal{O}(\delta \alpha^2),
\end{align*}
with axis $\mathbf{n}$ and angle $\delta \alpha \ll 1$. Hence
\[
f(\mathbf{q}, \mathbf{p}) \mapsto f(\mathbf{q} + \delta a \mathbf{n} \times \mathbf{q}, \mathbf{p} + \delta \alpha \mathbf{n} \times \mathbf{p}),
\]
so
\begin{align*}
	\delta f &= \delta \alpha (\mathbf{n} \times \mathbf{q}) \cdot \frac{\partial f}{\partial \mathbf{q}} + \delta \alpha (\mathbf{n} \times \mathbf{p}) \cdot \frac{\partial f}{\partial \mathbf{p}} \\
		 &= \delta \alpha \mathbf{n} \cdot \biggl( \mathbf{q} \times \frac{\partial f}{\partial \mathbf{q}} + \mathbf{p} \times \frac{\partial f}{\partial \mathbf{p}} \biggr) \\
		 &= \delta \alpha \{f, \mathbf{n} \cdot \mathbf{L}\},
\end{align*}
where $\mathbf{L} = \mathbf{q} \times \mathbf{p}$. Thus $\mathbf{n} \cdot \mathbf{L}$ is the generator of rotations around the $\mathbf{n}$ axis.

To any smooth function $Q(\mathbf{q}, \mathbf{p})$ we can associate a Hamiltonian field $- \{Q, \cdot\}$ which tells us how the system changes under a transformation generated by $Q$.
\begin{theorem}[Noether's Theorem]
	Suppose that
	\[
	\frac{\partial Q}{\partial t} = \frac{\partial H}{\partial t} = 0.
	\]
	Then $Q$ is conserved if and only if
	\[
		0 = \frac{\diff Q}{\diff t} = \{Q, H\} = \frac{\diff H}{\diff \alpha},
	\]
	where $\alpha$ is a parameter along the integral curves of $-\{Q, \cdot\}$.

	In other words, symmetries of $H$ correspond to conserved quantities $Q$.
\end{theorem}

\begin{exbox}
	If $H$ is invariant under rotations, i.e. $\{H, \mathbf{L}\} = 0$, then $\mathbf{L}$ is conserved.

	If $H$ is invariant under translations of configuration space, i.e. $\{H, \mathbf{p}\} = 0$, then $\mathbf{p}$ is conserved.

	We always have $\{H, H\} = 0$ if $H$ is independent of $t$, so the energy is always conserved.
\end{exbox}

\subsection{Dirac's Quantization}
\label{sub:dirac_quant}

Suppose $f, g, h \in C^{\infty}(P)$ such that $\{f, g\}= h$. Dirac thought that quantizing this classical system meant finding self-adjoint operators $\hat f, \hat g, \hat h$ on $L^2(\mathbb{C}, \diff^n q)$ such that
\[
	[\hat f, \hat g] = i \hbar \hat h.
\]
However,
\begin{theorem}[Groenwald-Van Hove]
	This is impossible in general.
\end{theorem}

\subsection{Liouville's Theorem}
\label{sub_lou_thm}

Suppose a system evolves in time via Hamilton's equations. Then after a short time $\delta t$,
\[
	(q^a(t), p_b(t)) \mapsto (\tilde q^a(t), \tilde p_b(t)) = \biggl( q^a + \delta t \frac{\partial H}{\partial p_a}, p_b - \delta t \frac{\partial H}{\partial q^b} \biggr) + \mathcal{O}(\delta t^2).
\]
Suppose initially our system occupies some region $R_0 \subset P$, that evolves to a region $R_t$ under Hamiltonian flow. Let
\begin{align*}
	V &= \mathrm{Vol}(R_0) = \delta^n q\, \delta^n p \\
	  &= \delta q^1 \cdots \delta q^n \delta p_1 \cdots \delta p_n.
\end{align*}
After a short time $\delta t$, the region $R_{\delta t}$ has volume
\[
\tilde V = \delta^n \tilde q\, \delta^n \tilde p = |J| \delta^n q\, \delta^n p,
\]
where
\[
J =
\begin{pmatrix}
	\partial \tilde q^a/\partial q^b & \partial \tilde q^a/\partial p_b \\
	\partial \tilde p_a/\partial q^b & \partial \tilde p_a/\partial p_b
\end{pmatrix}
\]
is the Jacobian matrix. We have that
\begin{align*}
	J &=
\begin{pmatrix}
	\delta^a_b + \delta t \, \partial^2 H/\partial q^b \partial p_a & \delta t \, \partial^2 H/ \partial p_b \partial p_a \\
	-\delta t \, \partial^2 H/\partial q^b \partial q^a & \delta_a^b - \delta t \, \partial^2 H/\partial q^a \partial p_b
\end{pmatrix}
+ \mathcal{O}(\delta t^2) \\
	  &= I + \delta t \, \delta J + \mathcal{O}(\delta t^2).
\end{align*}
Therefore, since $|J| = 1 + \delta t \tr(\delta J) + \mathcal{O}(\delta t^2)$, we have
\begin{align*}
	\tilde V &= V \biggl(1 + \delta t \biggl( \frac{\partial^2 H}{\partial q^a \partial p_a} - \frac{\partial^2 H}{\partial q^a \partial p_a} \biggr) \biggr) \\
		 &= V.
\end{align*}
This is Liouville's theorem\index{Liouville's theorem}: the volume of any region $R \subset P$ is preserved by time evolution.

For example, we could have a phase space probability distribution $\rho(\mathbf{q}, \mathbf{p}, t)$ normalised so that
\[
\int_P \rho(\mathbf{q}, \mathbf{p}, t) \diff^n \mathbf{q} \diff^n \mathbf{p} = 1,
\]
where $\rho$ represents the probability density to find a particle at $(\mathbf{q}, \mathbf{p}) \in P$ at time $t$.

Then Liouville's theorem says that
\begin{align*}
	0 &= \frac{\diff}{\diff t} \int_P \rho(\mathbf{q}, \mathbf{p}, t) \diff^n \mathbf{q} \diff^n \mathbf{p} \\
	  &= \int_P \biggl( \frac{\partial \rho}{\partial q^a} \dot q^a + \frac{\partial \rho}{\partial p_a} \dot p_a + \frac{\partial \rho}{\partial t} \biggr) \diff^n q \diff^n p \\
	  &= \int_P \biggl( \{\rho, H\} + \frac{\partial \rho}{\partial t} \biggr) \diff^n q \diff^n p.
\end{align*}
If particles are neither created nor destroyed, then Liouville's theorem holds locally, so we have a continuity equation
\[
	\frac{\partial \rho}{\partial t} = \{- \rho, H\}.
\]
%lecture 17
This can be viewed as
\[
\frac{\partial \rho}{\partial t} + \nabla \cdot \mathbf{j} = 0,
\]
where $\mathbf{j}$ is the probability current on the phase space, i.e. $\mathbf{j} = (\rho \dot q^a, \rho \dot p_a)$. Then,
\begin{align*}
	\frac{\partial \rho}{\partial t} + \nabla \cdot \mathbf{j} &= \frac{\partial \rho}{\partial t} + \frac{\partial}{\partial q^a}(\rho \dot q^a) + \frac{\partial}{\partial p_a}(\rho \dot p_a) \\
								   &= \frac{\partial \rho}{\partial t} + \frac{\partial}{\partial q^a} \biggl( \rho \frac{\partial H}{\partial p_a} \biggr) - \frac{\partial}{\partial p_a} \biggl( \rho \frac{\partial H}{\partial q^a} \biggr) \\
								   &= \frac{\partial \rho}{\partial t} + \frac{\partial \rho}{\partial q^a} \frac{\partial H}{\partial p_a} - \frac{\partial \rho}{\partial p_a} \frac{\partial H}{\partial q^a} + \rho \biggl( \frac{\partial^2 H}{\partial q^a \partial p_a} - \frac{\partial^2 H}{\partial q^a \partial p_a} \biggr) \\
								   &= \frac{\partial \rho}{\partial t} + \{\rho, H\} = 0.
\end{align*}

\subsection{Bohr-van Leeuwen Theorem}
\label{sub:bvl_thm}

Often we are intersected in time-independent distributions. For example, if a system of particles is constrained to have total energy $E$, i.e.
\[
\int H(\mathbf{q}, \mathbf{p}) \rho (\mathbf{q}, \mathbf{p}) \diff^{3N} q \diff^{3N} p = E,
\]
but is otherwise free, then it is governed by the Boltzmann distribution
\[
\rho(\mathbf{q}, \mathbf{p}) = \frac{e^{-H(\mathbf{q}, \mathbf{p})/kT}}{Z(E)},
\]
where $T$ is temperature, and $k, Z(E)$ are constants.

If the particles are free within the box, then
\[
H(\mathbf{q}, \mathbf{p}) = \sum_{i = 1}^{N} \frac{p_i^2}{2m}.
\]
If the particles are electrically charges and the system is subject to an external magnetic field, then
\[
H(\mathbf{q}, \mathbf{p}) = \sum_i \frac{(\mathbf{p}_i - e \mathbf{A}(\mathbf{q}_i))^2}{2m}.
\]
The magnetic moment $\bm{\mu}_i$ due to a charged particles is
\[
\bm{\mu}_i = \mathbf{q}_i \times \mathbf{j}_i = e (\mathbf{q} \times \mathbf{\dot q})_i,
\]
so for the whole system the total magnetic moment in some direction $\mathbf{n}$ is
\[
\mathbf{n} \cdot \bm{\mu} = e \sum_i \mathbf{n} \cdot (\mathbf{q}_i \times \mathbf{\dot q}_i) = e \sum_i (\mathbf{n} \times \mathbf{q}_i) \cdot (\mathbf{p}_i - e \mathbf{A}(\mathbf{q}_i)).
\]
Hence on average, we have
\begin{align*}
	\langle \mathbf{n} \cdot \bm{\mu} \rangle &= \int \mathbf{n} \cdot \bm{\mu} \rho(\mathbf{q}_i, \mathbf{p}_i) \diff^{3N} q \diff^{3N} p \\
						  &= \frac{e}{Z(E) m} \int \sum_i (\mathbf{n} \times \mathbf{q}_i) \cdot (\mathbf{p}_i - e \mathbf{A}(\mathbf{q}_i)) e^{\sum (\mathbf{p}_i - e \mathbf{A}_i)^2/2mkT} \diff^{3N} q \diff^{3N} p \\
						  &= 0,
\end{align*}
since this integral is an odd function of $\mathbf{\dot q} = \mathbf{p}_i - e \mathbf{A}_i$, Hence there cannot be net magnetic moments in classical dynamics. This is the Bohr-van Leeuwen theorem\index{Bohr-van Leeuwen theorem}, and shows that magnetism is a quantum property.

\subsection{Poincar\'e Recurrence}
\label{sub:pc_recur}

Suppose our system is confined to a bounded region of $P$. For example, if the system has fixed energy $E$ with $T \geq 0$ and $V$ bounded from below, then it cannot escape to regions where $V(\mathbf{q}) > E$.

Suppose our system is at some point $(\mathbf{q}_0, \mathbf{p}_0) \in P$ at $t = 0$. Then, for any open region $D_0 \subset P$ containing $(\mathbf{q}_0, \mathbf{p}_0)$, there exists some point $(\mathbf{q}_0', \mathbf{p}_0') \in D_0$ which eventually returns to $D_0$.

\begin{proofbox}
	Let $D_k$ be the time evolution of $D_0$ after time $kT$, for $k \in \mathbb{N}$. By Liouville's theorem,
	\[
	\mathrm{Vol}(D_k) = \mathrm{Vol}(D_0).
	\]
	Hence, if $D_k \cap D_{k'} = \emptyset$ for all $k, k'$, then
	\[
	\mathrm{Vol} \Biggl( \bigcup_{k = 0}^{n} D_k \Biggr) = \sum_{k = 0}^{n} \mathrm{Vol}(D_k) = (n+1) \mathrm{Vol}(D_0).
	\]
	As $n \to \infty$, this diverges, contradicting our assumption that the system could only explore a bounded region of $P$. Hence there exists some $D_k, D_{k'}$ with $D_k \cap D_{k'} \neq \emptyset$.

	As Hamiltonian evolution is reversible, we can trace these regions back to find $D_0 \cap D_{(k' - k)} \neq \emptyset$.
\end{proofbox}

This is the \emph{Poincar\'e recurrence theorem}\index{Poincar\'e recurrence}. Now Skinner draws some boxes with points in them; imagine I \TeX'd them up.

\subsection{Canonical Transformations}
\label{sub:can_trans}

Because $(\mathbf{q}, \mathbf{p})$ are on a more symmetric footing in Hamilton's equations, we can do more general coordinate transformation that just holonomic transforms $q^a \to \tilde q^a(q, t)$ that preserved the Euler-Lagrange equations.

Let $y^\alpha= (q^a, p_b)$ be $2n$ coordinates on $P$. A transformation $y^\alpha \mapsto Y^\alpha =  Y^\alpha(y, t)$ is called \emph{canonical}\index{canonical transformation} if it leaves Poisson brackets unchanged, in the sense that
\[
	\{f, g\}_y = \{f, g\}_Y,
\]
for all $f, g \in C^{\infty}(P)$. To understand this, note that
\[
	\{f, g\}_y = \frac{\partial f}{\partial q^a} \frac{\partial g}{\partial p_a} - \frac{\partial f}{\partial p_a} \frac{\partial g}{\partial q^a} = \frac{\partial f}{\partial y^\alpha} \Omega^{\alpha \beta} \frac{\partial g}{\partial y^\beta},
\]
where
\[
\Omega^{\alpha\beta} =
\begin{pmatrix}
	0 & \id \\
	-\id & 0
\end{pmatrix}.
\]
%lecture 18
Note that $\Omega^2 = -\id$, and $\Omega^{T} = -\Omega$. Viewing as $f(y(Y))$, we have
\[
\frac{\partial f}{\partial y^{\alpha}} = \frac{\partial f}{\partial Y^{\gamma}} \frac{\partial Y^{\gamma}}{\partial y^{\alpha}},
\]
so the Poisson bracket is
\[
	\{f, g\}_y = \frac{\partial \phi}{\partial Y^{\gamma}} \frac{\partial Y^{\gamma}}{\partial y^{\alpha}} \Omega^{\alpha \beta} \frac{\partial Y^{\delta}}{\partial y^{\beta}} \frac{\partial g}{\partial Y^{\delta}}.
\]
This is equal to $\{f, g\}_Y$ if and only if
\[
\frac{\partial Y^{\gamma}}{\partial y^{\alpha}} \Omega^{\alpha \beta} \frac{\partial Y^{\delta}}{\partial y^{\beta}} = \Omega^{\gamma \delta},
\]
or equivalently $J \Omega J^{T} = \Omega$, with
\[
J^{\gamma}_\alpha = \frac{\partial Y^{\gamma}}{\partial y^{\alpha}}.
\]
The forward direction is easy to see:
\[
	\{f, g\}_y = \frac{\partial f}{\partial Y^{\gamma}} \biggl( \frac{\partial Y^{\gamma}}{\partial y^{\alpha}} \Omega^{\alpha \beta} \frac{\partial Y^{\delta}}{\partial y^{\beta}} \biggr)  \frac{\partial g}{\partial Y^{\delta}} = \frac{\partial f}{\partial Y^{\gamma}} \Omega^{\gamma \delta} \frac{\partial g}{\partial Y^{\delta}} = \{f, g\}_Y.
\]
For the other direction, choose $f = Y^{\gamma}(y, t)$ and $g = Y^{\delta}(y, t)$. Then
\[
	\{Y^{\gamma}, Y^{\delta}\}_y = \frac{\partial Y^{\gamma}}{\partial y^{\alpha}} \Omega^{\alpha \beta} \frac{\partial Y^{\delta}}{\partial y^{\beta}} = \{Y^{\gamma}, Y^{\delta}\}_Y = \Omega^{\gamma \delta}.
\]
At each point $y \in P$, the matrix $J(y) \in \mathsf{GL}(2n, \mathbb{R})$. For canonical transformations, $J \Omega J^{T} = \Omega$. So $J \in \mathsf{Sp}(2n, \mathbb{R}) \subset \mathsf{GL}(2n, \mathbb{R})$, where $\mathsf{Sp}(2n, \mathbb{R})$ is called the \emph{symplectic group}\index{symplectic group}.

In the original coordinates, we have
\[
J^{\alpha}_\beta = \frac{\partial Y^{\alpha}}{\partial y^{\beta}} =
\begin{pmatrix}
	\partial Q^{a}/\partial q^b & \partial Q^a/\partial p_b \\
	\partial P_a/\partial q^b & \partial P_a/\partial p_b,
\end{pmatrix},
\]
and the condition $J \Omega J^{T} = \Omega$ becomes
\[
\begin{pmatrix}
	\{Q^a, Q^b\}_{(q, p)} & \{Q^a, P_b\}_{(q, p)} \\
	\{P_a, Q^b\}_{(q, p)} & \{P_a, P_b\}_{(q, p)}
\end{pmatrix}
=
\begin{pmatrix}
	0 & \delta^{a}_b \\
	-\delta_a^b & 0
\end{pmatrix}.
\]

\begin{exbox}
	Suppose we swap our positions and momenta, by defining
	\[
	\mathbf{Q} = \mathbf{p}, \qquad \mathbf{P} = - \mathbf{q}.
	\]
	Then,
	\[
	J =
	\begin{pmatrix}
		\partial Q/\partial q & \partial Q/\partial p \\
		\partial P/\partial q & \partial P/\partial p
	\end{pmatrix}
	=
	\begin{pmatrix}
		0 & \delta^a_b \\
		-\delta^a_b & 0
	\end{pmatrix} = \Omega,
	\]
	in this case. Then since
	\[
	J \Omega J^{T} = \Omega \Omega \Omega^{T} = \Omega,
	\]
	this transformation is canonical. Hence it is really just a matter of convention what we think of as a position coordinate, and what we think of as a momentum coordinate.
\end{exbox}

We say coordinates $(q^a, p_b)$ obeying usual Poisson bracket relations form a \emph{canonically conjugate pair}\index{canonically conjugate pair}.

\subsection{Hamilton's Equations and Canonical Transformations}
\label{sub:ham_eq_can_trans}

Hamilton's equations say that
\[
	\frac{\diff y^\alpha}{\diff t} = \Omega^{\alpha \beta} \frac{\partial H}{\partial y^{\beta}}.
\]
Under a canonical transformation $y^\alpha \mapsto Y^{\alpha}(y, t)$, we have
\begin{align*}
	\frac{\diff Y^{\alpha}}{\diff t} &= \frac{\partial Y^{\alpha}}{\partial y^{\beta}} \frac{\diff y^{\beta}}{\diff t} + \frac{\partial Y^{\alpha}}{\partial t} \\
					 &= \frac{\partial Y^{\alpha}}{\partial y^{\beta}} \Omega^{\beta \gamma} \frac{\partial H}{\partial y^{\gamma}} + \frac{\partial Y^{\alpha}}{\partial t}.
\end{align*}
Then since $J \Omega J^{T} = \Omega$, we have $J \Omega = \Omega (J^{T})^{-1}$, so
\[
\frac{\diff Y^{\alpha}}{\diff t} = \Omega^{\alpha \beta} \frac{\partial y^{\gamma}}{\partial Y^{\beta}} \frac{\partial H}{\partial y^{\gamma}} + \frac{\partial Y^{\alpha}}{\partial t} = \Omega^{\alpha \beta} \frac{\partial H}{\partial Y^{\beta}} + \frac{\partial Y^{\alpha}}{\partial t}.
\]
Hence for canonical transformations with no explicit $t$ dependence, we have
\[
\dot y^{\alpha} = \Omega^{\alpha \beta} \frac{\partial H}{\partial y^{\beta}} \iff \dot Y^{\alpha} = \Omega^{\alpha \beta} \frac{\partial H}{\partial Y^{\beta}}.
\]
\subsection{Generating Transformation}
\label{sub:gen_trans}

There are several ways to construct coordinate transformations that are generated to be canonical.

Recall we defined a Hamiltonian vector field associate to $f(\mathbf{q}, \mathbf{p})$ as
\[
	\{f, \cdot\} = \mathcal{D}_f.
\]
This obeys properties following from properties of the Poisson bracket:
\begin{itemize}
	\item $\mathcal{D}_f(g) = \{f, g\} = - \mathcal{D}_g(f)$.
	\item $\mathcal{D}_{(f_1+f_2)}(g) = \mathcal{D}_{f_1}(g) + \mathcal{D}_{f_2}(g)$.
	\item $\mathcal{D}_f(g_1+g_2) = \mathcal{D}_f(g_1) + \mathcal{D}_f(g_2)$.
	\item $\mathcal{D}_f(gh) = \mathcal{D}_f (g)h + g \mathcal{D}_f(h)$.
	\item $[\mathcal{D}_f, \mathcal{D}_g] = \mathcal{D}_{\{f, g\}}$:
		\begin{align*}
			[\mathcal{D}_f, \mathcal{D}_g] h &= \mathcal{D}_f(\mathcal{D}_g h) - \mathcal{D}_g(\mathcal{D}_f h) = \{f, \{g, h\}\} - \{g, \{f, h\}\} \\
							 &= \{\{f, g\}, h\} = \mathcal{D}_{\{f, g\}} h.
		\end{align*}
\end{itemize}
To construct a canonical transformation, pick any $f \in C^{\infty}(P)$. Then define $Y^{\alpha}(y, s)$ by
\[
Y^{\alpha}(y, s) = e^{-s \mathcal{D}_f}(y^{\alpha}) = \sum_{n = 0}^{\infty} \frac{(-s)^n}{n!} (\mathcal{D}_f)^n y.
\]
This will be a canonical coordinate transformation for any $s \in \mathbb{R}$. Indeed, notice
\begin{align*}
	\frac{\partial}{\partial s} Y^{\alpha}(y, s) &= - \sum_{n = 1}^{\infty} \frac{(-s)^{n-1}}{(n-1)!} (\mathcal{D}_f)^{n} y^{\alpha} = - \mathcal{D}_f Y^{\alpha}(y, s) \\
						     &= - \{f, Y^{\alpha}(y, s)\} = \{Y^{\alpha}(y, s), f\}.
\end{align*}
Also $Y^{\alpha}(y, 0) = y^{\alpha}$. Hence.
%lecture 19
\begin{align*}
	\frac{\partial}{\partial s} \{Y^\alpha, Y^\beta\} &= \biggl\{ \frac{\partial Y^{\alpha}}{\partial s}, Y^\beta \biggr\} + \biggl\{ Y^\alpha, \frac{\partial Y^\beta}{\partial s} \biggr\} \\
							  &= \{\{Y^\alpha(y, s), f\}, Y^\beta\} + \{Y^\alpha, \{Y^\beta(y, s), f\}\} \\
							  &= \{\{Y^\alpha(y, s), Y^\beta(y, s)\}, f\}.
\end{align*}
This has the same form, now for the function $\{Y^\alpha, Y^\beta\}(y, s)$. Hence we can solve this first order PDE as
\begin{align*}
	\{Y^\alpha(y, s), Y^\beta(y, s)\} &= e^{-s \mathcal{D}_f}\{Y^\alpha(y, 0), Y^\beta(y, 0)\} \\
					  &= e^{-s \mathcal{D}_f} \{y^\alpha, y^\beta\} = e^{-s \mathcal{D}_f}(\Omega^{\alpha\beta}) = \Omega^{\alpha\beta}.
\end{align*}
Hence $\{Y^\alpha, Y^\beta\} = \Omega^{\alpha\beta}$ if $Y^\alpha = e^{-s \mathcal{D}_f}(y^\alpha)$, so the transformation is canonical.

\begin{exbox}
	Let's choose $P = \mathbb{R}^2 \ni (q, p)$ and pick $f = ap$ for a constant $a$. Then
	\[
		\mathcal{D}_f = \{f, \cdot\} = - a \frac{\partial}{\partial q},
	\]
	so we have
	\begin{align*}
		Q(q, p) &= e^{-s \mathcal{D}_f}(q) = e^{s a \frac{\partial}{\partial q}}(q) = q + sa, \\
		P(q, p) &= e^{-s \mathcal{D}_f}(p) = e^{sa \frac{\partial}{\partial q}}(p) = p.
	\end{align*}
	So $f = ap$ generated translations, which are canonical coordinate transformations.
\end{exbox}

In particular, if we chose $f(q, p) = H(q, p)$ to be the system's Hamiltonian, then the transformation
\[
Q(q, p, t) = e^{-t \mathcal{D}_H}(q), \qquad P(q, p, t) = e^{-t \mathcal{D}_H}(p)
\]
is canonical. Since $\frac{\partial Y^\alpha}{\partial t} = \{Y^\alpha(y, t), H\}$, this is just time evolution by Hamilton's equations.

\begin{exbox}
	Consider a simple harmonic oscillator with $m = 1 = \omega$, so
	\[
	H = \frac{1}{2}(p^2 + q^2).
	\]
	Then we have
	\[
	\mathcal{D}_h = q \frac{\partial}{\partial p} - p \frac{\partial}{\partial q},
	\]
	and under time evolution/Hamiltonian flow,
	\begin{align*}
		Q(q, p, t) &= e^{-t \mathcal{D}_H}(q) = \sum_{n = 0}^{\infty} \frac{(-t)^n}{n!}\bigl(q \frac{\partial}{\partial p} - p \frac{\partial}{\partial q}\biggr)^{n} q \\
			   &= \sum_{n = 0}^{\infty} \biggl( \frac{(-t)^n}{(2n)!}(-1)^n q + \frac{(-t)^{2n+1}}{(2n+1)!}(-1)^{n+1}p \biggr) \\
			   &= q \cos t + p \sin t.
	\end{align*}
	Similarly we can show $P(q, p, t) = p \cos t - q \sin t$, which are indeed a new pair of canonically conjugate coordinates on $P$.
\end{exbox}

\subsection{Hamilton's Principle of Least Action}
\label{sub:ham_least_action}

Another way to generate canonical coordinate transformations uses the Hamiltonian versions of the principle of least action.

We view the action as a functional $S = S[\mathbf{q}, \mathbf{p}]$ and write
\[
S = \int_{t_0}^{t_1} L(q^a, \dot q^b(q, p), t) \diff t= \int_{t_0}^{t_1} p_a \dot q^a - H(q, p) \diff t.
\]
Extremizing with respect to changes of the curve in phase space, we have
\begin{align*}
	0 &= \delta S = \int_{t_0}^{t_1} \biggl[ \delta p_a \dot q^a + p_a \delta \dot q^a - \frac{\partial H}{\partial q^a} \delta q^a - \frac{\partial H}{\partial p_a} \delta p_a \biggr] \diff t\\
	  &= \int_{t_0}^{t_1} \biggl( \dot q^a - \frac{\partial H}{\partial p_a} \biggr) \delta p_a - \biggl( \dot p_a + \frac{\partial H}{\partial q^a} \biggr) \delta q^a \diff t + p_a \delta q^a \biggr|_{t_0}^{t_1}.
\end{align*}
So if $\delta q^a(t_0)=  \delta q^a(t_1) = 0$, we extremize by setting
\[
\dot q^a = \frac{\partial H}{\partial p_a}, \qquad \dot p_a = - \frac{\partial H}{\partial q^a},
\]
which are Hamilton's equations. However, there's more freedom, because now we are extremizing over curves in phase space. Since all our curves begin and end at the same two points in phase space, we also have $\delta p_a(t_0) = \delta p_a(t_1) = 0$.

Thus we can add the total time derivative of any $F \in C^{\infty}(P)$, i.e.
\[
	S \mapsto S + \int_{t_0}^{t_1} \frac{\diff}{\diff t}F(q, p, t) \diff t.
\]
This will not change the form of the equations of motion.

Suppose we have two coordinate systems $(q^a, p_b)$ and $(Q^a, P_b)$ on $P$, with Hamiltonians $H(q, p)$ and $K(Q, P) = H(q(Q,P), p(Q, P))$. These lead to the same equation if motion if
\[
P_a \diff Q^a - K \diff t = p_a \diff q^a - H \diff t - \diff F_1(q, Q).
\]
Here $F_1$ is called a \emph{generating function of the first kind}\index{generating function of the first kind}. We have
\[
	\diff F_1 = \frac{\partial F_1}{\partial q_a} \diff q^a + \frac{\partial F_1}{\partial Q^a} \diff Q^a,
\]
so comparing we have
\[
	\biggl( P_a + \frac{\partial F_1}{\partial Q^a} \biggr) \diff Q^a + (H-K) \diff t = \biggl( p_a - \frac{\partial F_1}{ \partial q^a} \biggr) \diff q^a.
\]
For these to hold for arbitrary variations, we should define
\[
P_a = - \frac{\partial F_1}{\partial Q^a}, \qquad p_a = \frac{\partial F_1}{\partial q^a},
\]
and $K = H$. Hence $p_a = \frac{\partial F_1}{\partial q} = p_a(q, Q)$, so we can invert to get $Q = Q(q,p)$.

%lecture 20

\begin{exbox}
	Let's choose $F_1 = q^aQ^a$. Then,
	\[
	P_a = - \frac{\partial F_1}{\partial Q^a} = -q^a, \qquad p_a = \frac{\partial F}{\partial q^a} = Q^a,
	\]
	i.e. $(Q^a, P_b) = (p_a, -q^b)$, so this switches the momentum coordinates.
\end{exbox}

Instead of viewing our generating function $F$ as depending on both sets of coordinates, it is sometimes helpful to think of $\tilde F_2 = \tilde F_2(q, P)$, in which case it is called a \emph{generating function of the second kind}\index{generating function of the second kind}. Hence we have
\[
\diff \tilde F_2 = \frac{\partial \tilde F_2}{\partial q^a} \diff q^a + \frac{\partial \tilde F_2}{\partial P_b} \diff P_b,
\]
and collecting terms now gives
\[
P_a \diff Q^a - \frac{\partial \tilde F_2}{\partial P_a} \diff P_a - K \diff t = \biggl( p_a - \frac{\partial \tilde F_2}{\partial q^a} \biggr) \diff q^a - H \diff t,
\]
and hence
\[
\diff (P_a Q^a) - \biggl( Q^a + \frac{\partial \tilde F_2}{\partial P_a} \biggr) \diff P_a = \biggl( p_a - \frac{\partial \tilde F_2}{\partial q^a} \biggr) \diff q^a - (H-K) \diff t.
\]
We can now define $F_2(q,P) = \tilde F_2 + P_aQ^a$. Then
\[
p_a = \frac{\partial F_2}{\partial q^a}, \qquad Q^a = \frac{\partial F_2}{\partial P_a},
\]
and $H = K$.
\begin{exbox}
	Consider the generating function
	\[
		F_2(q,P) = \int^q\sqrt{2P - u^2}\diff u,
	\]
	and hence
	\[
		p = \frac{\partial F_2}{\partial q} = \sqrt{2P - q^2}, \qquad Q = \frac{\partial F_2}{\partial P} = \int^q \frac{\diff u}{\sqrt{2P - u^2}} = \tan^{-1} \biggl( \frac{q}{\sqrt{2P - q^2}} \biggr),
	\]
	hence
	\[
		q = \sqrt{2P} \sin Q, \qquad p =\sqrt{2P} \cos Q.
	\]
	In particular, this transformation is useful for the simple harmonic oscillator, since
	\[
	H_{\mathrm{SHO}}(q, p) = \frac{1}{2}(p^2 + q^2) = P,
	\]
	so that in the now coordinates $K(Q,P) = P$. Hence in the $(Q,P)$ coordinates, Hamilton's equations becomes
	\[
	\dot Q = \frac{\partial K}{\partial P} = 1, \qquad \dot P = - \frac{\partial K}{\partial Q} = 0,
	\]
	so $P(t)$ is constant and $Q(t) = (t - t_0)$.
\end{exbox}

\subsection{Integrable Systems}
\label{sub:int_sys}

In the previous example, we found a coordinate transformation $(q^a, p_b) \to (\theta^a, I_b)$ such that in the new coordinates, $H = H(I)$.

These are called \emph{action-angle coordinates}\index{action-angle coordinates}. Any system in which such a transformation is possible is called \emph{integrable}\index{integrable system}.

A theorem of Liouville says that such a transformation is possible if and only if there exists $n$ algebraically independent quantities $I_a : P^{2n} \to \mathbb{R}$ that obey $\{I_a, I_b\} = 0$. These constants are said to be \emph{in involution}.

In action-angle coordinates, where $H = H(I)$, we will trivially have
\[
\dot I_a = - \frac{\partial H}{\partial \theta^a} = 0,
\]
so $I_a$ is constant, while
\[
\dot \theta^a = \frac{\partial H}{\partial I_a} = \omega_a,
\]
which is also constant, hence $\theta^a(t) = \omega^a(t - t_0)$. That is, we will have straightened out the flow in $P$.

Integrable systems are very special, and usually are hard to find. However, every one-dimensional system whose Hamiltonian obeys $\frac{\partial H}{\partial t} = 0$ is integrable.

\begin{exbox}
	Suppose that
	\[
	H(q,p) = \frac{p^2}{2m} + V(q),
	\]
	and we will suppose that motion is bounded, so $H = E$ with
	\[
	E < \lim_{q \to \infty}V(q).
	\]
	This motion will oscillate back and forth between $(q_1, q_2)$. In phase space, the orbits look like whatever Skinner wrote on the board.

	I claim that we can take
	\[
	I = \frac{1}{2 \pi} \oint p \diff q,
	\]
	which is the area enclosed by the orbit $p = \sqrt{2m(E - V(q))}$ in phase space. We know for this $H(q, p)$ that $p = m \dot q$, so
	\[
		\oint_{\mathrm{orbit}}\diff t = \sqrt{\frac{m}{2}} \oint_{\mathrm{orbit}} \frac{\diff q}{\sqrt{E - V(q)}},
	\]
	hence we have
	\begin{align*}
		\frac{2\pi}{\omega} &= \sqrt{\frac{m}{2}} \oint \frac{\diff q}{\sqrt{E - V(q)}} = \sqrt{2m} \oint \frac{\diff}{\diff E} \sqrt{E - V(q)} \diff q \\
				    &= \frac{\diff}{\diff E} \oint \sqrt{2m(E-V(q))} \diff q = \frac{\diff}{\diff E} \oint p \diff q \\
				    &= 2 \pi \frac{\diff I}{\diff E},
	\end{align*}
	and therefore
	\[
	\frac{\diff E}{\diff I} = \omega \implies \frac{\partial H}{\partial I} = \omega.
	\]
%lecture 21
	To justify exchanging the derivative and integral, note that as we change $E$, the curve $p(q, E)$ changes and the endpoints $q_1, q_2$ of the orbit change. We have
	\[
	\delta q_i = \biggl(\frac{\diff V}{\diff q} \biggr)^{-1} \delta E,
	\]
	so the part we neglected (i.e. the area of the ``new regions'') is
	\begin{align*}
		\sim \int_{q_1 + \delta q_1}^{q_2} &\sqrt{2m(E - V(q))} \biggl( \frac{\diff V}{\diff q} \biggr)^{-1} \delta E \\
						   &\approx \sqrt{2m (E - V(q_1 + \delta q_1/2))} \biggl( \frac{\diff V}{\diff q} \biggr)^{-1} \biggr|_{q_1 + \delta q_1/2} \delta E \\
						   &\sim \mathcal{O}(\delta E^2),
	\end{align*}
	since by definition $E - V(q) \to 0$ at the endpoints. Hence we are justified in exchanging the order of the derivative.
\end{exbox}

This definition of action variables also extends to higher dimensional phase spaces. If we have found $n$ constants of the motion, then the orbits must look like $(S^1)^n$, and we've foliated the phase space with these tori.

In a 4D phase space of an integrable system, $\dot I_a = 0$ and $\theta^a = \omega^a t$. We have a bunch of tori. I really don't know how to explain what Skinner is drawing, and also I can't hear him because he didn't turn on the mic.

We define the action variables to be
\[
I_a = \frac{1}{2 \pi} \oint_{\gamma_a} \mathbf{p} \cdot \diff \mathbf{q} = \frac{1}{2 \pi} \oint_{\gamma_a} p_b \diff q^b,
\]
where $\gamma_a$ are the non-contractible curves on our tori.

\begin{exbox}[Kepler Problem]
	If $V(\mathbf{r}) = -k/r$, then angular momentum is conserved, and the motion $\mathbf{r}(t) \in \mathbb{R}^3$ is confined to the plane $\mathbf{r}(t) \cdot \mathbf{L} = 0$.

	Let $(r, \phi)$ be 2D polar coordinates on this plane. The Hamiltonian in the plane is
	\[
	H = \frac{p_r^2}{2m} + \frac{p_\phi^2}{2mr^2} - \frac{k}{r}.
	\]
	The motion is periodic in both $\phi$ and $r$, and these two cycles give us our two action variables:
	\[
		I_\phi = \frac{1}{2\pi} \oint_{\phi \text{ cycle}} p_r \diff r + p_\phi \diff \phi = \frac{1}{2\pi} \int_0^{2\pi} p_\phi \diff \phi = p_\phi,
	\]
	\[
		I_r = \frac{1}{2\pi} \oint_{r \text{ cycle}} \mathbf{p} \cdot \diff \mathbf{q} = \frac{2}{2 \pi} \int_{r_{\mathrm{min}}}^{r_{\mathrm{max}}} p_r \diff r.
	\]
	We have that
	\begin{align*}
		p_r^2 &= 2m\biggl( E + \frac{k}{r} \biggr) - \frac{I_\phi^2}{r^2} \\
		      &= 2m|E| \biggl(-1 + \frac{k}{|E|r} - \frac{I_\phi^2}{2m|E|r^2} \biggr) \\
		      &= 2m|E| \biggl(1 - \frac{r_{\mathrm{min}}}{r} \biggr) \biggl( \frac{r_{\mathrm{max}}}{r} - 1 \biggr),
	\end{align*}
	where
	\[
	r_{\mathrm{min}} + r_{\mathrm{max}} = \frac{k}{|E|}, \qquad r_{\mathrm{min}}r_{\mathrm{max}} = \frac{I_\phi^2}{2m|E|}.
	\]
	Hence,
	\begin{align*}
		I_r &= \frac{\sqrt{2m|E|}}{\pi} \int_{r_{\mathrm{min}}}^{r_{\mathrm{max}}} \sqrt{\biggl(1 - \frac{r_{\mathrm{min}}}{r} \biggr) \biggl( \frac{r_{\mathrm{max}}}{r} - 1 \biggr)} \diff r \\
		&= \sqrt{2m|E|} \biggl( \frac{r_{\mathrm{min}} + r_{\mathrm{max}}}{2} - \sqrt{r_{\mathrm{min}} r_{\mathrm{max}}} \biggr) = \sqrt{\frac{m}{2|E|}} k - I_\phi.
	\end{align*}
	Rearranging, this gives
	\[
	I_r + I_\phi = k \sqrt{\frac{m}{2|E|}},
	\]
	or in other words,
	\[
	H(I) = - \frac{m k^2}{2(I_r + I_\phi)^2}
	\]
	for the Kepler problem. We see that the corresponding angle variables obey
	\[
	\dot \phi = \frac{\partial H}{\partial I_\phi} = \omega_\phi, \qquad \dot \theta = \frac{\partial H}{\partial I_r} = \omega_r,
	\]
	for frequencies in the angular and radial directions.

	Since this Kepler $H = H(I_r + I_\phi)$, we see that $\omega_\phi = \omega_r$. This is the fact that the orbits are elliptical in the Kepler problem.
\end{exbox}

\subsection{Adiabatic Invariants}
\label{sub:ad_inv}

Suppose that the Hamiltonian depends on some parameter $\lambda(t)$ which is very slowly varied, where very slowly means that $|\dot \lambda|/|\lambda| \ll |\omega|$.

A quantity $I(p, q;\lambda)$ is called an \emph{adiabatic invariant}\index{adiabatic invariant} if, for every $\kappa > 0$, there exists $\eps_0 > 0$ such that
\[
|I(p(t), q(t); \lambda(t)) - I(p(0), q(0); \lambda(0))| < \kappa,
\]
for all $t \in (0, 1/\eps)$ and all $\eps \in (0, \eps_0)$.

\begin{proposition}
	The action variable
	\[
	I = \frac{1}{2\pi} \oint p \diff q
	\]
	in a 2D phase space is an adiabatic invariant.
\end{proposition}

To see this, note
\[
\dot E = \frac{\partial H}{\partial \lambda} \dot \lambda,
\]
and
\[
	I = \frac{1}{2\pi} \oint \sqrt{2m(E(t) - V(q, \lambda(t)))} \diff q.
\]
Altogether,
\begin{align*}
	\dot I &= \frac{\partial I}{\partial E} \biggr|_{\lambda} \dot E + \frac{\partial I}{\partial \lambda} \biggr|_{E} \dot \lambda \\
	       &= \biggl( \frac{\partial I}{\partial E} \biggr|_{\lambda} \frac{\partial H}{\partial \lambda} + \frac{\partial I}{\partial \lambda} \biggr|_{E} \biggr) \dot \lambda.
\end{align*}
We know that, at fixed parameter $\lambda$,
\[
\frac{\partial I}{\partial E} \biggr|_{\lambda} = \frac{1}{\omega(\lambda)} = \frac{T(\lambda)}{2 \pi},
\]
for the orbit period $T(\lambda)$. For the remaining term, note our orbit in phase space changes as we vary the parameter $\lambda$. We have
\begin{align*}
	\frac{\partial I}{\partial \lambda} \biggr|_{E} &= \frac{1}{2\pi} \frac{\partial}{\partial \lambda} \biggr|_{E} \oint  p \diff q = \frac{1}{2 \pi} \oint \frac{\partial p}{\partial \lambda} \biggr|_E \diff q \\
							&= \frac{1}{2 \pi} \int_0^{T(\lambda)} \frac{\partial p}{\partial \lambda} \biggr|_E \dot q \diff t = \frac{1}{2\pi} \int_0^{T(\lambda)} \frac{\partial p}{\partial \lambda} \biggr|_E \frac{\partial H}{\partial p} \biggr|_\lambda \diff t.
\end{align*}
Also, on our orbit $H(p, q; \lambda) = E$, so taking the derivative with respect to $\lambda$ we have
\[
0 = \frac{\partial H}{\partial \lambda} \biggr|_{p, q, E} + \frac{\partial H}{\partial p} \biggr|_{\lambda, q, E} \frac{\partial p}{\partial \lambda} \biggr|_E,
\]
where $p = \sqrt{2m(E - V(\lambda, q))}$. Hence
\[
\frac{\partial I}{\partial \lambda} \biggr|_E = - \frac{1}{2 \pi} \int_0^{T(\lambda)} \frac{\partial H}{\partial \lambda} \biggr|_E \diff t.
\]
Combining both terms, we have
\begin{align*}
	\dot I &= \biggl( \frac{\partial H}{\partial \lambda} - \frac{1}{T(\lambda)} \int_0^{T(\lambda)} \frac{\partial H}{\partial \lambda} \diff t \biggr) \frac{T(\lambda)}{2\pi} \dot \lambda \\
	       &= \biggl( \frac{\partial H}{\partial \lambda} - \biggl\langle \frac{\partial H}{\partial \lambda} \biggr\rangle \biggr) \frac{T(\lambda)}{2 \pi} \dot \lambda.
\end{align*}
Consequently, so long as $\lambda$ varies sufficiently slowly that it's approximately constant over a period of oscillation, we have $\dot I \approx 0$.

\begin{exbox}
	A particle bounces back and forth between the walls of a rigid box. Take
	\[
	V(x) =
	\begin{cases}
		0 & 0 < x < a,\\
		\infty & \text{else}.
	\end{cases}
	\]
	If the collisions with the walls are perfectly elastic, then the phase space orbits are rectangles, with
	\[
	I = \frac{1}{2 \pi} \oint p \diff q = \frac{mva}{\pi},
	\]
	and this quantity will be an adiabatic invariant. In particular, if the separation between the walls is slowly varied, the velocity of the particle must change as
	\[
	v(t) = v(0) \frac{a(0)}{a(t)}.
	\]
\end{exbox}

\begin{exbox}
	Suppose we slowly vary the length $l$ of a simple pendulum. Then
	\[
	H = \frac{p_\theta^2}{2ml^2} - mgl \cos \theta \approx \frac{p_\theta^2}{2 m l^2} + \frac{mgl}{2}\theta^2.
	\]
	Then the action variable is
	\[
	I = \frac{1}{2 \pi} \oint p \diff q = \frac{ab}{2},
	\]
	where $a, b$ are the semi-axes of the ellipse of the phase in phase space, and we have
	\[
		a = \sqrt{\frac{2E}{mgl}}, \qquad b = \sqrt{2m E l}.
	\]
	Hence we have
	\[
		I = E\sqrt{\frac{l}{g}} = \frac{E}{\omega},
	\]
	where $\omega = \sqrt{g/l}$. We can write the energy in terms of the amplitude:
	\[
	E = \frac{mgl}{2} \theta_{\mathrm{max}}^2.
	\]
	Hence, as the length of the pendulum is altered, the amplitude will vary as
	\[
	\frac{E}{\omega} = \frac{mg^{1/2}}{2} l^{3/2} \theta_{\mathrm{max}}^2
	\]
	stays constant. Hence
	\[
		\theta_{\mathrm{max}}(t) = \theta_{\mathrm{max}}(0) \biggl[ \frac{l(0)}{l(t)} \biggr]^{3/4}.
	\]

\end{exbox}

%lecture 23

\newpage

\section{Field Theory}
\label{sec:ft}

A \emph{covector}\index{covector} (or a 1-form) on a space $M$ is a collection $\omega_a(x) \diff x^a$, where $\omega_a(x)$ for $a = 1, \ldots, n = \dim(M)$ are functions and $\{\diff x^a\}$ are a basis of the cotangent space to $M$ at $x$ 

A \emph{vector field}\index{vector field} $V$ on $M$ is likewise a collection $V^a(x) \partial/\partial x^a$.

Vectors and covectors are naturally dual by the pairing
\[
	(V, \omega) \mapsto V^a(x) \omega_a(x) : M \to \mathbb{R}.
\]

Similarly, a 2-form is a collection $\omega_{ab}(x) \diff x^a \wedge \diff x^b$, where $\diff x^a \wedge \diff x^b = - \diff x^b \diff x^a$, so $\omega_{ab}(x) = \omega_{[ab]}(x)$, and hence consist of $n(n-1)/2$ independent functions. We can generalize to $p$-forms
\[
	\phi_{ab\dots d}(x) \diff x^a \wedge \cdots \wedge \diff x^d \text{ consisting of } \binom{n}{p} \text{ functions}.
\]
The space of $p$-forms is denote $\Omega^p(M)$, and is an infinite dimensional vector space. There's a natural map called the \emph{exterior derivative}\index{exterior derivative}:
\[
\diff : \Omega^{p} \to \Omega^{p+1}.
\]
For a 1-form $A = A_a(x) \diff x^a$, we have
\[
\diff : A \mapsto \diff A = \partial_a A_b \diff x^a \wedge \diff x^b = \frac{1}{2}(\partial_a A_b - \partial_b A_a) \diff x^a \wedge \diff x^b.
\]
In electromagnetism $F = \diff A$ is shorthand for
\[
F_{ab} = \partial_a A_b - \partial_b A_a.
\]
Importantly, $\diff^2 = 0$, since
\[
\diff^2\omega = \partial_\alpha \partial_b \omega_{c,d,\ldots, e} \diff x^a \wedge \diff x^b \wedge \cdots \wedge \diff x^e = 0,
\]
since $\partial_a \partial_b = \partial_b \partial_a$. In electromagnetism, this is the gauge freedom $A \sim A + \diff \lambda$, which leaves
\[
F = \diff(A + \diff \lambda) = \diff A + \diff^2\lambda = \diff A.
\]
Now let's consider an action
\[
	S[x] = \int L(x, \dot x) \diff t.
\]
This is a function on the space $M$ of curves in our space $\mathbb{R}^3$. Let $\delta$ be the exterior derivative on this space of curves. When the equations of motions hold,
\begin{align*}
	\delta S &= \int (\text{bulk e.o.m.} \delta x \diff t + \frac{\partial L}{\partial \dot x} \delta x \biggr|_{t_0}^{t_1} \\
		 &= \frac{\partial L}{\partial \dot x} \delta x \biggr|_{t_0}^{t_1} = p \delta x \biggr|_{t_0}^{t_1}.
\end{align*}
It follows that
\[
0 = \delta^2 S = \delta p \wedge \delta x \biggr|_{t_0}^{t_1}.
\]
In other words, $\delta p \wedge \delta x (t_0) = \delta p \wedge \delta x (t_1)$, so as long as the equations of motions hold in between. This is
\[
	\omega = \delta p \wedge \delta x = \frac{1}{2} \omega_{\alpha \beta} \diff y^{\alpha} \wedge \diff y^{\beta},
\]
where
\[
\omega_{\alpha\beta} =
\begin{pmatrix}
	0&1\\-1&0
\end{pmatrix}.
\]
This is the symplectic form on the space of solutions. The inverse of $\omega$ is the Poisson bracket
\[
\omega^{-1} = - \frac{\partial}{\partial p} \wedge \frac{\partial}{\partial q},
\]
so
\[
\omega^{-1}(\diff f, \diff g) = \frac{\partial f}{\partial x} \frac{\partial g}{\diff p} - \frac{\partial f}{\partial p} \frac{\partial g}{\partial x}.
\]

In mechanics, we have considered Lagrangians and actions defined along the one-dimensional worldline of the particle. In field theory, we have further integrals.

The action is now a function on the space of field configurations, say
\[
	S[\phi] = \int_{\mathbb{R}^{1,3}} \frac{1}{2} \partial_{\mu} \phi \partial^{\mu} \phi - V(\phi) \Diff4 x.
\]
The Euler-Lagrange equations say $\delta S = 0$. In this case,
\begin{align*}
	\delta S &= \int \partial_\mu \delta \phi \partial^\mu \phi - \delta \phi \frac{\partial V}{\partial \phi} \Diff 4 x \\
		 &= \int - \biggl(\partial^\mu \partial_\mu \phi + \frac{\partial V}{\partial \phi} \biggr) \delta \phi \Diff4 x + \int n_\mu \partial^\mu \phi \delta \phi \biggr|_{\eps_i}^{\eps_F} \Diff3 x.
\end{align*}
Hence the equation of motion for fixed boundary conditions give
\[
\partial^\mu \partial_\mu \phi = - \frac{\partial V}{\partial \phi}
\]
at each $x^\mu \in \mathbb{R}^{1,3}$.

For example, if $V(\phi) = 0$, then $\partial^\mu \partial_\mu \phi = 0$, which is the wave equations.

If $V(\phi) = \frac{1}{2} m^2 \phi^2$, then $\partial^\mu \partial_\mu \phi = - m^2 \phi$, which is the Klein-Gordon equations. If we express
\[
\phi(x^\mu) = \int \frac{\Diff 4 k}{(2 \pi)^2} e^{i k \cdot x} \tilde \phi(k_\mu),
\]
then these equations require
\[
k^\mu k_\mu = E^2 - \mathbf{k} \cdot \mathbf{k} = m^2.
\]
The \emph{field momentum}\index{field momentum} is $\pi = \partial L/\partial \dot \phi = \dot \phi$. Again, if our boundaries are slices of constant time, then the boundary term from the variational principle is

\[
\int \partial_0 \phi \biggr|_{\eps_i}^{\eps_F} \Diff 3 x = \int \pi(\mathbf{x}, t) \delta \phi(\mathbf{x}, t) \biggr|_{t_i}^{t_F} \Diff3 x.
\]
We get a corresponding symplectic form and Poisson bracket
\[
	\{\phi(\mathbf{x}, t), \pi(\mathbf{y}, t)\} = \delta^3(\mathbf{x} - \mathbf{y}).
\]
In QFT, we can canonically quantize this system to give
\[
	[\hat \phi(\mathbf{x}, t), \hat \pi(\mathbf{y}, t)] = i \hbar \delta^3(\mathbf{x} - \mathbf{y}).
\]


\newpage

\printindex

\end{document}
