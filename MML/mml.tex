\documentclass[12pt]{article}

\usepackage{ishn}

\makeindex[intoc]

\begin{document}

\hypersetup{pageanchor=false}
\begin{titlepage}
	\begin{center}
		\vspace*{1em}
		\Huge
		\textbf{II Mathematics of Machine Learning}

		\vspace{1em}
		\large
		Ishan Nath, Lent 2024

		\vspace{1.5em}

		\Large

		Based on Lectures by Prof. Rajen Shah

		\vspace{1em}

		\large
		\today
	\end{center}
	
\end{titlepage}
\hypersetup{pageanchor=true}

\tableofcontents

\newpage

\section{Introduction}
\label{sec:into}

Consider a pair of random variables $(X, Y)$. Here $X$ is the input, or features, or predictors, and $Y$ is the response or output. We have $(X, Y) \in \mathcal{X} \times \mathcal{Y}$, where we assume $\mathcal{X} = \mathbb{R}^p$, with joint distribution $P_0$.

We wish to predict $Y$ from $X$ using a \emph{hypothesis}\index{hypothesis} $h : \mathcal{X} \to \mathcal{Y}$. We measure the quality of a prediction using a \emph{loss function}\index{loss function} $\ell : \mathcal{Y} \times \mathcal{Y} \to \mathbb{R}$. In a classification setting $\mathcal{Y} = \{-1, +1\}$. Typically, $\ell $ is the misclassification loss:
\[
	\ell(h(x), y) = \mathbbm{1}_{\{h(x) \neq y\}}.
\]
We refer to $h$ as a \emph{classifier}\index{classifier}. In a regression setting, $\mathcal{Y} \in \mathbb{R}$. Typically, $\ell $ is the squared error loss:
\[
\ell(h(x), y) = (h(x) - y)^2.
\]
We aim to pick $h$ with small \emph{risk}\index{risk}: $R(h) = \mathbb{E}[\ell(h(X), Y)]$, when $h$ is deterministic. However most of the time $h$ is not deterministic.

The squared error risk is minimized by the \emph{regression function}\index{regression function}
\[
x \mapsto \mathbb{E}[Y|X = x].
\]
The classifier $h_0$ that minimises the 0-1 risk is known as the \emph{Bayes classifier}\index{Bayes classifier}. Its risk is known as the \emph{Bayes risk}\index{Bayes risk}. In this context,
\[
\eta(x) = \mathbb{P}(Y = 1 |X = x)
\]
is the \emph{regression function}.

\begin{proposition}
	A Bayes classifier is given by
	\[
	h_0(x) =
	\begin{cases}
		1 & \eta(x) > 1/2, \\
		0 & \text{otherwise}.
	\end{cases}
	\]
\end{proposition}

Suppose we have iid copies $(X_1, Y_1), \ldots, (X_n, Y_n)$ of $(X, Y)$. These copies is what is known as our \emph{training data}\index{training data}, $D$. Our task is to use this data to construct $\hat h$ such that the risk
\[
R(\hat h) = \mathbb{E}[\ell(h(X), Y) | D]
\]
is small. Note that $R(\hat h)$ is now a random variable!

The classical statistics approach is as follows: model $P_0$ using some parametric family, and try to estimate the unknown parameters. The machine learning approach begins with a given class $\mathcal{H}$ from which we pick $\hat h$. Hence our task, in this course, is to find an algorithm to pick $\hat h$.

\begin{exbox}
	Linear regression is an example of this. We have
	\[
		\mathcal{H} = \{x \mapsto \mu + \beta^{T}x, \mu \in \mathbb{R}, \beta \in \mathbb{R}^p\}.
	\]
	We can generalize this a bit, and take
	\[
		\mathcal{H} = \left\{ x \mapsto \sum_{j = 1}^d \phi_j(x) \beta_j \mid \beta \in \mathbb{R}^d \right\},
	\]
	where $\phi_1, \ldots, \phi_d$ are fixed \emph{basis functions}\index{basis functions} $\phi_j : \mathbb{R}^p \to \mathbb{R}$. Or even more generally,
	\[
		\mathcal{H} = \left\{ x \mapsto \sum_{j = 1}^d \phi_j(x) \beta_j \mid \beta \in \mathbb{R}^d, \phi_j \in \mathcal{B} \right\}
	\]
	for a class $\mathcal{B}$ of functions $\phi : \mathbb{R}^p \to \mathbb{R}$.

	For classification, we can take a linear model and compose it with the sign function, where $\sgn(0) = -1$.
\end{exbox}

\subsection{Review of Conditional Expectation}
\label{sub:cond_ex}

Suppose $Z \in \mathbb{R}$, $W \in \mathbb{R}^d$ are random variables. There are two ways of thinking about $\mathbb{E}[Z|W]$.
\begin{itemize}
	\item We can think of it as
		\[
		\mathbb{E}[Z | W = w] = \int z f_{z|w}(z|w) \diff z = g(w),
		\]
		hence $\mathbb{E}[Z|W] = g(W)$.
	\item Or, $\mathbb{E}[Z|W]$ is our best guess of $Z$ knowing only $W$.
\end{itemize}

%lecture 2

Let's look at a couple of specific cases.
\begin{enumerate}[(i)]
	\item If $Z \indep W$ (meaning $Z$ independent of $W$), then
		\[
		\mathbb{E}[Z|W] = \mathbb{E}[Z].
		\]
		More generally, if $U \indep (Z, W)$ then
		\[
		\mathbb{E}[Z|W, U] = \mathbb{E}[Z|W].
		\]
	\item The tower property: if $f : \mathbb{R}^d \to \mathbb{R}^m$ then
		\[
		\mathbb{E}[\mathbb{E}[Z|W]|f(W)] = \mathbb{E}[Z|f(W)].
		\]
		In particular, if $f$ is constant, then
		\[
		\mathbb{E}[\mathbb{E}[Z|W]] = \mathbb{E}[Z].
		\]
	\item We can fix what we know.
		\begin{align*}
			\mathbb{E}[f(W_1&, \ldots, W_d) | W_1 = w_1, \ldots, W_r = w_r] \\
							 &= \mathbb{E}[f(w_1, \ldots, w_r, W_{r+1}, \ldots, W_d)|W_1 = w_1, \ldots, W_r = w_r].
		\end{align*}
		This is often used in a specific way. If $g : \mathbb{R}^d \to \mathbb{R}$, then
		\[
		\mathbb{E}[g(W) Z|W] = g(W) \mathbb{E}[Z|W].
		\]
	\item The best least squares predictor is
		\[
		\mathbb{E}[(Z - g(W))^2] = \mathbb{E}[(Z - \mathbb{E}[Z|W])^2] + \mathbb{E}[(\mathbb{E}[Z|W] - g(W))^2].
		\]
		We can prove this. Indeed,
		\begin{align*}
			\mathbb{E}[(Z - g(W))^2] &= \mathbb{E}[(Z - \mathbb{E}[Z|W] + \mathbb{E}[Z|W] - g(W))^2] \\
						 &= \mathbb{E}[(Z - \mathbb{E}[Z|W])^2] + \mathbb{E}[(\mathbb{E}[Z|W] - g(W))^2] \\
						 & \qquad + 2 \mathbb{E}[(Z - \mathbb{E}[Z|W])(\mathbb{E}[Z|W] - g(W))],
		\end{align*}
		but now we find that
		\begin{align*}
			\mathbb{E}[(Z - \mathbb{E}&[Z|W])(\mathbb{E}[Z|W] - g(W))] \\
			&= \mathbb{E}[\mathbb{E}[(Z - \mathbb{E}[Z|W])(\mathbb{E}[Z|W] - g(W))]|W] \\
										  &= \mathbb{E}[(\mathbb{E}[Z|W] - g(W))] \mathbb{E}[Z - \mathbb{E}[Z|W] | W] = 0,
		\end{align*}
		as this last term becomes 0.
\end{enumerate}

The idea of the tower law is the same as the law of total probability: we can break our probability space into a disjoint union of spaces where we know a given random variable, and weighting by their respective probabilities gives us the original expectation.

Probabilistic results can be applied conditionally. In particular, we have conditional Jensen.

Recall $f : \mathbb{R}\to \mathbb{R}$ is convex if
\[
t f(x) + (1-t)f(y) \geq f(tx + (1-t)y),
\]
for all $f \in [0, 1]$. If $f$ is convex, then Jensen gives
\[
\mathbb{E}[f(Z)|W] \geq f(\mathbb{E}[Z|W]).
\]

\subsection{Bayes Risk}
\label{sub:bay_r}

We will prove that $h_0$ given in the first proposition is indeed a Bayes classifier. Note we have that
\[
R(h) = \mathbb{P}(Y \neq h(X)) = \mathbb{E}[P(Y \neq h(X))|X].
\]
So $h_0(x)$ must minimiser over $h(x)$, the value
\begin{align*}
	\mathbb{P}(Y \neq h(x)|X = x) &= \mathbb{P}(Y = 1, h(x) = -1 | X = x) + \mathbb{P}(Y = -1, h(x) = 1 | X = x) \\
				      &= \mathbb{P}(Y = 1|X = x) \mathbbm{1}_{\{h(x) = -1\}} + \mathbb{P}(Y = -1 | X = x) \mathbbm{1}_{\{h(x) = 1\}} \\
				      &= \eta(x) \mathbbm{1}_{\{h(x) = -1\}} + (1 - \eta(x)) \mathbbm{1}_{\{h(x) = 1\}}.
\end{align*}
Now if $\eta(x) > 1/2$, then $h_0(x) = 1$, and if it is less than $1/2$, then $h_0(x) = -1$. If $\eta(x) = 1/2$, then either choice is constant, so any $h(x)$ minimises it.

\subsection{Empirical Risk Minimisation}
\label{sub:emp_r_min}

The risk $R(h)$ is the expectation of $\ell(h(X), Y)$, where $(X, Y) \sim P_0$.

The \emph{empirical risk}\index{empirical risk} or \emph{training error}\index{training error} is the expectation of $\ell(h(X), Y)$ with respect to the empirical measure, i.e.
\[
\hat R(h) = \frac{1}{n} \sum_{i = 1}^n \ell(h(X_i), Y_i).
\]
Given $\mathcal{H}$, then
\[
	\hat h \in \underset{h \in \mathcal{H}}{\mathrm{argmin}} \,\hat R(h)
\]
is the (a) \emph{empirical risk minimiser}\index{empirical risk minimiser}.

Here $\hat h$ is a random variable that is dependent on the training data $D$.

\begin{exbox}
	In the regression setting with 
	\[
		\mathcal{H} = \{x \mapsto \mu + \beta^{T}x \mid \mu \in \mathbb{R}, \beta \in \mathbb{R}^p\},
	\]
	the empirical risk minimiser is the ordinary least squares solution: $\hat (x) = \hat \mu + \hat \beta^{T}x$, where
	\[
		(\hat \mu, \hat \beta) = \underset{(\mu, \beta) \in \mathbb{R} \times \mathbb{R}^p}{\mathrm{argmin}} \, \frac{1}{n} \sum_{i = 1}^n (Y_i - \mu - X_i^{T}\beta)^2.
	\]
	More generally, if we consider
	\[
		\mathcal{H} = \left\{ x \mapsto \sum_{j = 1}^d \phi_j(x) \beta_j \mid \beta \in \mathbb{R}^d\right\},
	\]
	where each $\phi_j : \mathbb{R}^p \to \mathbb{R}$, then suppose we have $\Phi \in \mathbb{R}^{n \times d}$ with $\Phi_{ij} = \phi_j (x_i)$. Then writing $\phi(x) = (\phi_1(x), \ldots, \phi_d(x))$, the empirical risk minimiser is
	\[
	\hat h(x) = \phi(x)^{T}(\Phi^{T} \Phi)^{-1} \Phi^{T} Y_{1:n},
	\]
	where $Y_{1:n} = (Y_1, \ldots, Y_n)$.
\end{exbox}

\subsection{Bias-variance Tradeoff}
\label{sub:bv}

Consider $\hat h = \hat h_D$ trained on data $D = (X_i, Y_i)_{i=1}^n$. Let $(X, Y) \indep D$. Then we define
\[
\bar h(x) : x \mapsto \mathbb{E}[\hat h_D(x)].
\]
Recall we can split the squared error into two terms. Using this,
\begin{align*}
	\mathbb{E}[R(\hat h_D)] &= \mathbb{E}[(Y - \hat h_D(X))^2] \\
				&= \mathbb{E}[(Y - \mathbb{E}[Y|D,X])^2] + \mathbb{E}[(\mathbb{E}[Y|X] - \hat h_D(X))^2] \\
				&= \mathbb{E}[\mathbb{E}[(Y - \mathbb{E}[Y|X])^2]|X] + \mathbb{E}[(\hat h_D(X) - \mathbb{E}[\hat h_D(X)|X])^2] \\
				& \qquad \qquad + \mathbb{E}[(\bar h(X) - \mathbb{E}[Y|X])^2] \\
				&= \mathbb{E}[(\mathbb{E}[Y|X] - \bar h(X))^2] + \mathbb{E}[\Var(\hat h_D(X)|X)] + \mathbb{E}[\Var(Y|X)].
\end{align*}
Here we have three terms, the \emph{squared bias}, the \emph{variance of} $\hat h_D$, and the \emph{irreducible variance}.

%lecture 3

Consider $\hat h_D$, and empirical risk minimizer over the class $\mathcal{H}$. We expect that large $\mathcal{H}$ will lead to low bias, but high variance, and a small $\mathcal{H}$ will lead to high bias, but low variance.

It is instructive to consider a decomposition involving 
\[
\tilde h_{X_{1:n}} : x \mapsto \mathbb{E}[\hat h_D(x) | X_{1:n}].
\]
Then we can show that
\begin{align*}
	\mathbb{E}[(Y - \hat h_D(X))^2 | X = x] &= \mathbb{E}[(\mathbb{E}[Y|X = x]  - \tilde h_{X_{1:n}}(x))^2] \\
	+ \mathbb{E}[(\hat h_D(x) - \tilde h_{X_{1:n}}(x))^2] &+ \Var(Y | X = x).
\end{align*}
Consider the ERM $\hat h_D$, in where
\[
\hat h_D(x) = \phi(x)^{T}(\Phi^{T}\Phi)^{-1}\Phi^{T} Y_{1:n}.
\]
We further assume that $\Var(Y | X = x) = \sigma^2$, i.e. constant in $x$. Then
\begin{align*}
	\mathbb{E}[(\tilde h_D(x) - \tilde h_{X_{1:n}}(x))^2 | X_{1:n}] \\
	= \mathbb{E}[(\phi(x)^T (\Phi^T\Phi)^{-1}\Phi^T(Y_{1:n} - \mathbb{E}[Y_{1:n}|X_{1:n})]^2 \mid X_{1:n}] \\
	= \phi(x)^T(\Phi^T \Phi)^{-1} \Phi^T \mathbb{E}[(Y_{1:n} - \mathbb{E}[Y_{1:n} \mid X_{1:n}])(Y_{1:n} - \mathbb{E}[Y_{1:n}\mid X_{1:n}])^{T} \mid X_{1:n}] \\
	\times \Phi(\Phi^T \Phi)^{-1} \phi(x).
\end{align*}
Note that $\mathbb{E}[Y_j \mid X_{1:n}] = \mathbb{E}[Y_j \mid X_j]$, so if $A$ is the matrix in the middle,
\begin{align*}
	A_{jk} &= \mathbb{E}[(Y_j - \mathbb{E}[Y_j \mid X_j])(Y_k - \mathbb{E}[Y_k \mid X_k]) \mid X_j, X_k] \\
	       &= \mathbb{E}[Y_j Y_k \mid X_j, X_k] - \mathbb{E}[Y_j \mid X_j] \mathbb{E}[Y_k \mid X_k].
\end{align*}
If $j \neq k$, then
\begin{align*}
	\mathbb{E}[Y_j Y_k \mid X_j, X_k] &= \mathbb{E}[Y_j \mathbb{E}[Y_k \mid X_j, Y_j] \mid X_j, X_k] \\
					  &= \mathbb{E}[Y_j \mathbb{E}[Y_k \mid X_k] \mid X_j, X_k] \\
					  &= \mathbb{E}[Y_j \mid X_j] \mathbb{E}[Y_k \mid X_k].
\end{align*}
Thus $A_{jk} = 0$ if $j \neq k$. Moreover, $A_{jj} = \sigma^2$, by our assumption, hence $A = \sigma^2I$. Therefore we get
\[
\mathbb{E}[(\hat h_D(x) - \tilde h_{X_{1:n}}(x))^2 \mid X_{1:n}] = \sigma^2 \phi(x)^T (\Phi^T \Phi)^{-1} \phi(x).
\]
Consider averaging this quantity over $x = X_1, \ldots, X_n$. This is
\begin{align*}
	&\frac{1}{n} \tr ( \sum_{i = 1}^n \sigma^2 \phi(X_i)^2 (\Phi^T \Phi)^{-1} \phi(X_i)) \\
	= &\frac{\sigma^2}{n} \tr \left( \sum_{i = 1}^n \phi(X_i) \phi(X_i)^T (\Phi^{T}\Phi)^{-1} \right) = \frac{\sigma^2 d}{n}.
\end{align*}

So interestingly this quantity is on average linear in $d$, the dimension of basis functions.

In the above, we have the following classifiers:
\begin{itemize}
	\item $\hat h_D$. This is the empirical risk minimizer, and is a function of the data $D$.
	\item $\bar h$. This is the average value of $\hat h_D$, over training data $D$. For a large class $\mathcal{H}$ we expect that the variance of $\hat h_D$ will be large.
	\item  $\tilde h_{X_{1:n}}$. This is the average value of $\hat h_D$, given the values $X_1, \ldots, X_n$. We use it as a proxy for $\bar h$ in our calculation for the variance of $\hat h_D$.
\end{itemize}

In the above, it is important to know that $X_{1:n}\indep X$, as $X$ is part of $(X, Y)$, the independent sample we are trying to predict.

\subsection{Cross Validation}
\label{sub:x_val}

Let $\tilde h^{(1)}, \ldots, \tilde h^{(m)}$ be $m$ ``machine learning'' methods, that take in training data $D = (X_i, Y_i)_{i = 1}^n$, all iid, and output a hypothesis $\hat h_D^{(j)} : \mathcal{X} \to \mathcal{Y}$.

Given a loss $\ell$ and associated risk $R$, we may wish to pick $\hat h^{(j)}$ to minimize the risk $R(\hat h_D^{(j)}) = \mathbb{E}[\ell(\hat h_D^{(j)}(X), Y) \mid D]$, where $(X, Y) \indep D$, and the variable $(X, Y)$ is equal in distribution to $(X_1, Y_1)$.

It is easier to attempt to minimize the expected risk
 \[
\mathbb{E}[R(\hat h_D^{(j)})] = \mathbb{E}[\mathbb{E}[\ell(\hat h_D^{(j)}, Y)|D]].
\]
The idea of $v$-fold cross validation\index{cross validation} estimates this by splitting the data into $v$ folds (i.e. groups) of roughly equal size.

%lecture 4

We first split the data into $v$ folds, say $A_1, \ldots, A_v$. Let $D_{-v}$ denote all the data, except that in the $k$'th fold. For each $j$, we apply $\hat h^{(j)}$ to data $D_{-k}$, to get model $\hat h^{(j)}_{-k} = \hat h^{(j)}_{D_{-k}}$. Then we choose $j$ to minimize
\[
\mathrm{CV}(j) = \frac{1}{n} \sum_{k = 1}^v \sum_{i \in A_k} \ell(\hat h^{(j)}_{-k}(X_i), Y_i).
\]
Writing $\hat j$ for the minimizer, we select the hypothesis $\hat h^{(j)}_D$.

Note that for each $i \in A_k$,
\[
\mathbb{E}[\ell(\hat h^{(j)}_{-k}(X_i), Y_i)] = \mathbb{E}[R (\hat h^{(j)}_{D_{-k}})].
\]
This is similar to $\mathbb{E}[R(\hat h^{(j)})]$, except with a dataset of size $n$ replaced by one with size $n - |A_k|$. Thus $\mathrm{CV}(j)$ gives a biased estimate of this quantity, with this bias typically decreasing as $v$ increases.

The choice $v = n$ can give the least bias, but computationally may be expensive, and the variance may be large as the summands tend to be positively correlated. Typical choices are $v = 5$ or $10$.

One thing we can do is to cross validate with permuted data. This reduces the variance slightly.

\newpage

\section{Popular Machine Learning Methods}
\label{sec:pop_mac}

\subsection{Decision Trees}
\label{sub:dec_tree}

\emph{Decision trees}\index{decision tree} (also known as regression trees in the context we shall be interested in) construct hypotheses of the form
\[
T : x \mapsto \sum_{j = 1}^J \gamma_j \mathbbm{1}_{R_j}(x),
\]
where $R_j$ are rectangular regions that partition all of $\mathbb{R}^p$, and $\gamma_j \in \mathbb{R}$ are coefficients.

Decision trees are built to both choose the coefficients and the regions, hence we are sort of letting the data decide the basis functions to use. What decision trees don't do is empirical risk minimisation over the entire class of functions, because it is computationally difficult and due to overfitting.

One point is that given a region $R_j$, it is easy to find the coefficient $\gamma_j$ that minimizes the risk: it will be the average value of the output in that region. The idea behind decision trees is that the $J = 2$ case is an interesting and solvable problem, and using this case we are able to recurse and generate more trees.

The recursive binary partitioning scheme is as follows:
\begin{enumerate}
	\item Choose a maximum number of regions $J$, and initialise our regions $\mathcal{R} = \{\mathbb{R}^p\}$.
	\item For each region $R \in \mathcal{R}$ such that $I = \{ i \mid X_i \in R\}$ has $|I| \geq 2$, we do the following: for each $j = 1, \ldots, p$, let $S_j$ be the midpoints between adjacent $\{X_{ij}\}_{i \in I}$. Find the predictor $\hat j_R$ and split point $\hat s_R$ to minimize over $j = 1, \ldots, p$ and  $s \in S_j$ 
		\[
			\min_{\gamma_l \in \mathbb{R}} \sum_{\substack{i \in I \\ X_{ij} \leq s}} (Y_i - \gamma_l)^2 + \min_{\gamma_r \in \mathbb{R}}\sum_{\substack{i \in I \\ X_{ij} > s}} (Y_i - \gamma_R)^2 - \min_{\gamma \in \mathbb{R}} \sum_{i \in I} (Y_i - \gamma)^2.
		\]
	\item Let $\hat R$ be the region yielding the lowest value, and set
		\[
			\hat R_l = \{x \in \hat R \mid X_{\hat j_R} \leq \hat s_r\}, \qquad \hat R_r = \hat R \setminus \hat R_l.
		\]
	\item Update $\mathcal{R} = (\mathcal{R} \setminus \{\hat R\}) \cup \{\hat R_l, \hat R_r\}$.
	\item Repeat these steps until $|\mathcal{R}| = J$.
	\item Writing $\mathcal{R} = \{\hat R_1, \ldots, \hat R_J\}$, let $\hat I_j = \{i \mid X_i \in \hat R_j\}$, and set
		\[
		\hat \gamma_j = \frac{1}{|\hat I_j|} \sum_{i \in I_j} Y_i.
		\]
	\item Then our final tree is
		\[
		\hat T : x \mapsto \sum_{i = 1}^J \hat \gamma_j \mathbbm{1}_{\hat R_j}(x).
		\]
\end{enumerate}

Note while the computations seem expensive, they can be organized in an efficient manner as follows: consider for simplicity we have the first split, so $I = \{1, \ldots, n\}$, and suppose $X_1 < X_2 < \cdots < X_n$.

Then minimization is equivalent to finding the index $m$ to minimise the sum of
\begin{align*}
	\sum_{i \leq m} \left( Y_i - \frac{1}{m} \sum_{j \leq m} Y_j \right)^2 + \sum_{i > m} \left(Y_i - \frac{1}{n - m}\sum_{j > m} Y_j \right)^2 \\
	= \sum_{i = 1}^n Y_i^2 - \frac{1}{m} A_m^2 - \frac{1}{n - m} B_m^2,
\end{align*}
where $A_{m} = Y_1 + \cdots + Y_m$, and $B_m = \sum Y_i - A_m$. But then $A_{m+1} = A_m + Y_{m+1}$, and $B_{m+1} = B_m - Y_{m+1}$. Thus we can compute $A_1, \ldots, A_n$ and $B_1, \ldots, B_n$ in $\mathcal{O}(n)$ times, so the minimization may be performed in $\mathcal{O}(n)$ time.

The reason they are called decision trees is that, at each point, we can view the output of the model as a tree. Each time it forms a tree, we make a binary decision as to which region to put our data in, given a binary classification of our data.

One thing to note is our decision to use rectangular regions $R_j$. This is a form of bias-variance tradeoff: we could allow more types of regions if we wanted to, but the search would become computationally harder and we risk overfitting. In the end, it depends on the situation we are using this model in. If we, for example, expect rotational invariance then allowing a broader set of models makes sense, whereas if we expect $x_1$ and $x_2$ to correspond to variables which would have no linear relation, then it does not.

Another way to use this is to do the same decision tree, after rotating the data ourself, and then averaging over these rotations.

%lecture 5ish

\subsection{Random Forests}
\label{sub:rand_for}

While trees are fast to compute, they can be unstable, because the initial split can be sensitive to the data, and the piecewise constant fits may not be a good approximation to $x \mapsto \mathbb{E}[Y | X = x]$. Recall our bias-variance decomposition:
\[
\mathbb{E}[R(\hat T_D)] = \mathbb{E}[(\mathbb{E}[Y | X] - \bar T(X))^2] + \mathbb{E}[\Var(\hat T_D(X) \mid X)] + \mathbb{E}[\Var(Y|X)],
\]
where $\bar T(X) = \mathbb{E}[\hat T_D(X)]$. A \emph{random forest}\index{random forest} attempts to estimate $\bar T$. If we have multiple datasets $D_1, \ldots, D_B$, we could do this by finding
\[
\frac{1}{B} \sum_{b = 1}^B \hat T_{D_b}.
\]
Random forest replicates this by sampling $D$ with replacement to give datasets $D_1^\ast, \ldots, D_B^\ast$ and performs the following:
\begin{enumerate}
	\item For each $b = 1, \ldots, B$, fit a decision tree $\tilde T^{(b)} = \hat T_{D^\ast_b}$, but when searching for the best predictor to split on, randomly sample $m_{\mathrm{try}}$ of the $p$ predictors and choose the best split among these.
	\item Output the average
		\[
		f_{\mathrm{rf}} = \frac{1}{B} \sum_{b = 1}^B \tilde T^{(b)}.
		\]
\end{enumerate}
The rationale for sampling predictors is to make $\tilde T^{(b)}$ more independent. Indeed, consider $b_1 \neq b_2$, and $x \in \mathbb{R}^p$ with
\[
\Corr(\hat T^{(b_1)}(x), \hat T^{(b_2)}(x)) = \rho > 0.
\]
Then we find
\begin{align*}
	\Var(f_{\mathrm{rf}}(x)) &= \Var \left( \frac{1}{B} \sum_{b = 1}^B \hat T^{(b)}(x) \right) \\
				 &= \frac{1}{B^2} \sum_{b = 1}^B \Var(\hat T^{(b)}(x)) + \frac{1}{B^2} \sum_{b_1 \neq b_2} \Cov(\hat T^{(b_1)}(x), \hat T^{(b_2)}(x)) \\
				 &= \frac{1}{B} \Var(\hat T^{(1)}(x) ) + \frac{B(B-1)}{B^2} \rho \Var(\hat T^{(1)}(x)) \\
				 &= \left(\frac{1}{B}(1 - \rho) + \rho \right) \Var(\hat T^{(1)}(x)).
\end{align*}
So we expect that increasing $B$ will decrease the variance. This is true, up to a certain point, due to this factor of $\rho$. Hence while we may want to minimize the bias term by choosing the optimal split point for each tree, injecting randomness allows us to decrease $\rho$, and also tries to solve the greedy nature of the trees.

A small $m_{\mathrm{try}}$ reduces the variance, but increases the bias. We can choose a best $m_{\mathrm{try}}$ through cross validation.

%lecture 6

\newpage

\section{Statistical Learning Theory}
\label{sec:slt}

Recall that, in the regression setting, using ordinary least squares on a set of $d$ basis functions yields
\begin{align*}
	\mathbb{E}(\hat h_D(X) - \tilde h_{X_{1:n}}(X))^2 = \mathbb{E}R(\hat h_D) - \mathbb{E} R(\tilde h_{X_{1:n}}(X)) \approx \frac{\sigma^2 d}{n},
\end{align*}
when $\Var(Y|X = x) =\sigma^2$, a constant in $x$.

Consider $\hat h$ an ERM over a general class $\mathcal{H}$. We will study the \emph{excess risk}\index{excess risk}
\[
	R(\hat h) - R(h^\ast), \text{ where } h^\ast = \mathrm{argmin}_{h \in \mathcal{H}}R(h).
\]
We don't really have much to go off of, except that
\begin{align*}
	R(\hat h) - R(h^\ast) &= R(\hat h) - \hat R(\hat h) + \underbrace{\hat R(\hat h) - \hat R(h^\ast)}_{\leq 0} + \hat R(h^\ast)- R(h^\ast) \\
			      &\leq \sup_{h \in \mathcal{H}}\{ R(h) - \hat R(h)\} + \hat R(h^\ast) - R(h^\ast).
\end{align*}
Consider $|\mathcal{H}|$ finite. Then
\begin{align*}
	\mathbb{P} \left( \max_{h \in \mathcal{H}} \{R(h) - \hat R(h)\} > t \right) &= \mathbb{P} \left( \bigcup_{h \in \mathcal{H}} \{R(h) - \hat R(h) > t\} \right) \\
										    &\leq \sum_{h \in \mathcal{H}} \mathbb{P}(R(h) - \hat R(h) > t).
\end{align*}

\subsection{Sub-Gaussiannity and Hoeffding's Inequality}
\label{sub:sub_g_h_ineq}

Recall:

\begin{proposition}[Markov's Inequality]\index{Markov's inequality}
	Let $W$ be a non-negative random variable. Then $t \mathbbm{1}_{\{W \geq t\}} \leq W$. Taking expectation and dividing by $t$,
	\[
	\mathbb{P}(W \geq t) \leq \frac{\mathbb{E}[W]}{t}.
	\]
\end{proposition}

Applying this with a strictly increasing transformation $\phi : R \to (0, \infty)$ and any random variable $W$,
\[
\mathbb{P}(W \geq t) = \mathbb{P}(\phi(W) \geq \phi(t)) \leq \frac{\mathbb{E} \phi(W)}{\phi(t)}.
\]
Taking $\phi(t) = e^{\alpha t}$ for $\alpha > 0$, yields the so-called \emph{Chernoff bound}\index{Chernoff bound}
\[
\mathbb{P}(W \geq t) \leq \inf_{\alpha > 0} e^{-\alpha t} \mathbb{E}[e^{\alpha W}].
\]

\begin{exbox}
	Suppose $W \sim N(0, \sigma^2)$. Then
	\[
	\mathbb{E} e^{\alpha W} = e^{\alpha^2 \sigma^2/2},
	\]
	which gives, from the Chernoff bound,
	\begin{align*}
		\mathbb{P}(W \geq t) &\leq \exp \left( \inf_{\alpha > 0} \left[ - \alpha t + \frac{\alpha^2\sigma^2}{2} \right] \right) \\
		&= e^{-t^2/(2\sigma^2)}.
	\end{align*}
\end{exbox}

\begin{definition}
	Say a random variable $W$ is \emph{sub-Gaussian}\index{sub-Gaussian} with parameter $\sigma > 0$ if
	\[
	\mathbb{E} e^{\alpha(W - \mathbb{E} W)} \leq e^{\alpha^2 \sigma^2/2},
	\]
	for all $\alpha \in \mathbb{R}$.
\end{definition}

Then we have the following tail bound:

\begin{proposition}
	If $W$ is $\sigma$-sub-Gaussian, then
	\[
	\mathbb{P}(W - \mathbb{E} W \geq t) \leq e^{-t^2/(2\sigma^2)},
	\]
	for all $t \geq 0$.
\end{proposition}

Note that if $W$ is $\sigma$-sub-Gaussian, then it is also $\sigma'$-sub-Gaussian for $\sigma' \geq \sigma$. Moreover, $-W$ is also $\sigma$-sub-Gaussian, so
\[
\mathbb{P}(|W - \mathbb{E}W| \geq t) \leq \mathbb{P}(W - \mathbb{E}W \geq t) + \mathbb{P}(-(W - \mathbb{E} W) \geq t) \leq 2e^{-t^2/(2\sigma^2)}.
\]
Note we do not require $W$ to be non-negative, as $\phi(x) = e^{\alpha x}$ automatically makes our random variables positive.

\begin{exbox}
	A \emph{Rademacher random variable}\index{Rademacher random variable} $\epsilon$ takes values $+1$ or $-1$, each with equal probability $1/2$. It is $1$-sub-Gaussian, as
	\[
	\mathbb{E} e^{\alpha \epsilon} = \frac{1}{2}(e^\alpha + e^{-\alpha}) = \sum_{k = 0}^{\infty} \frac{(\alpha^2)^k}{(2k)!} \leq \sum_{k = 0}^\infty \frac{(\alpha^2/2)^k}{k!} = e^{\alpha^2/2}.
	\]
\end{exbox}

\begin{lemma}[Hoeffding's Lemma]\index{Hoeffding's lemma}
	If $W$ is a random variable taking values in $[a, b]$, then $W$ is $\sigma$-sub-Gaussian with parameter $(b-a)/2$.
\end{lemma}

\begin{proofbox}
	We prove a weaker version with $\sigma = b - a$. First, we claim if random variable $Z$ is symmetric with $|Z| \leq m$, then
	\[
	\mathbb{E} e^{\alpha Z} \leq e^{\alpha^2m^2/2}.
	\]
	Indeed, $Z \overset d= \epsilon Z$, where $\epsilon \indep Z$ is a Rademacher random variable:
	\begin{align*}
		\mathbb{E} e^{\alpha Z} &= \mathbb{E} \mathbb{E}[e^{\alpha \epsilon Z} \mid Z] \\
					&\leq \mathbb{E} e^{\alpha^2 Z^2/2} \leq e^{\alpha^2 m^2/2}.
	\end{align*}
	Let $W'$ be an independent copy of $W$. Then
	\begin{align*}
		\mathbb{E} e^{\alpha(W - \mathbb{E} W')} &= \mathbb{E} e^{\alpha \mathbb{E}[W - W'\mid W]} \\
							 &\leq \mathbb{E}\mathbb{E}[ e^{\alpha(W - W')} \mid W] \\
							 &= \mathbb{E} e^{\alpha(W - W')}.
	\end{align*}
	But $W - W'$ is symmetric, and $|W - W'| \leq b - a$.
\end{proofbox}

\begin{proposition}
	Suppose $W_1, \ldots, W_n$ are independent with each $W_i$ $\sigma_i$-sub-Gaussian. Then for $\gamma \in \mathbb{R}^n$, $\gamma^T W$ is $\sigma$-sub-Gaussian with
	\[
	\sigma = \left( \sum_{i = 1}^n \sigma_i^2 \gamma_i^2 \right)^{1/2}.
	\]
\end{proposition}

%lecture 7

\begin{proofbox}
	Let's look at
	\begin{align*}
		\mathbb{E} \exp \left( \alpha \sum_{i = 1}^n \gamma_i(W_i - \mathbb{E}W_i) \right) &= \mathbb{E} \prod_{i = 1}^n e^{(\alpha(W_i - \mathbb{E} W_i)\gamma_i} \\
												   &= \prod_{i= 1}^n \mathbb{E} e^{\alpha(W_i - \mathbb{E} W_i)\gamma_i} \\
												   &\leq \prod_{i = 1}^n e^{\alpha^2 \gamma_i^2 \sigma_i^2/2} \\
												   &= \exp \left( \frac{\alpha^2 \sum \gamma_i^2 \sigma_i^2}{2} \right).
	\end{align*}
\end{proofbox}

From the above, if $W_1, \ldots, W_n$ are independent random variables taking values in $[a_i, b_i]$ for all $i$, then for $t \geq 0$,
\[
\mathbb{P} \left( \frac{1}{n} \sum_{i = 1}^n (W_i-  \mathbb{E} W_i ) \geq t \right) \leq \exp \left( - \frac{t^2}{2 \frac{1}{n^2} \sum (\frac{b_i - a_i}{2})^2} \right) = \exp \left(- \frac{2n^2t^2}{\sum (b_i - a_i)^2} \right).
\]
This is known as \emph{Hoeffding's inequality}\index{Hoeffding's inequality}.

\begin{proposition}
	Suppose $W_1, \ldots, W_d$ are mean zero and $\sigma$-sub-Gaussian. Then
	\[
		\mathbb{E} \max_{j = 1, \ldots, d} W_j \leq \sigma \sqrt{2 \log d}.
	\]
\end{proposition}

\begin{proofbox}
	Let $\alpha > 0$. We have
	\begin{align*}
		\mathbb{E} \max_{j = 1, \ldots, d} e^{\alpha W_j} \leq \sum_{j = 1}^d \mathbb{E} e^{\alpha W_j} \leq d e^{\alpha^2 \sigma^2/2}.
	\end{align*}
	Now we know
	\[
	\mathbb{E} \exp(\alpha \max_j W_j) \geq \exp(\alpha \mathbb{E} \max_j W_j),
	\]
	by Jensen's inequality applied to $u \mapsto e^{\alpha u}$. Hence we have that
	\[
	\mathbb{E} \max_j W_j \leq \inf_{\alpha > 0} \left( \frac{\log d}{\alpha} + \frac{\alpha \sigma^2}{2} \right).
	\]
	Take $\alpha$ such that
	\[
		\frac{\log d}{\alpha^2} = \frac{\sigma^2}{2} \implies \alpha = \frac{\sqrt{2 \log d}}{\sigma}.
	\]
	This gives
	\[
		\mathbb{E} \max_j W_j \leq \sigma \sqrt{2 \log d}.
	\]
\end{proofbox}

Now we are done with the theory of concentration inequalities. We continue to applications.

\subsection{Finite Hypothesis Class}
\label{sub:fin_hyp}

\begin{theorem}
	Consider the classification setting (with $0$-$1$ loss), and suppose $|\mathcal{H}| < \infty$. With probability at least $1 - \delta$, and ERM $\hat h$ satisfies
	\[
		R(\hat h) - R(h^\ast) \leq \sqrt{\frac{2( \log |\mathcal{H}| + \log(1/\delta))}{n}}.
	\]
\end{theorem}

If we take $\delta = 0.01$ and $|\mathcal{H}| = 10^n$, then the right hand side is about $\sqrt{2 \log 10}\sqrt{\frac{r+2}{n}}$.

\begin{proofbox}
	Let $t > 0$. Then
	\begin{align*}
		\mathbb{P}(R(\hat h) - R(h^\ast) > t) &= \mathbb{P}(R(\hat h) - R(h^\ast) > t, \hat h \neq h^\ast) \\
						      &\geq \mathbb{P}(R(\hat h) - \hat R(\hat h)) > t/2, \hat h \neq h^\ast) \tag{1}\\
						      &\quad + \mathbb{P}(\hat R(h^\ast) - R(h^\ast)) > t/2). \tag{2}
	\end{align*}
	We bound (2) first. By Hoeffding's inequality,
	\[
		\mathbb{P}\left( \frac{1}{n} \sum_{i = 1}^n [\ell(h^\ast(X_i), Y_i) - \mathbb{E} \ell(h^\ast(X_i), Y_i)] > t/2 \right) \leq e^{-nt^2/2}.
	\]
	For (1), let $\mathcal{H}_- = \mathcal{H} \setminus \{h^\ast\}$. Then
	\[
		\{R(\hat h) - \hat R(\hat h) > t/2, \hat h \neq h^\ast\} \subseteq \bigcup_{h \in \mathcal{H}_-} \{R(h) - \hat R(h) > t/2\}.
	\]
	Then from a union bound, we can bound (1) by
	\[
	\sum_{h \in \mathcal{H}_-} \mathbb{P}(R(h) - \hat R(h) > t/2) \leq (|\mathcal{H}| - 1) e^{-nt^2/2},
	\]
	again by Hoeffding's inequality. Hence
	\[
	\mathbb{P}(R(\hat h) - R(h^\ast) > t) \leq |\mathcal{H}| e^{-nt^2/2} = \delta,
	\]
	if we set
	\[
		t = \sqrt{- \frac{2}{n} \log \frac{\delta}{|\mathcal{H}|}} = \sqrt{\frac{2 (\log |\mathcal{H}| + \log(1/\delta))}{n}}.
	\]
\end{proofbox}

\begin{exbox}
	Consider the classification setting with $X \in [0,1)^2$. Partition $[0, 1)^2$ into $m^2$ disjoint squares $R_1, \ldots, R_{m^2} \subset [0, 1)^2$ of the form $[\frac{r}{m}, \frac{r+1}{m}) \times [\frac{s}{m}, \frac{s+1}{m})$, for $r, s = 0, \ldots, m-1$. Let
	\[
	\bar Y_j = \sgn \left( \sum_{i \mid X_i \in R_j} Y_i \right),
	\]
	and the final classifier (the histogram classifier) will be
	\[
j	\hat h^{\mathrm{hist}}(x) = \sum_{i = 1}^{m^2} \bar Y_j \mathbbm{1}_{R_j}(x).
	\]
	Then $\hat h^{\mathrm{hist}}$ is an ERM over $\mathcal{H}$ consisting of the $2^{m^2}$ classifiers, each corresponding to a way of assigning labels $+1$ or $-1$ to the regions $R_1, \ldots, R_{m^2}$.

	Then our result gives us, with probability at least $1 - \delta$,
	\[
		R(\hat h^{\mathrm{hist}}) - R(h^\ast) \leq \sqrt{\frac{2m^2 \log 2 + 2 \log (1/\delta)}{n}}.
	\]

\end{exbox}
%lecture 8

\subsection{Rademacher Complexity}
\label{sub:r_comp}

Recall that we are minimizing
\begin{align*}
	R(\hat h) - R(h^\ast) &= R(\hat h) - \hat R(\hat h) + \hat R(h) - \hat R(h^\ast) + \hat R(h^\ast) - R(h^\ast) \\
			      &\leq \sup_{h \in \mathcal{H}} (R(h) - \hat R(h)) + \hat R(h^\ast) - R(h^\ast).
\end{align*}
Taking expected values, we get
\[
\mathbb{E}[R(\hat h) - R(h^\ast)] \leq \mathbb{E} G = \mathbb{E} \sup_{h \in \mathcal{H}} (R(h) - \hat R(h)).
\]
Let $Z_i = (X_i, Y_i)$ for all $i$, and define
\[
	\mathcal{F} = \{(x, y) \mapsto - \ell(h(x), y) \mid h \in \mathcal{H}\}.
\]
Then we can see
\[
	G = \sup_{f \in \mathcal{F}} \frac{1}{n}\sum_{i = 1}^n [f(Z_i) - \mathbb{E} f(Z_i)].
\]
To motivate the following definitions, consider the special case when $f(Z_i) \overset d = -f(Z_i)$. Then $\mathbb{E} f(Z_i) = 0$, and if $\eps_{1:n}$ are iid Rademacher random variables with $\eps_{1:n} \indep Z_{1:n}$, then $(f(Z_i))_{i = 1}^n \overset d = (\eps_i f(Z_i))_{i = 1}^n$. Thus
\begin{align*}
	\mathbb{E} G = \mathbb{E} \mathbb{E}\left[ \sup_{f \in \mathcal{F}} \frac{1}{n} \sum_{i = 1}^n \eps_i f(Z_i) \mid Z_{1:n} \right].
\end{align*}

\begin{definition}
	Let $\mathcal{F}$ be any class of functions $f : \mathcal{Z} \to \mathbb{R}$, and let $z_1, \ldots, z_n \in \mathcal{Z}$.
	\begin{itemize}
		\item Let $\mathcal{F}(z_{1:n}) = \{(f(z_i), \ldots, f(z_n))\ \mid f \in \mathcal{F}\} \subseteq \mathbb{R}^n$.
		\item Define the \emph{empirical Rademacher complexity}\index{Rademacher complexity}
			\[
				\hat{\mathcal{R}} (\mathcal{F}(z_{1:n})) = \mathbb{E} \sup_{f \in \mathcal{F}} \frac{1}{n} \sum_{i = 1}^n \eps_i f(z_i),
			\]
		where $\eps_{1:n}$ are iid Rademacher. Given iid $Z_1, \ldots, Z_n \indep \eps_{1:n}$, we can may also view the empirical Rademacher complexity as a random variable
		\[
			\hat{\mathcal{R}} (\mathcal{F}(Z_{1:n})) = \mathbb{E}\left[ \sup_{f \in \mathcal{F}} \frac{1}{n} \sum_{i = 1}^n f(Z_i) \eps_i \mid Z_{1:n} \right].
		\]
	\item Define the \emph{Rademacher complexity} of $\mathcal{F}$ as
		\[
			\mathcal{R}_n(\mathcal{F}) = \mathbb{E} \hat{\mathcal{R}}(\mathcal{F}(Z_{1:n})).
		\]
	\end{itemize}
\end{definition}

\begin{theorem}
	Let $\mathcal{F}$ and random variables $Z_{1:n}$ be as in the definition. Then
	\[
		\mathbb{E}\left[ \sup_{f \in \mathcal{F}} \frac{1}{n} \sum_{i = 1}^n [f(Z_i) - \mathbb{E} f(Z_i)] \right] \leq 2 \mathcal{R}_n(\mathcal{F}).
	\]
\end{theorem}

\begin{proofbox}
	Introduce an independent copy $(Z_i')_{i = 1}^n$ of $(Z_i)_{i = 1}^n$. Then,
	\begin{align*}
		\mathbb{E}\left[ \sup_{f \in \mathcal{F}}\frac{1}{n} \sum_{i = 1}^n [f(Z_i) - \mathbb{E} f(Z_i')] \right] &= \mathbb{E}\left[ \sup_{f \in \mathcal{F}} \mathbb{E} \left( \frac{1}{n} \sum_{i = 1}^n [f(Z_i) - f(Z_i')] \mid Z_{1:n}\right) \right] \\
															  &\leq \mathbb{E} \left( \sup_{f \in \mathcal{F}} \frac{1}{n} \sum_{i = 1}^n [f(Z_i) - f(Z_i')] \right).
	\end{align*}
	Indeed for any collection of random variables $V_t$, we have
	\[
	\mathbb{E} V_t \leq \mathbb{E} \sup_{t'} V_{t'} \implies \sup_t \mathbb{E}V_t \leq \mathbb{E} \sup_t V_t.
	\]
	Take $\eps_{1:n}$ Rademacher random variables with $\eps_{1:n} \indep (Z_{1:n}, Z_{1:n}')$. Then,
	\begin{align*}
		\sup_{f \in \mathcal{F}} \frac{1}{n} \sum_{i = 1}^n [f(Z_i) - f(Z_i')] &\overset d = \sup_{f \in \mathcal{F}} \frac{1}{n} \sum_{i = 1}^n \eps_i[f(Z_i) - f(Z_i')] \\
										       &\leq \sup_{f \in \mathcal{F}} \frac{1}{n} \eps_i f(Z_i) + \sup_{g \in \mathcal{F}} \frac{1}{n} \sum_{i = 1}^n (-\eps_i) g(Z_i).
	\end{align*}
	Thus we get
	\[
		\mathbb{E}\left(\sup_{f \in \mathcal{F}} \frac{1}{n} \sum_{i = 1}^n [f(Z_i) - \mathbb{E} f(Z_i)] \right) \leq 2 \mathcal{R}_n (\mathcal{F}).
	\]
\end{proofbox}

\begin{theorem}
	Let $\mathcal{F} = \{(x, y) \mapsto \ell(h(x), y) \mid h \in \mathcal{H}\}$ (here loss can be arbitrary). Then
	\[
	\mathbb{E}[R(\hat h) - R(h^\ast)] \leq 2\mathcal{R}_n (\mathcal{F}).
	\]
\end{theorem}

\begin{proofbox}
	Recall that
	\[
		\mathbb{E}[R(\hat h) - R(h^\ast)] \leq \mathbb{E} \sup_{h \in \mathcal{H}}[R(h) - \hat R(h)] \leq 2 \mathcal{R}_n (-\mathcal{F}),
	\]
	by our previous theorem, where $-\mathcal{F} = \{ -f \mid f \in \mathcal{F}\}$. But if $\eps_{1:n}$ are iid Rademacher random variables, then $\eps_{1:n} \overset d = - \eps_{1:n}$, so $\mathcal{R}_n(\mathcal{F}) = \mathcal{R}_n(-\mathcal{F})$.
\end{proofbox}

\subsection{VC Dimension}
\label{sub:vc_dim}

Consider a classification setting (with 0-1 loss). Our aim is to bound $\mathcal{R}_n(\mathcal{F})$ where
\[
	\mathcal{F} = \{(x, y) \mapsto \ell(h(x), y) \mid h \in \mathcal{H}\}.
\]
Note that if $x_{1:n} \in \mathcal{X}^n$, and $y_{1:n} \in \{-1, 1\}^n$ with $z_i = (x_i, y_i)$, then
\[
|\mathcal{F}(z_{1:n})| = |\mathcal{H}(x_{1:n})|,
\]
as we can match $(\ell(h(x_i), y_i))_{i = 1}^n$ to $(h(x_i))_{i = 1}^n$.
\begin{lemma}
	The empirical Rademacher complexity
	\[
		\hat{\mathcal{R}}(\mathcal{F}(z_{1:n})) \leq \sqrt{\frac{2 \log|\mathcal{H}(x_{1:n})|}{n}}.
	\]
\end{lemma}

\begin{proofbox}
	Let $|\mathcal{F}(z_{1:n})| = d$, and $\mathcal{F}' = \{f_1, \ldots, f_d\} \subseteq \mathcal{F}$ such that $\mathcal{F}'(z_{1:n}) = \mathcal{F}(z_{1:n})$. Given $\eps_{1:n}$ iid Rademacher random variables, let
	\[
	W_j = \frac{1}{n} \sum_{i = 1}^n \eps_i f_j(z_i).
	\]
	Then
	\[
		\hat{\mathcal{R}}(\mathcal{F}(z_{1:n})) = \mathbb{E} \max_{j = 1, \ldots, d} W_j.
	\]
	Recall that each $\eps_i$ is $1$-sub-Gaussian, so each $W_j$ is $n^{-1/2}$-sub-Gaussian. Hence
	\[
		\mathbb{E} \max_{j = 1, \ldots, d} W_j \leq \sqrt{\frac{2 \log d}{n}}.
	\]
\end{proofbox}

%lecture 9

\begin{definition}
	Let $\mathcal{H}$ be a class of functions with $h : \mathcal{X} \to \{a, b\}$, with $|\mathcal{H}| \geq 2$. Say $\mathcal{H}$ \emph{shatters}\index{shatter} $x_{1:n} \in \mathcal{X}^n$ if $|\mathcal{H}(x_{1:n})| = 2^n$, i.e. our hypothesis class contains any function on this set.

	Define the \emph{shattering coefficient}\index{shattering coefficient}
	\[
	s(\mathcal{H}, n) = \max_{x_{1:n} \in \mathcal{X}^n} |\mathcal{H}(x_{1:n})|.
	\]
	Define the \emph{VC dimension}\index{VC dimension} $\mathrm{VC}(\mathcal{H})$ to be the largest $n$ such that $x_{1:n}$ is shattered. If no such $n$ exists, say $\mathrm{VC}(\mathcal{H})= \infty$, i.e.
	\[
		\mathrm{VC}(\mathcal{H}) = \sup\{n \in \mathbb{N} \mid s(\mathcal{H}, n) = 2^n\}.
	\]
\end{definition}

\begin{remark}
	Note that if $|\mathcal{H}(x_{1:n})| = 2^n$, then $|\mathcal{H}(x_{1:m})| = 2^m$ for all $m \leq n$, so $\mathrm{VC}(\mathcal{H}) = n$ if and only if it shatters $x_{1:n}$, and it does not shatter any $x_{1:n+1}$.
\end{remark}

\begin{exbox}
	Take $\mathcal{H} = \{\mathbbm{1}_{[a, b)} \mid b > a\}$. This is an infinite class. For $n$ points, we find that $s(\mathcal{H}, n) \leq (n+1)^2$ since $(\mathbbm{1}_{[a,b)}(x_i)) = (\mathbbm{1}_{[a', b')}(x_i))$ if and only if $a, a'$ are in the same interval $[x_i, x_{i+1})$, and so are $b, b'$ (except maybe if they are degenerate).

	For $n = 2$, any set can be shattered, and no $x_{1:3}$ can be shattered, as for $x_1 < x_2 < x_3$, we cannot have $x_1, x_3 \mapsto 1$ and $x_2 \to 0$. So $\mathrm{VC}(\mathcal{H}) = 2$.
\end{exbox}

\begin{lemma}[Sauer-Shelah Lemma]
	Let $\mathcal{H}$ be as in the definition with $\mathrm{VC}(\mathcal{H}) < \infty$. Then $s(\mathcal{H}, n) \leq (n+1)^{\mathrm{VC}(\mathcal{H})}$.
\end{lemma}

An important consequence is that
\[
	\mathcal{R}_n(\mathcal{F}) \leq \sqrt{\frac{2 \mathrm{VC}(\mathcal{H}) \log(n+1)}{n}}.
\]

\begin{exbox}
	Let $\mathcal{X} = \mathbb{R}^p$, and consider
	\[
		\mathcal{A} = \left\{ \prod_{j = 1}^p (-\infty, a_j] \mid a_1, \ldots, a_p \in \mathbb{R}\right\},
	\]
	and $\mathcal{H} = \{\mathbbm{1}_{A} \mid A \in \mathcal{A}\}$. We claim that $\mathrm{VC}(\mathcal{H}) = p$. Indeed, the set of basis vectors $e_1, \ldots, e_p$ can be shattered. For $I \subseteq \{1, \ldots, p\}$, set $a_j = 1$ if $j \in I$, and $0$ else. Then
	\[
		e_j \in \prod_{j = 1}^p (-\infty, a_j] \iff j \in I.
	\]
	Now take $x_1, \ldots, x_{p+1} \in \mathbb{R}^p$. For each coordinate $j = 1, \ldots, p$, let $I_j = \{k \mid x_{kj} \geq x_{lj} \text{ for all } l\}$. There there must be some $k^\ast$ not a unique element of any $I_1, \ldots, I_p$, but then no $h \in \mathcal{H}$ has $h(x_{k^\ast}) = 0$ and $h(x_k) = 1$ for $k \neq k^\ast$, so $x_1, \ldots, x_{p+1}$ cannot be shattered and $\mathrm{VC}(\mathcal{H}) = p$.
\end{exbox}

Let $\mathcal{F}$ be a vector space of function $f : \mathcal{X} \to \mathbb{R}$, e.g. affine maps. We can form classifiers $\mathcal{H}=  \{ \sgn f \mid f \in \mathcal{F}$ \}.

\begin{proposition}
	Let $\mathcal{H}$ be as above, for some vector space $\mathcal{F}$. Then $\mathrm{VC}(\mathcal{H}) \leq \dim (\mathcal{F})$.
\end{proposition}

\begin{proofbox}
	Let $d = \dim (\mathcal{H}) + 1$, and take $x_1, \ldots, x_d \in \mathcal{X}$. We need to show these cannot be shattered. Consider the linear map $L : \mathcal{F} \to \mathbb{R}^d$ by $f \mapsto (f(x_1), \ldots, f(x_d))^T$. Then since $\dim (\mathcal{L}(\mathcal{F})) \leq \dim \mathcal{F} = d-1$, there exists $\gamma \in \mathbb{R}^d$ orthogonal to everything in $L(\mathcal{F})$ i.e.
	\[
	\sum_{i : \gamma_i > 0}\gamma_i f(x_i) + \sum_{i : \gamma_i \leq 0} \gamma_i f(x_i) = 0,
	\]
	for all $f \in \mathcal{F}$. Without loss of generality there is $i$ with $\gamma_i > 0$. But then we can't have $h(x_i) = 1$ for $f(x_i) > 0$, and $h(x_{i}) = -1$ for $f(x_i) < 0$, otherwise the left hand side of the above expression would be strictly positive.

	Thus $x_1, \ldots, x_d$ cannot be shattered, so $\mathrm{VC}(\mathcal{H}) \leq d - 1 = \dim \mathcal{F}$.
\end{proofbox}

\begin{exbox}
	Let $\mathcal{X} = [0, 1)^2$, and $\mathcal{F}$ to be the polynomials of degree at most $d$. Taking $\mathcal{H} = \{ \sgn \circ f \mid f \in \mathcal{F}\}$, we get
	\[
	\dim(\mathcal{F}) \leq (d+1) + (d) + (d-1) + \cdots + 1 = \frac{(d+1)(d+2)}{2}.
	\]
	Hence for $d = 5$, $\dim (\mathcal{F}) = 21$. For an ERM $\hat h$, we have
	\[
		\mathbb{E}[R(\hat h) - R(h^\ast)] \leq 2 \sqrt{\frac{2 \times 21 \log(n+1)}{n}}.
	\]
	Comparing this to $\hat h^{\mathrm{hist}}$,
	\[
		\mathbb{E}[R(\hat h^{\mathrm{hist}} - R(h^\ast)] \leq \frac{1}{2 \sqrt n} \sqrt{2 m^2 \log 2}.
	\]
\end{exbox}

%lecture 10

\newpage

\section{Computation for Empirical Risk Minimisation}
\label{sec:comp}

Computation of the ERM with $0$-$1$ can be computationally very hard. We thus aim to approximate the ERM via a convex optimisation problem, i.e. minimizing a convex function over a convex set.

\subsection{Convexity}
\label{sub:conv}

$C \subseteq \mathbb{R}^d$ is \emph{convex}\index{convex} if
\[
	x, y \in C \implies tx + (1 - t)y \in C \text{ for all } t \in (0, 1).
\]
The intersections of convex sets are convex.

\begin{definition}
	\begin{itemize}
		\item[]
		\item For $S \subseteq \mathbb{R}^d$, the \emph{convex hull}\index{convex hull} $\mathrm{conv}\, S$ of $S$ is the intersection of all convex sets containing $S$.
		\item $v \in \mathbb{R}^d$ is a \emph{convex combination}\index{convex combination} of $v_1, \ldots, v_n \in \mathbb{R}^d$ if
			\[
			v = \alpha_1 v_1 + \cdots + \alpha_m v_m
			\]
			for some $\alpha_1, \ldots, \alpha_m \geq 0$, and $\alpha_1 + \cdots + \alpha_m = 1$.
	\end{itemize}
\end{definition}

\begin{lemma}
	For $S \subseteq \mathbb{R}^d$, $v \in \mathrm{conv}\, S$ if and only if $v$ is a convex combination of some points in $S$.
\end{lemma}

\begin{proofbox}
	Let $D$ be the set of all convex combinations of points in $S$.

	We show $D \subseteq \mathrm{conv} \, S$. Intersections of convex sets are convex, so we know $\mathrm{conv} \, S$ is a convex set containing $S$. Thus any convex combinations of two points from $S$ is in $\mathrm{conv} \, S$.

	Let's induct. Suppose for $m \geq 2$, that any convex combination of $m$ points is in $\mathrm{conv} \, S$. Take $v_1, \ldots, v_{m+1} \in S$, and $\alpha_1, \ldots, \alpha_{m+1} \geq 0$ with $\alpha_1 + \cdots + \alpha_{m+1} = 1$. Now consider $v = \alpha_1 v_1 + \cdots + \alpha_{m+1} + v_{m+1}$.

	If $a_{m+1} = 1$, then $v = v_{m+1} \in S$. If not, call $t = a_1 + \cdots + a_m > 0$. Then
	\[
	v = t \left( \frac{a_1}{t} v_1 + \cdots + \frac{a_m}{t} v_m \right) + (1 - t) v_{m+1}.
	\]
	By our induction hypothesis, the first part is in $\mathrm{conv} \, (S)$, by our induction hypothesis. Hence the entire thing is in $\mathrm{conv} \, S$.
\end{proofbox}

\begin{lemma}
Let $S \subseteq \mathbb{R}^d$. For any linear map $L : \mathbb{R}^d \to \mathbb{R}^m$, $\mathrm{conv} \, L(S) = L(\mathrm{conv} \, S)$.
\end{lemma}

\begin{proofbox}
	$v \in \mathrm{conv} \, L(S)$ if and only if there exists $m \in \mathbb{N}$ and $v_1, \ldots, v_m \in S$, $\alpha_i \geq 0$ with $\alpha_1 + \cdots + \alpha_m = 1$ with
	\[
	v = \sum_{j = 1}^m \alpha_j L(v_j) = L \left( \sum_{i = 1}^m \alpha_j v_j \right) \in L(\mathrm{conv} \, S).
	\]
	Moreover, $v \in L(\mathrm{conv} \, S)$ if and only if $v$ takes the form $L(\sum_{i = 1}^m \alpha_j v_j)$.
\end{proofbox}

\subsection{Convex Functions}
\label{sub:convf}

Let $C \subseteq \mathbb{R}^d$ be convex. Then $f : C \to \mathbb{R}$ is \emph{convex}\index{convex function} if
\[
f( (1 - t) x + t y) \leq (1 - t) f(x) + t f(y),
\]
for $t \in (0, 1)$ and $x, y \in C$. It is \emph{strictly convex} if this inequality is strict.

By the triangle inequality, every norm is a convex function. Moreover, any local minimum of a convex function is a global minimum: if $x \in C$ is such that for all $y \in C$, there exists some $0 < t < 1$ with $f(x) \leq f((1 - t)x + ty)$, then
\[
f(x) \leq (1 - t)f(x) + t f(y) \implies f(x) \leq f(y).
\]

Convex functions satisfy a lot of properties. See notes.

\subsection{Convex Surrogates}
\label{sub:conv_s}

Consider $\mathcal{H} = \{x \mapsto \sgn(\beta^T x)  \mid \beta \in \mathbb{R}^p$ \}. Then ERM involves minimizing over $\beta \in \mathbb{R}^p$
\[
	\frac{1}{n} \sum_{i = 1}^n \mathbbm{1}_{Y_i \neq \sgn (X_i^T \beta)} \approx \frac{1}{n} \sum_{i = 1}^n \mathbbm{1}_{(-\infty, 0]}(Y_i X_i^T \beta),
\]
by ignoring when $X_i^T \beta = 0$. This is nasty to work with as $\mathbbm{1}_{(-\infty, 0]}$ is a non-convex, discontinuous function. If we replace it with a convex function, the resulting objective function will be convex. We therefore make the following changes to our setup:
\begin{itemize}
	\item Take $\mathcal{H}$ to be a family of functions $h : X \to \mathbb{R}$. We obtain the corresponding classifier by composition with $\sgn$.
	\item We consider losses of the form
		\[
		\ell(h(x), y) = \phi(yh(x)),
		\]
		where $\phi : \mathbb{R} \to (0, \infty)$ is convex.
\end{itemize}

%lecture 11

The $\phi$-risk\index{$\phi$-risk} is then
\[
R_\phi(h) = \mathbb{E} \phi(Y h(X)).
\]
We want to minimize the empirical $\phi$-risk\index{empirical $\phi$-risk}
\[
\hat R_\phi(h) = \frac{1}{n} \sum_{i = 1}^n \phi(Y_i h (X_i)).
\]
Classical choices of $\phi$ include the:
\begin{itemize}
	\item Hinge loss: $\phi(u) = \max(1 - u, 0)$.
	\item Logistic loss: $\phi(u) = \log_2(1 + e^{-u}) = \log(1 + e^{-u})/2$.
	\item Exponential loss $\phi(u) = e^{-u}$.
\end{itemize}
We would like the minimizer $h_{\phi, 0}$ of $R_\phi(h)$ over all functions $h$ (assuming it exists) to be such that $x \mapsto \sgn(h_{\phi, 0}(x))$ is equivalent to a Bayes classifier $x \mapsto \sgn(\eta(x) - 1/2)$, where $\eta(x) = \mathbb{P}(Y = 1 \mid X = x)$.

To study this, we examine the conditional $\phi$-risk
\[
\mathbb{E}[\phi(Yh(X)) \mid X = x] = \phi(h(x))\eta(x) + \phi(-h(x))(1 - \eta(x)).
\]
Let $C_\eta(\alpha) = \phi(\alpha) \eta + \phi(-\alpha)(1 - \eta)$, for generic $\eta \in [0, 1]$.

\begin{definition}
	We say $\phi : \mathbb{R} \to [0, \infty)$ is \emph{classification calibrated}\index{classification calibrated} if, for any $\eta \in [0, 1]$ with $\eta \neq 1/2$,
	\[
	\inf_{\alpha \in \mathbb{R}} C_\eta(\alpha) \leq \inf_{\alpha(2\eta - 1) \leq 0} C_\eta(\alpha).
	\]
\end{definition}

\begin{theorem}
	Let $\phi : \mathbb{R} \to [0, \infty)$ be convex. Then $\phi$ is classification calibrated if it is differentiable at $0$, with $\phi'(0) < 0$.
\end{theorem}

\begin{proofbox}
	$C_\eta$ is convex and differentiable at $0$ with
	\[
	C_\eta'(0) = \phi'(0) \eta - (1 - \eta) \phi'(0) = (2\eta - 1)\phi'(0).
	\]
	Suppose $\eta > 1/2$, so $C_\eta'(0) < 0$. By convexity,
	\[
	C_\eta(\alpha) \geq C_\eta(0) + \alpha C_n'(0) \geq C_n(0)
	\]
	for $\alpha \leq 0$, and also
	\[
	\lim_{\alpha \downarrow 0} \frac{C_\eta(\alpha) - C_\eta(0)}{\alpha} = C_\eta'(0) < 0,
	\]
	so there exists $\alpha^\ast > 0$ with $C_\eta(\alpha^\ast) < C_\eta(0) = \inf_{\alpha \leq 0} C_\eta(\alpha)$.

	Note that $C_{1/2 + \theta}(\alpha) = C_{1/2 - \theta}(-\alpha)$ for $\theta \in [0, 1/2]$, so when $\eta < 1/2$, there is $\alpha^\ast < 0$ with
	\[
	C_\eta(\alpha^\ast) < C_\eta(0) = \inf_{\alpha \geq 0}C_\eta(\alpha).
	\]
\end{proofbox}

Being classification calibrated means an optimal classifier will be a Bayes classifier.

\subsection{Rademacher Complexity}
\label{sub:conv_rc}

Recall theorem 8. It says when $\mathcal{F} = \{(x, y) \mapsto \phi(yh(x)) \mid h \in \mathcal{H}\}$, then
\[
\mathbb{E} R_\phi(\hat h) - R_\phi(h^\ast) \leq 2 \mathcal{R}_n(\mathcal{F}).
\]
\begin{lemma}[Contraction Lemma]
	Let $r = \sup |h(x)|$, over $h \in \mathcal{H}$ and $x \in \mathcal{X}$. Suppose there exists $L \geq 0$ with
	\[
	|\phi(u) - \phi(u')| \leq L|u - u'|,
	\]
	for all $u, u' \in [-r, r]$, so $f$ is $L$-Lipschitz on $[-r, r]$. Then for any $(x_1, y_1), \ldots, (x_n, y_n) \in \mathcal{X} \times \{-1, 1\}$, writing $z_i = (x_i, y_i)$, we have
	\[
		\hat{\mathcal{R}}(\mathcal{F}(z_{1:n})) \leq L \hat{\mathcal{R}}(\mathcal{H}(x_{1:n})),
	\]
	so $\mathcal{R}_n(\mathcal{F}) \leq L \mathcal{R}_n(\mathcal{H})$.
\end{lemma}

In the simple case $\mathcal{H} = \{x \mapsto x^T \beta \mid \beta \in \mathbb{R}^p\}$,
\[
\mathcal{R}_n(\mathcal{H}) = \mathbb{E} \left[ \sup_{\beta \in \mathbb{R}^p} \frac{1}{n} \sum_{i = 1}^n \epsilon_i X_i^T \beta \right]
\]
may not be finite.

\subsection{Constraints}
\label{sub:const}

Suppose $\xi = \{x \in \mathbb{R}^p \mid \|x\|_2 \leq C\}$, and consider
\[
	\mathcal{H} = \{x \mapsto x^T \beta \mid \|\beta\|_2 \leq \lambda\},
\]
for some $\lambda > 0$. Then
\begin{align*}
	\hat{\mathcal{R}} (\mathcal{H}(x_{1:n})) &= \frac{1}{n} \mathbb{E}\left[ \sup_{\|\beta\|_2 \leq \lambda} \sum_{i = 1}^n \epsilon_i x_i^T \beta \right] \\
						 &= \frac{\lambda}{n} \mathbb{E} \left( \biggl\| \sum_{i =1}^n \epsilon_i x_i\biggr\|_2 \right) \\
						 &\leq \frac{\lambda}{n} \left( \mathbb{E} \biggl\|\sum_{i= 1}^n \epsilon_i x_i\biggr\|_2^2 \right)^{1/2}.
\end{align*}
This thing can be found:
\begin{align*}
	\mathbb{E} \biggl\|\sum_{i = 1}^n \epsilon_i x_i \biggr\|_2^2 = \mathbb{E} \left( \sum_{i = 1}^n \sum_{j = 1}^n \epsilon_i \epsilon_j x_i^T x_j \right) = \sum_{i = 1}^n \|x_i\|_2^2 \leq nC^2,
\end{align*}
so we finally get
\[
	\hat{\mathcal{R}}(\mathcal{H}(x_{1:n})) \leq \frac{\lambda C}{\sqrt n}.
\]
%lecture 12

This also extends to variables in which $\mathbb{E} \|X_i\|_2^2 < \infty$.

\begin{exbox}
	Take $\phi$ to be the hinge loss, which is Lipschitz with constant 1, and $\mathcal{H}$ as above. Then $\sgn \circ \hat h$ is known as the \emph{support vector classifier}\index{support vector classifier}.

	Then the above results give
	\[
	\mathbb{E} R_\phi (\hat h) - R(h^\ast) \leq \frac{2\lambda C}{\sqrt n}.
	\]
\end{exbox}

\subsection{\texorpdfstring{$\ell_1$}{l\_1} constraint}
\label{sub:l1_con}

Recall the $\ell_1$ norm of $u \in \mathbb{R}^p$ is
\[
\|u\|_1 = \sum_{j = 1}^p |u_j|.
\]
Now consider
\[
	\mathcal{H} = \{x \mapsto x^T \beta \mid \|\beta\|_1 \leq 1\}.
\]
This looks like a diamond in the $x,y$ plane. To bound $\mathcal{\hat R}(\mathcal{H}(x_{1:n}))$, we may use the following fact:

\begin{lemma}
	For any $A \subseteq \mathbb{R}^n$,
	\[
	\mathcal{\hat R}(A) = \mathcal{\hat R}(\mathrm{conv} \, A).
	\]
\end{lemma}

Note that $\{\beta \mid \|\beta\|_1 < 1\} = \mathrm{conv}(S)$, where
\[
	S = \bigcup_{j = 1}^p \{-e_j, e_j\}.
\]
Indeed, if $\|\beta\|_1 = 1$, then
\begin{align*}
	\beta = \sum_{j = 1}^p \beta_j e_j = \sum_{j =1}^p |\beta_j| \sgn(\beta_j) e_j \in \mathrm{conv}\, S.
\end{align*}
Next, if $\|\beta\|_1 < 1$, then
\[
\beta = t \frac{\beta}{\|\beta\|_1} + (1 - t) \left( \frac{-\beta}{\|\beta\|_1} \right) = (2t - 1) \frac{\beta}{\|\beta\|_1},
\]
where $t = (\|\beta\|_1 + 1)/2$. Now recall that
\[
\mathcal{H}(x_{1:n}) = \{
	\begin{pmatrix}
		x_1^T \beta & \cdots & x_n^T \beta
\end{pmatrix}^T \mid \|\beta\|_1 \leq \lambda\},
\]
so letting $L(\beta) = 
\begin{pmatrix}
	x_1^T \beta & \cdots & x_n^T \beta
\end{pmatrix}^T$, then
\[
\mathcal{H}(x_{1:n}) = L(\mathrm{conv}\, S) = \mathrm{conv}\, L(S).
\]
Thus,
\begin{align*}
	\mathcal{\hat R}(\mathcal{H}(x_{1:n})) &= \mathcal{\hat R}(L(S)) = \frac{\lambda}{n} \mathbb{E} \left( \max_{j = 1, \ldots, p} \left| \sum_{i = 1}^n \epsilon_i x_{ij} \right| \right),
\end{align*}
where $\epsilon_1, \ldots, \epsilon_n$ are iid Rademacher. Now, note
\[
\pm \sum_{i = 1}^n \epsilon_i x_{ij}
\]
is $\sigma$-sub-G, with
\[
\sigma = \left( \sum_{i = 1}^n x_{ij}^2 \right)^{1/2}.
\]
Therefore,
\begin{align*}
	\mathcal{\hat R}(\mathcal{H}(x_{1:n})) &\leq \frac{\lambda}{n} \max_{j = 1, \ldots, p} \left( \sum_{i = 1}^n x_{ij}^2 \right)^{1/2} \sqrt{2 \log (2p)}.
\end{align*}
If $\mathcal{X} = [-1, 1]^p$, then
\[
	\mathcal{\hat R}(\mathcal{H}(x_{1:n})) \leq \lambda \sqrt{\frac{2 \log(2p)}{n}}.
\]

\begin{exbox}
	Take $\phi$ to be the hinge loss, and $\mathcal{H}_1 = \{x^T \beta \mid \|\beta\|_1 \leq \lambda_1\}$. Suppose $\mathcal{X} = [-1, 1]^p$. Then the ERM $\hat h_1$ and $h_1^\ast = \mathrm{argmin}\, R_\phi(h)$ satisfy
	\[
		\mathbb{E} R_\phi(\hat h_1) - R_\phi(h_1^\ast) \leq 2 \lambda_1 \sqrt{ \frac{2 \log(2p)}{n}}.
	\]
	Recall if $\mathcal{H}_2 = \{x \mapsto x^T \beta \mid \|\beta\|_2 \leq \lambda\}$, then the ERM $\hat h_2$ and $h_2^\ast = \mathrm{argmin}\, R_\phi(h)$ satisfy
	\[
		\mathbb{E} R_\phi(\hat h_2) - R_\phi(h_2^\ast) \leq 2 \lambda_2 \sqrt{\frac{p}{n}}.
	\]
	Suppose there is $\beta^0 \in \mathbb{R}^p$ such that $h_0 : x \mapsto x^T \beta^0$ minimises $R_\phi(h)$ over all functions $h \in \{x \mapsto x^T \beta\}$.

	\begin{itemize}
		\item If $\beta^0 = \sqrt{p}^{-1}
			\begin{pmatrix}
				1 & \cdots & 1
			\end{pmatrix}^T
			$, then taking $\lambda_1 = \sqrt{p}$ gives $h_1^\ast = h_0$, so
			\[
				\mathbb{E} R_\phi(\hat h_1) - R_\phi(h_0) = \mathcal{O} \left( \sqrt{\frac{p \log p}{n}} \right).
			\]
		\item If $\beta^0 = \sqrt{s}^{-1}
			\begin{pmatrix}
				1 & \cdots & 1 & 0 & \cdots & 0
			\end{pmatrix}
			$, then taking $\lambda_1 = \sqrt{s}$ gives $h_1^\ast = h_0$, so
			\[
				\mathbb{E} R_\phi(\hat h_1) - R_\phi(h_0) = \mathcal{O}\left(\sqrt{ \frac{s \log p}{n}} \right).
			\]
	\end{itemize}
	In both cases, taking $\lambda_2 = 1$ gives $h_2^\ast = h_0$, so
	\[
		\mathbb{E} R_\phi(\hat h_2) - R(h_0) \leq 2 \sqrt{\frac{p}{n}}.
	\]
\end{exbox}

Conclusion: if every predictor is equally important, we expect ERM with $\mathcal{H}_2$ to outperform $\mathcal{H}_1$. However, if only $s \ll p$ predictors are important, then we expect ERM with $\mathcal{H}_1$ to do much better than $\mathcal{H}_2$.

%lecture 13

\subsection{Projections onto Convex Sets}
\label{sub:proj_conv}

Suppose we wish to minimize a convex function $f : C \to \mathbb{R}$, where $C \subseteq \mathbb{R}^d$ is a convex set. Given a candidate minimizer $\beta \in C$, consider the Taylor expansion
\[
f(z) \approx f(\beta) + \nabla f(\beta)^T (z - \beta).
\]
Taking $z = \beta + \eta \delta$ some some $\eta > 0$ and direction $\delta$, we have
\[
\min_{\delta \mid \|\delta\|_2 \leq 1} f(\beta + \eta \delta) \approx f(\beta) + \eta \min_{\delta \mid \|\delta\|_2 = 1} \nabla f(\beta)^T \delta.
\]
This suggest updating $\beta$ by moving in the direction $- \nabla f(\beta)$; however, this may result in leaving $C$, so we need a sensible way of getting back into $C$.

\begin{proposition}
	Let $C \subseteq \mathbb{R}^d$ be a closed convex set. Then for each $x \in \mathbb{R}^d$, the minimizer $\pi_C(x)$ of $z \mapsto \|z - x\|_2$ over $z \in C$ exists, and is unique.

	We call $\pi_C(x)$ the \emph{projection}\index{projection} of $x$ on $C$. Also,
	\begin{itemize}
		\item $(x - \pi_C(x))^T (z - \pi_C(x)) \leq 0$, for all $z \in C$, $x \in \mathbb{R}^d$.
		\item $\|\pi_C(x) - \pi_C(z)\|_2 \leq \|x - z\|_2$, for all $x, z \in \mathbb{R}^d$.
	\end{itemize}
\end{proposition}

\begin{proofbox}
	For existence, let $\mu = \inf_{z \in C}\|x - z\|_2$, and $B = \{z \mid \|x - z\|_2 \leq \mu + 1\}$. Then we have
	\[
	\inf_{z \in C} \|x - z\|_2 = \inf_{z \in C \cap B} \|x - z\|_2.
	\]
	But the right hand side is a minimum of a continuous function $z \mapsto \|z - x\|_2$ over a compact set, so the infimum is achieved.

	For uniqueness, notice that $z \mapsto \|z - x\|_2^2$ is strictly convex, as the Hessian is positive definite. Hence its minimum on a convex set is unique, if it exists. Now we prove the two properties.

	\begin{itemize}
		\item For arbitrary $x$, let $\pi_C(x) = \pi$. If $z \in C$, then $(1 - t)\pi + t z \in C$ for all $t \in [0, 1]$ by convexity, so
			\begin{align*}
				\|x - \pi\|_2^2 & \leq \|x - (1 - t)\pi - tz\|_2^2 = \|x - \pi + t(\pi - z)\|_2^2 \\
						&= \|x - \pi\|_2^2 + 2t(x - \pi)^T(\pi - z) + t^2 \|\pi - z\|_2^2,
			\end{align*}
			hence we get
			\[
				(x - \pi)^T(z - \pi) \leq \frac{1}{2} t \|\pi - z\|_2^2 \leq 0,
			\]
			as this holds for all $t$.
		\item Let $z \in \mathbb{R}^d$ be arbitrary. Then from the previous,
			\[
				(x - \pi_C(x))^T(\pi_C - \pi_C(x)) \leq 0,
			\]
			and moreover by interchanging $x$ and $z$,
			\[
				(\pi_C(z) - z)^T(\pi_C(z) - \pi_C(x)) \leq 0.
			\]
			Adding these, we get
			\[
			\|\pi_C(z) - \pi_C(x)\|_2^2 \leq |(x - z)^T(\pi_C(z) - \pi_C(x)| \leq \|x-z\|_2 \|\pi_C(z) - \pi_C(x)\|_2.
			\]
			Then we get our inequality by dividing, or if not the projections are equal and we get our result anyway.
	\end{itemize}
\end{proofbox}

\subsection{Subgradients}
\label{sub:subg}

Recall that for a convex function $f : \mathbb{R}^d \to \mathbb{R}$ differentiable at $x$, we have
\[
f(z) \geq f(x) + \nabla f(x)^T (z - x).
\]

\begin{definition}
	A vector $g \in \mathbb{R}^d$ is a \emph{subgradient}\index{subgradient} of a convex function $f : \mathbb{R}^d \to \mathbb{R}$ at $x \in \mathbb{R}^d$ if
	\[
	f(z) \geq f(x) + g^T(z - x),
	\]
	for all $z \in \mathbb{R}^d$. The set of subgradients of $f$ at $x$ is called the \emph{subdifferential}\index{subdifferential} of $f$ at $x$, and is denoted by $\partial f(x)$.
\end{definition}

\begin{exbox}
	If $\phi(u) = \max(1 - u, 0)$, then
	\[
		\partial \phi(1) = [-1, 0].
	\]
\end{exbox}

These satisfy the following.

\begin{proposition}
	If $f : \mathbb{R}^d \to \mathbb{R}$ is convex, then $\partial f(x)$ is non-empty, for all $x \in \mathbb{R}^d$.
\end{proposition}

\begin{proposition}
	If $f : \mathbb{R}^d \to \mathbb{R}$ is convex and differentiable at $x$, then $\partial f(x) = \{ \nabla f(x)\}$.
\end{proposition}

\begin{proposition}
	Let $f, f_1, f_2 : \mathbb{R}^d \to \mathbb{R}$ be convex.
	\begin{enumerate}[\normalfont(i)]
		\item For $\alpha > 0$, $\partial(\alpha f)(x) = \{\alpha g \mid g \in \partial f(x)\}$.
		\item $\partial (f_1 + f_2)(x) = \{g_1 + g_2 \mid g_1 \in \partial f_1(x), g_2 \in \partial f_2(x)\}$.
		\item If $h : \mathbb{R}^m \to \mathbb{R}$ is $x \mapsto f(Ax + b)$, where $A \in \mathbb{R}^{d \times m}$, $b \in \mathbb{R}^d$, then
			\[
				\partial h(x) = \{ A^T g \mid g \in \partial f(Ax + b)\}.
			\]
	\end{enumerate}
\end{proposition}

\begin{exbox}
	Consider
	\[
	f(\beta) = \frac{1}{n} \sum_{i = 1}^n \phi(y_i x_i^T \beta),
	\]
	where $\phi(u) = \max(1 - u, 0)$. Let $h_i(\beta) = \phi(y_i x_i^T \beta)$. Then
	\[
		\partial h_i (\beta) = \{y_i x_i t \mid t \in [-1, 0]\}
	\]
	when $y_i x_i^T \beta = 1$. By these previous properties, the subgradients of $f$ and $\beta$ take the form
	\[
	\frac{1}{n} \sum_{i = 1}^n y_i x_i t_i,
	\]
	where $t_i \in [-1, 0]$, and
	\[
	t_i =
	\begin{cases}
		-1 & y_i x_i^T \beta < 1, \\
		0 & y_i x_i^T \beta > 1.
	\end{cases}
	\]
\end{exbox}

\subsection{Gradient Descent}
\label{sub:grad_d}

Our goal is to minimize $f(\beta)$ subject to $\beta \in C$, where $C$ is closed and convex, and $f$ is convex. Our input is a guess $\beta_1 \in C$, and we have fixed $k \in \mathbb{N}$, the number of step, and step sizes $(\eta_s)_{s = 1}^{k-1}$.

Then for $s = 1$ to $k-1$, we find $g_s \in \partial f(\beta_s)$, and pick
\[
z_{s + 1} = \beta_s - \eta_s g_s, \qquad \beta_{s + 1} = \pi_C(z_{s + 1}).
\]
We could return the last element, but instead we will return the average
\[
\bar \beta = \frac{1}{k} \sum_{s = 1}^k \beta_s.
\]

%lecture 14

\begin{theorem}
	Suppose $\hat \beta$ minimises the convex function $f : C \to \mathbb{R}$, where $C$ is a closed, convex set. Suppose:
	\begin{itemize}
		\item $\sup_{\beta \in C} \|\beta\|_2 \leq R < \infty$.
		\item $\sup_{\beta \in C} \sup_{g \in \partial f(\beta)} \|g\|_2 \leq L < \infty$.
	\end{itemize}
	If $\eta_s = \eta = 2R/(L\sqrt{k})$, then
	 \[
	f(\hat \beta) - f (\bar \beta) \leq \frac{2LR}{\sqrt k}.
	\]
\end{theorem}

\begin{proofbox}
	We have that
	\[
	f(\hat \beta) \geq f(\beta_s) + g_s^T(\hat \beta - \beta_s).
	\]
	Thus, rearranging,
	\begin{align*}
		f(\hat \beta_s) - f(\hat \beta) &\leq g_s^T(\beta_s - \hat \beta) \\
						&= \frac{1}{\eta}(\beta_s - z_{s+1})^T(\beta_s - \hat \beta).
	\end{align*}
	We now use the relation
	\[
	u^Tv = \frac{1}{2} (\|u\|_2^2 + \|v\|_2^2 - \|u-v\|_2^2).
	\]
	Passing this along, we get
	\begin{align*}
		\frac{1}{2\eta}(\|\beta_s - z_{s+1}\|_2^2 + \|\beta_s - \hat \beta\|_2^2 - \|z_{s+1} - \hat \beta\|_2^2 ).
	\end{align*}
	Now note that
	\[
	\|z_{s+1} - \hat \beta\|_2^2 \geq \|\pi_C(z_{s+1}) - \pi(\hat \beta)\|_2^2 = \|\beta_{s+1} - \hat \beta\|_2^2,
	\]
	so we get
	\[
	f(\beta_s) - f(\hat \beta) \leq \frac{1}{2 \eta}(\eta^2 \|g\|_2^2 + \|\beta_s - \hat \beta\|_2^2 - \|\beta_{s+1} - \hat \beta\|_2^2),
	\]
	hence taking the sum, by using Jensen's inequality,
	\begin{align*}
		f(\bar \beta) - f(\hat \beta) &\leq \frac{1}{k} \sum_{s = 1}^k f(\beta_s) - f(\hat \beta) \\
					      &\leq \frac{\eta L^2}{2} + \frac{\|\beta_s - \hat \beta\|_2^2}{2 \eta k} \leq \frac{\eta L^2}{2} + \frac{2R^2}{2\eta k} = \frac{2RL}{\sqrt k}.
	\end{align*}
\end{proofbox}

\begin{exbox}
	Doing empirical risk minimization with hinge loss $\phi(u) = \max(1 - u, 0)$, where
	\begin{align*}
		\mathcal{H} &= \{ x \mapsto x^T \beta \mid \|\beta\|_2 \leq \lambda\}, \\
				\chi &= \{x \in \mathbb{R}^p \mid \|x\|_2 \leq C\},
	\end{align*}
	then given data $(x_i, y_i)$, recall that the subgradient of the objective function of the ERM problem
	\[
	f(\beta) = \frac{1}{n} \sum_{i = 1}^n \phi(y_i x_i^T \beta)
	\]
	takes the form
	\[
	g = \frac{1}{n} \sum_{i = 1}^n y_i x_i t_i,
	\]
	where $t_i \in [-1, 0]$. By the triangle inequality, $\|g\|_2 \leq C$. Gradient descent with $\eta_s = 2\lambda/(C\sqrt k)$ gives
	\[
	f(\bar \beta) - f(\hat \beta) \leq \frac{2 C \lambda}{\sqrt k}.
	\]
\end{exbox}

\subsection{Stochastic Gradient Descent}
\label{sub:sgd}

An issue with gradient descent is that computation of the gradient can involve a sweep over the entire dataset. \emph{Stochastic gradient descent}\index{stochastic gradient descent} (SGD) instead works with unbiased estimates of subgradients as are available when minimizing convex functions of the form
\[
f(\beta) = \mathbb{E} \tilde f(\beta; u),
\]
where $\tilde f : \mathbb{R}^p \times U \to \mathbb{R}$ is such that $\beta \to \tilde f(\beta; u)$ is convex for all $u \in U$, and $u$ is a random variable taking values in $U$.

\begin{exbox}[Important]
	Taking $u \sim \mathrm{Unif}\{1, 2, \ldots, n\}$, then the ERM objective with $\mathcal{H} = \{h_\beta \mid \beta \in C\}$ may be written as
	\[
	\frac{1}{n} \sum_{i = 1}^n \ell(h_\beta(x_i), y_i) = \mathbb{E} \ell(h_\beta(x_u), y_u),
	\]
	when $\beta \mapsto \ell(h_\beta(x), y)$ is convex for all $x \in \mathcal{X}$, $y \in \mathcal{Y}$.
\end{exbox}

The stochastic gradient descent algorithm is as follows: we are given a guess $\beta_1 \in C$, and $k \in \mathbb{N}$, the number of steps, and step sizes $(\eta_s)_{s = 1}^{k-1}$, as well as iid copies of $u_1, u_2, \ldots, u_{k-1}$ of $u$. For $s = 1$ to $k-1$, we pick $\tilde g_s \in \partial \tilde f(\beta_s; u_s)$ and let
\begin{align*}
	z_{s+1} = \beta_s - \eta \tilde g_s, \qquad \beta_{s+1} = \pi_C(z_{s+1}).
\end{align*}
Then we again return the average
\[
\bar \beta = \frac{1}{k}\sum_{s = 1}^k \beta_s.
\]
The key point is that in this ERM example, computing $\tilde g_s$ involves only a single data point $(x_{u_s}, y_{u_s})$.

\begin{theorem}
	Suppose $\hat \beta$ minimizes $f : \mathbb{R}^p \to \mathbb{R}$ over a closed convex set $C \subseteq \mathbb{R}^p$. Suppose
	\begin{itemize}
		\item $\sup_{\beta \in C} \|\beta\|_2 \leq R < \infty$, and
		\item $\sup_{\beta \in C} \mathbb{E} \sup_{g \in \partial f(\beta, u)} \|g\|_2^2 \leq L^2$.
	\end{itemize}
	Then taking $\eta_s = \eta = 2R/(L\sqrt k)$, then
	\[
	\mathbb{E} f(\bar \beta) - f(\hat \beta) \leq \frac{2LR}{\sqrt k}.
	\]
\end{theorem}

%lecture 15

\begin{proofbox}
	Let $g_s = \mathbb{E}[\tilde g_s \mid \beta_s]$. Then $g_s \in \partial f(\beta_s)$. Indeed,
	\begin{align*}
		\tilde f(\beta; u_s) \geq \tilde f(\beta_s; u_s) + \tilde g_s^T (\beta - \beta_s),
	\end{align*}
	so taking the expectation and conditioning on $\beta_s$, and that $u_s \indep \beta_s$,
	\begin{align*}
		\mathbb{E} \tilde f(\beta; u_s) = f(\beta) \geq f(\beta_s) + g_s^T (\beta - \beta_s),
	\end{align*}
	for all $\beta$. Thus
	\begin{align*}
		f(\beta_s) - f(\hat \beta) &\leq g_s^T(\beta_s - \hat \beta) \\
					   &= \mathbb{E}[ \tilde g_s^T(\beta_s - \hat \beta) \mid \beta_s] = \cdots \\
					   &\leq \frac{1}{2 \eta} \mathbb{E}[\eta^2 \|\tilde g_s\|_2^2 + \|\beta_s - \hat \beta\|_2^2 - \|\beta_{s+1} - \hat \beta\|_2^2 \mid \beta_s],
	\end{align*}
	arguing the same way as in the previous proof. Taking expectations and averaging,
	\begin{align*}
		\mathbb{E} f(\bar \beta) - f(\hat \beta) &\leq \mathbb{E}\left[ \frac{1}{k}\sum_{s = 1}^k f(\beta_s)\right] - f(\hat \beta) \\
					 &\leq \frac{\eta L^2}{2} + \frac{\|\beta_1 - \hat \beta\|_2^2}{2 \eta k} = \frac{2RL}{\sqrt k},
	\end{align*}
	as $\|\beta_1 - \hat \beta\|_2^2 \leq 4R^2$.
\end{proofbox}

\newpage

\section{Popular Machine Learning Methods II}
\label{sec:pop}

\subsection{Adaboost}
\label{sub:adb}

Given a set $\mathcal{B}$ of base classifiers $h : \mathcal{X} \to \{-1, 1\}$ such that $h \in \mathcal{B} \implies -h \in \mathcal{B}$, consider
\[
	\mathcal{H} = \left\{ \sum_{m = 1}^M \beta_m h_m \mid \beta_m \geq 0, h_m \in \mathcal{B} \text{ for } m = 1, \ldots, M\right\}.
\]

Empirical risk minimization over this class is not easy. \emph{Adaboost}\index{Adaboost} can be motivated as a greedy ERM over $\mathcal{H}$ using the exponential loss $\phi(u) = e^{-u}$, and it works as follows:

Set $\hat f_0$ to be $x \mapsto 0$. Then for $m = 1, \ldots, M$, we pick
\[
	(\beta_m, \hat h_m) = \mathrm{argmin}_{\beta \geq 0, h \in \mathcal{B}} \frac{1}{n} \sum_{i = 1}^n \exp (- y_i(\hat f_{m-1}(x_i) + \beta h(x_i))),
\]
and then let $\hat f_m = \hat f_{m-1} + \hat \beta_m \hat h_m$. We then output the classifier $\sgn \circ \hat f_m$. The key property of Adaboost is that this minimization is easy to compute. Define weights
\[
w_i^{(m)} = \frac{1}{n} \exp (-y_i \hat f_{m-1} (x_i)),
\]
then note
\begin{align*}
	\frac{1}{n} \sum_{i = 1}^n \exp(-y_i(\hat f_{m-1}(x_i) &+ \beta h(x_i))) = \sum_{i = 1}^n w_i^{(m)} \exp(- \beta y_i h(x_i)) \\
	&= e^{\beta} \sum_{i = 1}^n w_i^{(m)} \mathbbm{1}_{\{y_i \neq h(x_i)\}} + e^{-\beta} \sum_{i = 1}^n w_i^{(m)} (1 - \mathbbm{1}_{\{y_i \neq h(x_i)\}}) \\
	&= (e^\beta - e^{-\beta}) \sum_{i = 1}^n w_i^{(m)} \mathbbm{1}_{\{y_i \neq h(x_i)\}} + e^{-\beta} \sum_{i = 1}^n w_i^{(m)}.
\end{align*}
We are sort of minimizing a weighted misclassification loss. Let
\[
	\mathrm{err}_m(h) = \sum_{i = 1}^n w_i^{(m)} \mathbbm{1}_{\{y_i \neq h(x_i)\}} / \sum_{i = 1}^n w_i^{(m)}.
\]
Then $\hat h_m = \mathrm{argmin}_{h \in \mathcal{B}} \mathrm{err}_m(h)$, so $\hat h_m$ is a weighted ERM over the class $\mathcal{B}$. If $\mathrm{err}_m(\hat h_m) > 0$, then
\[
	(e^{\hat \beta_m} + e^{-\hat \beta_m}) \mathrm{err}_m(\hat h_m) = e^{- \hat \beta_m}.
\]
Let $x = e^{\hat \beta_m}$, and $a = \mathrm{err}_m(\hat h_m)$. We have
\[
	(x^2 + 1)a = 1 \implies x = \sqrt{a^{-1} - 1},
\]
which we can solve to get
\[
\hat \beta_m = \frac{1}{2} \log \left( \frac{1 - \mathrm{err}_m(\hat h_m)}{\mathrm{err}_m(\hat h_m)} \right).
\]
To use Adaboost effectively, we need the weighted ERM step to be computationally fast.

\begin{exbox}
	Let $\mathcal{X} \in \mathbb{R}^p$, and
	\[
		\beta = \{x \mapsto \sgn(x_j- a), x \mapsto - \sgn(x_j - a) \mid a \in \mathbb{R}, j = 1, \ldots, p\},
	\]
	the class of \emph{decision stumps}\index{decision stump}.
\end{exbox}

%lecture 16

\subsection{Gradient Boosting}
\label{sub:gb}

Imagine applying gradient descent directly to minimise $R(h) = \mathbb{E} \ell(h(X), Y)$ over functions $h$.

We start with an initial guess $f_0 : \mathcal{X} \to \mathbb{R}$. Then for $m = 1, \ldots, M$, we compute
\begin{align*}
	g_m(x) &= \frac{\partial}{\partial \alpha} \mathbb{E}[ \ell(\alpha, Y) \mid X = x] \biggr|_{f_{m-1}(x)} \\
	       &= \mathbb{E}\left[ \frac{\partial}{\partial \alpha} \ell(\alpha, Y)\biggr|_{f_{m-1}(x)} \mid X = x \right],
\end{align*}
and then let $f_m = f_{m-1} - \eta g_m$ (for some $\eta > 0$). The problem is that we don't have access to this conditional expectation. Recall that
\[
x \mapsto \mathbb{E}\left[ \frac{\partial}{\partial \alpha} \ell(\alpha, Y) \biggr|_{f_{m-1}(x)} \mid X = x\right]
\]
minimizes
\[
\mathbb{E}\left[ \left( \frac{\partial}{\partial \alpha} \ell(\alpha, Y) \biggr|_{f_{m-1}(x)} - g(X) \right)^2 \right],
\]
over all functions $g : \mathcal{X} \to \mathbb{R}$. \emph{Gradient boosting}\index{gradient boosting} estimates this conditional expectation function using some bass regression procedure $\hat h$ that takes in training data $\mathcal{D}$ and outputs a hypothesis $\hat h_{\mathcal{D}} : \mathcal{X} \to \mathbb{R}$, which works as follows:

Our input data is $Y_{1:n}$, $X_{1:n}$, and we set $\eta > 0$. The base regression method is $\hat h$, and the number of iterations is $M$. We compute
\[
\hat \mu = \mathrm{argmin}_{\mu \in \mathbb{R}} \frac{1}{n} \sum_{i = 1}^n \ell(\mu, Y_i),
\]
and set $\hat f_0(x) = \hat \mu$. For $m = 1, \ldots, M$, we find
\[
W_i = \frac{\partial \ell(\alpha, Y_i)}{\partial \alpha} \biggr|_{\hat f_{m-1}(X_i)},
\]
and then we set $\hat g_m = \hat h(X_{1:n}, W_{1:n})$, i.e. we regress $W_{1:n}$ onto $X_{1:n}$, and then set
\[
\hat f_m = \hat f_{m-1} - \eta \hat g_m.
\]
We then return $\hat f_M$, or $\sgn \circ \hat f_M$.

\begin{exbox}
	For least squares, $\ell(\alpha, Y) = (\alpha - Y)^2$, and so
	\[
	\frac{\partial \ell(\alpha, Y_i)}{\partial \alpha} \biggr|_{f_{m-1}(X_i)} = -2(Y_i - f_{m-1}(X_i)).
	\]
	For least-squares, we can consider $\hat h$ as the ERM of $\{x \mapsto \mu + x_j \beta \mid \mu \in \mathbb{R}, \beta \in \mathbb{R}, j = 1, \ldots, p\}$.
\end{exbox}

\subsection{Feedforward Neural Networks}
\label{sub:nn}

\emph{Feedforward neural networks}\index{neural networks} are based on a class of hypotheses $h : \mathbb{R}^p \to \mathbb{R}$ of the form
\[
h(x) = A^{(d)} \circ g \circ A^{(d-1)} \circ g \circ \cdots \circ g \circ A^{(2)} \circ g \circ A^{(1)} (x),
\]
where we have
\begin{itemize}
	\item $d$, the depth.
	\item $A^{(k)}(v) = \beta^{(k)} v + \mu^{(k)}$, where $v \in \mathbb{R}^{m_k}$, $\beta^{(k)} \in \mathbb{R}^{m_{k+1} \times m_k}$, $\mu^{(k)} \in \mathbb{R}^{m_{k+1}}$.
	\item $g : \mathbb{R}^m \to \mathbb{R}^m$ (for arbitrary $m$) is
		\[
		g(v) = 
		\begin{pmatrix}
			\psi(v_1) \\ \cdots \\ \psi(v_m)
		\end{pmatrix},
		\]
		for a non-linear \emph{activation function}\index{activation function} $\psi : \mathbb{R} \to \mathbb{R}$. Usually, $\psi$ is the rectified linear unit (ReLU): $u \mapsto \max(0, u)$, but historically people used the sigmoid $u \mapsto 1/(1 + e^{-u})$.
\end{itemize}

The parameters $(\beta^{(k)}, \mu^{(k)})$ are learned from the data $(x_1, y_1), \ldots, (x_n, y_n)$ by ``attempting'' to minimise the empirical $\phi$ risk using stochastic gradient descent. Suppose that $\phi$ and $\psi$ are differentiable.

To compute the gradient of $z = \phi(y h(x))$ with respect to the parameters, we use the chain rule as follows. Let $h^{(0)} = x$, the input layer. Define inductively for $k = 1, \ldots, d-1$
\[
x^{(k)} = A^{(k)}( h^{(k-1)}), \qquad h^{(k)} = g(x^{(k)}).
\]
Hence $h^{(k)}$ is the $k$'th hidden layer. Then $x^{(d)}$ is the output layer. Now,
\[
\frac{\partial z}{\partial x^{(d)}} = y \phi' (y x(d)),
\]
\[
\frac{\partial z}{\partial \mu^{(d)}} = \frac{\partial z}{\partial x^{(d)}}, \qquad \frac{\partial z}{\partial \beta^{(d)}_{1k}} = \frac{\partial z}{\partial x^{(d)}} h^{(d-1)}_k,
\]
\[
\frac{\partial z}{\partial h^{(d-1)}_j} = \frac{\partial z}{\partial x^{(d)}} \beta_{1j}^{(d)}, \qquad \frac{\partial z}{\partial x_{j}^{(d-1)}} = \frac{\partial z}{\partial h_j^{(d-1)}} \psi'(x_j^{(d-1)}),
\]
\[
	\frac{\partial z}{\partial \mu^{(d-1)}_j} = \frac{\partial z}{\partial x_j ^{(d-1)}}, \qquad \frac{\partial z}{\partial \beta^{(d-1)}_{jk}} = \frac{\partial z}{\partial x_j^{(d-1)}} h_k^{(d-2)}.
\]



\printindex

\end{document}
