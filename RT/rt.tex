\documentclass[12pt]{article}
\usepackage{amsmath}
\usepackage{mathtools}
\usepackage[a4paper]{geometry}
\usepackage{fancyhdr}
\usepackage{tikz}
\usepackage{amssymb}
\usepackage{graphicx}
\usepackage{amsthm}
\usepackage{import}
\usepackage{xifthen}
\usepackage{pdfpages}
\usepackage{transparent}
\usepackage{adjustbox}
\usepackage[shortlabels]{enumitem}
\usepackage{bbm}
\usepackage{parskip}
\makeatletter
\newcommand{\@minipagerestore}{\setlength{\parskip}{\medskipamount}}
\makeatother
\usepackage{imakeidx}

\DeclareMathOperator{\Ker}{Ker}
\DeclareMathOperator{\Img}{Im}
\DeclareMathOperator{\rank}{rank}
\DeclareMathOperator{\nullity}{null}
\DeclareMathOperator{\spn}{span}
\DeclareMathOperator{\tr}{tr}
\DeclareMathOperator{\adj}{adj}
\DeclareMathOperator{\id}{id}
\DeclareMathOperator{\Sym}{Sym}
\DeclareMathOperator{\Orb}{Orb}
\DeclareMathOperator{\Stab}{Stab}
\DeclareMathOperator{\ccl}{ccl}
\DeclareMathOperator{\Aut}{Aut}
\DeclareMathOperator{\Syl}{Syl}
\DeclareMathOperator{\sgn}{sgn}
\DeclareMathOperator{\Fit}{Fit}
\DeclareMathOperator{\Ann}{Ann}
\DeclareMathOperator{\epi}{epi}
\DeclareMathOperator{\Hom}{Hom}
\DeclareMathOperator{\chr}{char}
\DeclareMathOperator{\Irr}{Irr}
\DeclareMathOperator{\Ind}{Ind}
\DeclareMathOperator{\Res}{Res}


\newcommand{\incfig}[1]{%
	\def\svgwidth{\columnwidth}
	\import{./figures/}{#1.pdf_tex}
}

\setlength\parindent{0pt}

\newtheorem{theorem}{Theorem}[section]
\newtheorem{corollary}{Corollary}[section]
\newtheorem{lemma}{Lemma}[section]
\newtheorem{proposition}{Proposition}[section]

\theoremstyle{definition}
\newtheorem{definition}{Definition}[section]
\newtheorem{example}{Example}[section]

\theoremstyle{remark}
\newtheorem*{remark}{Remark}

\pagestyle{fancy}
\fancyhf{}
\rhead{\leftmark}
\lhead{Page \thepage}
\setlength{\headheight}{15pt}

\makeindex[intoc]

\usepackage{hyperref}
\hypersetup{
    colorlinks,
    citecolor=black,
    filecolor=black,
    linkcolor=black,
    urlcolor=black
}

\newcommand{\mapsfrom}{\mathrel{\reflectbox{\ensuremath{\mapsto}}}}

\begin{document}

\hypersetup{pageanchor=false}
\begin{titlepage}
	\begin{center}
		\vspace*{1em}
		\Huge
		\textbf{II Representation Theory}

		\vspace{1em}
		\large
		Ishan Nath, Michaelmas 2022

		\vspace{1.5em}

		\Large

		Based on Lectures by Dr. Simon Wadsley

		\vspace{1em}

		\large
		\today
	\end{center}
	
\end{titlepage}
\hypersetup{pageanchor=true}

\tableofcontents

\newpage

\section{Complete reducibility and Maschke's Theorem}%
\label{sec:complete_reducibility_and_maschke_s_theorem}

Informally, last time we say that if $W$ is a subrepresentation of $V$ we can choose bases so the matrices all look like
\[
\begin{pmatrix}
	\rho_w(g) & \ast \\
	0 & \rho_{V/W}(g)
\end{pmatrix}
.\]
This decomposes $V$ into two pieces $W$ and $V/W$, but we lose data in $\ast$.

The question is, given a representation $V$, and $W$ a $G$-invariant subspace, when can we find a vector space complement to $W$ that is also $G$-invariant?

\begin{example}
	\begin{enumerate}[(1)]
		\item[]
		\item If $g = C_2 = \{\pm 1\}$ and $V = \mathbb{R}^2$, and we let
			\[
				\rho(-1) = 
				\begin{pmatrix}
					-1 & 0 \\
					0 & 1
				\end{pmatrix}
			,\]
			then let $W$ be the invariant subspace
			\[
			W = \left\langle
				\begin{pmatrix}
					1 \\
					0
				\end{pmatrix}
			\right \rangle
			,\]
			there are many vector space complements, but only one of these, namely
			\[
			\left \langle
				\begin{pmatrix}
					0 \\
					1
				\end{pmatrix}
			\right \rangle
			,\]
			is $G$-invariant.
		\item If $G = (\mathbb{Z},+)$ and $V = \mathbb{R}^2$, let $\rho$ be determined by
			\[
				\rho(1) =
				\begin{pmatrix}
					1 & 1 \\
					0 & 1
				\end{pmatrix}
			.\]
			Again
			\[
			W = \left \langle
				\begin{pmatrix}
					1 \\
					0
				\end{pmatrix}
			\right \rangle
			\]
			is $G$-invariant, but there is no $G$-invariant complement.
	\end{enumerate}
\end{example}

\begin{definition}
	We say a representation $V$ is a \textbf{direct sum of} $(U_i)_{i = 1}^{d}$\index{internal direct sum} if each $U_i$ is a subrepresentation of $V$ and
	\[
	V = \bigoplus_{i = 1}^{d} U_i
	\]
	as vector spaces. This is the internal direct sum.

	Given a family of representations $(\rho_i, V_i)_{i = 1}^{d}$, we can define a representation (the external direct sum)\index{external direct sum} with underlying vector space
	\[
		V = \bigoplus_{i = 1}^{d} V_i = \left\{(v_i)_{i = 1}^{d} \mid v_i \in V_i \right\}
	\]
	and
	\[
		\rho(g) ((v_i)) = (\rho_i(g) v_i)
	\]
	for all $g \in G$, $(v_i) \in V$. We write
	\[
		(\rho, v) = \bigoplus_{i = 1}^{d}(\rho_i, V_i) = \bigoplus_{i = 1}^{d} \rho_i = \bigoplus_{i = 1}^{d} V_i
	.\]
\end{definition}

\begin{example}
	\begin{enumerate}[(1)]
		\item[]
		\item Suppose $G$ acts on a set $X$ and $X$ can be written as a disjoint union $X = X_1 \cup X_2$ with each $X_i$ $G$-invariant. Then
	\[
		kX = \{f \in kX \mid f|_{X_1} = 0\} \oplus \{f \in kX \mid f|_{X_2} = 0\}
	.\]
	Or externally, $kX \cong kX_1 \oplus kX_2$.

	More generally, if the $G$-action decomposes into orbits
	\[
	X = \bigcup_{i - 1}^{d} O_i
	,\]
	then
	\[
		kX = \bigoplus_{i = 1}^{d} \mathbbm{1}_{O_i}(kX) \cong \bigoplus_{i = 1}^{d} kO_i
	,\]
	where
	\begin{align*}
		\mathbbm{1}_{O_i} : kX &\to kX \\
		x &\mapsto
		\begin{cases}
			f(x) & x \in O_i, \\
			0 & x \not \in O_i.
		\end{cases}
	\end{align*}
\item If $G$ acts transitively on a finite set $X$, then
	\[
		U = \left\{ f \in kX \mid \sum_{x \in X}f(x) = 0 \right\}
	\]
	is a subrepresentation of $kX$, and so is
	\[
		W = \{f \in kX \mid f \text{ is constant on } X \}
	.\]
	Indeed, if $f \in U$ and $g \in G$,
	\[
		\sum_{x \in X} (g \cdot f)(x) = \sum_{x \in X}f(g^{-1} \cdot x) = \sum_{x' \in X}f(x) = 0
	,\]
	so $gf \in U$. Similarly, if $f \in W$, $gf$ is constant, $gf \in W$. Note if the characteristic $k = 0$, then $U \cap W = 0$.
\item We saw before that if $G = C_2 = \{\pm 1\}$, then every representation is a direct sum of one-dimensional representations, as we can diagonalize $\rho(-1)$. We can generalize: suppose $G$ is a finite abelian group, and $(\rho, V)$ is a rep of $G / \mathbb{C}$. Each $g \in G$ has finite order so the minimal polynomial of $\rho(g)$ divides $X^{|G|} - 1$, and this has distinct roots. Thus there is a basis for $V$ for which $\rho(g)$ is diagonal.

	Because $\rho(g)$ and $\rho(h)$ commute for all $g, h \in G$, it follows that we can simultaneously diagonalize all elements, so $\rho$ is a direct sum of one-dimensional representations.

\end{enumerate}
\end{example}

\begin{proposition}
	Suppose $\rho : G \to GL(V)$ is a representation and $V = U \oplus W$ as vector spaces. The following are equivalent:
	\begin{enumerate}[(i)]
		\item $U$ and $W$ are subrepresentations;
		\item There is a basis $v_1, \ldots, v_d$ of $V$, with $v_1, \ldots, v_i$ a basis of $U$ and $v_{i+1}, \ldots, v_{d}$ a basis of $W$, such that all matrices $\rho(g)$ are block diagonal;
		\item For every basis $v_1, \ldots, v_d$ of $V$ such that $v_1, \ldots, v_i$ is a basis for $U$ and $v_{i+1}, \ldots, v_{d}$ is a basis for $W$, all matrices are block diagonal.
	\end{enumerate}
	
\end{proposition}

Notice that for example, take $\rho : C_2 \to GL_2(\mathbb{R})$ by
\[
-1 \mapsto
\begin{pmatrix}
	-1 & -2 \\
	0 & 1
\end{pmatrix}
.\]
This gives a representation, but $\mathbb{R}^2$ breaks up as a direct sum of representations $\langle e_1 \rangle$ and $\langle e_1 - e_2 \rangle$, even though $\rho(-1)$ is upper triangular in the standard bases.

We've seen that there are representations of $(\mathbb{Z}, +)$ such that not every subrepresentation has a $G$-invariant complement, but we can prove the following remarkable theorem:

\begin{theorem}[Maschke's Theorem]\index{Maschke's theorem}
	Let $G$ be a finite group and $(\rho, V)$ a representation of $G$ over $k$, a field of characteristic $0$. Suppose $W \leq V$ is a subrepresentation of $V$, then $W$ has a $G$-invariant complement in $V$.
\end{theorem}

\begin{corollary}[Complete reducibility]\index{complete reduciblilty}
	If $G$ is a finite group and $V$ is a $k$-vector space where the characteristic $k = 0$, then every representation $(\rho, V)$ of $G$ is isomorphic to a direct sum of irreducible representations.
\end{corollary}

\begin{adjustbox}{minipage = \columnwidth - 25.5pt, margin=1em, frame=1pt, margin=0em}
	\textbf{Proof:} We proceed by induction on $\dim V$. If $\dim V = 0$ or $V$ is irreducible, then the result is clear ($r = 0$ or $1$ representation). Otherwise $V$ has a proper subrepresentation $W$. By Maschke's theorem, there is a $G$-invariant complement $U$. Since $\dim U, \dim W < \dim V$, by the induction hypothesis,
	\[
	U \cong \bigoplus_{i = 1}^{s}U_i, \quad W \cong \bigoplus _{i = 1}^{t}W_j
	,\]
	where $U_i, W_i$ are irreducible. Hence since $V = U \oplus W$, we get
	\[
	V \cong \bigoplus _{i = 1}^{s} U_i \oplus \bigoplus_{i = 1}^{t} W_i
	.\]
\end{adjustbox}

To prove Maschke's theorem, we look at the following example:
\begin{example}
	Let $G$ act on a finite set $X$ and consider the real permutation representation $\mathbb{R}X$. We prove Maschke's theorem for $\mathbb{R}X$.

	With respect to the bases $\delta_{x}$, all the matrices are orthogonal, i.e. they preserve the standard inner product
	\[
		(f_1, f_2) = \sum_{x \in X}f_1(x) f_2(x)
	.\]
	Since for $g \in G$,
	\[
		(gf_1, gf_2) = \sum_{x \in X}(g \cdot f_1)(x) (g \cdot f_2) (x) = \sum_{x \in X} f_1(g^{-1} \cdot x) f_2 (g^{-1} \cdot x) = (f_1, f_2)
	,\]
	now if $W \subset \mathbb{R}X$ is a subrepresentation then we can define $W^{\perp} = \{f \in \mathbb{R}X \mid (f, w) = 0\; \forall w \in W \}$. Then $W^{\perp}$ is a vector space complement by standard linear algebra, and for $g \in G$, $f \in W^{\perp}$,
	\[
		(gf, w) = (f, g^{-1}w) = 0
	,\]
	since $g^{-1}w \in W$. This gives $W^{\perp}$ is a $G$-invariant complement.
\end{example}

Recall that if $V$ is a $\mathbb{C}$-vector space, then a \textbf{(Hermitian) inner product}\index{Hermitian inner product} is a positive-definite Hermitian sequilinear form, that is $(-,-) : V \times V \to \mathbb{C}$ such that
\begin{enumerate}[(i)]
	\item $(ax + by, z) = \overline{a}(x, z) + \overline{b}(y, z)$ for all $a, b \in \mathbb{C}$, $x, y, z \in V$;
	\item $(x, ay + bz) = a(x, y) + b(x, z)$ for all $a, b \in \mathbb{C}$, $x, y, z \in V$;
	\item $(x, y) = \overline{(y, x)}$ for all $x, y \in V$;
	\item $(x, x) > 0$ for all $x \in V \setminus \{0\}$.
\end{enumerate}
Note that (i) and (ii) are equivalent if (iii) is true.

The standard inner product on $\mathbb{C}^{n}$ is given by
\[
	\langle x, y \rangle = \sum_{i = 1}^{n} \overline{x_i}y_i
.\]
Recall the \textbf{unitary group}\index{unitary group} $U(n)$ is the subgroup of $GL_n(\mathbb{C})$ given by
\begin{align*}
	U(n) &= \{A \in GL_n(\mathbb{C}) \mid A^{\dagger}A = I \} \\
	     &= \{A \in GL_n(\mathbb{C}) \mid \langle Ax, Ay \rangle = \langle x, y\rangle \}.
\end{align*}

\begin{definition}
	We say $(\rho, V)$, a representation of $G$ is \textbf{unitary}\index{unitary representation} if there is a basis for $V$ such that the corresponding map $\rho : G \to GL_n(\mathbb{C})$ has image contained in $U(n)$.
\end{definition}

\begin{lemma}
	If $\rho : G \to U(n) \leq GL_n(\mathbb{C})$ is a unitary representation and $W \leq \mathbb{C}^{n}$ is a $G$-invariant subspace, then $W^{\perp}$ is a $G$-invariant subspace.
\end{lemma}

\begin{adjustbox}{minipage = \columnwidth - 25.5pt, margin=1em, frame=1pt, margin=0em}
	\textbf{Proof:} First, $W^{\perp}$ is a vector space complement by standard linear algebra. Since for all $g \in G$, $\rho(g) \in U(n)$, if $v \in W^{\perp}$ and $w \in W$, then
	\[
		\langle \rho(g)v, w \rangle = \langle v, \rho(g)^{-1} w \rangle = 0
	,\]
	as $W$ is $G$-invariant. Hence $\rho(g)v \in W^{\perp}$, as required.
\end{adjustbox}

Thus the conclusion of Maschke's theorem is true if $k = \mathbb{C}$ and $(\rho, V)$ is unitary.

\begin{definition}
	We say an inner product on a representation $(\rho, V)$ of $G$ is $G$-invariant if $(gx, gy) = (x, y)$ for all $x, y \in V$ and $g \in G$. Equivalently, if $(gx, gx) = (x, x)$ for all $x \in V$ and $g \in G$.
\end{definition}

\begin{proposition}
	A representation $(\rho, V)$ of $G$ is unitary if and only if there is a $G$-invariant inner product on $V$.
\end{proposition}

\begin{adjustbox}{minipage = \columnwidth - 25.5pt, margin=1em, frame=1pt, margin=0em}
	\textbf{Proof:} If $(\rho, V)$ is unitary, there exists a basis for $V$ such that the induced matrices $\rho(g) \in U(n)$ for all $g \in G$. Now
	\[
		\left( \sum_{i = 1}^{n} \lambda_i \rho_i, \sum_{j = 1}^{n} \mu_j \rho_j \right) = \sum_{i = 1}^{n} \overline{\lambda_i} \mu_i
	\]
	is $G$-invariant. Conversely, if $(-,-)$ is $G$-invariant, we can choose an orthonormal basis for $V$ with respect to $(-,-)$ by Gram-Schmidt, and now the matrices $\rho(g)$ with respect to the bases are all unitary. Since $(-,-)$ is the standard inner product with respect to the chosen basis, $(\rho, V)$ is unitary.
\end{adjustbox}

\begin{proposition}[Weyl's unitary trick]\index{Weyl's unitary trick}
	If $V$ is a complex representation of a finite group then $V$ has a $G$-invariant inner product.
\end{proposition}

\begin{adjustbox}{minipage = \columnwidth - 25.5pt, margin=1em, frame=1pt, margin=0em}
	\textbf{Proof:} Pick any inner product in $V$, $\langle -, - \rangle$, say. We define a new inner product by taking the average:
	\[
		(x, y) = \frac{1}{|G|} \sum_{g \in G} \langle gx, gy \rangle
	.\]
	It is easy to verify that $(-, -)$ is an inner product. Moreover, if $h \in G$, $x, y \in V$, then
	\[
		(hx, hy) = \frac{1}{|G|} \sum_{g \in G} \langle ghx, ghy \rangle = \frac{1}{|G|} \sum_{g \in G} \langle g'x, g'y \rangle = (x, y)
	.\]
	So $(-, -)$ is a $G$-invariant inner product.
\end{adjustbox}

\begin{corollary}
	Every finite subgroup of $GL_n(\mathbb{C})$ is conjugate to a subgroup of $U(n)$.
\end{corollary}

\begin{adjustbox}{minipage = \columnwidth - 25.5pt, margin=1em, frame=1pt, margin=0em}
	\textbf{Proof:} If $G \leq GL_n(\mathbb{C})$ is finite then the map $G \to GL_n(\mathbb{C})$ is a representation and by the unitary trick, $\rho$ has an invariant inner product. Thus, there exists $P \in GL_n(\mathbb{C})$ such that
	\[
		P g P^{-1} \in U(n)
	.\]
\end{adjustbox}

We now generalise our idea to field of characteristic 0. We can frame our argument for when the orthogonal projection $V \to W$ is $G$-linear (when $W$ is $G$-linear) and has $G$-invariant kernel as a $G$-invariant complement to $W$.

\begin{theorem}[Maschke's theorem]\index{Maschke's theorem}
	Let $G$ be a finite group and $V$ a representation of $G$ over a field of characteristic $0$. Then every subrepresentation $W$ of $V$ has a $G$-invariant complement.
\end{theorem}

\begin{adjustbox}{minipage = \columnwidth - 25.5pt, margin=1em, frame=1pt, margin=0em}
	\textbf{Proof:} Let $W$ be an invariant subspace of $V$ and choose a projection $\pi : V \to V$ with image $W$. Recall that $\Hom_k(V, V)$ is a representation of $G$ via $g \cdot \phi = g \circ \phi \circ g^{-1}$, and let 
	\[
		\pi^{G} \equiv \frac{1}{|G|} \sum_{g \in G}(g \cdot \pi)
	.\]
	Then
	\begin{enumerate}[(i)]
		\item If $w \in W$,
			\[
				\pi^{G}(w) = \frac{1}{|G|} \sum_{g \in G} g \circ \phi \circ g^{-1}(w) = \frac{1}{|G|} \sum_{g \in G} g \circ g^{-1}(w) = w
			,\]
			as $g^{-1}(w) \in W$.
		\item If $v \in V$,
			\[
				\pi^{G}(v) = \frac{1}{|G|} g \circ \pi \circ g^{-1}(v) \in W
			.\]
		\item Moreover, for $h \in G$,
			\[
				h \cdot \pi^{G} = \frac{1}{|G|} \sum_{g \in G}(hg) \cdot \pi = \frac{1}{|G|} \sum_{g' \in G} g' \cdot \pi = \pi^{G}
			,\]
			so $\pi^{G}$ is $G$-linear. So $\Ker \pi^{G}$ is a $G$-invariant complement to $\Img \pi^{G} = W$.
	\end{enumerate}
	
\end{adjustbox}

\begin{remark}
	\begin{enumerate}[(1)]
		\item[]
		\item We can explicitly compute $\pi^{G}$ and $\ker \pi^{G}$ given $(\rho, V)$ and $W$.
		\item We only used the characteristic being 0 to make sense of  $1/|G|$, i.e. we only need $|G| \neq 0 \in k$.
		\item For any $G$-representation $V$ where the characteristic $k \nmid G$, the map
			\[
				\pi(v) = \frac{1}{|G|} \sum_{g \in G} g \cdot v
			\]
			is a $G$-linear projection onto $V^{G}$, the elements fixed by $g$. Note that $\dim V^{G} = \tr \pi$.
	\end{enumerate}
	
\end{remark}

\newpage

\section{Schur's Lemma}%
\label{sec:schur_s_lemma}

Recall that if a vector space $V$ has dimension $d$ then $\Aut(V) \cong GL_d(k)$. This group parameterises bases of $V$. Given $e_1, \ldots, e_d$ a basis of $V$,
\begin{align*}
	\Aut(V) &\leftrightarrow \{\text{bases of } V\} \\
	\alpha &\mapsto (\alpha e_1, \ldots, \alpha e_d)
\end{align*}
The decomposition of $V$ as a direct sum $\bigoplus V_i$ of 1-dimensional subspaces can be parametrised as
\begin{align*}
	GL_d(k)/T &\leftrightarrow \{\text{decomposition } V = \bigoplus_{i = 1}^{d} k v_i \} \\
	GL_d(k)/N_G(T) &\leftrightarrow \{\text{decomposition } V = \bigoplus_{i = 1}^{d} k v_i \text{ up to reordering}\},
\end{align*}
where $T$ is the set of diagonal non-singular matrices, and $N_G(T)$ is the set of matrices with one non-zero element in every row and column.

\begin{theorem}[Schur's lemma]\index{Schur's lemma}
	Suppose $V$ and $W$ are irreducible representations of a group $G$ over $k$. Then:
	\begin{enumerate}[\normalfont(i)]
		\item Every element of $\Hom_G(V, W)$ is either $0$ or an isomorphism;
		\item If $k$ is algebraically closed then $\dim_{k} \Hom_G(V, W)$ is either $0$ or $1$.
	\end{enumerate}
\end{theorem}

In particular, $\Hom_G(V, V) = k \id_{V}$. This entails that irreducible representations are rigid as 1-dimensional vector spaces.

\begin{adjustbox}{minipage = \columnwidth - 25.5pt, margin=1em, frame=1pt, margin=0em}
\textbf{Proof:}
\begin{enumerate}[(i)]
	\item Suppose $\phi \in \Hom_G(V, W) \setminus 0$. Then $\ker \phi \leq V$ is a $G$-invariant subspace of $V$. Since $V$ is simple, $\ker \phi = 0$. Similarly, $0 \neq \Img \phi \leq W$ is a $G$-invariant subspace, so $\Img \phi = W$. Thus $\phi$ is both injective and surjective, so it is an isomorphism.
	\item Suppose $\phi_1, \phi_2 \in \Hom_G(V, W)$ are non-zero. Then they are both homomorphisms, so consider $\phi = \phi_1^{-1}\phi_2$. Since $k$ is closed, we may find an eigenvalue $\lambda$ such that $\phi - \lambda \id_V$ has non-zero kernel, hence is $0$. This gives $\phi_1^{-1}\phi_2 = \lambda \id_V$, so $\phi_2 = \lambda \phi_1$.
\end{enumerate}

\end{adjustbox}

\begin{proposition}
	If $V, V_1, V_2$ are $k$-representations of $G$, then there are isomorphisms of $k$-vector spaces
	\begin{enumerate}[\normalfont(i)]
		\item $\Hom_G(V, V_1 \oplus V_2) \cong \Hom_G(V, V_1) \oplus \Hom_G(V, V_2)$,
		\item $\Hom_G(V_1 \oplus V_2, V) \cong \Hom_G(V_1, V) \oplus \Hom_G(V_2, V)$.
	\end{enumerate}
\end{proposition}

\begin{adjustbox}{minipage = \columnwidth - 25.5pt, margin=1em, frame=1pt, margin=0em}
	\textbf{Proof:} Consider the natural inclusion maps $V_i \to V_1 \oplus V_2$ for $i = 1, 2$. This induces linear maps (by post-composition)
	\[
		\Hom_k(V, V_i) \to \Hom_k(V, V_1 \oplus V_2)
	.\]
	Together, these induces isomorphisms induce a linear isomorphism
	\[
		\Hom_k(V, V_1) \oplus \Hom_k(V, V_2) \to \Hom_k(V, V_1 \oplus V_2)
	,\]
	by $(f_1, f_2) \mapsto f_1 + f_2$. Since this is a $G$-linear map, this is an intertwining map. Since in general, $\Hom_G(U, W)$ consists of the $G$-fixed points of $\Hom_k(U, W)$, it follows that there is an induced map
	\[
		\Hom_G(V, V_1) \oplus \Hom_G(V, V_2) \to \Hom_G(V, V_1 \oplus V_2)
	.\]
	Similarly, the natural projection maps $\pi_i : V_1 \oplus V_2 \to V_i$ are $G$-linear and induce a $G$-linear isomorphism
	\begin{align*}
		\Hom_k(V_1 \oplus V_2, V) &\to \Hom_k(V_1, V) \oplus \Hom_k(V_2, V) \\
		f &\mapsto (\pi_1 \circ f, \pi_2 \circ f).
	\end{align*}
	Similarly, this gives a induced map.
\end{adjustbox}

\begin{corollary}
	If
	\[
	V \cong \bigoplus_{i = 1}^{r} V_i, \quad W \cong \bigoplus_{j = 1}^{s} W_j
	,\]
	then
	\[
		\Hom_G(V, W) \cong \bigoplus_{i = 1}^{r} \bigoplus_{j = 1}^{s} \Hom_G(V_i, W_j)
	.\]
\end{corollary}

This follows from an induction argument.

\begin{corollary}
	Suppose $k$ is algebraically closed and
	\[
	V \cong \bigoplus_{i = 1}^{r} V_i
	\]
	is a decomposition of a $k$-representation of $G$ into irreducible components. Then for each irreducible representation $W$ of $G$,
	\[
		|\{i \mid V_i \cong W\}| = \dim \Hom_G(W, V)
	.\]
\end{corollary}

\begin{adjustbox}{minipage = \columnwidth - 25.5pt, margin=1em, frame=1pt, margin=0em}
\textbf{Proof:} From the previous result,
\[
	\Hom_G(W, V) \cong \bigoplus_{i = 1}^{r} \Hom_G(W, V_i)
,\]
and by taking dimensions,
\[
	\dim \Hom_G(W, V) = \sum_{i = 1}^{r} \dim \Hom_G(W, V_i)
.\]
It suffices that $\dim \Hom_G(W, V_i) = 1$ if $W \cong V_i$, and $0$ if $W \not \cong V_i$. But this is Schur's lemma.
\end{adjustbox}

This raises the questions: how do we compute the number $\dim \Hom_G(V, W)$? We saw in the proof of Maschke's theorem that if $k$ is the trivial representation then
\[
	\dim \Hom_G(k, V) = \dim V^{G} = \frac{1}{|G|} \sum_{g \in G}\tr \rho(g)
.\]

\begin{corollary}
	If $G$ is a finite group that has a faithful irreducible representation over an algebraically closed field, then the centre $Z(G)$ is cyclic.
\end{corollary}

\begin{adjustbox}{minipage = \columnwidth - 25.5pt, margin=1em, frame=1pt, margin=0em}
	\textbf{Proof:} Let $V$ be a faithful irreducible representation of $G$ and $z \in Z$. Then $\rho(g) \rho(z) = \rho(z) \rho(g)$ for all $g \in G$. So $\rho(z) \in \Hom_G(V, V) = k \id_V$. 

	Let  $\phi : Z \to k^{\times}$ such that $\rho(z) = \phi(z) \id_V$. This is an injective homomorphism since $V$ is faithful. In particular, $Z$ is isomorphic to a finite subgroup of $k^{X}$, and so it is cyclic.
\end{adjustbox}

\begin{corollary}
	Every irreducible complex representation of a finite abelian group $G$ is one-dimensional.
\end{corollary}

\begin{adjustbox}{minipage = \columnwidth - 25.5pt, margin=1em, frame=1pt, margin=0em}
	\textbf{Proof:} Let $(\rho, V)$ be a complex irreducible representation of $G$. For each $g \in G$, $\rho(g) \in \Hom_G(V, V)$. So by Schur, $\rho(g) = \lambda_g \id_V$. Then for $v \in V$, $\rho(g)v \in \langle v \rangle$, so $\langle v \rangle$ is a $G$-invariant subspace, so $\langle v \rangle = V$.
\end{adjustbox}

\begin{proposition}
	Every finite abelian group has $|G|$ irreducible complex representations.
\end{proposition}

\begin{adjustbox}{minipage = \columnwidth - 25.5pt, margin=1em, frame=1pt, margin=0em}
	\textbf{Proof:} Let $(\rho, V)$ be an irreducible complex representation. By the last corollary, $\dim \rho = 1$, so $\rho : G \to \mathbb{C}^{\times}$ is a homomorphism, and every such homomorphism gives an irreducible representation.

	If $G = G_1 \times G_2$ is a direct product of subgroups $G_1, G_2$ then there is a bijection
	\[
		\{\text{homs. } G \to \mathbb{C}^{\times}\} \leftrightarrow \{\text{homs. } G_1 \to \mathbb{C}^{\times}\} \times \{\text{homs. } G_2 \to \mathbb{C}^{\times}\}
	.\]
	Thus we may reduce to the case $G$ is cyclic. Then $G = \langle x \rangle = C_n$, say. Now $\rho$ is determined by $\rho(x)$, and $\rho(x)^{n} = 1$, so $\rho(x)$ is an $n$'th root of unity. There are $n$ such roots of unity, so $G$ has $n = |G|$ irreducible representations.
\end{adjustbox}

\begin{lemma}
	If $(\rho_1, V_1), (\rho_2, V_2)$ are non-isomorphic one-dimensional complex representations of a finite group $G$, then
	\[
		\sum_{g \in G}\rho_1(g^{-1}) \rho_2(g) = \sum_{g \in G}\overline{\rho_1(g)}\rho_2(g) = 0
	.\]
\end{lemma}

\begin{adjustbox}{minipage = \columnwidth - 25.5pt, margin=1em, frame=1pt, margin=0em}
	\textbf{Proof:} We have seen that $\Hom_{\mathbb{C}}(V_1, V_2)$ is a $G$-representation via $(g \cdot \phi) = \rho_2(g) \phi \rho_1(g^{-1})$, and
	\[
		\sum_{g \in G}g \cdot \phi \in \Hom_G(V_1, V_2) = 0
	,\]
	for all $\phi \in  \Hom_{\mathbb{C}}(V_1, V_2)$, by Schur's. Pick an isomorphism $\phi \in \Hom_{\mathbb{C}}(V_1, V_2)$. Then
	\[
		0 = \sum_{g \in G}\rho_2(g) \phi \rho_1(g^{-1}) = \sum_{g \in G} \overline{\rho_1(g)}\rho_2(g) \phi
	,\]
	as required.
\end{adjustbox}

If $V$ is a representation of a group $G$ that is completely reducible and $W$ is any irreducible representation of $G$, then the $W$-isotypic component\index{isotypic component} of $V$ is the smallest subrepresentation of $V$ containing all simple subrepresentations isomorphic to $W$. This exists, as if $(V_i)$ are subrepresentations of $V$ containing all simple subrepresentations isomorphic to $W$, then so is $\bigcap V_i$.

We say that $V$ has a unique isotypical decomposition\index{unique isotypical decomposition} if $V$ is the direct sum of its $W$-isotypic components as $W$ varies over all simple representations of $V$ (up to isomorphism).

\begin{corollary}
	Suppose $G$ is a finite abelian group. Then every complex representation $V$ of $G$ has a unique isotypical decomposition.
\end{corollary}

\begin{adjustbox}{minipage = \columnwidth - 25.5pt, margin=1em, frame=1pt, margin=0em}
\textbf{Proof:} For each $\theta_i : G \to \mathbb{C}^{\times}$, we can define $W_i$ to be the subspace of $V$ defined by
\[
	W_i = \{v \in V \mid \rho(g)v = \theta_i(g)v \text{ for all } g \in G\}
.\]
Since $V$ is completely reducible and every irreducible representation of $G$ is one dimensional, $V = \sum W_i$. We now show the sum is direct. But this gives
\[
	0 = \rho(g) \sum w_i = \sum \theta_i (g) w_i
.\]
Then for each $j = 1, \ldots, |G|$,
\[
	\sum_{i} \sum_{g \in G} (\theta_j(g^{-1})\theta_i(g))w_i = 0
.\]
But this gives $|G| w_j = 0$. Note the proof shows that
\[
	\frac{1}{|G|} \sum_{g \in G} \theta_j(g^{-1}) \rho(g)
\]
is a projection of $V$ onto the $\theta_j$-isotypic component.
\end{adjustbox}

\newpage

\section{Characters}%
\label{sec:characters}

We have seen that if $G$ is finite and $k$ has characteristic zero, then every representation $V$ decomposes into simple representations. Moreover, we have seen when $k$ is algebraically closed, then $n_i = \dim \Hom_G(V, V_i) = \dim \Hom_G(V_i, V)$. Our next goal is to try and classify irreducible representations of a finite group and understand how to compute the $n_i$.

\begin{definition}
	Given a representation $\rho : G \to GL(V)$ the \textbf{character}\index{character} of $\rho$ is the function $\chi = \chi_{\rho} = \chi_{V} : G \to k$ given by $g \mapsto \tr \rho(g)$. Since for matrices $\tr(AB) = \tr(BA)$, the character does not depend on the choice of basis for $V$. The same argument gives that isomorphic representations have the same trace.
\end{definition}

\begin{adjustbox}{minipage = \columnwidth - 25.5pt, margin=1em, frame=1pt, margin=0em}
\begin{example}
	Let $G = D_6$ act on $\mathbb{R}^2$ by the symmetries of a triangle. To compute the character of the rep, it suffies to compute the eigenvalues of each matrix $\rho(g)$.

	Each reflection $\{s, st, st^2\}$ has eigenvalues $1, -1$, so $\chi(g) = 0$. Then for the rotations,
	\[
		\rho(t^{n}) =
		\begin{pmatrix}
			\cos \frac{2 \pi n}{3} & \sin \frac{2 \pi n}{3} \\
			- \sin \frac{2 \pi n}{3} & \cos \frac{2 \pi n}{2}
		\end{pmatrix}
	,\] 
	so
	\[
		\chi(t^{n}) = 2 \cos \frac{2 \pi n}{3} =
		\begin{cases}
			-1 & r = 1, 2, \\
			2 & r = 0.
		\end{cases}
	\]
\end{example}

\end{adjustbox}

\begin{proposition}
	Let $(\rho, V)$ be a representation of $G$ with character $\chi$. Then:
	\begin{enumerate}[\normalfont(i)]
		\item $\chi(e) = \dim V$;
		\item $\chi(g) - \chi(h g h^{-1})$, $g, h \in G$;
		\item If $\chi'$ is the character of $(\rho', W)$, then $\chi + \chi'$ is the character of $V \oplus W$;
		\item If $k = \mathbb{C}$ and $o(g) < \infty$, then $\chi(g^{-1}) = \overline{\chi(g)}$.
	\end{enumerate}
	
\end{proposition}

\begin{adjustbox}{minipage = \columnwidth - 25.5pt, margin=1em, frame=1pt, margin=0em}
\textbf{Proof:}
\begin{enumerate}[(i)]
\item $\chi(e) = \tr \id_V = \dim V$.
\item $\rho(h)\rho(g)\rho(h)^{-1} = \rho(hgh^{-1})$, so $\rho(g)$ and $\rho(hgh^{-1})$ are conjugate, so have the same trace.
\item This follows from
	\[
	\tr
	\begin{pmatrix}
		A & 0 \\
		0 & B
	\end{pmatrix}
	= \tr A + \tr B
	.\]
\item If $\rho(g)$ has eigenvalues $\lambda_1, \ldots, \lambda_n$, then $\chi(g) = \sum \lambda_i$, but if $o(g)$ is finite then all the $\lambda_i$ are roots of unity, and so have multitude 1. So
	\[
		\overline{\chi(g)} = \sum_{i = 1}^{n} \overline{\lambda_i} = \sum_{i = 1}^{n} \lambda_i^{-1}
	,\]
	and the $\lambda_i^{-1}$ are the eigenvalues of $\rho(g^{-1})$.
\end{enumerate}

\end{adjustbox}

\begin{lemma}
	If $U, V$ and $W$ are representations of $G$, and $u_1 \in \Hom_k(V, W)$, $u_2 \in \Hom_k(U, V)$, then for all $g \in G$,
	\[
		g \cdot (\phi_1 \circ \phi_2) = (g \cdot \phi_1) \circ (g \cdot \phi_2)
	.\]
	In particular,
\[\phi_1 \in \Hom_G(V, W) \implies g\cdot(\phi_1 \circ \phi_2) = \phi_1 \circ(g \circ \phi_2),\]
\[
	\phi_2 \in \Hom_G(U, V) \implies g \cdot(\phi_1 \circ \phi_2) = (g \cdot \phi_1) \circ \phi_2
,\]
and
\[
	\phi_1 \in \Hom_G(V, W), \phi_2 \in \Hom_G(U, V) \implies \phi_1 \circ \phi_2 \in \Hom_G(U, W)
.\]
\end{lemma}

\begin{adjustbox}{minipage = \columnwidth - 25.5pt, margin=1em, frame=1pt, margin=0em}
\textbf{Proof:} 
\[
	(g \cdot \phi_1) \circ (g \cdot \phi_2) = (g \circ \phi_1 \circ g^{-1}) \circ (g \circ \phi_2 \circ g^{-1}) = g \circ (\phi_1 \circ \phi_2) \circ g^{-1} = g \cdot (\phi_1 \circ \phi_2)
.\]
\end{adjustbox}

\begin{definition}
	We say $f : G \to k$ is a \textbf{class function}\index{class function} if $f(hgh^{-1}) = f(g)$ for all $g, h \in G$. We write $ \mathcal{C}_G$ for the $k$-vector space of class functions. Notice that if $\mathcal{O}_1, \ldots, \mathcal{O}_r$ is a list of conjugacy classes in $G$, then the indicator function
	\begin{align*}
		\mathbbm{1}_{\mathcal{O}_i} : G &\to k \\
		g &\mapsto
		\begin{cases}
			1 & g \in \mathcal{O}_i, \\
			0 & g \not \in \mathcal{O}_i,
		\end{cases}
	\end{align*}
	is a class function.
\end{definition}

\subsection{Orthogonality of Characters}%
\label{sub:orthogonality_of_characters}

We will assume $G$ is a finite group and $k = \mathbb{C}$. Note $k = \bar k$ has characteristic $0$ will suffice for the main result.

We can make $\mathcal{C}_G$ into a Hermitian inner product space, by defining
\[
	\langle f_1, f_2 \rangle = \frac{1}{|G|} \overline{f_1(g)}f_2(g)
.\]
This defines a Hermitian inner product, and it is easy to check that the indicator functions $\mathbbm{1}_{\mathcal{O}_i}$ are pairwise orthogonal. Note that
\[
	\langle \mathbbm{1}_{\mathcal{O}_i}, \mathbbm{1}_{\mathcal{O}_i} \rangle = \frac{|\mathcal{O}_i|}{|G|} = \frac{1}{|C_G(x_i)|}
,\]
for any $x_i \in \mathcal{O}_i$. Thus if $x_1, \ldots, x_r$ are conjugacy class representatives,
\[
	\langle f_1, f_2 \rangle = \sum_{i = 1}^{r} \frac{1}{|C_G(x_i)|}\overline{f_1(x_i)}f_2(x_i)
.\]
\begin{adjustbox}{minipage = \columnwidth - 25.5pt, margin=1em, frame=1pt, margin=0em}
\begin{example}
	$G = D_6 = \langle s, t \mid s^2 = t^3 = 1, sts = t^{-1}\rangle$ has conjugacy classes $\{e\}, \{s, st, st^2\}, \{t, t^2\}$. Hence,
	\[
		\langle f_1, f_2 \rangle = \frac{1}{6}\overline{f_1(e)}f_2(e) + \frac{1}{2}\overline{f_1(s)}f_2(s) + \frac{1}{3}\overline{f_1(t)}f_2(t)
	.\]
\end{example}

\end{adjustbox}

\begin{theorem}[Orthogonality of Characters]\index{orthogonality of characters}
	If $V, V'$ are irreducible representations of $G$, then
	\[
		\langle \chi_V, \chi_{V'} \rangle =
		\begin{cases}
			1 & V \cong V', \\
			0 & \text{else}.
		\end{cases}
	\]
\end{theorem}

\begin{lemma}
	If $V$ and $W$ are unitary representations of a group $G$, then
	\[
		\chi_{\Hom_k(V, W)}(g) = \overline{\chi_V(g)}\chi_{W}(g)
	.\]
\end{lemma}

\begin{adjustbox}{minipage = \columnwidth - 25.5pt, margin=1em, frame=1pt, margin=0em}
	\textbf{Proof:} Given $g \in G$, we may choose $v_1, \ldots, v_n$ a basis of $V$ consisting of $g$-eigenvectors and $w_1, \ldots, w_m$ a basis of $W$ consisting of $g$-eigenvectors. Then the function $\alpha_{ij}(v_k) = \delta_{jk}w_i$ extend to linear maps that form a basis for $\Hom_k(V, W)$ and
	\begin{align*}
		(g \cdot \alpha_{ij})(v_k) = g \circ (\alpha_{ij}(g^{-1} (v_k))) = \delta_{jk}\lambda_k^{-1} \mu_i w_i.
	\end{align*}
	Thus $g \cdot \alpha_{ij} = \lambda_k^{-1} \mu_i \alpha_{ij}$ and
	\[
		\chi_{\Hom_k(V, W)}(g) = \sum_{i, j} \lambda_j^{-1} \mu_i = \chi_V(g^{-1}) \chi_W(g) = \overline{\chi_V(g)}\chi_W(g)
	.\]
\end{adjustbox}

\begin{lemma}
	If $U$ is a representation of $G$, then
	\[
		\dim U^{G} = \dim \{ u \in U \mid gu=u\, \forall g \in G\} = \langle 1, \chi_U\rangle = \frac{1}{|G|} \sum_{g \in G}\chi_U(g)
	.\]
\end{lemma}

\begin{adjustbox}{minipage = \columnwidth - 25.5pt, margin=1em, frame=1pt, margin=0em}
\textbf{Proof:} Define $\pi : U \to U$ by
\[
	\pi(u) = \frac{1}{|G|} \sum_{g \in G}gu
.\]
Then $\pi(u) \in U^{g}$ for all $u \in U$. Moreover, $\pi_{U^{G}} = \id_{U^{G}}$, so
\[
	\dim U^{G} = \tr \id_{U^{G}} = \tr \pi = \frac{1}{|G|} \sum_{g \in G}\chi_U(g)
.\]
\end{adjustbox}

\begin{proposition}
	If $V$ and $W$ are representations of $G$, then
	\[
		\dim \Hom_G(V, W) = \langle \chi_V, \chi_W \rangle
	.\]
\end{proposition}

\begin{adjustbox}{minipage = \columnwidth - 25.5pt, margin=1em, frame=1pt, margin=0em}
	\textbf{Proof:} By the lemmas, $\dim \Hom_G(V, W) = \{\mathbbm{1}, \overline{\chi_V}\chi_W \rangle = \langle \chi_V, \chi_W \rangle$.
\end{adjustbox}

\begin{corollary}
	If $V, W$ are irreducible representations of $G$, then
	\[
		\langle \chi_V, \chi_W \rangle =
		\begin{cases}
			1 & V \cong W, \\
			0 & \text{else}.
		\end{cases}
	\]
	In particular, $\chi_V = \chi_W \implies V \cong W$.
\end{corollary}

\begin{adjustbox}{minipage = \columnwidth - 25.5pt, margin=1em, frame=1pt, margin=0em}
\textbf{Proof:} By the proposition and Schur's Lemma,
\[
	\langle \chi_V, \chi_W \rangle = \dim \Hom_G(V, W) =
	\begin{cases}
		1 & V \cong W, \\
		0 & \text{else}.
	\end{cases}
\]
If $\chi_V= \chi_W$, then $\langle \chi_V, \chi_V \rangle_{G} > 0$, so $V \cong W$.
\end{adjustbox}

\begin{corollary}
	If $(\rho, V)$ is a representation of $G$, then
	\[
		V \cong \bigoplus_{W \text{ irr.}} \langle \chi_W, \chi_{\rho} \rangle W
	.\]
	In particular, if $(\rho', V')$ is another representation of $G$, then $\rho \cong \rho'$ if and only if $\chi_\rho = \chi_{\rho'}$.
\end{corollary}

\begin{adjustbox}{minipage = \columnwidth - 25.5pt, margin=1em, frame=1pt, margin=0em}
\textbf{Proof:} By Maschke's theorem, there exists $n_W \geq 0$ such that
\[
	V \cong \bigoplus_{W \text{ irr.}} n_w W
,\]
and we have seen that $n_w = \dim \Hom_G(W, V) = \langle \chi_W, \chi_V \rangle_W$. So
\[
	V \cong \bigoplus_{W \text{ irr.}} \langle \chi_W, \chi_\rho \rangle_G W
.\]
This only depends on $\chi_{\rho}$.
\end{adjustbox}

Note that complete reducibility was crucial in this proof. For example, the representation of $(\mathbb{Z}, +)$ by
\[
	\rho(1) =
	\begin{pmatrix}
		1 & 1 \\
		0 & 1
	\end{pmatrix}
	, \quad \rho(1) =
	\begin{pmatrix}
		1 & 0 \\
		0 & 1
	\end{pmatrix}
,\]
have the same character but are not isomorphic.

\begin{corollary}
	If $\rho$ is a representation of $G$ with character $\chi$, then $\rho$ is irreducible if and only if $\langle \chi, \chi \rangle = 1$.
\end{corollary}

\begin{adjustbox}{minipage = \columnwidth - 25.5pt, margin=1em, frame=1pt, margin=0em}
	\textbf{Proof:} The forward direction follows from the orthogonality of characters. Otherwise, assume that $\langle \chi, \chi \rangle = 1$. Then we may write $\chi = \sum n_i \chi_i$. But then
	\[
		\langle \chi, \chi \rangle = \sum_{W, W'} n_W n_{W'} \langle \chi_W, \chi_{W'} \rangle_g = \sum_{W}n_W^2
	.\]
	Hence $n_W = 1$ for some $W$, and so $\chi = \chi_j$.
\end{adjustbox}

\begin{adjustbox}{minipage = \columnwidth - 25.5pt, margin=1em, frame=1pt, margin=0em}
\begin{example}
	Consider the action of $D_6$ on $\mathbb{C}^2$ by the symmetries of a triangle. Then $\chi(1) = 2$, $\chi(t) = -1$, $\chi(s) = 0$. We have
	\[
		\langle \chi, \chi \rangle_G = \frac{1}{6} \overline{\chi(1)}\chi(1) + \frac{1}{3} \overline{\chi(t)} \chi(t) + \frac{1}{2} \overline{\chi(s)}\chi(s) = \frac{4}{6} + \frac{1}{3} = 1
	.\]
	So this representation is irreducible.
\end{example}

\end{adjustbox}

\begin{theorem}
	The characters of irreducible representations of a finite group $G$ form an orthonormal basis for $\mathcal{C}_G$ with respect to
	\[
		\langle f_1, f_2 \rangle_G = \frac{1}{|G|} \sum_{g \in G}\overline{f_1(g)}f_2(g)
	.\]
\end{theorem}

\begin{adjustbox}{minipage = \columnwidth - 25.5pt, margin=1em, frame=1pt, margin=0em}
\textbf{Proof:} We have already seen the irreducible characters form an orthonormal set. We now show they span $\mathcal{C}_G$. Let
\[
	I = \langle \chi_1, \ldots, \chi_r \rangle
\]
be the $\mathbb{C}$-linear span of the sets of irreducible characters. We need to show that $I^{\perp} = 0$. Suppose $f \in \mathcal{C}_G$. For each representation $(\rho, V)$ of $G$, we can define
\[
	\phi = \phi_{\rho, V} = \frac{1}{|G|} \sum_{g \in G} \overline{f(g)}\rho(g) \in \Hom_G(V, V)
.\]
For for $h \in G$,
\begin{align*}
	\rho(h^{-1}) \circ \phi \circ \rho(h) &= \frac{1}{|G|} \sum_{g \in G} \overline{f(g)}\rho(h^{-1}gh) = \frac{1}{|G|} \sum_{g \in G} \overline{f(hg'h^{-1})}\rho(g') \\
					      &= \frac{1}{|G|} \sum_{g \in G} \overline{f(g')} \rho(g') = \phi.
\end{align*}
\end{adjustbox}

\begin{adjustbox}{minipage = \columnwidth - 25.5pt, margin=1em, frame=1pt, margin=0em}
Thus if $V$ is irreducible, then $\phi_{\rho, V} = \lambda \id_V$ for some $\lambda \in \mathbb{C}$. If additionally $f \in I^{\perp}$, then $\lambda \dim V = \tr(\phi_{\rho, V}) = \langle f, \chi_V \rangle = 0$, so $\lambda = 0$.

But every representation $V$ decomposes as a direct sum of irreducible representations, and
\[
\phi_{\rho, V} = \bigoplus_{i = 1}^{r} \phi_{\rho, V_i}
.\]
Thus $\phi_{\rho, V} = 0$ for every representation $V$. In particular, if $V = \mathbb{C} G$, then
\[
	0 = \phi_{\rho, V} \delta_e = \frac{1}{|G|} \sum_{g \in G} \overline{f(g)} \delta_{g} = \frac{1}{|G|} \overline{f}
,\]
so $f = 0$.

\end{adjustbox}


\begin{corollary}
	The number of irreducible representations of $G$ (up to isomorphism) is the number of conjugacy classes in $G$.
\end{corollary}

\begin{corollary}
	For $g \in G$, $\chi(g) \in \mathbb{R}$ for every character $\chi \iff g$ is conjugate to $g^{-1}$ in $G$.
\end{corollary}

\begin{adjustbox}{minipage = \columnwidth - 25.5pt, margin=1em, frame=1pt, margin=0em}
	\textbf{Proof:} Since $\chi(g^{-1}) = \chi(g)$, $\chi(g) = \mathbb{R}$ for every character $\chi$ if and only if $f(g) = f(g^{-1})$ for every class function. But if $g$ is conjugate to $g^{-1}$, then $f(g) = f(g^{-1})$ for every class function, and if $\mathcal{O}$ is a conjugacy class in $G$, then consider class function
	\begin{align*}
		\mathbbm{1}_{\mathcal{O}_i} : G &\to \mathbb{C} \\
		h &\mapsto 
		\begin{cases}
			1 & h \in \mathcal{O}_i, \\
			0 & h \not \in \mathcal{O}_i.
		\end{cases}
	\end{align*}
	So if $f(g) = f(g^{-1})$ for all $f \in \mathcal{C}_G$, then $\mathbbm{1}_{\mathcal{O}_i}(g) = \mathbbm{1}_{\mathcal{O}_i}(g^{-1})$ for all $\mathcal{O}_i$, hence $g$ and $g^{-1}$ are in the same conjugacy class.
\end{adjustbox}

\subsection{Character Tables}%
\label{sub:character_tables}

Since characters determine representations, and representations are all direct sums of irreducible representations, to classify representations is (roughly) the same as classifying irreducible characters. The difference is finding an explicit irreducible representation with a given irreducible character.

The character table\index{character table} of a group is defined as follows. Let the conjugacy classes of $G$ be $\mathcal{O}_1, \ldots, \mathcal{O}_r$, with $\mathcal{O}_1 = \{e\}$, and choose $g_i \in \mathcal{O}_i$. We then list the irreducible characters $\chi_1, \ldots, \chi_r$, with convention that $\chi_1 = \chi_{\mathbb{C}}$. Then we write down the matrix

\begin{center}
\begin{tabular}{c|cccccc}
	 & $e$ & $g_2$ & $\cdots$ & $g_j$ & $\cdots$ & $g_r$ \\
	 \hline
	$\chi_1$ & 1 & 1 & $\cdots$ & 1 & $\cdots$ & 1 \\
	$\chi_2$ &  &  &  &  &  &  \\
	$\vdots$ &  &  &  &  &  &  \\
	$\chi_i$ &  &  &  & $\chi_i(g_j)$ &  &  \\
	$\vdots$ &  &  &  &  &  &  \\
	$\chi_r$ &  &  &  &  &  &  \\
\end{tabular}
\end{center}
This is the character table. Sometimes we write the size of $\mathcal{O}_i$ above $g_i$, and sometimes the equivalent data $|C_G(g_i)|$.

\begin{adjustbox}{minipage = \columnwidth - 25.5pt, margin=1em, frame=1pt, margin=0em}
\begin{example}
	\begin{enumerate}[1.]
		\item[]
		\item Let $C_3 = \langle x \rangle$ and $\omega \in \mathbb{C}$ such that $\omega^2 = \overline{\omega}$. Then the character table is
			\begin{center}
\begin{tabular}{c|ccc}
	 & $e$ & $x$ & $x^2$ \\
	 \hline
	$\chi_1$ & $1$ & $1$ &  \\
	$\chi_2$ & $1$ & $\omega$ & $\overline{\omega}$ \\
	$\chi_3$ & $1$ & $\overline{\omega}$ & $\omega$ \\
\end{tabular}
			\end{center}
		Here, the rows are pairwise orthogonal, and the columns are as well.
	\item Consider $G = S_3$. There are three conjugacy classes $\mathcal{O}_1 = \{e\}$, $\mathcal{O}_2 = \{(12), (23), (13)\}$ and $\mathcal{O}_3 = \{(123),(132)\}$. Thus there are three irreducible representations. The trivial representation has character $1$ for all $g \in G$. There is also the sign character $\epsilon : S_3 \to \{\pm 1\}$ given by
		\[
			\epsilon(g) =
			\begin{cases}
				1 & g \text{ even}, \\
				-1 & g \text{ odd}.
			\end{cases}
		\]
	\end{enumerate}
\end{example}
\end{adjustbox}

\begin{adjustbox}{minipage = \columnwidth - 25.5pt, margin=1em, frame=1pt, margin=0em}
\begin{enumerate}
	\item[]
		We can compute the last character using orthogonality of characters. Let $\chi(e) = a$, $\chi((12)) = b$, $\chi((123)) = c$. Then by orthogonality,
		\begin{align*}
			0 = \langle \mathbf{1}, \chi \rangle &= \frac{1}{6} (a + 3b + 2c), \\
			0 = \langle \epsilon, \chi \rangle &= \frac{1}{6} (a - 3b + 2x), \\
			1 = \langle \chi, \chi \rangle &= \frac{1}{6}(a^2 + 3b^2 + 2c^2).
		\end{align*}
		Solving, we get $a = 2, b = 0, c = -1$. Hence the character table is
\begin{center}
\begin{tabular}{c|ccc}
	$|\mathcal{O}_i|$ & $1$ & $3$ & $2$ \\
			  & $e$ & $(12)$ & $(123)$ \\
			  \hline
	$\mathbf{1}$ & $1$ & $1$ & $1$ \\
	$\varepsilon$ & $1$ & $-1$ & $1$ \\
	$\chi$ & $2$ & $0$ & $-1$ \\
\end{tabular}
\end{center}
		In fact, we have seen the last representation from the symmetries of a triangle in $\mathbb{R}^2$. The rows are orthogonal under
		\[
			\langle f_1, f_2 \rangle = \sum_{g_i} \frac{1}{|C_G(g_i)|}\overline{f_1(g_i)}f_2(g_i)
		,\]
		and the columns are orthogonal with respect to the standard inner product. The lengths of the columns are simply the size of the centralizer.

\end{enumerate}

\end{adjustbox}

\begin{proposition}[Column Orthogonality]
	If $G$ is a finite group and $\chi_1, \ldots, \chi_r$ are the irreducible characters of $G$, then for $g, h \in G$,
	\[
		\sum_{i = 1}^{r} \overline{\chi_i(g)}\chi_i(h) =
		\begin{cases}
			0 & g, h \text{ not conjugate}, \\
			|C_G(g)| & g, h \text{ conjugate}.
		\end{cases}
	\]
\end{proposition}

\begin{adjustbox}{minipage = \columnwidth - 25.5pt, margin=1em, frame=1pt, margin=0em}
	\textbf{Proof:} Let $X$ be the character table of $G$ as a matrix $X_{ij} = \chi_i(g_j)$, and let $D$ be the diagonal matrix with $D_{ii} = |C_G(g_i)|$. The orthogonality of characters table is equivalent to
	\[
		\sum_{k} \frac{1}{|C_G(g_i)|} \overline{X_{ik}}X_{jk} = \delta_{ij}
	,\]
	or $\overline{X} D^{-1} X^{T} = I$. Since $X$ is square and $D$ is real, we can compute this as $D^{-1} \overline{X}^{T} = X^{-1}$, or $D = \overline{X}^{T} X$, or
	\[
		\sum_{k = 1}^{r} \overline{\chi_k(g_i)} \chi_k(g_j) = \delta_{ij} |C_G(g_i)|
	,\]
	as desired.
\end{adjustbox}

\newpage

\begin{definition}
	The character ring\index{character ring} of a group $G$ is the subset of $\mathcal{C}_G$ with
	\[
		R(G) = \{\chi_1 - \chi_2 \mid \chi_1, \chi_2 \text{ characters of } G\}
	.\]
\end{definition}

Since $\chi_{V_1 \oplus V_2} = \chi_{V_1} + \chi_{V_2}$, $R(G)$ is an additive subgroup of $\mathcal{C}_G$.

Since $\chi_{triv} = \mathbbm{1}$, $R(G)$ has a multiplicative unit.

Since $\chi_{V_1 \otimes V_2} = \chi_{V_1} \cdot \chi_{V_2}$, $R(G)$ is a subring of $\mathcal{C}_G$.

\begin{remark}
	\begin{enumerate}[(i)]
		\item[]
		\item Tensor products for general representations for general representation coincide with our earlier definition when one representation is one-dimensional.
		\item If $(\rho, V)$ is a representation of $G$ and $(\rho', W)$ is a representation of another group $H$, then we can make $V \otimes W$ a representation of $G \times H$ via
			\[
				\rho_{V \otimes W}(g, h) = \rho(g) \otimes \rho'(h)
			.\]
			This is a representation due to our last lemma, and it satisfies $\chi_{V \otimes W}(g, h) = \chi_V(g) \chi_W(h)$. Thus
			\begin{align*}
				R(G) \times R(H) &\to R(G \times H) \\
				(\chi_V, \chi_W) &\mapsto \chi_{V \otimes W}
			\end{align*}
			is a ring homomorphism. Moreover when $G = H$, we refined the representation $V \times W$ of $G \times G$ to the diagonal subgroup $G \sim \{(g, g) \mid g\in G\} \leq G \times G$.
		\item If $X$ and $Y$ are finite sets with $G$-action, then
			\begin{align*}
				kX \otimes kY &\to k(X \times Y) \\
				(\delta_x \otimes \delta_y) &\mapsto \delta_{(x, y)}
			\end{align*}
			is an isomorphism of representations of $G$.
	\end{enumerate}
\end{remark}

\begin{proposition}
	Suppose $G$, $H$ are finite groups, $(\rho_1, V_1), \ldots, (\rho_s, V_s)$ are all the irreducible $\mathbb{C}$-representations of $G$, and $(\rho_1', W_1), \ldots, (\rho_t', W_t)$ are the irreducible $\mathbb{C}$-representations of $H$.

	Then for each $1 \leq i \leq s$, $1 \leq j \leq t$, $(\rho_1 \otimes \rho_j', V_i \otimes W_j')$ is an irreducible representation of $G \times H$.

	Moreover all irreducible representations of $G \times H$ arise in this was, so $R(G) \times R(H) \cong R(G \times H)$.
\end{proposition}

\begin{adjustbox}{minipage = \columnwidth - 25.5pt, margin=1em, frame=1pt, margin=0em}
	\textbf{Proof:} Let $\chi_1, \ldots, \chi_s$ be the characters of $V_1, \ldots, V_s$, and $\psi_1, \ldots, \psi_t$ be the characters of $W_1, \ldots, W_t$. Then we know $\chi_{V_i \otimes W_j}(g, h) = \chi_i(g) \psi_j(h)$. Thus,
	\begin{align*}
		\langle \chi_{V_i \otimes W_j}, \chi_{V_l \otimes W_m} \rangle &= \frac{1}{|G \times H|} \sum_{(g, h) \in G \times H} \overline{\chi_i(g) \psi_j(h)}\chi_l(g) \psi_m(h) \\
									       &= \frac{1}{|G|}\frac{1}{|H|} \Biggl( \sum_{g \in G} \overline{\chi_i(g)}\chi_l(g) \Biggr)  \sum_{h \in H} \Biggl( \sum_{h \in H} \overline{\psi_j(h)}\psi_m(h) \Biggr) \\
									       &= \langle \chi_i, \chi_l \rangle \langle \psi_j, \psi_m \rangle = \delta_{il} \delta_{jm}.
	\end{align*}
	Thus $\chi_{V_i \otimes W_j}$ are all irreducible and pairwise distinct. Since $\dim \mathcal{C}_{G \times H} = (\dim \mathcal{C}_G) (\dim \mathcal{C}_H)$, we are done.
\end{adjustbox}

Suppose we want to extend this: If $V, W$ are irreducible representations of $G$, then must $V \otimes W$ be an irreducible representation of $G$. If $\dim V$ or $\dim W = 1$, then the answer is yes. But usually, this is false.

\begin{adjustbox}{minipage = \columnwidth - 25.5pt, margin=1em, frame=1pt, margin=0em}
\begin{example}
	Take $G = S_3$, then recall the character table
\begin{center}
\begin{tabular}{c|ccc}
		& $e$ & $(12)$ & $(123)$ \\
			  \hline
	$\mathbf{1}$ & $1$ & $1$ & $1$ \\
	$\varepsilon$ & $1$ & $-1$ & $1$ \\
	$\chi_{V}$ & $2$ & $0$ & $-1$ \\
\end{tabular}
\end{center}
Then we have $\chi_{V \otimes V}$ is given by $\chi_V^2(e) = 4$, $\chi_V^2((12)) = 0$, $\chi^2((123)) = 1$. Hence $\chi_{V \otimes V} = \mathbf{1} + \varepsilon + \chi_{V}$.
\end{example}
\end{adjustbox}

In general,
\[
\chi_i \otimes \chi_j = \sum_{k} a_{ij}^{k} \chi_k
,\]
with $a_{ij}^{k} \in \mathbb{N}$. These coefficients determine $R(G)$ as a ring. In fact $V \otimes V, V \otimes V \otimes V, \ldots$ are never irreducible if $\dim V > 1$.

\subsection{Symmetric and Exterior Products}%
\label{sub:symmetric_and_exterior_products}

For any vector space $V$, we can define $\sigma  V \otimes V \to V \otimes V$ by $\sigma(v \otimes w) = w \otimes v$, for all $v, w \in V$.

Note $\sigma^2 = \id_{V \otimes V}$, so if $\chr k \neq 2$, then $\sigma$ decomposes $V \otimes V$ into two $\sigma$-eigenspaces:
\begin{align*}
	S^2V &= \{ a \in V \otimes V \mid \sigma(a) = a\}, \text{ the symmetric square of } V,\\
	\Lambda^2V &= \{ a \in V \otimes V \mid \sigma(a) = -a\}, \text{ the exterior/alternating square of } V.
\end{align*}
This is the isotypical decomposition of a $C_2$ action of $V \otimes V$ given by $\sigma$.

\begin{lemma}
	Suppose $v_1, \ldots, v_m$ is a basis for $V$.
	\begin{enumerate}[\normalfont(i)]
		\item $S^2V$ has basis
			\[
				v_iv_j = \frac{1}{2} (v_i \otimes v_j + v_j \otimes v_i)
			,\]
			for $1 \leq i \leq j \leq m$.
		\item $\Lambda^2V$ has basis
			\[
				v_i \wedge v_j = \frac{1}{2} (v_i \otimes v_j - v_j \otimes v_i)
			,\]
			for $1 \leq i < j \leq m$.
	\end{enumerate}
	Thus $\dim S^2V = \frac{1}{2}m(m+1)$ and $\dim \Lambda^2V = \frac{1}{2}m(m-1)$.
\end{lemma}

\begin{adjustbox}{minipage = \columnwidth - 25.5pt, margin=1em, frame=1pt, margin=0em}
\textbf{Proof:} It is easy to see each $v_iv_j \in S^2V$ and $v_i \wedge v_j \in \Lambda^2V$, and the union of the claimed bases span $V \otimes V$ and has $m^2$ elements. So it is a basis for $V \otimes V$, and they are indeed bases.
\end{adjustbox}

\begin{proposition}
	Let $(\rho, V)$ be a representation of $G$ over $\mathbb{C}$, with $\chr k \neq 2$.
	\begin{enumerate}[\normalfont(i)]
		\item $V \otimes V = S^2 V \oplus \Lambda^2 V$ as representations of $G$.
		\item For $g \in G$,
			\begin{align*}
				\chi_{S^2V}(g) &= \frac{1}{2} (\chi_V(g)^2 + \chi_V(g^2)), \\
				\chi_{\Lambda^2V}(g) &= \frac{1}{2} (\chi_V(g)^2 - \chi_V(g^2)).
			\end{align*}
	\end{enumerate}
\end{proposition}

\begin{adjustbox}{minipage = \columnwidth - 25.5pt, margin=1em, frame=1pt, margin=0em}
	\textbf{Proof:} For (i), we need to show if $a \in V \otimes V$ with $\sigma(a) = \lambda a$, then $\sigma(\rho_{V \otimes V}(g)(a)) = \lambda \rho_{V \otimes V}(g)(a)$. It is enough to show that $\sigma \rho_{V \otimes V} = \rho_{V \otimes V} \sigma$, i.e. $\sigma \in \Hom_G(V \otimes V, V \otimes V)$.

	But $\sigma g(v \otimes w) = \sigma(gv \otimes gw) = gw \otimes gv$, and $g \sigma(v \otimes w) = g(w \otimes v) = gw \otimes gv$, as desired.

	For (ii), note that $\chi_{S^2V} + \chi_{\Lambda^2V} = \chi_{V \otimes V} = \chi_{V}^2$, so it suffices to prove the formula for $\chi_{S^2V}$.

	Let $v_1, \ldots, v_m$ be a basis for $V$ consisting of $g$-eigenvectors $gv_i = \lambda_i v_i$. Then
	\[
		g(v_iv_j) = \frac{1}{2} (gv_i \otimes gv_j + gv_j \otimes gv_i) = \frac{\lambda_i \lambda_j}{2}(v_i \otimes v_j + v_j \otimes v_i) = \lambda_i \lambda_j v_i v_j
	.\]
	So the sum of the eigenvalues are
	\[
		\chi_{S^2V}(g) = \sum_{1 \leq i \leq j \leq m} \lambda_i \lambda_j
	.\]
	But we can write
	\[
		\frac{1}{2} (\chi_V(g)^2 + \chi_V(g^2)) = \frac{1}{2} \left( \Biggl( \sum_{i = 1}^{n} \lambda_i \Biggr)^2 + \sum_{i = 1}^{n} \lambda_i^2 \right) = \frac{1}{2} \cdot 2 \Biggl( \sum_{i \leq j} \lambda_i \lambda_j \Biggr) = \chi_{S^2 V}(g)
	.\]
\end{adjustbox}

\begin{adjustbox}{minipage = \columnwidth - 25.5pt, margin=1em, frame=1pt, margin=0em}
\begin{example}
	Take $G = S_4$. The character table for $S_4$ is

\begin{center}
\begin{tabular}{c|ccccc}
		& $e$ & $(12)(34)$ & $(12)$ & $(123)$ & $(1234)$ \\
			  \hline
	$\mathbf{1}$ & $1$ & $1$ & $1$ & $1$ & $1$ \\
	$\varepsilon$ & $1$ & $1$ & $1$ & $-1$ & $-1$ \\
	$\chi_{3}$ & $3$ & $-1$ & $0$ & $1$ & $-1$\\
	$\chi_4$ & $3$ & $-1$ & $0$ & $-1$ & $1$ \\
	$\chi_5$ & $2$ & $2$ & $-1$ & $0$ & $0$ \\
	\hline
	$\chi_(s)$ & $4$ & $1$ & $0$ & $1$ & $1$ \\
	$\chi_{(g^2)}$ & $3$ & $3$ & $0$ & $3$ & $-1$ \\
	$S^2\chi_3$ & $6$ & $2$ & $0$ & $2$ & $0$ \\
	$\Lambda^2\chi_3$ & $3$ & $-1$ & $0$ & $-1$ & $1$
\end{tabular}
\end{center}
So $S^2\chi_3 = 1 + \chi_3 + \chi_5$ and $\Lambda^2 \chi_3 = \chi_4 = \varepsilon \chi_3$. Hence we can compute $\chi_4, \chi_5$ fro $\chi_1$, $\chi_2$, $S^2 \chi_3$ and $\Lambda^2 \chi_3$.
\end{example}
\end{adjustbox}

More generally for every vector space $V$, we can consider $V^{\otimes n} = \underbracket{V \otimes \cdots \otimes V}_{n \text{ times}}$. Then for any $\omega \in S_n$ we can define $\sigma(\omega) : V^{\otimes} \to V^{\otimes}$ by
\[
	v_1 \otimes \cdots \otimes v_n \mapsto v_{\omega^{-1}(1)} \times \cdots \times v_{\omega^{-1}(n)}
.\]
We can show that
\begin{align*}
	S_n &\to GL(V^{\otimes n}) \\
	\omega &\mapsto \sigma(\omega)
\end{align*}
defines a representation $S_n$ on $V^{\otimes n}$. Moreover, if $V$ is a representation of $G$, then the $G$-action on $V^{\otimes n}$ and $S_n$ commute, so $S_n \to GL(V^{\otimes})$ has image inside $\Hom_G(V^{\otimes n}, V^{\otimes n})$.

Now we can decompose $V^{\otimes n}$ into $S_n$ isotypical components, and each one will be a $G$-invariant subspace of $V^{\otimes n}$. In particular, we can consider the components for $\mathbf{1}$ and $\varepsilon$.

\begin{definition}\index{symmetric power}\index{alternating power}
	Suppose $V$ is a vector space. We define
	\begin{align*}
		S^{n}V &= \{a \in V^{\otimes n} \mid \sigma(\omega)(a) = a \, \forall w \in S_n\}, \text{ the $n$'th symmetric power of } V,\\
		\Lambda^{n}V &= \{a \in V^{\otimes n} \mid \sigma(\omega)(a) = \varepsilon(\omega) a \, \forall w \in S_n\}, \text{ the $n$'th alternating power of }V.
	\end{align*}
	Note that $S^{n}V \oplus \Lambda^{n}V = \{a \in V^{\otimes n} \mid \sigma(\omega)(a) = a \, \forall a \in A_n\} \neq V^{\otimes n}$ in general.

	We define
	\begin{align*}
		v_1\cdots v_n &= \frac{1}{n!} \sum_{\omega \in S_n} v_{\omega(1)} \otimes \cdots \otimes v_{\omega(n)} \in S^{n}V, \\
		v_1 \wedge \cdots \wedge v_n &= \frac{1}{n!} \sum_{\omega \in S_n} \varepsilon(\omega) v_{\omega(1)} \otimes \cdots v_{\omega(n)} \in \Lambda^{n}V.
	\end{align*}
\end{definition}

We can show that if $v_1, \ldots, v_d$ is a basis for $V$, then
\begin{align*}
	\{v_{i_1} \cdots v_{i_n} \mid 1 \leq i_1 \leq i_2 \leq \cdots \leq i_n \leq d\} \text{ is a basis for } &S^{n}V, \\
	\{v_{i_1} \wedge \cdots \wedge v_{i_n} \mid 1 \leq i_1 < i_2 < \cdots < i_n \leq d\} \text{ is a basis for } & \Lambda^{n}V.
\end{align*}
Hence given $g \in G$ and $V$ unitary we can compute the character values of $g$ on $S^{n}V$ and $\Lambda^{n}V$ in terms of the eigenvalues of $V$.

For any vector space $V$, we have $\Lambda^{n}V = 0$ if $n > d$. Moreover, if $(\rho, V)$ is a representation of $G$, then $\lambda^{\dim V}V = \det \rho$.

\begin{definition}
	Suppose $\chr(k) = 0$. Given a vector space $V$, we may define the \textbf{tensor algebra}\index{tensor algebra} of $V$ as
	\[
	TV = \bigoplus_{n \geq 0} V^{\otimes n}
	.\]
	Then this is a non-commutative graded ring with product characterised by
	\[
		(v_1 \otimes \cdots \otimes v_r) \times (w_1 \otimes \cdots \otimes w_s) = v_1 \otimes \cdots \otimes v_r \otimes w_1 \otimes \cdots \otimes w_s
	.\]
	Thus the ring is graded by the degree of the product. There is a graded quotient given by the \textbf{symmetric algebra}\index{symmetric algebra} of $V$,
	\[
		SV =  TV/(x \otimes y - y \otimes x \mid x, y \in V)
	,\]
	and the \textbf{exterior algebra}\index{exterior algebra} of $V$,
	\[
		\Lambda V = TV/(x \otimes y + y \otimes x \mid x, y \in V)
	.\]
	One can show
	\[
	SV \cong  \bigoplus_{n \geq 0} S^{n}V, \quad \Lambda V \cong \bigoplus_{n \geq 0} \Lambda^{n}V
	.\]
	Then $SV$ is a commutative ring and $\Lambda V$ is graded-commutative, that is if $x \in \Lambda^{r}V$ and $y \in \Lambda^{s}V$, then $x \wedge y = (-1)^{rs}y \wedge x$.
\end{definition}

\subsection{Duality}%
\label{sub:duality}

\begin{definition}
	If $G$ is a group and $(\rho, V)$ is a representation of $G$, then the dual representation\index{dual representation} $(\rho^{\ast}, V^{\ast})$ is given by
	\[
		\rho^{\ast}(g) (\theta)(v) = \theta(\rho(g^{-1})v)
	.\]
\end{definition}

\begin{lemma}
	$\chi_{V^{\ast}}(g) = \chi_{V}(g^{-1})$.
\end{lemma}

\begin{adjustbox}{minipage = \columnwidth - 25.5pt, margin=1em, frame=1pt, margin=0em}
	\textbf{Proof:} If $\rho(g)$ is represented by $A$ with respect to a basis $v_1, \ldots, v_d$ of $V$ and $\varepsilon_1, \ldots, \varepsilon_d$ is the dual basis, then
	\[
		\rho(g)^{-1}v_i = \sum (A^{-1})_{ji} v_j
	.\]
	Thus
	\[
		\rho^{\ast}(g) \varepsilon_k(v_i) = \varepsilon_k \Biggl( \sum_{j} (A^{-1})_{ji} v_j \Biggr) = (A^{-1})_{ki}
	,\]
	and so we get
	\[
		\rho^{\ast}(g) \varepsilon_k = \sum_{i}(A^{-1})^{T}_{ij} \varepsilon_i
	.\]
	This gives the result.
\end{adjustbox}

Note that there is an involution $\ast$ of $\mathcal{C}_G$ given by $f^{\ast}(g) = f(g^{-1})$. We have proven that this induces an involution on $R(G)$ since $\chi_V^{\ast} - \chi_{V^{\ast}}$.

\begin{definition}
	We say $V$ is \textbf{self-dual}\index{self-dual} if $V \cong V^{\ast}$ as representations of $G$.
\end{definition}

When $G$ is finite and $k = \mathbb{C}$, then $\chi_{V^{\ast}} = \chi_{V}^{\ast} = \overline{\chi_V}$ if and only if $\chi_V(g) \in \mathbb{R}$ for all $g \in G$.

\begin{adjustbox}{minipage = \columnwidth - 25.5pt, margin=1em, frame=1pt, margin=0em}
\begin{example}
	\begin{enumerate}[1.]
		\item[]
		\item If $G = C_3 = \langle x \rangle$ and $V = \mathbb{C}$, then if $\rho(x) = \omega$, then $\rho^{\ast}(x) = w^2$ and $V$ is not self-dual.
		\item If $G = S_n$, then since $g$ is always conjugate to its inverse in $S_n$, $\chi^{\ast} = \chi$ always and so every representation is self-dual.
		\item Permutation representations $\mathbb{C}X$ are always self-dual as $\chi_{\mathbb{C}X}(g) \in \mathbb{N}$.
	\end{enumerate}
\end{example}
\end{adjustbox}

\newpage

\section{Induction}%
\label{sec:induction}

\subsection{Construction}%
\label{sub:construction}


Suppose $H \leq G$ is a subgroup. The restriction makes representations of $G$ into representations of $H$. We would like a way to use representations of $H$ to build representations of $G$.

Given a group $G$, we will write $[g]_{G}$ for the conjugacy class of $g \in G$. So $\mathbf{1}_{[g]_{G}} : G \to k$ given by
\[
	\mathbf{1}_{[g]_{G}}(x) =
	\begin{cases}
		1 & x \text{ conjugate to } g \in G, \\
		0 & \text{otherwise}.
	\end{cases}
\] 
We note that for $g \in G$, $[g]_{G}^{-1} = [g^{-1}]_{G}$, since $(xgx^{-1})^{-1} = xg^{-1}x^{-1}$, and so $(\mathbf{1}_{[g]_{G}})^{ast} = \mathbf{1}_{[g^{-1}]_{G}}$.

If $H \leq G$, then $[g]_{G} \cap H$ is a union of $H$-conjugacy classes:
\[
	[g]_{G} \cap n H = \bigcup_{[h]_H \subset [g]_{G}}[h]_{H}
,\]
and so $r : \mathcal{C}_G \to \mathcal{C}_H$ by $f \mapsto f|_H$ is a well-defined linear maps, with
\[
	r^{\ast}(\mathbf{1}_{[g]_{G}}) = \sum_{[h]_H \subset [g]_{G}} \mathbf{1}_{[h]_H}
.\]
Since for every finite group, $\langle f_1, f_2 \rangle_{G} = \frac{1}{|G|}\sum_{g \in G}f_1^{\ast}(g) f_2(g)$ defines a non-degenerate bilinear form on $\mathcal{C}_G$, the map $r$ has an adjoint $r^{\ast}$ characterised by
\[
	\langle r(f_1), f_2 \rangle_{H} = \langle f_1, r^{\ast}(f_2) \rangle_{G}
,\]
for $f_1 \in \mathcal{C}, f_2 \in \mathcal{H}$. In particular, if $f \in \mathcal{C}_H$,
\[
	\langle \mathbf{1}_{[g^{-1}]_{G}}, r^{\ast}(f) \rangle_{G} = \langle r (\mathbf{1}_{[g^{-1}]_{G}}), f \rangle_{H} = \frac{1}{|H|} \sum_{[h]_{H} \subset [g]_{G}} |[h]_{H}| f(h)
.\]
On the other hand, we have
\[
	\langle \mathbf{1}_{[g^{-1}]_{G}}, r^{\ast}(f) \rangle_{G} = \frac{1}{|G|} \sum_{x \in [g]_{G}}r^{\ast}(f)(x) = \frac{|[g]_{G}|}{|G|} r^{\ast}(f)(g)
.\]
Thus, by comparing these, we get
\[
	r^{\ast}(f)(g) = \sum_{[h]_H \subset [g]_{G}} \frac{|C_G(g)|}{|C_H(h)|}f(h) = \frac{1}{|H|}\sum_{x \in G}f^{\circ}(x^{-1}gx)
,\]
where $f^{\circ}(g) = f(g)\mathbf{1}_{[g]_G}$.

Suppose that $\chi$ is a $\mathbb{C}$-character of $H$, and $\psi$ is an irreducible $\mathbb{C}$-character of $G$. Then
\[
	\langle \psi, r^{\ast}(\chi)\rangle_{G} = \langle r(\psi), \chi \rangle_{H} \in \mathbb{N}
,\]
by the orthogonality of characters, since $r(\psi)$ is a character of $H$. Writing $\Irr(G)$ for the set of irreducible characters of $G$, we get
\[
	r^{\ast}(\chi) = \sum_{\psi \in \Irr(G)} \langle \psi_H, \chi]_{H} \psi
\]
is a character in $R(G)$. This formula is only useful to computer $r^{\ast}(\chi)$ if we understand $\Irr(G)$. Since we usually want to use $\Irr(H)$ to understand $\Irr(G)$, the first formula will be more useful.

\begin{adjustbox}{minipage = \columnwidth - 25.5pt, margin=1em, frame=1pt, margin=0em}
\begin{example}
	If $G = S_3$ and $H = A_3 = \{1, (123), (132)\}$, then if $f \in \mathcal{C}_H$,
	\begin{align*}
		r^{\ast}(f)(e) &= 2f(e), \\
		r^{\ast}(f)((12)) &= 0, \\
		r^{\ast}(f)((123)) &= f((123)) + f((132)).
	\end{align*}
	Thus, $r^{\ast}(\chi_1) = \mathbf{1} + \varepsilon$ and $r^{\ast}(\chi_2) = r^{\ast}(\chi_3)$ is the 2-dimensional irreducible representation. Note that if $\chi \in \Irr(H)$, then $r^{\ast}(H)$ may be irreducible, but not necessarily. Note that $r^{\ast}(\chi)(e) = \frac{|G|}{|H|} \chi(e)$.
\end{example}
\end{adjustbox}

We would like to build a representation of $G$ with character $r^{\ast}(\chi)$, given a representation $W$ of $H$ with character $\chi$. Suppose $G$ is a finite group and $W$ is a vector space
\[
	\Hom(G, W) = \{f : G \to W\}
,\]
a vector space under pointwise operations. Then $\Hom(G, W)$ is a representation of $G$ via
\[
	(g \cdot f)(x) = f(g^{-1}x)
.\]
Then if $W \cong \bigoplus_{i = 1}^{\dim W} k$, then
 \[
	 \Hom(G, W) \cong \bigoplus_{i = 1}^{\dim W}\Hom(G, k) \cong (\dim W) k G
.\]

\begin{definition}
	If $H$ is a subgroup of $G$, and $W$ is a representation of $H$, then
	\[
		\Hom_H(G, W) = \{f \in \Hom(G, W) \mid f(xh) = h^{-1}f(x)\}
	\]
	is a $k$-linear subspace of $\Hom(G, W)$.
\end{definition}

\begin{adjustbox}{minipage = \columnwidth - 25.5pt, margin=1em, frame=1pt, margin=0em}
\begin{example}
	If $W = \mathbf{1}$ is the trivial representation, and $F \in \Hom(G, \mathbf{1})$, then
	\begin{align*}
		f \in \Hom_H(G, \mathbf{1}) &\iff f(xh) = f(h) \iff f \text{ constant on left cosets}.
	\end{align*}
	Thus $\Hom_H(G, \mathbf{1}) \cong k G/H$.
\end{example}
\end{adjustbox}

\begin{lemma}
	$\Hom_H(G, W)$ is a $G$-invariant subspace of $\Hom(G, W)$ with dimension $|G/H| \dim W$.
\end{lemma}

\begin{adjustbox}{minipage = \columnwidth - 25.5pt, margin=1em, frame=1pt, margin=0em}
	\textbf{Proof:} Let $f \in \Hom_H(G, W)$, $g, x \in G$ and $h \in H$. We show that $(g \cdot f)(xh) = h^{-1}(g \cdot f)(x)$. This follows from
	\[
		(g \cdot f)(xh) = f(g^{-1}xh) = h^{-1}f(g^{-1}x) = h^{-1}(g \cdot f)(x)
	.\]
	Moreover, if $x_1, \ldots, x_{|G/H|}$ are coset representations of $G/H$, then $f \in \Hom_H(G, W)$ is determined by its values $f(x_1), \ldots, f(x_{|G/H|})$, and these values may be chosen freely. So $\Hom_H(G, W) \cong \Hom(G/H, W)$ as vector spaces, and the latter has dimension $|G/H| \dim W$.
\end{adjustbox}

\begin{definition}
	If $H \leq G$ is a subgroup and $W$ is a representation of $H$, then
	\[
		\Ind_{H}^{G}W = \Hom_H(G, W)
	\]
	is the induced representation.\index{induced representation} Let $\Res_H^{G}(V)$ denote the restriction of a representation $V$ of $G$ to $H$.\index{restriction}
\end{definition}

\begin{theorem}[Frobenius Reciprocity]
	Let $V$ be a representation of $G$ and $W$ a representation of $H$. Then
	\[
		\Hom_G(V, \Ind_H^{G}(W)) \cong \Hom_H(\Res_{H}^{G}(V), W)
	.\]
\end{theorem}

\begin{corollary}
	$\langle \chi_V, \Ind_{H}^{G}(W) \rangle = \langle \chi_V|H, \chi_W \rangle_H$. In particular, $\chi_{\Ind_H^{G}(\chi_W)} = r^{\ast}(\chi_W)$.
\end{corollary}



\newpage

\printindex

\end{document}

