\documentclass[12pt]{article}

\usepackage{ishn}

\makeindex[intoc]

\begin{document}

\hypersetup{pageanchor=false}
\begin{titlepage}
	\begin{center}
		\vspace*{1em}
		\Huge
		\textbf{II Principles of Quantum Mechanics}

		\vspace{1em}
		\large
		Ishan Nath, Michaelmas 2023

		\vspace{1.5em}

		\Large

		Based on Lectures by Prof. Enrico Pajer

		\vspace{1em}

		\large
		\today
	\end{center}
	
\end{titlepage}
\hypersetup{pageanchor=true}
\setcounter{tocdepth}{2}

\tableofcontents

\newpage

%lecture 1

\setcounter{section}{-1}

\section{Postulates of Quantum Mechanics}
\label{sec:postulates}

The following are the key postulates we take for quantum mechanics:
\begin{enumerate}[label = Postulate \Roman*.]
	\item The state of the system is a vector in a Hilbert space. Moreover the probabilities for events to happen are obtained from vector products.
	\item Observables are Hermitian linear operators on $\mathcal{H}$.
	\item Time evolution is governed by
		\[
		i \hbar \partial_t \psi = H\psi,
		\]
		the time-dependent Schr\"{o}dinger equation.
\end{enumerate}

\newpage

\section{Hilbert Spaces}
\label{sec:hilbert_spaces}

Quantum mechanics lives in a Hilbert space\index{Hilbert space}. This is a vector space $\mathcal{H}$ over $\mathbb{C}$, equipped with an inner product $(\cdot, \cdot) : \mathcal{H} \times \mathcal{H} \to \mathbb{C}$. The inner product satisfies:
\begin{itemize}
	\item $(\phi, \psi) = (\psi, \phi)^{\ast}$,
	\item $(\phi, a\psi) = a (\phi, \psi)$ (but $(a \phi, \psi) = a^{\ast} (\phi, \psi)$),
	\item $(\phi, \phi) \geq 0$ with equality if and only if $\phi = 0$.
\end{itemize}

This inner product induces a norm $\|\psi\| = \sqrt{(\psi, \psi)}$, which satisfies the Cauchy-Schwarz inequality
\[
|(\psi, \phi)| \le \|\phi\|\|\psi\|.
\]
Moreover we assume that $\mathcal{H}$ is complete, i.e. every Cauchy sequence converges inside $\mathcal{H}$.

\begin{exbox}[Examples of Hilbert Spaces]
	Finite dimensional $\mathcal{H}$ are simple: $\mathcal{H} \cong \mathbb{C}^{n}$ for some $n \in \mathbb{N}$, with
	\[
	v, u \in \mathcal{H} \implies(v, u) = \sum_{i=1}^{n} v_i^{\ast} u_i.
	\]
	In quantum mechanics these arise in various circumstances; either as a toy model, an approximation, or a subspace.

	However most of the time $\dim \mathcal{H} = \infty$. In quantum mechanics, these Hilbert spaces are all isomorphic to $\ell^2$, which is the space of all square-summable infinite sequences of complex numbers. Here the inner product is
	\[
		(v, u) = \sum_{i=1}^{\infty} v_i^{\ast} u_i < \infty,
	\]
	by Cauchy-Schwarz.

	For example, for a particle on a circle, the wavefunction is
	\[
	\psi(\theta) = \sum_n a_n e^{i n \theta},
	\]
	with
	\[
	\|\psi\|^2 = \int_0^{2\pi} |\phi(\theta)|^2\diff\theta = 2\pi\sum_{n=1}^{\infty}|a_n|^2.
	\]
	For the one-dimensional quantum harmonic oscillator, the wavefunctions are
	\[
	\psi(x)=\sum_{n=1}^{n} a_n H_n(x) e^{-\alpha x^2},
	\]
	where $H$ are the Hermite polynomials. Again, $\|\phi\|^2=\sum |a_n|^2$.

	Another example is $L^2(\mathbb{R},\diff x)$, i.e. the functions $f : \mathbb{R} \to \mathbb{C}$ such that
	\[
		\int |f(x)|^2\diff x < \infty.
	\]
	Then the inner product is
	\[
		(\phi,\psi) = \int \phi(x)^{\ast}\psi(x) \diff x < \infty.
	\]
	$\psi$ not need to be smooth, or even differentiable or continuous.
\end{exbox}

A Hilbert space $\mathcal{H}$ is \emph{separable}\index{separable} if it has a dense and countable subspace.

All separable Hilbert spaces are isomorphic to each other. All Hilbert spaces in quantum mechanics are separable, including $\ell^2$ and $L^2$. Hence $\ell^2$ is isomorphic to $L^2$.

\subsection{Dual Spaces}
\label{sub:dual_spaces}

The dual $\mathcal{H}^{\ast}$ of a Hilbert space $\mathcal{H}$ is the space of linear maps $f : \mathcal{H} \to \mathbb{C}$. Under various assumptions, $\mathcal{H}^{\ast}$ is isomorphic to $\mathcal{H}$.

\begin{itemize}
	\item For $\dim \mathcal{H} < \infty$, this is true by taking a dual basis. If we define $f_\phi = (\phi, \cdot)$ as a linear map, then we can show each $f \in \mathcal{H}^{\ast}$ is of this type. Here $\mathcal{H}^{\ast}$ inherits an inner product from $\mathcal{H}$: $\|f_\phi\|_{\mathcal{H}^{\ast}} = \|\phi\|_{\mathcal{H}}$.
	\item For $\dim \mathcal{H} = \infty$, this is not true in general. But it is true for $\mathcal{H}^{\ast}_{\text{continuous}}$, the space of continuous linear maps (Riesz representation theorem).
\end{itemize}

\subsection{Dirac Notation}
\label{sub:dirac_not}

Here is a useful piece of notation by Dirac:
\begin{itemize}
	\item if $\psi \in \mathcal{H}$, we place it inside a `ket': $\ket{\psi}$.
	\item if $\chi \in \mathcal{H}^{\ast}$, we place it inside a `bra': $\bra{\chi}$.
\end{itemize}
Hence the action of $\bra{\chi}$ on $\ket{\psi}$ is written $\braket{\chi|\psi} \in \mathbb{C}$, and is called a bra-ket.

You may think of this as follows: given basis $\ket{n}$ for  $n = 1, \ldots$
\begin{itemize}
	 \item we have vectors $\ket{v}$,
	 \item we have linear maps $\bra{u}$,
	 \item and we have scalars $\braket{u|v}$.
\end{itemize}

%lecture 2

\subsection{Continuum States}
\label{sub:continuum}

In this course, either $\mathcal{H} \cong \mathbb{C}^{m}$ or $\mathcal{H} \cong \ell^2$. In these spaces we can expand in an orthonormal basis $\{\ket{\phi_n}\} \in \mathcal{H}$, where $\braket{\phi_n|\phi_m} = \delta_{m,n}$.

For any $\ket{\psi} \in \mathcal{H}$,
\[
	\ket{\psi} = \sum_{n} a_n \ket{\phi_n}.
\]
We can generalise this to \emph{continuum states}. For example, in $L^2(\mathbb{R}, \diff x)$ and for all $x \in \mathbb{R}$ we have a continuum state\index{continuum state} $\ket{x}$, normalized such that
\[
	\braket{x|x'} = \delta(x - x').
\]
Expanding $\ket{\psi} \in L^2(\mathbb{R}, \diff x)$ as
\[
	\ket{\psi} = \int_ \mathbb{R} \psi(x')\ket{x'} \diff x',
\]
we get that
\[
	\braket{x|\psi} = \int \braket{x |\psi(x')| x'}\diff x' = \int \psi(x') \braket{x|x'} \diff x' = \psi(x).
\]
The normalization condition ensures that
\[
	\braket{\phi|\psi} = \int \diff x' \bra{x'}\phi(x')^{\ast} \int \diff x\, \psi(x) \ket{x} = \int_ \mathbb{R} \phi(x)^{\ast} \psi(x) \diff x.
\]
Note that $\braket{x|x} = \delta(x - x)$, which is undefined. Hence continuum states are not normalizable in $L^2(\mathbb{R}, \diff x)$. We can formally define them in further courses, see II Analysis of Functions.

\subsection{Different Bases}
\label{sub:different_bases}

The Dirac notation frees us from having to pick a basis. For example, we could use the momentum basis $\{\ket{p}\}$,
\[
	\ket{\psi} = \int \psi(x) \ket{x} \diff x = \int \tilde \psi(p) \ket{p} \diff p.
\]
These are related. We will later show that
\[
	\braket{x|p} = e^{ixp/\hbar} \frac{1}{\sqrt{2\pi \hbar}} = \braket{p|x}^{\ast},
\]
and hence
\[
	\psi(x) = \braket{x|\psi} = \int \braket{x|\tilde \psi(p)|p} \diff p = \int \tilde \psi(p) \frac{e^{ipx/\hbar}}{\sqrt{2 \pi \hbar}} \diff p.
\]
So $\psi(x)$ and $\tilde \psi(p)$ are related by a Fourier transform.

\subsection{Tensor Products}
\label{sub:tensor_products}

More complicated systems require larger Hilbert spaces. We can build them by taking tensor products.

Let $\dim(\mathcal{H}_1), \dim(\mathcal{H}_2) < \infty$, and let $\{\ket{e_a}\} \in \mathcal{H}_1$, $\{\ket{f_\alpha}\} \in \mathcal{H}_2$ be orthonormal bases. Then,
\[
	\mathcal{H}_1 \otimes \mathcal{H}_2 \ni \ket{\psi} = \sum_{a,\alpha} c_{a\alpha}\ket{e_a} \otimes \ket{f_\alpha}.
\]
The inner product is defined on basis elements:
\[
	(\bra{e_a} \otimes \bra{f_\alpha}) (\ket{e_b} \otimes \ket{f_\beta}) = \braket{e_a|e_b} \braket{f_\alpha|f_\beta},
\]
then we can extend to $\mathcal{H}_1 \otimes \mathcal{H}_2$ by linearity.

\begin{exbox}
	If $\{\ket{x}\}$ and $\{\ket{y}\}$ are continuum states of a particle moving in the $x$ and $y$ directions, then a particle in $\mathbb{R}^2$ is
	\[
		\ket{\psi} = \int_{\mathbb{R} \times \mathbb{R}} \psi(x, y) \ket{x} \otimes \ket{y} \diff x \diff y.
	\]
	Note that $\psi(x, y) \neq \psi_1(x)\psi_2(y)$. Again the inner product is
	\begin{align*}
	\braket{\phi|\psi} &= \int \phi^{\ast}(x', y') \psi(x, y) (\bra{x'} \otimes \bra{y'})(\ket x \otimes \ket y) \diff x \diff x' \diff y \diff y' \\
			   &= \int_{\mathbb{R}^2} \phi^{\ast}(x, y) \psi(x, y) \diff x \diff y.
	\end{align*}
\end{exbox}

A shorthand notation is
\[
	\ket \psi = \int \psi(\mathbf{x}) \ket{\mathbf{x}} \Diff3 \mathbf{x},
\]
where $\psi(\mathbf{x}) = \psi(x_1, x_2, x_3)$, and $\ket{\mathbf{x}} = \ket{x, y, z} = \ket x \otimes \ket y \otimes \ket z$.

\subsection{Quantum Systems with Internal Structure}
\label{sub:internal_structure}

Take a simple model: a hydrogen atom consisting of an electron $\elec$ and a proton $\prot$, so
\[
	\ket \psi = \int \psi (\mathbf{x}_{\elec} \mathbf{x}_{\prot}) \ket{\mathbf{x}_{\elec}} \otimes \ket{\mathbf{x}_{\prot}} \Diff3 \mathbf{x}_{\elec} \Diff3 \mathbf{x}_{\prot} \in L^2(\mathbb{R}^{6}, \Diff3\mathbf{x}_{\elec}, \Diff3\mathbf{x}_{\prot}).
\]
We can also introduce $\mathbf{x}_{\mathrm{rel}} = \mathbf{x}_{\elec} - \mathbf{x}_{\prot}$ and
\[
\mathbf{x}_{\mathrm{CM}} = \frac{m_{\elec} \mathbf{x}_{\elec} + m_{\prot} \mathbf{x}_{\prot} }{m_{\elec} + m_{\prot}},
\]
and instead we can write
\[
	\ket \psi = \int \psi'(\mathbf{x}_{\mathrm{CM}}, \mathbf{x}_{\mathrm{rel}}) \ket{\mathbf{x}_{\mathrm{CM}}} \otimes \ket{\mathbf{x}_{\mathrm{rel}}} \Diff3 \mathbf{x}_{\mathrm{CM}} \Diff3 \mathbf{x}_{\mathrm{rel}}.
\]
This is convenient, for example $\psi' = \approx e^{i \mathbf{k} \mathbf{x}_{\mathrm{CM}}} \psi_{\mathrm{rel}}(\mathbf{x}_{\mathrm{rel}})$.

For $N$-particles,
\[
L^2(\mathbb{R}^{3N},\Diff3 x_1, \ldots, \Diff3 x_n) = \bigotimes_{a = 1}^{N} L^2(\mathbb{R}^3, \Diff3 x_a).
\]
Spin means electrons are actually described by a pair of wavefunctions:
\[
\ket \psi \in L^2(\mathbb{R}^3, \Diff3 x) \otimes \mathbb{C}^2 \implies \psi(\mathbf{x}) =
\begin{pmatrix}
	\psi_{\uparrow}(\mathbf{x})\\ \psi_{\downarrow}(\mathbf{x})
\end{pmatrix}.
\]

\newpage

\section{Linear Operators}
\label{sec:linear_operators}

The main thing to note is that observables are represented by Hermitian linear operators.

For $\dim \mathcal{H} = n < \infty$, a linear operator $A$ is an $n \times n$ matrix $A : \mathcal{H} \to \mathcal{H}$.

The linear operators form an associative, but not commutative, algebra over the field $\mathbb{C}$ (called the Universal Covering Algebra, or simply the Operator Algebra). The linear sum of two operators is simply
\[
	(\alpha A + \beta B) \ket \psi = \alpha A \ket \psi + \beta B \ket \psi,
\]
for all $\alpha, \beta \in \mathbb{C}$ and $\ket \psi \in \mathcal{H}$. We can also define the product under composition:
\[
AB \ket \psi = A( B \ket \psi).
\]
We define a \emph{commutator}\index{commutator} of $A$ and $B$ as $[A,B] = AB - BA$. The commutator satisfies:
\begin{itemize}
	\item Antisymmetric: $[A, B] = - [B, A]$.
	\item Linear: $[\alpha A + \beta B, C] = \alpha[A, C] + \beta[B, C]$.
	\item Leibniz rule: $[A,BC] = B[A,C] + [A,B]C$.
	\item Jacobi identities: $[A,[B,C]] + [B,[C,A]] + [C,[A,B]] = 0$.
\end{itemize}

%lecture 3

The inner product allows us to define the \emph{adjoint} $A^{\dagger}$ of $A$, by\index{adjoint}\index{Hermitian conjugate}
\[
	\braket{\phi|A^{\dagger}|\psi} = \braket{\psi|A|\phi}^{\ast},
\]
for all $\ket{\phi}$ and $\ket{\psi} \in \mathcal{H}$. It obeys:
\begin{itemize}
	\item $(A+B)^{\dagger} = A^{\dagger} + B^{\dagger}$.
	\item $(AB)^{\dagger} = B^{\dagger} A^{\dagger}$.
	\item $(A^{\dagger})^{\dagger} = A$.
\end{itemize}
An operator satisfying $A^{\dagger} = A$ is called \emph{Hermitian}\index{Hermitian} or \emph{self-adjoint}.

\subsection{Eigenstates and Eigenvalues}
\label{sub:eigen}

A state $\ket{\psi} \in \mathcal{H}$ is an \emph{eigenstate}\index{eigenstate} of $A$ if $A \ket \psi = a \psi$ for $a \in \mathbb{C}$, where $a \in \mathbb{C}$ is the \emph{eigenvalue}\index{eigenvalue}.
\begin{itemize}
	\item Often we label the states by their eigenvalues, for example we would write
		\[
		A \ket a = a \ket a.
		\]
	\item If a linear operator $Q$ is Hermitian and $Q \ket q = q \ket q$, then $q \in \mathbb{R}$. Indeed,
		\[
			q \braket{q|q} = \braket{q|Q|q} = \braket{q|Q|q}^{\ast} = q^{\ast} \braket{q|q}.
		\]
	\item If $q_1 \neq q_2$, then $\braket{q_1|q_2} = 0$. Indeed, 
		\begin{align*}
			0 &= \braket{q_2|Q-Q^{\dagger}|q_1} = \braket{q_1|Q|q_1} - \braket{q_2|Q^{\dagger}|q_1} \\
			  &= q_1 \braket{q_2|q_1} - q_2 \braket{q_1|q_2}^{\ast} \\
			  &= (q_1 - q_2) \braket{q_2|q_1},
		\end{align*}
		hence $\braket{q_2|q_1} = 0$.
	\item Let $\{\ket n\}$ form an orthonormal basis of eigenvalues, where $Q \ket n = q_n \ket n$. Then we can easily compute $Q \ket \psi$ for all $\ket \psi \in \mathcal{H}$: Writing
		\[
		\ket \psi = \sum_{n} c_n \ket n,
		\]
		we get
		\[
		Q \ket \psi = Q \Biggl( \sum_n c_n \ket n \Biggr) = \sum_n c_n q_n \ket n.
		\]
	\item Notice that we can write
		\[
		\mathbbm{1}_{\mathcal{H}} = \sum_n \ket n \bra n.
		\]
	\item We can also define $f(Q)$ by
		\[
		f(Q) = \sum_n f(q_n) \ket n \bra n,
		\]
		which exists as long as $f(q_n)$ exists for all $n$.
	\item The matrix elements of $A$ in the basis is
		\[
			A_{kn} = \braket{k|A|n}.
		\]
		Then
		\[
			(AB)_{km} = \braket{k|AB|n} = \sum_n \braket{k|A|n} \braket{n|B|m} = \sum_n A_{kn} B_{nm}.
		\]
		So composition is matrix multiplication. Moreover,
		\[
			(A^{\dagger})_{mn} = \braket{m|A^{\dagger}|n} = \braket{n|A|m}^{\ast} = (A_{nm})^{\ast}.
		\]
\end{itemize}

\subsection{Infinite Dimensions}
\label{sub:inf_dim}

Let's mention (and ignore) some issues of $\dim \mathcal{H} = \infty$,such as $L^2$.
\begin{itemize}
	\item An operator $A$ is \emph{bounded}\index{bounded operator} if there exists $M \in \mathbb{R}$ such that $\|A \ket \psi\| < M \|\ket \psi\|$, for all $\ket \psi \in \mathcal{H}$.
	\item If $\dim \mathcal{H} < \infty$, then all operators are bounded. This is not true if $\dim \mathcal{H} = \infty$.
	\item If $A$ is unbounded, then it is only defined on a subset of $\mathcal{H}$, called the \emph{domain}\index{domain} of $A$. Here $\Dom(A) \subseteq \mathcal{H}$.
\end{itemize}

The position $X$ and momentum $P$ operators are unbounded. For example, we could have $\psi(x) \in L^2$, but not $x \cdot \psi (x)$ or $-i \partial_x \psi \hbar$.
\begin{itemize}
	\item Continuum states hide this computation:
		\begin{align*}
			\braket{x|P|\psi} &= \int \frac{\diff p}{\sqrt{2 \pi \hbar}} \braket{x|p} \braket{p|P|\psi} = \int e^{ipx/\hbar} p \tilde \psi(p) \frac{\diff p}{\sqrt{2 \pi \hbar}} \\
					  &= -i \hbar \frac{\partial}{\partial x} \Biggl[ \int e^{ipx/\hbar} \tilde \psi(p) \frac{\diff p}{\sqrt{2 \pi \hbar}} \Biggr] = -i \hbar \partial_x \braket{x|\psi} = -i \partial_x \hbar \psi(x).
		\end{align*}
\end{itemize}

\subsection{Operators for Composite Systems}
\label{sub:comp_ops}

Let $\{\ket{e_a}\}$ be a basis of $\mathcal{H}_1$ and $A : \mathcal{H}_1 \to \mathcal{H}_1$, and $\{\ket{f_\alpha}\}$ be a basis of $\mathcal{H}_2$ and $B:\mathcal{H}_2 \to \mathcal{H}_2$. Then we can define
\[
A \otimes B : \mathcal{H}_1 \otimes \mathcal{H}_2 \to \mathcal{H}_1 \otimes \mathcal{H}_2
\]
by
\[
	(A \otimes B)(\ket{e_a} \otimes \ket{f_\alpha}) = (A\ket{e_a}) \otimes (B\ket{f_\alpha}).
\]
This can be extended linearly to the whole $\mathcal{H} = \mathcal{H}_1 \otimes \mathcal{H}_2$. Note that
\[
	[A \otimes \mathbbm{1}_{\mathcal{H}_1}, \mathbbm{1}_{\mathcal{H}_1} \otimes B] = 0
\]
for all $A$ and $B$.

\begin{exbox}
	The hydrogen atom is described by
	\[
	H = \frac{\mathbf{p}_\prot^2}{2m_\prot} \otimes \mathbbm{1}_{\elec} + \mathbbm{1}_{\prot} \otimes \frac{\mathbf{p}_\elec^2}{2m_\elec} - \frac{q^2}{4 \pi \eps_0} \frac{1}{|\mathbf{X}_\elec - \mathbf{X}_\prot|}.
	\]
	We can also rewrite this as
	\[
		H = \frac{\mathbf{p}_{\mathrm{CM}}^2}{2M} \otimes \mathbbm{1}_{\mathrm{rel}} + \mathbbm{1}_{\mathrm{CM}} \otimes \biggl[ \frac{\mathbf{p}_{\mathrm{rel}}^2}{2\mu} - \frac{q^2}{4 \pi \eps_0} \frac{1}{|\mathbf{X}_{\mathrm{rel}}} \biggr],
	\]
	where $\mathbf{p}_{\mathrm{CM}} = \mathbf{p}_\prot + \mathbf{p}_\elec$,
	\[
	\mathbf{p}_{\mathrm{rel}} = \frac{m_\prot \mathbf{p}_\elec - m_\elec \mathbf{p}_\prot}{M},
	\]
	$M = m_\elec + m_\prot$ and $\mu = \frac{m_\elec m_\prot}{M}$ is the reduced mass.
\end{exbox}

\subsection{Postulates of Quantum Mechanics}
\label{sub:postulates_again}

Recall the postulates of quantum mechanics:
\begin{itemize}
	\item The state of a system is specified by $\ket \psi \in \mathcal{H}$, with $\ket \psi \neq 0$.
	\item Any complete system of orthogonal eigenstates $\{\ket {\phi_1}, \ket {\phi_2}, \ldots\}$ is in one-to-one correspondence with possible outcomes of measurements.
	\item The probability to observe the outcomes corresponding to $\ket {\phi_n}$ in state $\ket \psi$ is
		\[
			\mathbb{P}(\ket \psi \to \ket {\phi_n}) = \frac{|\braket{\phi_n|\psi}|^2}{\braket{\phi_n|\phi_n}\braket{\psi|\psi}}.
		\]
\end{itemize}

This is the probabilistic interpretation of quantum mechanics, given by Born.

%lecture 4

Let's look at these postulates in more detail.

Postulate I says that
\[
	\mathbb{P}(\ket \psi \to \ket{\phi_n}) = \frac{|\braket{\phi_n|\psi}|^2}{\braket{\phi_n|\phi_n}\braket{\psi|\psi}}.
\]

Note that physical states correspond to rays in $\mathcal{H}$ namely $\ket {\psi_1}$ and $\ket{\psi_2}$ belong to the same ray if and only if $\ket {\psi_1} = \lambda \ket{\psi_2}$, for $\lambda \in \mathbb{C}^{\ast}$.

Often we simply pick $\|\psi\| = 1$ on the representative. Note that probability begin positive required a positive norm for the states, or unitarity.

Postulate II says that observables are represented by linear operators in $\mathcal{H}$. The \emph{expectation value}\index{expectation value} $\braket Q_\psi$ of a property corresponding to $Q$ in state $\ket \psi$ is
\[
	\braket Q_\psi = \frac{\braket{\psi|Q|\psi}}{\braket{\psi|\psi}}.
\]
Note that $Q$ acting on $\ket \psi$ has nothing to do with the process of measuring.

We have two options for how to build our framework of quantum mechanics:
\begin{enumerate}
	\item We can start from $\mathcal{H}$, and define observables as operators.
	\item We start from an algebra of observables, and define/build $\mathcal{H}$ as a representation (see Von Neumann, and the Gelfand-Naimark-Segal construction).
\end{enumerate}

Postulate III says that the dynamical evolution of a system if governed by
\[
i \hbar \frac{\partial}{\partial t} \ket \psi = H \ket \psi.
\]
This is the time-dependent Schr\"odinger equation. The operator $H$ is called the \emph{Hamiltonian}\index{Hamiltonian}.

The form of $H$ depends on the system. Note that $\mathcal{H}$ does not involve time.

\subsection{The Generalized Uncertainty Principle}
\label{sub:gen_unc_pri}

We define the root mean square deviation or \emph{uncertainty}\index{uncertainty} $\Delta_\psi(Q)$ of $Q = Q^{\dagger}$ in state $\ket \psi$ as
\[
	\Delta_\psi Q = \sqrt{\braket{Q^2}_\psi - \braket Q_\psi^2}.
\]
Since $Q$ is Hermitian,
\[
\braket{Q^2}_\psi - \braket Q_\psi^2 = \braket{\psi|(Q-\braket Q_\psi)^2|\psi} = \|(Q - \braket Q_\psi)\ket \psi\|^2 \geq 0,
\]
so $\Delta_\psi Q$ is well defined, and vanishes only if $Q \ket \psi = q \ket \psi$ for some $q$.

The \emph{uncertainty principle}\index{uncertainty principle} for Hermitian operators $A, B$ is as follows:

Let $\ket{\psi_A} = A \ket \psi - \braket A_\psi \ket \psi$, and the same for $B$. Then, $\|\ket{\psi_A}\| = \Delta_\psi A$, and $\|\ket{\psi_B}\| = \Delta_\psi B$. We get:
\[
	2 i \,\mathrm{Im} \braket{\psi_A|\psi_B} = \braket{\psi|AB|\psi}  - \braket{\psi|AB|\psi}^{\ast} = \braket{\psi|[A,B]|\psi}.
\]
By Cauchy-Schwarz, we get
\[
	\Delta_\psi A \Delta_\psi B = \|\ket{\psi_A}\|\|\ket{\psi_B}\| \geq |\braket{\psi_A|\psi_B}| \geq \frac{1}{2}|\braket{[A,B]}_\psi|.
\]

\newpage

\section{The Quantum Mechanic Oscillator}
\label{sec:qmo}

Take the Hamiltonian for the one-dimensional quantum mechanic oscillator:
\[
H = \frac{P^2}{2m} + \frac{1}{2} m\omega^2 X^2.
\]
Introducing the \emph{raising} and \emph{lowering} operators\index{raising operator}\index{lowering operator}
\[
	A = \frac{1}{\sqrt{2 m \hbar \omega}}(m \omega X + iP), \qquad A^{\dagger} = \frac{1}{\sqrt{2m \hbar \omega}} (m\omega X - iP).
\]
These are convenient because
\begin{align*}
	A^{\dagger}A &= \frac{1}{2m\hbar \omega}(m\omega X - iP)(m\omega X + iP) \\
		     &= \frac{1}{2m\hbar \omega} (P^2 + m^2 \omega^2 X^2 + m \omega i [X,P]) = \frac{1}{\hbar \omega}H - \frac{1}{2},
\end{align*}
as $[X,P] = i \hbar$. Thus
\[
H = \hbar \omega \biggl(A^{\dagger}A + \frac{1}{2} \biggr) = \hbar \omega \biggl(N + \frac{1}{2} \biggr),
\]
where $N = A^{\dagger}A$ is called the \emph{number operator}\index{number operator}, and $N^{\dagger} = N$.

The commutation relations become
\begin{align*}
	[A, A^{\dagger}] &= \frac{1}{2m\hbar \omega}([\omega m X + iP], \omega m X - iP]) \\
			 &= \frac{-i m \omega}{2 m \hbar \omega} ([X, P] - [P,X]) = 1,
\end{align*}
and $[A,A] = [A^{\dagger}, A^{\dagger}] = 0$. Consequently,
\[
	[N,A^{\dagger}] = [A^{\dagger}A,A^{\dagger}] = A^{\dagger}[A,A^{\dagger}] + [A^{\dagger},A^{\dagger}]A = A^{\dagger},
\]
and similarly we can get $[N,A] = -A$.

\subsection{Building the Hilbert Space}
\label{sub:build_H}

Let $\ket n$ be a normalized eigenstate of $N$, so $N \ket n = n \ket n$. Then
\[
	N(A^{\dagger}\ket n) = ([N,A^{\dagger}] + A^{\dagger}N)\ket n = (A^{\dagger} + A^{\dagger}n)\ket n = (1+n) A^{\dagger} \ket n.
\]
Similarly, $N(A \ket n) = (n-1) (A \ket n)$.

Provided that $A^{\dagger} \ket n$ and $A \ket n$ are non-zero, these are new eigenvectors of $N$.

Now lets put some restrictions on $n$. Note that
\[
	n = \braket{n|N|n} = \braket{n|A^{\dagger}A|n} = \|A\ket n\|^2 \geq 0.
\]
Since $A$ decreases the eigenvalue of an eigenvector, there exists $n_{\min}$ such that $A \ket{n_{\min}} = 0$. Then $n_{\min} = 0$. Hence $n \in \mathbb{N}_0$.

To normalize, we let $A^{\dagger}\ket n = c_n \ket{n+1}$ for some $c_n \in \mathbb{C}$. Then
\[
	|c_n|^2 = \|A^{\dagger} \ket n\| = \braket{n|AA^{\dagger}|n} = \braket{n|N+[A,A^{\dagger}]|n}=n+1,
\]
so we can choose $c_n = \sqrt{n+1}$.

In summary,
\begin{itemize}
	\item We have a ground state $\ket 0$, such that $A \ket 0 = 0$.
	\item There are normalized eigenvectors of $H$ (and $N$)
		\[
			\ket n = \frac{1}{\sqrt{n!}} (A^{\dagger})^{n} \ket 0.
		\]
	\item The span of these vectors is called \emph{Fock} (Hilbert) space\index{Fock space}.
\end{itemize}

%lecture 5

\subsection{Position Space Wavefunctions}
\label{sub:pos_wavefn}

We can derive the wavefunction of energy eigenstates as follows. Let $\psi_0(x) = \braket{x|0}$ be the wavefunction of the ground state. Then
\[
	0 = \braket{x|A|0} \propto \braket{x|m\omega X + i P|0} = m \omega x \psi_0(x) + \hbar \partial_x \psi_0(x).
\]
Solving this ODE, we find that
\[
\psi_0(x) = \biggl(\frac{m \omega}{\pi \hbar} \biggr)^{1/4} \exp \biggl( - \frac{m \omega}{2 \hbar} x^2 \biggr).
\]
This is normalized such that
\[
	\braket{0|0} = \int_{-\infty}^{\infty} |\psi_0(x)|^2 = 1.
\]
Then the higher energy states are obtained by differentiation:
\[
	\psi_1(x) = \braket{x|A^{\dagger}|0} = m \omega x \psi_0(x) - \hbar \partial_x \psi_0(x) \propto 3x \exp \biggl( - \frac{m \omega}{2\hbar} x^2 \biggr).
\]
Repeating this action generates the Hermite polynomials.

\subsection{Uncertainty}
\label{sub:uncertainty}

The uncertainty principle says that
\[
	\Delta_\psi X \Delta_\psi P \geq \frac{1}{2}|\braket{[X,P]}_\psi| = \frac{\hbar}{2},
\]
for all $\ket \psi \in \mathcal{H}$. Let's compute this on $\ket 0$: Recall that $(\Delta_\psi X)^2 = \braket{X^2}_\psi -  \braket X_\psi^2 = \braket{X^2}_\psi$. In terms of $A$ and $A^{\dagger}$, we have
\[
	X = \sqrt{\frac{\hbar}{2 m \omega}}(A + A^{\dagger}).
\] 
Note that for any $\ket n$, we have
\[
	\braket{n|X|n} \propto \braket{n|A+A^{\dagger}|n} = 0.
\]
However
\begin{align*}
	\frac{2 m\omega}{\hbar} \braket{0|X^2|0} &= \braket{0|(A+A^{\dagger})^2|0} = \braket{0|AA^{\dagger}|0} \\
						 &= \braket{0|[A,A^{\dagger}] + A^{\dagger}A|0} = 1.
\end{align*}
Similarly, $\braket{0|P|0} = 0$, and $\braket{0|P^2|0} = \frac{m\hbar \omega}{2}$. Combining this, we get
\[
	\Delta_0 X \Delta_0P = \sqrt{\braket{0|X^2|0}\braket{0|P^2|0}} = \frac{\hbar}{2}.
\]
Hence the ground state has minimal uncertainty. We will see later that there are other states with minimum uncertainty.

\newpage

\section{Transformations in Quantum Mechanics}
\label{sec:transformations}

A \emph{transformation}\index{transformation} is a map from configuration $R$ to $R$' that preserves probabilities:

In quantum mechanics, these transformations are represented by linear unitary operators:
\[
	\braket{U\phi|U\psi} = \braket{\phi|\psi},
\]
or $U^{\dagger}U = 1$. It could also be an anti-linear, anti-unitary operator. The proof of this is Wagner's theorem.

We let $U \ket \psi = \ket{\psi'}$. The fact that the transformation must be unitary comes from the conservation of probability, Note that the transformations form a group $G$, and for all $g_1, g_2 \in G$, $U(g_1) \cdot U(g_2) = U(g_1 g_2)$.

The identity operator is $U(e) = \mathbbm{1}_{\mathcal{H}}$.

\subsection{Transformations of Operators}
\label{sub:transform_ops}

Let $A : \mathcal{H} \to \mathcal{H}$ be an operator. Then under a transformation $U$, $\braket{\psi|A|\psi}$ becomes
\[
	\braket{\psi'|A|\psi'} = \braket{\psi|U^{\dagger}AU|\psi}.
\]
Hence we can equivalently transform operators instead of states, by letting
\[
A \to A' = U^{\dagger} A U = U^{-1} A U.
\]
Then $A'B' = (AB)'$, and $[A',B'] = [A,B]'$. Note that similarity preserves the spectrum of $A$. Indeed, if $A \ket a = a \ket a$, then $U^{\dagger} \ket a$ is an eigenvector of $A'$ with the same eigenvalue.

\subsection{Continuous Transformations}
\label{sub:cont_trans}

Let $U$ depend smoothly on $\theta \in \mathbb{R}$, with $U(0) = \mathbbm{1}_{\mathcal{H}}$. For $\theta \ll 1$, by Taylor expansion,
\[
U(\theta) = \mathbbm{1}_{\mathcal{H}} - i \theta T + \mathcal{O}(\theta^2).
\]
$T$ is called the \emph{generator}\index{generator} of the transformation. The $-i$ term is included by convention.

As $U$ is unitary, $T$ is Hermitian:
\[
1 = U^{\dagger}U = (1+i\theta T^{\dagger})(1 - i\theta T) + \mathcal{O}(\theta^2) = 1 + i \theta(T^{\dagger} - T) + \mathcal{O}(\theta^2).
\]
So $T^{\dagger} = T$. Therefore, generators can be observables.

The action of $U$ on states is given by
\[
	\ket{\psi'} = \ket \psi = (1-i\theta T)\ket\psi \implies \ket \psi \mapsto \ket \psi + \delta \ket \psi,
\]
where $\delta \ket \psi = -i \theta T \ket \psi$, and the action on operators is given by
\[
A' = U^{\dagger}AU = (1+i\theta T^{\dagger})A(1-i\theta T) = A + \delta A,
\]
where $\delta A = i \theta[T, A]$. For finite transformations,
\[
U = \lim_{N \to \infty} \biggl(1 - i \frac{\theta}{N} T \biggr)^{N} = e^{-i \theta T} = \sum_{n = 0}^{\infty} (-i \theta T)^{n} \frac{1}{n!}.
\]

%lecture 6

\subsection{Spatial Transformations}
\label{sub:space_trans}

We will look at translations in $\mathbb{R}^3$. Let the translation by $\mathbf{a} \in \mathbb{R}^3$ be represented by $U(\mathbf{a})$. Then as translations commute,
\[
U(\mathbf{a})U(\mathbf{b}) = U(\mathbf{a}+\mathbf{b})=U(\mathbf{b}+\mathbf{a})=U(\mathbf{b})U(\mathbf{a}),
\]
and moreover $U^{\dagger}(\mathbf{a})=U(-\mathbf{a})$. For infinitesimal terms $\delta \mathbf{a}$, we get
\[
U(\delta \mathbf{a}) = \mathbbm{1}_{\mathcal{H}} - \frac{i}{\hbar} \delta \mathbf{a} \cdot \mathbf{P} + \mathcal{O}(\delta \mathbf{a}^2),
\]
and so
\[
U(\mathbf{a}) = e^{-i \mathbf{a} \cdot \mathbf{P}}.
\]
Here the operator $\mathbf{P}/\hbar$ generates translations, and as $U(\delta \mathbf{a}), U(\delta \mathbf{b})$ commute, we get $[\mathbf{P}_i, \mathbf{P}_j] = 0$ for Cartesian coordinates $i = x, y, z$.

\subsection{Canonical Commutation Relations}
\label{sub:can_com_rel}

If $\ket{\psi'} = U(\mathbf{a})\ket \psi$, we must have
\[
	\braket{\mathbf{X}}_{\psi'} = \braket{\psi|U^{\dagger}(\mathbf{a})\mathbf{X}U(\mathbf{a})|\psi} = \braket{\mathbf{X}}_\psi + \mathbf{a}.
\]
This must be true for all $\ket \psi \in \mathcal{H}$, so
\[
U^{\dagger}(\mathbf{a})\mathbf{X} U(\mathbf{a}) = \mathbf{X} + \mathbf{a} \mathbbm{1}_{\mathcal{H}}.
\]
Infinitesimally, we get
\[
\biggl(1 + i \frac{\delta \mathbf{a}}{\hbar} \cdot \mathbf{P} \biggr) \mathbf{X} \biggl( 1 - \frac{i}{\hbar} \delta \mathbf{a} \cdot \mathbf{P} \biggr) + \mathcal{O}(\delta \mathbf{a}^2) = \mathbf{X} + \delta \mathbf{a}.
\]
For this to be true at $\mathcal{O}(\delta \mathbf{a})$, we need
\[
	[X_i, P_j] = i \hbar \delta_{ij} \mathbbm{1}_{\mathcal{H}}.
\]
Note the algebra of observables needs $\dim \mathcal{H} = \infty$, as in finite dimensions,
\[
\dim \mathcal{H} = \Tr (\mathbbm{1}_{\mathcal{H}}) = -\frac{i}{\hbar} \Tr(XP - PX) = 0.
\]
This argument fails for $\dim \mathcal{H} = \infty$, as the trace is not necessarily defined. This is why the quantum mechanics of a particle needs to be defined on functional spaces, such as $L^2(\mathbb{R}, \diff x)$.

\subsection{Translations on the Wavefunction}
\label{sub:trans_wave}

$\psi(\mathbf{x}) = \braket{\mathbf{x}|\psi}$ is the wavefunction in position space. Let $\ket{\mathbf{x}} \to \ket{\mathbf{x}'} = U(\mathbf{a})\ket{\mathbf{x}}$. Then
\begin{align*}
	\mathbf{X} \ket{\mathbf{x}'} &= \mathbf{X} U(\mathbf{a})\ket{\mathbf{x}} = ([\mathbf{X},U(\mathbf{a}] + U(\mathbf{a}) \mathbf{X})\ket{\mathbf{x}}\\
				     &= (\mathbf{a} + \mathbf{x})U(\mathbf{a})\ket{\mathbf{x}} = (\mathbf{a} +\mathbf{x}) \ket{\mathbf{x}'}.
\end{align*}
After a transformation $\ket \psi \to U(\mathbf{a})\ket \psi$,
\[
	\psi_{\mathrm{trasnform}}(\mathbf{x})=\braket{\mathbf{x}|U(\mathbf{a}|\psi}=\braket{\mathbf{x}-\mathbf{a}|\psi}=\psi(\mathbf{x}-\mathbf{a}).
\]
For infinitesimal translations,
\[
\psi(\mathbf{x} - \delta \mathbf{a}) - \psi(\mathbf{x}) = - \delta \mathbf{a} \cdot \nabla \psi(\mathbf{x}).
\]
Also we know
\[
	\braket{\mathbf{x} | 1 - \frac{i}{\hbar} \delta \mathbf{a} \cdot \mathbf{P} | \psi} - \braket{\mathbf{x}|\psi} = -\frac{i}{\hbar} \braket{\mathbf{x}|\delta \mathbf{a} \cdot \mathbf{P} | \psi}.
\]
Comparing these two expressions, we get that
\[
	\braket{\mathbf{x}|\mathbf{P}|\psi}= - i \hbar \nabla \psi (\mathbf{x}).
\]

\subsection{Examples}
\label{sub:com_ex}

Take an eigenstate $\ket{\mathbf{p}}$ obeying $\mathbf{P} \ket{\mathbf{p}} = \mathbf{p} \ket{\mathbf{p}}$. After a translation,
\begin{align*}
	\psi_{\mathbf{p}}(\mathbf{x}) = \braket{\mathbf{x}|\mathbf{p}} \to \psi_{\mathbf{p}}(\mathbf{x} - \mathbf{a}) &= \braket{\mathbf{x}|U(\mathbf{a})|\mathbf{p}} = \braket{\mathbf{x}|e^{-i\mathbf{a} \cdot \mathbf{P}/\hbar}|\mathbf{p}} \\
														      &= e^{-i\mathbf{a}\cdot \mathbf{p}/\hbar} \braket{\mathbf{x}|\mathbf{p}}.
\end{align*}
Hence we get $\psi_{\mathbf{p}}(\mathbf{x} - \mathbf{a}) = e^{-i \mathbf{a} \cdot \mathbf{p}/\hbar}\psi_{\mathbf{p}}(\mathbf{x})$, which is solved by
\[
\psi_{\mathbf{p}}(\mathbf{x}) = \frac{1}{(2 \pi \hbar)^{3/2}} e^{i \mathbf{p} \cdot \mathbf{x}/\hbar},
\]
where we normalise such that $\braket{\mathbf{p}|\mathbf{p}'} = \delta^3(\mathbf{p} - \mathbf{p}')$.

Now consider the ground state of a quantum mechanical oscillator in 1 dimension. We have
\[
	\psi_0 = \braket{x|0} = C e^{-x^2 \alpha^2/2},
\]
for some $C$ and $\alpha$. The translated state $U(a)\ket 0$ is called a \emph{coherent state}\index{coherent state}
\[
	\psi_{a}^{\mathrm{coherent}}(x) = \braket{x|U(a)|0} = C e^{-\alpha^2(x-a)^2/2}.
\]
This has the same variance as $\ket 0$, but is centred at $x = a$. This is not an energy eigenstate; it will oscillate in time.

\subsection{\texorpdfstring{Rotations in $\mathbb{R}^3$}{Rotations in R3}}
\label{sub:rotations}

An anti-clockwise rotation $|\bm{}{\alpha}|$ around the $\bm{\hat \alpha}$ axis corresponds to
\[
\mathbf{v} \mapsto \mathbf{v}' = R(\bm{\alpha})\mathbf{v}.
\]
Since $\mathbf{v}' \cdot \mathbf{v}' = \mathbf{v} \cdot \mathbf{v}$, we have $\det R = 1$, so this is in $\mathsf{SO}(3)$. This is not abelian:
\[
R(\bm{\alpha})R(\bm{\beta}) \neq R(\bm{\beta})R(\bm{\alpha}).
\]
For infinitesimal rotations,
\[
R(\delta \bm{a}) \mathbf{v} = \mathbf{v} + \delta \bm{\alpha} \times \mathbf{v} + \mathcal{O}(\delta \bm{a}^2).
\]
Composing small rotations,
\[
R(\delta \bm{\alpha})(R \delta \bm{\beta}) \mathbf{v} = \mathbf{v} + \delta \bm{\beta} \times \mathbf{v} + \delta \bm{\alpha}\times (\mathbf{v} + \delta \bm{\beta} \times \mathbf{v}) + \mathcal{O}(\delta \bm{\alpha}^2, \delta \bm{\beta}^2).
\]
Hence we see that
\[
	[R(\delta \bm{\alpha}), R(\delta \bm{\beta})] \mathbf{v} = \delta \bm{\alpha} \times(\delta \bm{\beta} \times \mathbf{v}) - \delta \bm{\beta} \times (\delta \bm{\alpha} \times \mathbf{v}) = (\delta \bm{\alpha} \times \delta \bm{\beta}) \times \mathbf{v}.
\]
This is $R(\delta \bm{\alpha} \times \delta \bm{\beta}) - \mathbbm{1}_{\mathbb{R}^3}$ infinitesimally.

\subsection{Rotation Operator}
\label{sub:rot_op}

Let $U(\bm{\alpha})$ denote rotations on $\mathcal{H}$. Hence,
\[
U(\bm{\alpha}) = \mathbbm{1}_{\mathcal{H}} - i \frac{\bm{\alpha}}{\hbar} \mathbf{J} + \mathcal{O}(\bm{\alpha}^2) \implies U(\bm{\alpha}) = e^{-i \bm{\alpha} \cdot \mathbf{J} /\hbar}.
\]
The commutation relation on $\mathbf{v}$ implies
\[
	[U(\bm{\alpha}), U(\bm{\beta})] = U(\bm{\alpha} \times \bm{\beta}) - \mathbbm{1}_{\mathcal{H}}.
\]
In terms of the generator,
\[
	[J_i, J_j] = i \hbar \eps_{ijk} J_k.
\]
This is $\mathsf{SO}(3)$ algebra.
%lecture 7
For $U(\bm{\alpha})$ to be a rotation operator, we must have:
\[
	\mathbf{X}' = U^{\dagger}(\bm{\alpha})\mathbf{X} U(\bm{\alpha}) = R(\bm{\alpha})\mathbf{X}.
\]
Here $U$ is a scalar operator on a Hilbert space, and $R$ is the usual matrix of rotations in $\mathbb{R}^3$. For an infinitesimal rotation, this says:
\[
	\biggl(1 + \frac{i}{\hbar} \delta \bm{\alpha} \cdot \mathbf{J} + \cdots \biggr) \mathbf{X} \biggl(1 - \frac{i}{\hbar} \delta \bm{\alpha} \cdot \mathbf{J} + \cdots \biggr) = \mathbf{X} + \delta \bm{\alpha} \times \mathbf{X} + \cdots,
\]
so $[J_i, X_j] = i\hbar \eps_{ijk} X_k$.

A \emph{vector operator}\index{vector operator} $\mathbf{V}$ is any operator such that $U^{\dagger}(\bm{\alpha})\mathbf{V} U(\bm{\alpha}) = R(\bm{\alpha})\mathbf{V}$, or equivalently
\[
	[J_i, V_j] = i \hbar \eps_{ijk} V_k.
\]
A \emph{scalar operator}\index{scalar operator} $S$ transforms trivially under rotations: $U^{\dagger}(\bm{a})S U(\bm{\alpha}) = S$, or equivalently
\[
	[J_i, S] = 0.
\]
If $\mathbf{V}, \mathbf{W}$ are vector operators
\begin{align*}
	U^{\dagger}(\bm{\alpha})(\mathbf{V} \cdot \mathbf{W}) U(\bm{\alpha}) &= (U^{\dagger}(\bm{\alpha})\mathbf{V} U(\bm{\alpha})) \cdot (U^{\dagger}(\bm{\alpha}) \mathbf{V} U(\bm{\alpha})) \\
									     &= (R(\bm{\alpha}) \mathbf{V}) \cdot (R(\bm{\alpha}) \mathbf{W}) = \mathbf{V} \cdot \mathbf{W},
\end{align*}
so $\mathbf{V} \cdot \mathbf{W}$ is a scalar.

\subsection{Translations around a Circle}
\label{sub:trans_circle}

For a point particle, we can build up a rotation by successive translations along edges of an $N \gg 1$ sided regular polygon in the plane with a unit normal $\mathbf{n}$.

Each translation is through $\delta \mathbf{a} = \frac{2\pi}{N} \mathbf{n} \times \mathbf{X}$. Using the translation operator,
\[
U(\delta \mathbf{a}) = 1 - \frac{i}{\hbar} \frac{2\pi}{N} (\mathbf{n} \times \mathbf{X}) \cdot \mathbf{P} + \cdots = 1 - \frac{i}{\hbar} \frac{2\pi}{N} \mathbf{n} \cdot \mathbf{L},
\]
where $\mathbf{L} = \mathbf{X} \times \mathbf{P}$. As $N \to \infty$, these give translations around a circular path:
\[
	[L_i, L_j] = i \hbar \eps_{ijk} L_k, \qquad [L_i, X_j] = i\hbar \eps_{ijk} X_k, \qquad [L_i, P_j] = i\hbar \eps_{ijk} P_k.
\]
So if $\mathcal{H} = L^2(\mathbb{R}^3, \diff^3 x)$ then $\mathbf{L}$ and $\mathbf{J}$ act exactly the same way.

\subsection{Rotating Composite Systems}
\label{sub:rot_comp}

When $\mathcal{H}=L^2(\mathbb{R}^3, \diff^3 x) \otimes \mathcal{H}_{\mathrm{internal}}$, then $\mathbf{J} \neq \mathbf{L}$. For a composite system, the composition of translations is not necessarily equal to a rotation about the same axis.

Hence we can define the \emph{spin operator}\index{spin operator} $\mathbf{S} = \mathbf{J} - \mathbf{L}$.

Using the conjugation relations of $\mathbf{J}$ and $\mathbf{L}$, we see that
\[
	[S_i, X_j] = 0 = [S_i, P_j].
\]
This shows that $\mathbf{S}$ has nothing to do with the wavefunction. Moreover, we can calculate
\[
	[S_i, S_j] = ih \eps_{ijk} S_k.
\]
This means that $\mathbf{S}$ generates some form of rotations, as some $\mathsf{SO}(3)$ algebra.

\subsection{Parity}
\label{sub:parity}

Not every transformation has a Hermitian generator. The parity $P$ transformation $\mathbf{x} \mapsto -\mathbf{x}$ represents $\mathbb{Z}_2$. There is no `small parameter' here, instead parity is represented by a unitary operator $\Pi : \mathcal{H} \to \mathcal{H}$, obeying $\Pi^2 = \id_{\mathcal{H}}$. Thus the eigenvalues of $\Pi$ are $\pm1$.

Position transforms under parity as
\[
\Pi^{\dagger} \mathbf{X} \Pi = \Pi \mathbf{X} \Pi = - \mathbf{X},
\]
or equivalently $\{\Pi, \mathbf{X}\} = \Pi \mathbf{X} + \mathbf{X} \Pi = 0$.

However, we have $\Pi^{\dagger} \mathbf{L} \Pi = \mathbf{L}$ and $\Pi^{\dagger} \mathbf{J} \Pi = \mathbf{J}$. We say $\mathbf{L}$ and $\mathbf{J}$ are \emph{pesudo-vector operators}\index{pseudo-vector operators}.

Just as for rotations, $\Pi$ can act subtly on composite systems where $\mathcal{H} = L^2(\mathbb{R}^3, \Diff3x) \otimes \mathcal{H}_{\mathrm{internal}}$.

Physical particles can have intrinsic parity: if $\ket \psi \in \mathcal{H}$ is a parity eigenstate, then
\[
	\braket{\mathbf{x}|\Pi|\psi} = \eta\braket{-\mathbf{x}|\psi} = \eta \psi(-\mathbf{x}),
\]
where $\eta^2 = 1$.

%lecture 8

\newpage

\section{Time Evolution}
\label{sec:time_evols}

The Schr\"odinger equation determines time evolution:
\[
	i \hbar \frac{\partial \ket{\psi(t)}}{\partial t} = H \ket{\psi(t)},
\]
where $H = H^{\dagger}$ is the Hamiltonian. Note that $H$ generates time translations, assuming $\partial_t H = 0$:
\[
	\ket{\psi(t)} = U(t) \ket{\psi(0)},
\]
where $U(t) = e^{-iHt/\hbar}$. This is called the Schr\"odinger picture.

\subsection{Heisenberg Picture}
\label{sub:walter_pic}

Alternatively, we can think of operators as evolving in time, while the states do not.

The expected value of $Q$ at time $t$ is
\[
	\braket{\psi(t)|Q|\psi(t)} = \braket{\psi(0)|U^{\dagger}(t)Q U(t)|\psi(0)} = \braket{\psi(0)|Q_H(t)|\psi(0)}.
\]
The Heisenberg picture says that
\[
Q_H(t) = U^{\dagger}(t) Q_S U(t),
\]
where $U(t) = e^{-iHt/\hbar}$ and $Q_s = Q = Q_H(t = 0)$. Moreover we have $\ket \psi = \ket{\psi(0)}$ for all of time.

Differentiating, we obtain
\begin{align*}
	\frac{\diff}{\diff t}Q_H(t) &= \frac{i}{\hbar}U^{\dagger}(t)[H, Q_S] U(t) + U^{\dagger}(t) \frac{\partial Q_S}{\partial t} U(t) \\
				    &= \frac{i}{\hbar} [H,Q_H] + \frac{\partial Q}{\partial t}(t) \biggr|_H.
\end{align*}
This is called the \emph{Heisenberg equation}\index{Heisenberg equation}.

\subsection{Dynamics}
\label{sub:dynamics}

The Heisenberg equation (and the Schr\"odinger equation) is dependent on the Hamiltonian $H$. For motion in $\mathbb{R}^3$, we should choose $H = H(\mathbf{X}, \mathbf{P}, \ldots)$.

The simplest non-trivial Hamiltonian is $H = \mathbf{P}^2/2m$, which is rotation invariant. To model obstacles, we can add a potential $V(\mathbf{X})$. Then,
\[
H = \frac{\mathbf{P}^2}{2m} + V(\mathbf{X}).
\]
In the Heisenberg picture, the operators transform as
\[
\frac{\diff \mathbf{X}(t)}{\diff t} = \frac{\mathbf{P}(t)}{m}, \qquad \frac{\diff \mathbf{P}(t)}{\diff t} = - \nabla V(t),
\]
where
\[
\nabla V(t) = U^{\dagger}(t) \nabla V(\mathbf{X}) U(t).
\]
The translation generator $\mathbf{P}$ can be interpreted as classical momentum.

\subsection{Example: QHO}
\label{sub:ex_qmo}

Consider a translation by $x_0$ of the ground state of the quantum mechanics oscillator:
\[
	\ket{0,x_0} = e^{-ix_0 P/\hbar}\ket 0.
\]
In the Heisenberg picture, we get
\[
P(t) = U^{\dagger}(t) P U(t) = P \cos \omega t - X m \omega \sin \omega t.
\]
Evaluating for time $t$ gives a new state
\begin{align*}
	U(t) \ket{0;x_0} &= U(t) e^{-i x_0 P/\hbar} U^{\dagger}(t) U(t) \ket 0 \\
			 &= e^{-i x_0 P(-t)/\hbar} e^{-i\omega t/2} \ket 0 \\
			 &= e^{-i(m\omega x_0 X \sin \omega t + x_0 P \cos \omega t)/\hbar} e^{-i \omega t/2} \ket 0.
\end{align*}
Using the commutation of exponentials, we can get
\[
	U(t) \ket{0, x_0} = e^{-i(m\omega x_0/\hbar)X \sin \omega t} e^{-i(x_0/\hbar) P \cos \omega t)} e^{i \phi(t)} \ket 0,
\]
This is a Gaussian centered at $x(t) = x_0 \cos \omega(t)$ with momentum $m \dot x(t) = - m \omega x_0 \sin \omega t$. This behaves as a lump of motion released at time $t$ from rest.

\subsection{Conserved Quantities}
\label{sub:conserved_qs}

If $Q_H$ is time independent, then we say it is \emph{conserved}\index{conserved}. Suppose $\partial_t Q_S = 0$. Then in the Heisenburg picture,
\[
	\frac{\diff Q_H}{\diff t} = \frac{i}{\hbar} [H, Q_H] = U^{\dagger}(t) [H,Q] U(t).
\]
Hence conserved quantities commute with the Hamiltonian. Also, consider a conserved quantity $Q$ with eigenstates $Q \ket q = q \ket q$ at $t = 0$.

At time $t$, we have $\ket{q,t=0}$ becomes $\ket{q, t} = U(t) \ket{q, 0}$, and
\[
	Q U(t) \ket{q, 0} = U(t) Q\ket{q,0} = q \ket{q, t},
\]
so these are still $Q$ eigenstates. Hence it is useful to choose these as bases.

Note that if the eigenspace has dimension more than 1, then it may be the case that the state will change.

\subsection{Symmetries}
\label{sub:symms}

Consider a transformation $U(\theta) = e^{-i\theta T}$, so that
\[
H \to U^{\dagger}(\theta) H U(\theta)
\]
is the transformed Hamiltonian.

\begin{definition}
	A transformation is a \emph{symmetry}\index{symmetry} if $H$ remains unchanged, so 
	\[
		U^{\dagger}(\theta) H U(\theta) = H \iff [T,H] = 0.
	\]
\end{definition}

Hence symmetries of $H$ correspond to conserved quantities $T$.

%lecture 9

\newpage

\section{Angular Momentum}
\label{sec:ang_mom}

The generator $\mathbf{J}$ obeys the $\mathsf{SO}(3)$ algebra:
\[
	[J_i, J_k] = i\hbar \eps_{ijk}J_k.
\]
We can find a finite dimensional representation of this algebra. First notice
\[
	0 = \Tr_{\mathcal{H}}([J_i, J_j]) = i\hbar \eps_{ijk} \Tr_{\mathcal{H}}(J_k),
\]
so $J_i$ is represented by traceless matrices for $i = 1, 2, 3$. We can build $\mathcal{H}$, by first choosing how to label our states.

We can choose one, say $J_z$. However, then $J_{x}$ and $J_y$ do not commute with it, so we cannot find a simultaneous eigenbasis. However, note
\[
	[J_i, \mathbf{J} \cdot \mathbf{J}] = [J_i, J_j]J_j + J_j[J_i, J_j] = i\hbar (\eps_{ijk} J_k J_j + \eps_{ijk} J_k J_k) = 0.
\]
So we may label the eigenstates as $\ket{\beta, m}$ where
\[
	\mathbf{J}^2\ket{\beta,m} = \beta \hbar^2\ket{\beta, m}, \qquad J_z\ket{\beta, m} = m \hbar \ket{\beta, m},
\]
and also $\braket{\beta', m'|\beta,m} = \delta_{\beta\beta'}\delta_{mm'}$.

\subsection{Lowering and Raising}
\label{sub:low_raise}

We proceed as we did with the quantum mechanic oscillator. Define a new raising and lowering operator by
\[
J_{\pm} = J_x \pm i J_y.
\]
Their commutators are
\[
	[J_z, J_{\pm}] = [J_z, J_x] \pm i [J_z, J_y] = i\hbar(J_y \mp iJ_x) = \pm \hbar J_{\pm}.
\]
We learn that
\begin{align*}
	\mathbf{J}^2 (J_{\pm}\ket{\beta, m}) &= ([\mathbf{J}^2, J_{\pm}] + J_{\pm} \mathbf{J}^2) \ket{\beta, m} = \beta \hbar^2 J_{\pm} \ket{\beta, m}, \\
	J_z(J_{\pm}\ket{\beta, m}) &= ([J_z,J_{\pm}] + J_{\pm} J_z)\ket{\beta,m} = (m \pm 1)\hbar J_{\pm} \ket{\beta, m}.
\end{align*}

If $J_{\pm}\ket{\beta, m}$ is non-vanishing, then these are eigenvectors of $\mathbf{J}^2$ with eigenvalue $\beta$, and of $J_z$ with eigenvalues $m \pm 1$. To fix $\beta$ and $m$, we examine the norm:

\begin{align*}
	\|J_{+}\ket{\beta,m}\|^2 &= \ket{\beta,m|J_-J_+|\beta,m} = \ket{\beta,m|(J_x-iJ_y)(J_x+iJ_y)|\beta,m}\\
				 &=\ket{\beta,m|J_x^2+J_y^2 + i[J_x,J_y]|\beta,m} = \ket{\beta,m|\mathbf{J}^2-J_z^2-\hbar J_z|\beta,m}\\
				 &=\hbar^2(\beta-m(m+1))\geq 0.
\end{align*}
As $J_+$ increases $m$ and preserved $\beta$, we must have $\beta = j(j+1)$ for $j$ the maximum value of $m$.

States with $m = j$ are called the highest weight states in representation theory. Similarly, for $J_-$,
\[
	\|J_-\ket{\beta,m}\|^2 = \hbar^2(\beta - m(m-1)) \geq 0.
\]
So we also need $\beta = j'(j' - 1)$ for $j'$ the minimum of $m$. Comparing $j'(j'-1) = j(j+1)$, we either have $j = -j'$ or $j' = j+1$. The latter is impossible, so it must be the former.

Also $j - j' = 2j \in \mathbb{N}$, because $J_{\pm}$ is changing $m$ by one. Summarising,
\[
\beta = j(j+1), \qquad 2j \in \mathbb{N}_0.
\]
We relabel states by $\ket{\beta,m} \to \ket{j,m}$.

From $-j$ to $j$ we take $2j$ steps, so $\dim \mathcal{H} = 2j+1 < \infty$. Hence
\[
	j = \{0, 1/2, 1, 3/2, 2, \ldots\}, \qquad m = \{-j, -j+1, \ldots, j-1, j\}.
\]
Then the operators act as $J_z\ket{j,m} = \hbar m\ket{j,m}$ and $\mathbf{J}^2\ket{j,m} = \hbar^2 j(j+1)\ket{j,m}$. We also get
\begin{align*}
	J_{+}\ket{j,m} &= \sqrt{j(j+1)-m(m+1)} \hbar \ket{j,m+1}, \\
	J_-\ket{j,m} &= \sqrt{j(j+1)-m(m-1)}\hbar \ket{j,m-1},
\end{align*}
so that $\braket{j,m|j,m}=1$. Since $J_x = (J_+ + J_-)/2$ and $J_y = (J_+ + J_-)/2i$, we can see the action of all rotations $J_i$.

These operators stay inside $\mathcal{H}_j$, which is the Hilbert space for some $j$.

\begin{remark}
	\begin{itemize}
		\item[]
		\item We could have also worked with $J_x$ or $J_y$.
		\item The eigenvectors of $J_z$ are never eigenvectors of $J_x$ or $J_y$.
		\item $\ket{j,j}$ have maximal angular momentum along $\hat z$.
		\item Since $\braket{j,j|J_{x,y}|j,j} \neq 0$, we have some uncertainty
			\[
				\braket{j,j|J_x^2+J_y^2|j,j} = \braket{j,j|\mathbf{J}^2-J_z^2|j,j} = \hbar^2(j(j+1)-j^2) = \hbar^2j \neq 0.
			\]
			Compare this to $\braket{j,j|J_z^2|j,j} = \hbar^2 j^2$. Thus,
			\[
				\frac{\braket{j,j|J_x^2+J_y^2|j,j}}{\braket{j,j|J_z^2|j,j}} = \frac{1}{j}.
			\]
			This is negligible for macroscopic systems with $j \gg 1$, but not for $j$ in quantum mechanics.
	\end{itemize}	
\end{remark}

\subsection{Classical Limit}
\label{sub:classical_lim}

Consider measuring $\mathbf{J}$ along $\mathbf{x} = (\sin\theta, 0, \cos\theta)$. Classically, we would find this to be $\hbar j \cos \theta$, and in quantum mechanics we find the same thing on average:
\begin{align*}
	\braket{j,j|\mathbf{n} \cdot \mathbf{J} | j,j} &= \braket{j,j|\cos\theta J_z + \sin\theta J_x|j,j} \\
						       &= \cos \theta \hbar j.
\end{align*}
We can also compute
\begin{align*}
	\braket{j,j|(\mathbf{n} \cdot \mathbf{J})^2|j,j} &= \cos^2\theta \braket{J_z^2} + \sin^2\theta \braket{J_x^2} + \cos \theta \sin \theta \braket{J_xJ_z+J_zJ_x} \\
							 &= \cos^2\theta \hbar^2 j^2 + \sin^2\theta \frac{1}{4} \braket{j,j|J_{+}J_{-}|j,j} \\
							 &= \hbar^2(j^2 \cos^2\theta + j \sin^2\theta/2),
\end{align*}
hence the functional uncertainty is
\[
	\frac{\sqrt{\Delta_{\ket{j,j}}(\mathbf{n} \cdot \mathbf{J})}}{\braket{\mathbf{n} \cdot \mathbf{J}}} \propto \frac{1}{\sqrt j},
\]
which goes to $0$ as $j \gg 1$.

\begin{exbox}
	Consider a carbon-monoxide atom. The Hamiltonian can be modelled as (with $I = I_x = I_y \neq I_z$):
	\[
	H = \frac{J_x^2}{2I_x} + \frac{J_y^2}{2I_y} + \frac{J_z^2}{2I_z} = \frac{\mathbf{J}^2}{2I} + J_z^2\biggl( \frac{1}{2I_z} - \frac{1}{2I} \biggr).
	\]
	Hence $\ket{j,m}$ are energy eigenstates with
	\begin{align*}
		&H\ket{j,m} = E_{j,m}\ket{j,m} \\
		\implies& E_{j,m} = \frac{j(j+1)\hbar^2}{2I} + m^2\hbar^2 \biggl(\frac{1}{2I_z} - \frac{1}{2I} \biggr)
	\end{align*}
	For carbon monoxide, $I_z \ll I$, hence only $m = 0$ is accessible at low energies.

	Then the energy difference is
	\[
	|E_j - E_{j-1}| = \frac{j\hbar^2}{I},
	\]
	with corresponding frequency
	\[
		\nu_j = \frac{2 \pi \hbar j}{I} \approx j \cdot \qty{113}{\giga\hertz}.
	\]
	This is used in astronomy to observe gas.
\end{exbox}

\subsection{Integers and Half Integers}
\label{sub:ints_half_ints}

We can place further restrictions on $\mathcal{H}$. Suppose $\ket \psi \in \mathcal{H}_j$, and
\[
	\ket \psi = \sum_{m = -j}^{j} a_m \ket{j,m},
\]
then
\[
	U(\alpha \hat z) \ket \psi = \sum_{m = -j}^{j} a_m e^{-i\alpha J_z/\hbar} \ket{j,m} = \sum_{m = -j}^{j} a_m e^{-i \alpha m} \ket{j,m}.
\]
When $j \in \mathbb{N}_0$, we get that $U(2 \pi \hat z) = \mathbbm{1}_{\mathcal{H}_j}$, as expected for $\mathsf{SO}(3)$.

However when $j \in \mathbb{N}_0 + \frac{1}{2}$, we find that $U(2\pi \hat z) = -\mathbbm{1}_{\mathcal{H}_j}$.

Here half-integer values of $j$ are allowed because $\lambda \mathbbm{1}_{\mathcal{H}}$ acts trivially on projective $\mathcal{H}$. Mathematically, for $j = N + \frac{1}{2}$, these are not a representation of $\mathsf{SO}(3)$, but it is a projective representation.

\subsection{Spinors}
\label{sub:spinors}

Recall the spin operator, $\mathbf{J} = \mathbf{L} + \mathbf{S}$. Then $\mathbf{S}$ is the spin operator, and obeys the same algebra $[S_i, S_j] = i \hbar \eps_{ijk} S_k$ as $\mathbf{J}$.

Hence $\mathbf{S}$ has some representation. Our notation will be that representations of spin $\mathbf{S}$ will have eigenstates $\ket{s, \sigma}$, where
\begin{align*}
	\mathbf{S}^2 \ket{s, \sigma} &= s(s+1)\hbar^2 \ket{s, \sigma}, \\
	S_z\ket{s, \sigma} &= \sigma \hbar \ket{s, \sigma},
\end{align*}
with $\sigma \in \{-s, \ldots, s\}$.

Spin is an intrinsic property of a particle, like the charge. So a particle of spin $s$ lives in
\[
\mathcal{H} = L^2(\mathbb{R}^3) \otimes \mathbb{C}^{2s+1}.
\]

If $s = 0$, then we have the scalar configuration. Here there is only one state $\ket{s = 0, \sigma = 0}$ and
\[
	e^{-i\alpha \mathbf{S}/\hbar} \ket{0,0} = \ket{0,0}
\]
is invariant under rotation. Examples of scalars include sound, and the Higgs boson.

For $s = 1/2$, we have many examples, such as electrons, quarks, leptons, neutron, protons and many more. Here we have two linearly independent states. We focus on $\mathcal{H}_{1/2}$.

We say that $\ket{1/2, 1/2} = \ket{\uparrow}$, and $\ket{1/2,-1/2} = \ket{\downarrow}$.

Any state can be written as $\ket \psi = a \ket{\uparrow} + b \ket{\downarrow}$, with $|a|^2 + |b^2| = 1$ for $\|\psi\| = 1$.

In this basis,
\[
S_z =
\begin{pmatrix}
	\braket{\uparrow|S_z|\uparrow}& \braket{\uparrow|S_z|\downarrow} \\
	\braket{\downarrow|S_z|\uparrow} & \braket{\downarrow|S_z|\downarrow}
\end{pmatrix}
=\frac{\hbar}{2}
\begin{pmatrix}
	1&0\\0&-1
\end{pmatrix}
=\frac{\hbar}{2}\sigma_z.
\]
Using $S_x = \frac{1}{2} (S_+ + S_-)$, we find that
\[
S_x = \frac{\hbar}{2}
\begin{pmatrix}
	0&1\\1&0
\end{pmatrix}
=\frac{\hbar}{2}\sigma_x, \qquad S_y = \frac{\hbar}{2}
\begin{pmatrix}
	0&-i\\i&0
\end{pmatrix}
=\frac{\hbar}{2}\sigma_y.
\]
We write $\mathbf{S} = \frac{\hbar}{2} \bm{\sigma}$, where $\bm{\sigma} = (\sigma_x, \sigma_y, \sigma_z)$ are the Pauli matrices.

\subsection{Stern-Gerlach Experiment}
\label{sub:stern_gerlach}

This was an experiment that showed that the spins of particles are quantized, and for the silver atom found that $s = 1/2$.

Schematically, silver atoms are passed through a slowly varying magnetic field, and hit a detector. The Hamiltonian for this is
\[
H = \frac{\mathbf{P}^2}{2M} - \bm{\mu} \cdot \mathbf{B},
\]
where we have the magnetic dipole moment
\[
\bm{\mu} = \frac{\mu \mathbf{S}}{\hbar s}.
\]

Classically, a magnetic dipole $\bm{\mu}= \gamma \mathbf{S}$ in a magnetic field $\mathbf{B}$ experiences a torque:
\[
\mathbf{\dot S} = \bm{\mu} \times \mathbf{B}.
\]
If $\bm{\mu}$ and $\mathbf{B}$ are constant, then $\mathbf{S}$ precesses around $\mathbf{B}$ with angular velocity
\[
\bm{\omega} = -\gamma \mathbf{B}.
\]
Consider a proton $\prot$ with
\[
	\gamma = g_{\prot} \frac{e}{2 m_{\prot} c}, \qquad g_{\prot} = \qty{5.5}{\gram}.
\]
If $\prot$ cannot move, then the Hamiltonian is
\[
H = - \gamma \mathbf{S} \cdot \mathbf{B},
\]
with $\mathbf{B} = B \hat z$. Suppose initial state is
\[
	\ket{\mathbf{n} \uparrow} = e^{-i \phi/2} \cos \frac{\theta}{2} \ket{\uparrow} + e^{+i \phi/2} \sin \frac{\theta}{2} \ket{\downarrow},
\]
with the spin maximally aligned along
\[
\mathbf{n} = (\cos \phi \sin \theta, \sin \phi \sin \theta, \cos \theta).
\]
Then we get that
\begin{align*}
	\ket{\mathbf{n} \uparrow, t} &= U(t) \ket{\mathbf{n} \uparrow} = e^{-i H t} \ket{\mathbf{n} \uparrow} \\
				     &= e^{i(\omega t - \phi)/2} \cos \frac{\theta}{2} \ket{\uparrow} + e^{-i(\omega t - \phi)/2}\sin \frac{\theta}{2} \ket{\downarrow},
\end{align*}
which is maximally aligned along $\phi(t) = \phi - \omega t$, where
\[
\omega = \gamma B.
\]
This still precesses (on average).

Spin 1 particles include the W$^{(\pm)}$ and $Z$ vector boson, which mediate electroweak interactions (such as nuclear fusion and radioactivity).

Here $\dim \mathcal{H}_s = 2s + 1 = 3$, so $\mathcal{H}_{s=1} = \mathbb{C}^3$. The basis of $\mathbf{S}^2$ and $S_z$ eigenvectors are
\[
	\ket{1,1} = \ket +, \qquad \ket{1, 0} = \ket 0, \qquad \ket{1, -1} = \ket -.
\]
In this basis, we get
\begin{align*}
	S_x &= \frac{\hbar}{\sqrt 2}
	\begin{pmatrix}
		0&1&0\\1&0&1\\0&1&0
	\end{pmatrix}
	,\\
	S_y&=\frac{\hbar}{\sqrt2}
	\begin{pmatrix}
		0&-i&0\\i&0&-i\\0&i&0
	\end{pmatrix}
	,\\
	S_z&=\hbar
	\begin{pmatrix}
		1&0&0\\0&0&0\\0&0&-1
	\end{pmatrix}.
\end{align*}
These are found using
\begin{align*}
	S_+ \ket 0 &= \sqrt 2\hbar \ket +, & S_- \ket 0 &= \sqrt 2 \hbar \ket -. \\
	S_+ \ket - &= \sqrt 2 \hbar \ket 0, & S_-\ket + &= \sqrt 2 \hbar \ket 0.
\end{align*}

\subsection{Orbital Angular Momentum}
\label{sub:orb_ang_mom}

Consider the orbital angular momentum $\mathbf{L} = \mathbf{X} \times \mathbf{P}$, for a single particle. Recall that $\mathbf{J} = \mathbf{L} + \mathbf{S}$. Assume that this particle has 0 spin (hence is scalar). Then $\ket \psi \in L^2(\mathbb{R}^3)$.

We want $U(\bm{\alpha}) = e^{-i \alpha \mathbf{L}/\hbar}$ to describe rotations. Then we need
\[
	U^{\dagger}(\bm{\alpha}) \mathbf{X} U(\bm{\alpha}) = R(\bm{\alpha})\mathbf{X} \implies U(\bm{\alpha})\ket{\mathbf{x}} = \ket{R(\bm{\alpha})\mathbf{x}}.
\]
Under rotations $\ket \psi \to \ket{\psi'} = U(\bm{\alpha}) \ket \psi$, we have wavefunctions
\[
	\braket{\mathbf{x}|\psi} = \psi(\mathbf{x}) \to \braket{\mathbf{x}|U(\bm{\alpha})|\psi} = \braket{R^{-1}(\bm{\alpha})\mathbf{x}|\psi}=\psi((R^{-1}(\bm{\alpha})\mathbf{x})).
\]
This representation is unitary:
\begin{align*}
	\braket{\phi'|\psi'} &= \int_{\mathbb{R}^3} \Diff3 \mathbf{x} \phi(R^{-1}(\bm{\alpha})\mathbf{x})^{\ast} \psi(R^{-1}(\bm{\alpha})\mathbf{x}) \\
			     &= \int_{\mathbb{R}^3} \Diff3 \mathbf{\tilde x} \phi(\mathbf{\tilde x})^{\ast} \psi(\mathbf{\tilde x}) = \braket{\phi, \psi},
\end{align*}
where $\mathbf{\tilde x} = R^{-1} \mathbf{x}$.

\subsection{Irrepresentation of Orbital Angular Momentum}
\label{sub:irrep_oam}

Recall that the algebra for $\mathbf{L}$ is $[L_i, L_j] = i \hbar \eps_{ijk} L_k$, so we can simultaneously diagonalize $\mathbf{L}^2$ and $L_z$, with
\[
	\mathbf{L}^2 \ket{\ell, m} = \hbar^2 \ell(\ell+1) \ket{\ell,m}, \qquad L_z\ket{\ell,m} = \hbar m \ket{\ell, m}.
\]
For $R(\bm{\alpha}) \in \mathsf{SO}(3)$, we have $R(2 \pi \hat z) = \mathbbm{1}_{\mathcal{H}}$, so
\[
e^{-2 \pi i L_z/\hbar} = \mathbbm{1}_{L^2(\mathbb{R}^3)}.
\]
This is only true for $J$ and $m$ integers, which is unlike spin which can be a half integer.

Let us embed $H_\ell$, which is the $(2\ell+1)$ dimensional representation into $L^2(\mathbb{R}^3)$. Note that $L_z = X_xP_y - X_yP_x$, then
\[
	\braket{\mathbf{x}|L_z|\psi} = -i\hbar \biggl( x \frac{\partial}{\partial y} - y \frac{\partial}{\partial x} \biggr) \psi(\mathbf{x}) = -i \hbar \frac{\partial}{\partial \phi} \psi(r, \theta, \phi).
\]
Then we get
\[
	\braket{\mathbf{x}|\ell,m} = e^{-i m \phi} K_{\ell,m}(r, \theta)
\]
for some function $K_{\ell,m}$. The raising and lowering operators $L_{\pm}$ act as
\[
	\braket{\mathbf{x}|L_{\pm}|\psi} = i \hbar \biggl( (y \mp ix) \frac{\partial}{\partial z} - z \biggl( \frac{\partial}{\partial y} \mp i \frac{\partial}{\partial x} \biggr) \biggr) \psi(\mathbf{x}).
\]
The highest weight state for some $L$ is defined by $L_+\ket{\ell,\ell} = 0$. Hence, since
\[
	\braket{\mathbf{x}|L_{\pm}|\ell,m} = \pm e^{\pm i \phi} \biggl( \frac{\partial}{\partial \theta} \pm i \cot \theta \frac{\partial}{\partial \phi} \biggr) e^{im \phi} K_{\ell,m}(r, \theta),
\]
we get that this is solved by
\[
	\psi_{\ket{\ell,\ell}} = R_\ell(r) e^{i \ell \phi} \sin^{\ell} \theta,
\]
for some $R_\ell(r)$. The other states $\ket{\ell,m}$ are obtained by acting with $L_-$ on $\ket{\ell, \ell}$. The eigenstates we have constructed are called \emph{spherical harmonics}\index{spherical harmonics}, and are denoted by $Y_\ell^m(\theta, \phi)$. Hence
\[
	\psi_{\ket{\ell,m}} = R_\ell(r) Y_\ell^m (\theta, \phi).
\]
The eigenstate correlation
\[
	\mathbf{L}^2 \ket {\ell, m} = \ell(\ell+1)\hbar^2 \ket{\ell,m},
\]
implies that
\[
- \nabla^2_{S^2} Y_\ell^m (\theta, \phi) = \ell(\ell+1) Y_\ell^m(\theta, \phi).
\]
Hence $Y_\ell^m$ are orthonormal, as
\[
\int_{S^2} Y_{\ell'}^{m'} (\theta, \phi)^{\ast} Y_\ell^m(\theta, \phi) \sin \theta \diff \theta \diff \phi = \delta_{\ell\ell'} \delta_{mm'}.
\]

%lecture 12

Under parity, these spherical harmonics satisfy
\[
Y_\ell^m(-\mathbf{\hat x}) = (-1)^\ell Y_\ell^m(\mathbf{\hat x}) \iff Y_\ell^m(\pi - \theta, \pi + \phi) = (-1)^\ell Y_\ell^m(\theta, \phi).
\]

\newpage

\section{Central Potentials}
\label{sec:cent_potent}

Consider a particle of mass $M$ moving in a central potential
\[
H = \frac{\mathbf{P}^2}{2m} + V(|\mathbf{X}|).
\]
Since $[H, \mathbf{L}^2] = 0 = [H, L_z] = [\mathbf{L}^2, L_z]$, we can simultaneously diagonalize all of these operators. We use a basis $\ket{n, \ell, m}$ such that
\[
	H\ket{n, \ell, m} = E_{n, \ell, m} \ket{n, \ell, m},
\]
and as usual
\begin{align*}
	L_z\ket{n, \ell, m} &= \hbar m \ket{n, \ell, m}, \\
	\mathbf{L}^2 \ket{n, \ell, m} &= \hbar^2 \ell(\ell+1)\ket{n, \ell, m}.
\end{align*}
Note that $[H, L_{\pm}] = 0$, hence $E_n$ must be independent of $m$. Thus $E_{n, \ell, m} = E_{n, \ell}$. We say that there is a $(2\ell+1)$ degeneracy from reaising and lowering.

Generally, we do expect $E_{n, \ell}$ to depend on $\ell$.

\subsection{Hydrogen Atom}
\label{sub:h_atom}

With the Coulomb potential, the Hamiltonian is.
\[
H = \frac{\mathbf{P}^2}{2 \mu} - \frac{\kappa}{|\mathbf{X}|},
\]
we define the reduced mass
\[
\mu = \frac{m_{\elec}m_{\prot}}{m_{\elec} + m_{\prot}}.
\]
Here we have
\[
\kappa = \frac{e^2}{4 \pi \eps_0}.
\]
This obeys $[H, \mathbf{L}] = 0$. The energy levels are
\[
E_{n, \ell , m} = - \frac{R}{n^2},
\]
where $R$ is the Rydberg constant
\[
	R = \frac{\mu e^{4}}{32 \pi^2 \eps_0^2 \hbar^2} \approx \qty{13.6}{\electronvolt}.
\]
The $n$'th energy level is degenerate. For fixed $n$, we can have $m \in \{-\ell , \ldots, \ell \}$ and we can have $\ell \in \{0, 1, 2, \ldots, n-1\}$, so in total we have
\[
\sum_{\ell =0}^{n-1}(2\ell +1)=2\cdot\frac{n(n-1)}{2}+n=n^2.
\]
This is a lot of degeneracy. As we will see, this is because we have an extra symmetry.

Classically, Keplerian orbits close because the Runge-Lenz vector satisfies
\[
\mathbf{r} = \frac{1}{\mu} \mathbf{p} \times \bm{\ell} - \kappa \frac{\mathbf{x}}{|\mathbf{x}|} \implies \mathbf{\dot r} = 0.
\]
Hence we have an extra relation
\[
|\mathbf{r}|^2 = \kappa^2 + \frac{2E}{\mu}|\bm{\ell }|^2.
\]
In quantum mechanics, we can define the Runge-Lenz operator $\mathbf{R}^{\dagger} = \mathbf{R}$, where
\[
\mathbf{R} = \frac{1}{2 \mu} (\mathbf{P} \times \mathbf{L} - \mathbf{L} \times \mathbf{P}) - \kappa \frac{\mathbf{X}}{|\mathbf{X}|},
\]
then $\mathbf{R}$ is conserved for bound orbits. This means $[H, \mathbf{R}] = 0$, and also $\mathbf{L} \cdot \mathbf{R} = \mathbf{R} \cdot \mathbf{L} = 0$.

Then we find that
\[
\mathbf{R}^2 = \kappa^2 + \frac{2H}{\mu}(\mathbf{L}^2 + \hbar^2),
\]
so the eigenvalues of $H$ are fixed by those of $\mathbf{L}^2$ and $\mathbf{R}$. Notice also that
\[
	[L_i, R_j] = i \hbar \eps_{ijk} R_k,
\]
and
\[
	[R_i, R_j] = \frac{2i \hbar}{\mu} H \eps_{ijk} R_k.
\]
Let's solve the eigenstate problem again using $\mathbf{R}$. The right choice of operators are
\[
	\mathbf{A}_{\pm} = \frac{1}{2} \biggl( \mathbf{L} \pm \sqrt{\frac{\mu}{-2H}} \mathbf{R} \biggr).
\]
Here $\mathbf{A}_{\pm}$ are Hermitian for $E < 0$, and obey
\begin{align*}
	[A_{i+}, A_{j+}] &= i \hbar\eps_{ijk} A_{k+}, \\
	[A_{i-}, A_{j-}] &= i \hbar \eps_{ijk} A_{k-},\\
	[A_{i+}, A_{j-}] &= 0.
\end{align*}
So we find a symmetry algebra of $\mathsf{SU}(2) \times \mathsf{SU}(1)$, which is the same algebra as angular momentum. The eigenvalues are  $(a_{\pm})(a_{\pm}+1)\hbar^2$, for $|\mathbf{A}_{\pm}|^2$, with $a_{\pm} \in \{0, \frac{1}{2}, 1, \frac{3}{2}, \ldots \}$.

Since $\mathbf{L} \cdot \mathbf{R} = \mathbf{R} \cdot \mathbf{L} = 0$, we get
\[
|\mathbf{A}_+|^2 = |\mathbf{A}_-|^2 = \frac{1}{2}\mathbf{L}^2 - \frac{\mu}{2H} \mathbf{R}^2 = - \frac{\mu \kappa^2}{8H} - \frac{\hbar^2}{4},
\]
hence $a_+ = a_- = a$. Hence we can label the states as $\ket{a, m_+, m_-}$, where $m_{\pm}$ are the eigenvalues of $A_{z\pm}$. The energy levels are then
\[
E = - \frac{\mu \kappa^2}{8} \frac{1}{a(a+1)\hbar^2 + \hbar^2/4} = - \frac{\mu \kappa^2}{2} \frac{1}{(2a + 1)^2} = - \frac{\mu \kappa}{2} \frac{1}{n^2}.
\]
The degeneracy is $(2a+1)^2 = n^2$, by looking at all possible values of $m_+$, $m_-$.

\subsection{3D Harmonic Oscillator}
\label{sub:3d_ho}

For the 3D simple harmonic oscillator, we have
\[
H = \frac{1}{2m}(P_x^2 + P_y^2 + P_z^2) + \frac{1}{2}m\omega^2(X^2 + Y^2 + Z^2) = H_X + H_Y + H_Z.
\]
The raising and lowering operators
\[
	\mathbf{A}^{\dagger} = \frac{1}{\sqrt{2m \omega}} (m\omega \mathbf{X} - i \mathbf{P}),
\]
and $\mathbf{A} = (\mathbf{A}^{\dagger})^{\dagger}$, show that the energy eigenstates are
\[
	\ket{\mathbf{n}} = \ket{n_x, n_y, n_z} = \frac{1}{\sqrt{n_x!n_y!n_z!}}(A_x^{\dagger})^{n_x}(A_y^{\dagger})^{n_y}(A_z^{\dagger})^{n_z} \ket 0,
\]
and the energy is
\[
E_n = \biggl(N + \frac{3}{2}\biggr) \hbar \omega,
\]
where $N = n_x + n_y + n_z$. Hence the degeneracy is $(N+2)(N+1)/2$, which is also quite large.

%lecture 13

Note that we can also write the Hamiltonian as
\[
H = \hbar \omega \biggl(\mathbf{A}^{\dagger} \mathbf{A} + \frac{3}{2} \biggr),
\]
which was more symmetry. Note that $H$ is invariant under unitary transformation $A_i \to U_{ij} A_j$, where $U^{\dagger}_{ki} U_{ij} = \delta_{ij}$.

Note that $U_{ij}$ mixes $\mathbf{X}$ and $\mathbf{P}$. $U_{ij}$ has 9 real parameters, hence there are 9 corresponding conserved generators. Note there's a unitary operator $U_{(U, T)} : \mathcal{H} \to \mathcal{H}$ with
\[
U = \mathbbm{1}_{\mathcal{H}} - i \eps_{ij} T_{ij},
\]
with $T^{\dagger} = T$ the Hermitian generator, with $[T_{ij}, H] = 0$. Demanding that $U$ generates the rotation of $\mathbf{A}$ and $\mathbf{A}^{\dagger}$, we find
\[
T_{ij} = A_{i}^{\dagger} A_j.
\]
Indeed, note that
\begin{align*}
	\frac{1}{\hbar \omega} [T_{ij}, H] &= [A_i^{\dagger} A_j, A_k^{\dagger} A_k] = A_i^{\dagger} [A_j, A_k^{\dagger}] A_k + A_k^{\dagger} [A_i^{\dagger}, A_k] A_j\\
					   &= - A_i^{\dagger} \delta_{jk} A_k + A_k^{\dagger} \delta_{ik} A_j = 0.
\end{align*}
Decomposing $T_{ij}$ into the trace, antisymmetric and traceless symmetric parts, we have
\[
	T_{ij} = \frac{\delta_{ij}}{3} \mathbf{A}^{\dagger} \cdot \mathbf{A} + \frac{A_i^{\dagger} A_j - A_j^{\dagger} A_i}{2} + \biggl[ \frac{A_i^{\dagger} A_j - A_j^{\dagger} A_i}{2} - \frac{1}{3} \delta_{ij} \mathbf{A}^{\dagger} \mathbf{A} \biggr],
\]
we get that
\[
	\Tr(T) = H + \text{constant},
\]
is clearly conserved. The remaining 8 components generate $\mathsf{SU}(3) = \mathsf{U}(3)/\mathsf{U}(1)$. The antisymmetric part is just $\mathsf{SO}(3) \subset \mathsf{SU}(3)$:
\begin{align*}
	\eps_{ijk} A_j^{\dagger} A_k &= \frac{1}{2m \hbar \omega} \eps_{ijk}(m \omega X_j - i P_j)(m \omega X_k + i P_k) \\
				     &= \frac{i}{2\hbar} \eps_{ijk} (X_j P_k - P_j X_k) = \frac{i}{\hbar} L_i.
\end{align*}
This is just the orbital angular momentum:
\[
\mathbf{L} = \mathbf{X} \times \mathbf{P} = -i \hbar \mathbf{A}^{\dagger} \times \mathbf{A}.
\]
The traceless symmetric part is new to us. It mixes $\mathbf{X}$ and $\mathbf{P}$, and is responsible for the extra degeneracy.

\newpage

\section{Addition of Angular Momentum}
\label{sec:add_mom}

Classically, when we want to find the total angular momentum of two systems, we have
\[
\mathbf{J}_{\mathrm{total}} = \mathbf{J}_1 + \mathbf{J}_2.
\]
From the triangle inequality, this gives
\[
|\mathbf{J}_1| + |\mathbf{J}_2| \geq |\mathbf{J}_{\mathrm{total}}| \geq |\mathbf{J}_1| - |\mathbf{J}_2|,
\]
where $|\mathbf{J}_1| \geq |\mathbf{J}_2|$. Equality occurs if our systems are aligned or anti-aligned.

Let's see what happens in quantum mechanics. Consider $\mathcal{H}_{j_1}, \mathcal{H}_{j_2}$ with basis $\ket{j_i, m_i}$, eigenvectors of $(\mathbf{J}_i^2, \mathbf{J}_{i,z})$ for $i = 1, 2$. Then, $\ket{j_1, m_1} \otimes \ket{j_2, m_2}$ is a basis for $\mathcal{H}_{j_1} \otimes \mathcal{H}_{j_2} = \mathcal{H}_{\mathrm{total}}$.

We want to find this in terms of $\mathbf{J}_{\mathrm{total}} = \mathbf{J}_1 + \mathbf{J}_2$. Mathematically, we want to deduce $\mathcal{H}_{j_1}\otimes \mathcal{H}_{j_2}$ as $\bigoplus \mathcal{H}_j$.

The total angular momentum is
\[
\mathbf{J}^2 = \mathbf{J}_1^2 + \mathbf{J}_2^2 + 2 \mathbf{J}_1 \mathbf{J}_2.
\]
We want to figure out how
\[
\mathbf{J}_1 \cdot \mathbf{J}_2 = J_{1x}J_{2x} + J_{1y}J_{2y} + J_{1z}J_{2z}
\]
acts. Using
\[
J_x = \frac{J_+ + J_-}{2}, \qquad J_y = \frac{J_+ - J_-}{2i},
\]
we get that
\begin{align*}
	2 \mathbf{J}_1 \cdot \mathbf{J}_2 &= 2 \biggl( \frac{J_{1+} + J_{1-}}{2} \frac{J_{2+} + J_{2-}}{2} + \frac{J_{1+}-J_{1-}}{2i} \frac{J_{2+}-J_{2-}}{2i} + J_{1z}J_{2z} \biggr) \\
					  &= J_{1+}J_{2-} + J_{1-}J_{2+} + 2J_{1z}J_{2z},
\end{align*}
hence we get
\[
\mathbf{J}^2 = \mathbf{J}_1^2 + \mathbf{J}_2^2 + J_{1+} J_{2-} + J_{1-} J_{2+} + 2 J_{1z} J_{2z}.
\]
Considering the maximally aligned state $\ket{j_1, j_1} \ket{j_2, j_2}$, then
\[
	J_z \ket{j_1, j_1} \ket{j_2, j_2} = (j_1 + j_2) \hbar \ket{j_1, j_1} \ket{j_2, j_2},
\]
hence this is an eigenvector. Moreover
\begin{align*}
	\mathbf{J}^2 \ket{j_1, j_1} \ket{j_2, j_2} &= \bigl[ j_1(j_1+1) + j_2(j_2+1) + 2j_1j_2\bigr] \ket{j_1,j_1}\ket{j_2,j_2} \\
						   &= (j_1+j_2)(j_1+j_2+1)\ket{j_1,j_1}\ket{j_2,j_2}.
\end{align*}
Hence we can identify $\ket{j_1,j_1}\ket{j_2,j_2} = \ket{j,j}$, an eigenvector of the combined system.

We can find more states by acting with $J_- = J_{1-} + J_{2-}$ on $\ket{j,j}$. Recall that
\begin{align*}
	J_-\ket{j,m} &= \sqrt{j(j+1)-m(m-1)} \hbar \ket{j,m-1}, \\
	\implies J_-\ket{j,j} &= \sqrt{2j} \hbar \ket{j,j-1}, \\
	J_{i-} \ket{j_{i}, j_i} &= \sqrt{2j_i} \hbar \ket{j_i, j_{i-1}}
\end{align*}
for $i = 1, 2$. Comparing these results, we see that
\[
	\ket{j,j-1} = \sqrt{\frac{j_1}{j}} \ket{j_1,j_1-1}\ket{j_2,j_2} + \sqrt{\frac{j_2}{j}} \ket{j_1,j_1}\ket{j_2,j_2-1}.
\]
Repeating this builds all $\ket{j, m}$ with $j = j_1 + j_2$ and $m \in \{-j, -j+1, \ldots, j-1, j\}$.

But there are also imperfectly aligned states. For example, we will try to find $\ket{j-1,j-1}$. Note that this must be orthogonal to $\ket{j, j-1}$ as they are eigenstates of $\mathbf{J}^2$ with different eigenvalues. Hence we can find
\[
	\ket{j-1,j-1}= \sqrt{\frac{j_2}{j}} \ket{j_1,j_1-1}\ket{j_2,j_2} - \sqrt{\frac{j_1}{j}}\ket{j_1,j_1}\ket{j_2,j_2-1}.
\]
We can then obtain the full multiplet $\ket{j-1,m}$ by acting with $J_-$.

We expect this process to provide an eigenvalue for $\mathbf{J}^2$, at least until $|j_1 - j_2|$. If this was true, the total number of states would be $(2j+1)$ for each $j \in \{j_1 - j_2, \ldots, j_1+j_2\}$ giving
\[
\sum_{j = j_1 - j_2}^{j_1+j_2} (2j+1) = 2 \cdot \frac{2j_1(2j_2+1)}{2} + 2j_2 + 1 = (2j_1 +1)(2j_2+1).
\]
Since $\dim(\mathcal{H}_{j_1} \otimes \mathcal{H}_{j_2}) = \dim(\mathcal{H}_{j_1}) \cdot \dim(\mathcal{H}_{j_2}) = (2j_1+1)(2j_2+1)$, this shows we will have all states.

%lecture 14

\subsection{Examples of Addition}
\label{sub:ex_add}

We will look at some examples of tensor products of Hilbert spaces and the corresponding irreducible decomposition.

\subsubsection{$\frac{1}{2} \otimes \frac{1}{2} = 1 \oplus 0$}
\label{subsub:add_ex_1}

Take $j_1 = j_2 = \frac{1}{2}$. Then $j = 1$ or $j = 0$.

This is the case for a Hydrogen atom, so we use the notation $\ket \uparrow_\elec, \ket \downarrow_\elec$ for the basis of $\mathcal{H}_{1/2}$ of the electron $\elec$, and $\ket \uparrow_\prot, \ket \downarrow_\prot$ for the basis of the proton $\prot$.

We start from the highest weight
\[
	\ket{j, j} = \ket{1, 1} = \ket \uparrow_{\elec} \ket \uparrow_{\prot}.
\]
To build the rest of $\mathcal{H}_1$, we act on this with $J_-$. Using $J_- = J_{\elec -} + J_{\prot -}$, we find
\[
	\ket{1, 0} = \frac{1}{\sqrt 2} J_- \ket \uparrow_\elec \ket \uparrow_\prot = \frac{1}{\sqrt 2} (\ket \downarrow_\elec \ket \uparrow_\prot + \ket \uparrow_\elec + \ket\downarrow_\prot),
\]
and finally
\[
	\ket{1,-1} = \ket\downarrow_\elec\ket\downarrow_\prot.
\]
If we apply $J_-$ again, we get zero. Note that this spin 1 multiplet is symmetric under permutations $\elec \leftrightarrow \prot$.

The remaining state is
\[
	\ket{0,0} = \frac{1}{\sqrt 2} (\ket\downarrow_\elec\ket\uparrow_\prot - \ket\uparrow_\elec\ket\downarrow_\prot).
\]
This vector is antisymmetric under $\elec \leftrightarrow \prot$, and indeed we can see that $\ket{0,0}$ is annihilated by $J_z = J_{\elec z} + J_{\prot z}$.

This shows that
\[
\mathcal{H}_{1/2} \otimes \mathcal{H}_{1/2} = \mathcal{H}_1 \oplus \mathcal{H}_0.
\]
Often $(\ket{1, 1}, \ket{1, 0}, \ket{1, -1})$ is called the \emph{triplet}, and $\ket{0,0}$ is called the \emph{singlet}.

\subsubsection{$1 \otimes \frac{1}{2} = \frac{3}{2} \oplus \frac{1}{2}$}
\label{subsub:add_ex_2}

We can check that the dimensions are correct.

For the $j = 3/2$ subspace, we obtain
\begin{align*}
	\ket{3/2, 3/2} &= \ket{1, 1} \ket\uparrow, \\
	\ket{3/2, 1/2} &= \sqrt{\frac{2}{3}} \ket{1,0} \ket\uparrow + \sqrt{\frac{1}{3}} \ket{1,1} \ket\downarrow,\\
	\ket{3/2, -1/2} &= \sqrt{\frac{1}{3}}\ket{1,-1}\ket\uparrow + \sqrt{\frac{2}{3}}\ket{1,0}\ket\downarrow,\\
	\ket{3/2,-3/2} &= \ket{1,-1}\ket\downarrow.
\end{align*}
Moreover we can find
\begin{align*}
	\ket{1/2,1/2} &= \sqrt{\frac{1}{3}} \ket{1,0}\ket\uparrow - \sqrt{\frac{2}{3}}\ket{1,1}\ket\downarrow,\\
	\ket{1/2,-1/2}&= \sqrt{\frac{2}{3}} \ket{1,-1}\ket{\uparrow} - \sqrt{\frac{1}{3}} \ket{1,0}\ket\downarrow.
\end{align*}
The numbers in the linear combinations are called \emph{Clebsch-Gordon coefficients}\index{Clebsch-Gordon coefficients}:
\[
	C(j,m;j_1,m_1,j_2,m_2) = \braket{j,m|j_1,m_2; j_2,m_2}.
\]
For example, reading from the above we get
\[
	C(1/2, -1/2; 1, -1, 1/2, 1/2) = \sqrt{\frac{2}{3}}.
\]
\subsection{Wigner-Eckart Theorem}
\label{sub:we_thm}

Using $J_{\pm}$, we can align the system by changing $m$, but not $j$. Moreover, scalar operators don't change $j$ or $m$, as $\mathbf{J}$ commutes with $S$, hence $S \ket{j,m}$ satisfies
\begin{align*}
	J_z(S\ket{j,m}) &= ([J_z, S] + S J_z)\ket{j,m} = \hbar m S \ket{j,m}, \\
	\mathbf{J}^2 (S \ket{j,m}) &= \hbar^2 j(j+1) S\ket{j,m}.
\end{align*}
To find an operator to find $j$, we need to act with vector operators.

Recall that a vector operator $\mathbf{V}$ transforms as
\[
U^{\dagger}(\bm{\alpha}) \mathbf{V} U(\bm{\alpha}) = R(\bm{\alpha}) \mathbf{V}.
\]
Consider the simplest case: we have the state $\ket{0,0}$ with $j = m = 0$. Acting with $\mathbf{V} \neq \mathbf{J}$,
\begin{align*}
	U(\bm{\alpha}) \mathbf{V} \ket {0,0} &= U(\bm{\alpha}) \mathbf{V} U^{\dagger}(\bm{\alpha}) U(\bm{\alpha}) \ket{0,0} \\
					     &= R(-\bm{\alpha}) \mathbf{V} \ket{0,0} \neq \ket{0,0}.
\end{align*}
More systematically, we introduce the \emph{spherical basis}\index{spherical basis}, as for $J_{\pm}$ and $J_z$.

We treat $\mathbb{R}^3$ as $\mathbb{C} \otimes \mathbb{R}$. Then,
\[
\mathbf{v} = (v_1, v_2, v_3) \implies v^+ = \frac{v_1 + i v_2}{\sqrt 2}, \qquad v^- = \frac{v_1 - i v_2}{\sqrt 2}, \qquad v^0 = v_z.
\]
Now, we consider the vector operators (check these):
\[
V^{+ 1}  = - (V_x + i V_y) \frac{1}{\sqrt 2}, \qquad V^{-1} = (V_x - i V_y) \frac{1}{\sqrt 2}, \qquad V^0 = V_z.
\]
The commutation relations now imply
\begin{align*}
	[J_i, V_j] = i \hbar \eps_{ijk} V_k \implies & [J_z, V^m] = m \hbar V^m, \\
		   &[J_{\pm}, V^m] = \hbar \sqrt{2 - m(m\pm 1)} V^{m \pm 1}.
\end{align*}
More generally, if we have a $2j+1$ tensor $T_{(j)}^{m}$ with $m \in \{-j, \ldots, j\}$, then
\[
	[J_z, T_{(j)}^{m}] = m \hbar T_{(j)}^{m}, \qquad [J_{\pm}, T_{(j)}^m] = \hbar \sqrt{j(j+1)-m(m\pm 1)} T_{(j)}^{m \pm 1}.
\]
For example, we can build $j = 2$ as follows:
\begin{align*}
	T_{(2)}^{\pm 2} &= V_{(1)}^{\pm1} W_{(1)}^{\pm1}, \\
	T_{(2)}^{\pm 1} &= \frac{V_{(1)}^{0}W_{(1)}^{\pm1} + V_{(1)}^{\pm1}W_{(1)}^0}{\sqrt 2},\\
	T_{(2)}^0 &= \frac{V_{(1)}^1W_{(1)}^{-1} + V_{(1)}^{-1} W_{(1)}^{1} + 2 V_{(1)}^0 W_{(1)}^0}{\sqrt 6}.
\end{align*}

%lecture 15

\subsection{Combining Angular Momentum of States and Operators}
\label{sub:comb_ang_mom}

We can similarly combine angular momentum of states and numbers:
\begin{align*}
	J_z(T_{(j_1)}^{m_1} \ket{j_2, m_2}) &= ([J_z, T_{(j_1)}^{m_1}] + T_{(j_1)}^{m_1} J_z) \ket{j_2, m_2} \\
					    &= \hbar (m_1 + m_2) (T_{(j_1)}^{m_1} \ket{j_2, m_2}), \\
	J_{\pm}(T_{(j_1)}^{m_1} \ket{j_2,m_2}) &= ([J_{\pm}, T_{(j_1)}^{m_1}] + T_{(j_1)}^{m_1} J_{\pm} ) \ket{j_2, m_2} \\
					       &= \hbar \sqrt{j_1(j_1+1)-m_1(m_1\pm1)}T_{(j_1)}^{m_1\pm 1} \ket{j_2, m_2} \\
					       &\qquad + \hbar\sqrt{j_2(j_2+1)-m_2(m_2\pm1)}T_{(j_1)}^{m_1}\ket{j_2,m_2\pm1}.
\end{align*}
Using this, we can show that $T_{(j_1)}^{m_1} \ket{j_2, m_2}$ decompose into states of total angular momentum $j \in \{|j_1-j_2|, \ldots, j_1 + j_2\}$, just like for states $\ket{j_1, m_1} \otimes \ket{j_2, m_2}$. We can extend each component by taking the inner products
\[
	\braket{j,m|T_{(j_1)}^{m_1}|j_2, m_2}.
\]
Consider a tensor $T_{(j_2)}^{m_2}$, and two states $\ket{\alpha, j_3, m_3}$ and $\ket{\beta, j_1, m_1}$, where $\alpha$ and $\beta$ represent other possible quantum numbers. Here $\alpha$ and $\beta$ are implicit.

Then the \emph{Wigner-Eckart theorem}\index{Wigner-Eckart theorem} says that there exists a so called \emph{reduced matrix element} $\braket{j_3 \|T_{(j_2)}\| j_1}$ such that
\[
	\braket{j_3, m_3|T_{(j_2)}^{m_2}|j_1,m_1} = \braket{j_3,m_3|j_2,m_2,j_1,m_1} \braket{j_3\|T_{(j_2)}\|j_1}.
\]
For each $j_1, j_2, j_3$, the original object has $(2j_1+1)(2j_2+1)(2j_3+1)$ elements, but we can compute it from a single one of them using the Clepsch-Gordon coefficients.

As a sketch of the proof, using the equation about transformations of $T_{(j_2)}^{m_2} \ket{j_3, m_3}$, we can relate each value with $m_1, m_2, m_3$ with another value with $\tilde m_1, \tilde m_2, \tilde m_3$, and recognize the solution as the Clepsch-Gordon coefficients.

\subsection{Dipole Transition}
\label{sub:dip_trans}

Suppose an atom interacts with an electric field $\mathbf{E}$ with
\[
\Delta H \approx \mathbf{E} \cdot \mathbf{X}.
\]
If $\mathbf{E}$ is constant, then there are transitions between the $\ket{n, j, m}$ states of the atom. These are proportional to
\[
	|\braket{n',\ell', m' | \mathbf{E} \cdot \mathbf{X} |n, \ell , m}|^2.
\]
Note that $\mathbf{X}$ is a spin-1 operators, with spherical components $X^0$, $X^{\pm 1}$. So $\mathbf{X} \ket{n,\ell , m}$ can have total angular momentum equal to $\ell, \ell + 1$ or $\ell - 1$, and angular momentum $m+1, m, m-1$ along $\hat z$.

Hence we have so-called ``reduction rules'' $|\ell' - \ell| \leq 1$, $\ell +\ell' \geq 1$, $|m' - m| \leq 1$. If these are violated, then transition cannot occur.

For example, $\braket{2, 1, m' | \mathbf{X} | 4, 2, m}$ contains $45$ components, but it suffices to just compute one. We will take it to be the simplest: $m = m' = 0$ and $X^0 = X_z$. Then,
\[
	\braket{2, 1, m'| X^i | 4, 2,m} = \frac{\braket{1, m'|1, i; 2, m}}{\braket{1, 0|1, 0; 2, 0}} \braket{2, 1, 0 | X^0 | 4, 2, 0}.
\]

\newpage

\section{Indistinguishable Particles}
\label{sec:ind_part}

An important fact about quantum physics is that all electrons are all indistinguishable, as are all protons, quarks, and most particles.

These electrons have the same Hilbert spaces $L^2(\mathbb{R}^3) \otimes \mathbb{C}^2$, hence any two electrons have the same set of quantum numbers. Theoretically, this means we can't distinguish two different electrons.

This poses some interesting problems when we try to build a Hilbert space of the composite of electrons. For two particles, we will let $\ket \psi \in \mathcal{H}_1 \otimes \mathcal{H}_2$, with basis $\{\ket{\alpha_1, \alpha_2} = \ket{\alpha_1} \otimes \ket{\alpha_2} \}$, and where $\alpha_1, \alpha_2$ are a collection of all quantum numbers.

Because these particles are indistinguishable, we want the two spaces to be physically identical, i.e. $\ket{\alpha_1, \alpha_2} = \lambda \ket{\alpha_2, \alpha_1}$. Exchanging twice, we find that $\lambda^2 = 1$, so $\lambda = \pm1$.

If $\lambda = +1$, the particle is called a \emph{boson}\index{boson}. If $\lambda = -1$, the particle is called a \emph{fermion}\index{fermion}.

These are named after Satyendra Bose and Enrico Fermi, respectively.

These two indistinguishable bosons have state
\[
	\ket \phi = \frac{\ket{\alpha_1, \alpha_2} + \ket{\alpha_2, \alpha_1}}{\sqrt 2},
\]
whereas for two indistinguishable fermions, we have
\[
	\ket \psi = \frac{\ket{\alpha_1, \alpha_2} - \ket{\alpha_2, \alpha_1}}{\sqrt 2}.
\]
If $\alpha_1 = \alpha_2$, then $\ket \psi = 0$, so no two fermions can be in the same quantum states. This is \emph{Pauli's exclusion principle}.

%lecture 16

\subsection{Pauli Exclusion Principle}
\label{sub:pauli_exc}\index{Pauli exclusion principle}

If all quantum numbers of two indistinguishable fermions are the same, i.e. $\alpha_1 = \alpha_2$, then $\ket{\alpha_1, \alpha_2} = 0$.

It is important to exchange all quantum numbers. For example,
\[
\psi_{\sigma_1, \sigma_2}(\mathbf{x}_1, \mathbf{x}_2) = - \psi_{\sigma_2, \sigma_1}(\mathbf{x}_2, \mathbf{x}_1),
\]
but there is no relation between $\psi_{\sigma_1, \sigma_2}(\mathbf{x}_1, \mathbf{x}_2)$ and $\psi_{\sigma_2, \sigma_1}(\mathbf{x}_1, \mathbf{x}_2)$.

\subsection{Many Identical Particles}
\label{sub:many_part}

For $N$ identical particles, let $\sigma$ be a permutation. Then, if the operator $W_\sigma$ implements this permutation of the quantum state, i.e.
\[
	W_\sigma \ket{\alpha_1, \alpha_2, \ldots, \alpha_N} = \lambda_\sigma \ket{\alpha_{\sigma(1)}, \alpha_{\sigma(2)}, \ldots, \alpha_{\sigma(N)}}.
\]
Here $\lambda_\sigma = \pm 1$. For bosons, $\lambda_\sigma = 1$ for all $\sigma$, where for fermions $\lambda_\sigma = \sgn(\sigma)$.

Hence the Hilbert space $\mathcal{H}_N$ for $N$ indistinguishable particles is
\begin{align*}
	\mathcal{H}_{N \text{ bosons}} &= \Sym \otimes \mathcal{H}_{1 \text{ boson}},\\
	\mathcal{H}_{N \text{ fermions}} &= \mathrm{Alt} \otimes \mathcal{H}_{1 \text{ fermion}}.
\end{align*}
where we quotient out by symmetries.

\subsection{Spin and Statistic}
\label{sub:spin_stats}

It turns out that particles with integer spin $s \in \{0, 1, 2, \ldots\}$ are bosons, where those with half integer spin $s \in \{\frac{1}{2}, \frac{3}{2}, \ldots\}$ are fermions.

This applies to composite particles. For example, the Hydrogen atom is a composite of fundamental particles, hence it is a boson.

Note this definition is self consistent; a (composite) particle has integer spin if and only if it is a sum of an even number of fermions.

This fact relies on relativity, and will be seen in Part III Quantum Field Theory.

\subsection{Ideal Quantum Gas}
\label{sub:ideal_gas}

Consider $N$ indistinguishable free fermions in a box of size $L$. Then
\[
H = \sum_{i = 1}^{N} \frac{\mathbf{P}_i^2}{2m}.
\]
A single particle state has energy eigenstates $\ket{\mathbf{K}}$ with $E = \frac{\hbar \mathbf{K}^2}{2 m}$ and $\mathbf{K} = \frac{2\pi}{L} (n_1, n_2, n_3)$. Then
\[
	\braket{\mathbf{x} |\mathbf{K}} = \frac{e^{-i \mathbf{K} \cdot \mathbf{x}}}{L^{3/2}}.
\]
By Pauli exclusion principle, there can only be 1 fermion in state $\mathbf{n}$. At low energy, the first $N$ energy levels are filled. The highest filled level is called the \emph{Fermi level}\index{Fermi level} and has \emph{Fermi energy}
\[
E_F = \frac{\hbar^2 \mathbf{K}_F^2}{2m},
\]
and \emph{Fermi momentum} $\mathbf{K}_F$.

Let us see how this energy changes with the volume. For $N \gg 1$, we go to the continuum. Each electron occupies a volume of
\[
\biggl( \frac{2 \pi}{L}\biggr)^{3} \implies \frac{4 \pi |\mathbf{K}_f|^3}{3} = \biggl( \frac{2 \pi}{L} \biggr)^3 N,
\]
hence the total energy is
\[
E_{\mathrm{total}} = \int_0^{|\mathbf{K}|_F} \diff K \frac{\hbar^2 K^2}{2 m_\elec} \frac{4 \pi K^2}{2\pi/L^3} = \frac{\hbar^2 |\mathbf{K}|_F^5 L^3}{20 \pi^2 m_\elec} = \frac{\hbar^2}{20 \pi^2 m_\elec}(6 \pi^2 N)^{5/3} V^{-2/3}.
\]
Any reduction in the size of the box (at fixed $N$) is opposed by a ``degeneracy pressure''
\[
P_{\mathrm{deg}} = \frac{\partial E_{\mathrm{total}}}{\partial V} \biggl|_N = \frac{\hbar^2 (6 \pi^2 N)^{5/3}}{30 \pi^2 m_\elec} \frac{1}{V^{5/3}}.
\]

\subsection{Exchange and Parity}
\label{sub:ex_par}

Exchanging two indistinguishable particles is related to parity. Define quantities
\begin{align*}
	\mathbf{X}_{\mathrm{com}} &= \frac{\mathbf{X}_1 + \mathbf{X}_2}{2}, & \mathbf{P}_{\mathrm{com}} &= \mathbf{P}_1 + \mathbf{P}_2, \\
	\mathbf{X}_{\mathrm{rel}} &= \mathbf{X}_1 - \mathbf{X}_2, & \mathbf{P}_{\mathrm{rel}} &= \frac{\mathbf{P}_1 - \mathbf{P}_2}{2}.
\end{align*}
Exchanging particles 1 and 2 leaves $\mathbf{X}_{\mathrm{com}}$ and $\mathbf{P}_{\mathrm{com}}$ unchanged, but acts like parity on $(\mathbf{X}_{\mathrm{rel}}, \mathbf{P}_{\mathrm{rel}})$.

Recall the spherical harmonics, and that $Y_\ell^m(- \mathbf{\hat x}) =  (-1)^{\ell }Y_\ell^m(\mathbf{\hat x})$. Then if two indistinguishable particles have relative angular momentum $\ell $ then their wavefunction must be symmetric (resp. antisymmetric) if $\ell$ is even (resp. odd).

Complete behaviour under exchange also involves spin.
\begin{itemize}
	\item For bosons, the spin state must have the same symmetry as the spatial wavefunction.
	\item For fermions, the spin state must have the opposite symmetry as the spatial wavefunction.
\end{itemize}
Consider the dynamics of a pion $\pi^{-}$ orbiting a deuterium nucleus $D^{+}$. This has the same energy levels as the Hydrogen atom. We suppose the atom is in the ground state $\ket{n, \ell,m} = \ket{1, 0,0}$, and also the relevant spins are $s_{\pi} = 0$ and $s_D = 1$, so the atom has total angular momentum $j = 1$.

The atom decays into two neutrons $n$, so
\[
\pi^{-} + D^{+} \to n + n.
\]
Since $s_n = \frac{1}{2}$, neutrons are fermions. So the final state must be anti-symmetric.

The total angular momentum is conserved, so $j= 1$ in the final state. Recall that $\frac{1}{2} \otimes \frac{1}{2} = 1 \oplus 0$. Note if $s_{n,n} = 0$, then $\ell$ is even, but then $j$ cannot be 1.

%lecture 17

Hence we must have $s_{n,n} = 1$, and then $j \in \{\ell+1,\ell,\ell-1\}$. As $j = 1$ and $\ell$ must be odd, we have $\ell = 1$.

Assuming parity is conserved, we can find the parity $\eta_\pi$ of the pion. From conservation of parity,
\[
	\eta_\pi \eta_D = (-1)^{\ell}\eta_n^2 = -1.
\]
We know that $\eta_D = 1$ for physical reasons (it is the $s$-wave bound state of a proton and neutron, which have the same parity).

We conclude that $\eta_\pi = -1$, hence it is a pseudo-scalar.

\newpage

\section{Time Independent Perturbation Theory}
\label{sec:ti_pert_thy}

For realistic systems, we usually cannot solve the dynamics exactly. In this section, we will develop approximate methods to solve these problems.

Suppose that
\[
H = H_0 + \Delta H,
\]
and that we can solve for $H_0$. For $\lambda \in [0, 1]$, let
\[
H_\lambda = H_0 + \lambda \Delta H.
\]
Clearly $H_{1} = H$. We will try to Taylor expand in $\lambda$ around $\lambda = 0$.

We seek energy eigenstates $\ket{E_\lambda}$ of $H_\lambda$. If $\lambda$-dependence is analytic at $\lambda = 0$, then so are the eigenstates:
\begin{align*}
	\ket{E_\lambda} &= \ket{\alpha} + \lambda \ket{\beta} + \lambda^2 \ket{\gamma} + \cdots \\
	E(\lambda) &= E^{(0)} + \lambda E^{(1)} + \lambda^2 E^{(2)} + \cdots
\end{align*}
Plugging this expansion into the exact equation $H_\lambda \ket{E_\lambda} = E(\lambda) \ket{E_\lambda}$ gives
\begin{align*}
	(H_0 + \lambda \Delta H)(\ket \alpha &+ \lambda \ket \beta + \lambda^2 \ket \gamma + \cdots) \\
									      &= (E^{(0)} + \lambda E^{(1)} + \lambda^2 E^{(2)} + \cdots)(\ket \alpha + \lambda \ket \beta + \lambda^2 \ket \gamma + \cdots).
\end{align*}
To zeroth order in $\lambda$, we get
\[
H_0 \ket \alpha = E^{(0)} \ket \alpha.
\]
To first order in $\lambda$, we find
\[
H_0 \ket \beta + \Delta H \ket \alpha = E^{(0)} \ket \beta + E^{(1)} \ket \alpha.
\]
To second order in $\lambda$, we find
\[
H_0 \ket \gamma + \Delta H \ket \beta = E^{(0)} \ket \gamma + E^{(1)} \ket \beta + E^{(2)} \ket \alpha.
\]
The zeroth order solution says that $\ket \alpha$ is an eigenstate of $H_0$ with eigenvalue $E^{(0)}$. Hence we change notation to $\ket\alpha = \ket n$, and $E^{(0)} = E_n$.

For each $n$, the first order equation says
\[
	H_0 \ket{\beta_n} + \Delta H \ket n = E_n \ket{\beta_n} + E_n^{(1)} \ket n.
\]
Contracting both sides with $\bra n$, we find
\[
	\braket{n|H_0|\beta_n} + \braket{n|\Delta H|n} = E_n \braket{n|\beta_n} + E_n^{(1)}.
\]
Since $\braket{n|H_0|\beta_n} = \overline{\braket{\beta_n|H_0|n}} = \overline{\braket{\beta_n|E_n|n}} = E_n \braket{n|\beta_n}$, this becomes
\[
	\braket{n|\Delta H|n} = E_n^{(1)}.
\]
If we instead contract with $\bra m \neq \bra n$, we get
\[
	\braket{m|\Delta H| n} = (E_n - E_m) \braket{m|\beta_n}.
\]
Writing $\ket{\beta_n}$ in the basis $\{ \ket k\}$, i.e.
\[
	\ket{\beta_n} = \sum b_k \ket k,
\]
we get that if $E_n \neq E_m$, then
\[
	b_m = \frac{\braket{m|\Delta H|n}}{E_n - E_m} \implies \ket{\beta_n} = \sum_{m \neq n} \frac{\braket{m|\Delta H|n}}{E_n - E_m} \ket m.
\]
We can check that if we impose $\braket{E_\lambda|E_\lambda} = 1$, we must have that $b_n = \braket{n|\beta_n} = 0 + \mathcal{O}(\lambda^2)$.

At the second order $\mathcal{O}(\lambda^2)$, we have
\[
	H_0 \ket{\gamma_n} + \Delta H \ket{\beta_n} = E_n \ket{\gamma_n} + \braket{n|\Delta H| n} \ket{\beta_n} + E_n^{(2)} \ket n.
\]
Contracting with $\bra n$ again, we get
\[
	E_n^{(2)} = \braket{n|\Delta H| \beta_n} = \sum_{m \neq n} \frac{|\braket{n|\Delta H|m}|^2}{E_n - E_m}.
\]
Note that this is always negative for $n = 0$, the ground state. From this we can see that nearby states matter the most, and nearby levels are driven apart.

\subsection{Examples of Perturbation}
\label{sub:ex_pert}

We take the Hamiltonian
\[
H = \frac{P^2}{2m} + \frac{1}{2} m \omega^2 X^2 - \lambda m \omega^2 x_0 X,
\]
which we can rewrite as
\[
H = \frac{P^2}{2m} + \frac{1}{2} m \omega^2 (X - x_0 \lambda)^2 - \frac{1}{2} \lambda^2 m \omega^2 x_0^2.
\]
The exact energy levels are
\[
E_n(\lambda) = h \omega\biggl(n + \frac{1}{2}\biggr) - \frac{1}{2} \lambda^2 m \omega^2 x_0^2.
\]
These are indeed analytic in $\lambda$. Now let's compute this perturbation theory. Letting $\Delta H = - \lambda m \omega^2 x_0 X$ as a perturbation,
\begin{align*}
	E_n(\lambda) &= E_n + \lambda \braket{n|\Delta H|n} + \lambda^2 m^2 \omega^{4} x_0^2 \sum_{n \neq m} \frac{|\braket{n|X|m}|^2}{(n-m)\hbar \omega} + \mathcal{O}(\lambda^3) \\
	&= \hbar \omega \biggl(n + \frac{1}{2} \biggr) - \frac{1}{2} \lambda^2 m^2 \omega^2 x_0^2 + \mathcal{O}(\lambda^3),
\end{align*}
which is exact.

Now instead let us try replacing $\omega \to \omega(\lambda) = \omega\sqrt{1 + \lambda}$, then
\[
H = \frac{P^2}{2m} + \frac{1}{2} m \omega^2(1+\lambda)X^2.
\]
The exact energy levels are
\[
	E_n(\lambda) = \hbar \omega \sqrt{1 + \lambda} \biggl(n + \frac{1}{2} \biggr).
\]
Taking $\Delta H = \frac{1}{2} m \omega^2 \lambda X^2$ as a perturbation,
\begin{align*}
	E_n(\lambda) &= E_n + \frac{\lambda}{2} m \omega^2 \braket{n|X^2|n} + \frac{\lambda^2}{4} m^2 \omega^{4} \sum_{n \neq m} \frac{|\braket{n|X^2|m}|^2}{(n-m)\hbar \omega} \\
		     &= \biggl(n + \frac{1}{2} \biggr) \hbar \omega \biggl(1 + \frac{\lambda}{2} - \frac{\lambda^2}{8} + \mathcal{O}(\lambda^3) \biggr),
\end{align*}
which matches the Taylor expansion of $\sqrt{1 + \lambda}$.

%lecture 18

Perturbation theory converges to the exact $E_n(\lambda)$ for $|\lambda| < 1$. Hence perturbation theory has a finite radius of convergence.

For our third example, consider
\[
H = \frac{P^2}{2m} + \lambda X^4.
\]
This is hard to solve analytically. Using perturbation, we get
\[
E_0(\lambda) = \frac{1}{2} \hbar \omega + \sum_{n = 1}^{\infty} \lambda^n a_n,
\]
where
\[
	a_n = \frac{(-1)^{n+1} 3^n \sqrt 6}{\pi^{3/2}} \Gamma\biggl(n + \frac12\biggr) \biggl[1 - \frac{95}{72} \frac{1}{n} + \mathcal{O}\biggl(\frac{1}{n^2}\biggr)\biggr].
\]
For $n \to \infty$, note that $\Gamma(n + \frac12) \sim n! \sim n^n$. Then,
\[
\Delta E_0(\lambda) \sim \sum_{n}^{\infty} (\lambda n)^n.
\]
Hence the radius of convergence of the perturbation theory is $0$. Physically, taking $\lambda X^4$ for any $\lambda < 0$ makes the potential unbounded.

Usually, perturbation theory doesn't converge, but is \emph{asymptotic}:
\[
	\lim_{\lambda \to 0^+} \frac{1}{\lambda^N} \biggl[E(\lambda) - \sum_{n = 1}^{N} \lambda^n E^{(n)} \biggr] = 0.
\]
However, we have
\[
\lim_{N \to \infty} \sum_{n = 0}^{N} \lambda^n E^{(n)} = \infty.
\]
The optimal $N$ gives a good approximation for small $\lambda$.

\subsection{Fine Structure of Hydrogen}
\label{sub:fine_h}

The spectrum of $H$ that we studied so far is called the \emph{gross structure}\index{gross structure}. The bound states are $\ket{n, \ell, m}$ with
\[
E_n = - \frac{1}{2} \mu c^2 \frac{\alpha^2}{n^2},
\]
where
\[
\mu = \frac{m_\elec m_\prot}{m_\elec + m_\prot}
\]
is the reduced mass, and
\[
\alpha = \frac{e^2}{4 \pi \eps_0 \hbar c}
\]
is called the fine structure constant. Let's estimate the relativistic corrections. The electron's velocity is roughly
\[
v^2 \mu \sim \frac{\mathbf{p}^2}{\mu} \sim E \sim \mu c^2 \frac{\alpha^2}{n^2} \implies \frac{v}{c} \sim \frac{\alpha}{n}.
\]
There are relativistic effects controlled by this velocity. Since $\alpha \approx \frac1{137}$, we can treat this as a problem in perturbation theory. There are two effects.

\textbf{Kinetic correction:} The relativistic dispersion relation for massive particles is
\[
	E = \sqrt{\mu^2 c^4 + \mathbf{p}^2 c^2} = \mu c^2 + \frac{\mathbf{p}^2}{2 \mu} - \frac{\mathbf{p}^4}{8 \mu^3 c^2} + \cdots
\]
We treat this as a perturbation
\[
\Delta H = - \frac{\mathbf{p}^4}{8 \mu^3 c^2}
\]
to $H_0 = \mathbf{p}^2/2\mu$. We expect that
\[
	\frac{\braket{\Delta H}}{\braket{H_0}} \sim \biggl(\frac{v}{c} \biggr)^2 \propto \alpha^2.
\]
The energy is corrected by
\[
	E_{n, \ell, m}^{(1)} = \braket{n, \ell, m|\Delta H|n, \ell, m}.
\]
Can we use non-degenerate perturbation theory? To answer this, note that $[\mathbf{L}, \Delta H] = 0$, hence it follows
\[
	0 = \braket{n, \ell', m'|[\mathbf{L}^2, \Delta H] | n, \ell, m} = \hbar^2(\ell'(\ell'+1) - \ell(\ell+1)) \braket{n, \ell', m'|\Delta H|n, \ell, m},
\]
and also
\[
	0 = \braket{n, \ell', m'|[L_z, \Delta H]|n, \ell, m} = \hbar(m' - m) \braket{n, \ell', m'|\Delta H|n, \ell, m},
\]
so $\braket{n, \ell', m'|\Delta H|n, \ell, m} \propto \delta_{\ell \ell'}\delta_{m m'}$ and we can use non-degenerate perturbation theory.

Now we use some tricks to compute $\braket{\Delta H}$: 
\begin{align*}
	\braket{\Delta H}_{n, \ell, m} &= \frac{\braket{(H_0 - V)^2}_{n, \ell, m}}{2 \mu c^2} \\
				       &= -\frac{E_n^2 - 2 E_n \braket{V}_{n, \ell, m} + \braket{V^2}_{n, \ell, m}}{2 \mu c^2}.
\end{align*}
Also, the virial theorem says that $2 \braket{T} = - \braket{V}$, so
\[
	E_n = \braket{T} + \braket{V} = \frac{\braket{V}}{2},
\]
hence we get
\begin{align*}
	\braket{\Delta H}_{n, \ell, m} &= \frac{3 E_n^2}{2 \mu c^2} + \frac{\braket{V^2}_{n, \ell, m}}{2 \mu c^2} \\
				       &= \frac{1}{2} \mu c^2 \frac{3}{4} \frac{\alpha^{4}}{n^4} - \frac{\hbar^2}{2 \mu} \braket{ \frac{\alpha^2}{r^2} }_{n, \ell, m}.
\end{align*}
The remaining term $\braket{r^{-2}}$ can be reabsorbed into the effective potential
\[
	V_{\mathrm{eff}}(r) = - \frac{e^2}{4 \pi \eps_0} \frac{1}{r} + \frac{\hbar}{2 \mu} \biggl[ \frac{\ell(\ell+1)}{r^2} + \frac{\alpha^2}{^2} \biggr],
\]
where the fictitious angular momentum $\ell'$ is such that
\[
\ell'(\ell'+1) = \ell(\ell+1) + \alpha^2.
\]
We already know the energy levels of $V_{\mathrm{eff}}$ are
\[
E_n(\ell') = -\frac{1}{2} \mu \alpha^2 c^2 \frac{1}{(\ell'+1)^2}.
\]
To leading order, we have $\ell' = \ell + \delta \ell$, where
\[
\delta \ell = \frac{\alpha^2}{(2\ell + 1)},
\]
so we get
\[
	E_n(\ell + \delta \ell) = -\frac{1}{2} \mu \alpha^2 c^2 \biggl[ \frac{1}{(\ell + 1)^2} - \frac{2 \delta \ell }{(\ell + 1)^3} + \cdots \biggr]
\]
Combining all the corrections, we obtain
\[
E_{n, \ell, m}^{(1)} = - \frac{1}{2} \mu c^2 \frac{\alpha^{4}}{n^4} \biggl( \frac{n}{\ell + 1/2} - \frac{3}{4} \biggr).
\]
%lecture 19
\textbf{Spin-orbit coupling:} Another relativistic effect is spin-orbit coupling. The electron in the Coulomb field of the proton with velocity $\mathbf{v}$ feels a magnetic field
\[
\mathbf{B} = \frac{\gamma}{c^2} \mathbf{v} \times \mathbf{E} = \frac{1}{\mu c^2} \mathbf{p} \times \biggl( \frac{e}{4 \pi \eps_0} \frac{\mathbf{\hat x}}{|\mathbf{x}|^2} \biggr) = - \frac{e}{4 \pi \eps_0 \mu c^2} \frac{\mathbf{L}}{|\mathbf{x}|^3}.
\]
Here $\gamma = (1 - (v/c)^2)^{-1/2}$ is the Lorentz factor, and $\mathbf{p} = m \mathbf{v} \gamma$.

Spin-orbit coupling between the electron's magnetic dipole moment $\mathbf{m}$ and this electric field gives another alteration to the Hamiltonian:
\[
\Delta H_{\mathrm{SO}} = - \mathbf{m} \cdot \mathbf{B} = - \frac{e}{2 \mu} \mathbf{B} \cdot \mathbf{S} = - \frac{e^2}{8 \pi \eps_0 \mu c^2} \frac{\mathbf{L} \cdot \mathbf{S}}{|\mathbf{X}|^3}.
\]
This depends on $j = \ell \pm 1/2$ for $\ket{\uparrow}$ and $\ket{\downarrow}$.

For $\ell = 0$, this is trivial, but for $\ell \neq 0$ it depends on the spin. Let's write
\[
\mathbf{L} \cdot \mathbf{S} = (\mathbf{J}^2 - \mathbf{L}^2 - \mathbf{S}^2)\frac{1}{2},
\]
then indexing our states by $\ket{n, j, m_j; \ell}$, we have
\begin{align*}
	\mathbf{L} \cdot \mathbf{S} \ket{n, j, m_j; \ell} &= \frac{\hbar^2}{2} \biggl(j(j+1) - \ell(\ell+1) - \frac{3}{4} \biggr) \ket{n, j, m_j; \ell} \\
							  &= \frac{\hbar^2}{2}
							  \begin{cases}
								  \ell \ket{n, j, m_j; \ell} & j = \ell + 1/2,\\
								  -(\ell+1)\ket{n, j, m_j; \ell} & j = \ell - 1/2.
							  \end{cases}
\end{align*}
After a bit of work, we can add this to obtain
\[
	E_{n, j, \ell } = - \frac{1}{2} \mu \alpha^2 c^2 \biggl[ \frac{1}{n^2} - \frac{\alpha^2}{n^3} \biggl(\frac{3}{4n} - \frac{1}{j + 1/2} \biggr) + \mathcal{O}(\alpha^{4}) \biggr].
\]
\begin{remark}
	\begin{itemize}
		\item[]
		\item The degeneracy in $\ell$ is now lifted, however the Runge-Lenz vector is no longer conserved.
		\item For the general periodic table, the electron feels the attraction of $Z$ protons, so we should rescale $\alpha \to Z \alpha$.

			This rescaling shows relativistic corrections are most important for heavier atoms, for example
			\[
			E_{n, \ell+1/2, \ell } - E_{n, \ell - 1/2, \ell } = \frac{1}{2} \mu c^2 \frac{1}{n^3} \frac{Z^4 \alpha^4}{\ell(\ell+1)}.
			\]
			 This is good for the $2p$ state of hydrogen, but for iron for example is only a 10\% correction.
	\end{itemize}
\end{remark}

\subsection{Degenerate Perturbation Theory}
\label{sub:deg_pt}

If $H_0$ has degenerate eigenstates, even a tiny perturbation can have a large effect. We saw this problem in our expansion
\[
	\ket{n_\lambda} = \ket{n} + \lambda \sum_{m \neq n} \frac{\braket{m|\Delta H| n}}{E_n - E_m} + \mathcal{O}(\lambda^2).
\]
Assume there are finitely many degenerate states forming the $N$-dimensional subspace $V \in \mathcal{H}$, and $H_0 \ket \psi = E_V \ket \psi$ for all $\ket \psi \in V$. Let $\{\ket r\}$ be an orthonormal basis for $V$.

Then divergences are avoided if $\braket{r'|\Delta H|r} = \delta_{r r'}$, that is, if $\Delta H$ is diagonal in $V$. Since $\Delta H = \Delta H^{\dagger}$, we can always diagonalize it.

Let $\{\ket n\}$ be a new basis of eigenvectors of $H_0$ such that $\Delta H$ is diagonal in every degenerate subspace, that is, $\braket{m|\Delta H|n} = 0$ if $E_n = E_m$. Then the perturbation theory formula applies.

\begin{exbox}[Linear Stark Effect]
	Put the hydrogen atom in an electric field $\mathbf{E} = |E| \hat x_3$. Then,
	\[
	\Delta H = - e | E| X_3.
	\]
	By parity, $\braket{1, 0, 0|X_3|1, 0, 0} = 0$, so $E_0^{(1)} = 0$. The first excited state $n = 2$ has degeneracy 4.

	Parity implies that $\braket{2, \ell', m' | X_3| 2, \ell, m}$ vanishes unless $|\ell'-\ell|$ is odd. Also, $[L_z, X_3] = 0$, so we have
	\[
		\braket{2, 0, 0 | X_3 | 2, 1, \pm 1} = 0.
	\]
	Within $V$, the perturbation $\Delta H$ acts as
	\[
	\Delta H = e |E|
	\begin{pmatrix}
		0&0&a&0 \\
		0&0&0&0 \\
		a^{\ast}&0&0&0\\
		0&0&0&0,
	\end{pmatrix}
	\]
	where $a = \braket{2, 0, 0|X_3|2, 1, 0} = -3 a_0$. The eigenvalues of $\Delta H$ in $V$ are $e |E| a_0 \times \{-3, 0, 0, 3\}$. Moreover $\Delta H$ is diagonal in basis
	\[
		\frac{\ket{2, 0, 0} - \ket{2, 1, 0}}{\sqrt 2}, \qquad \ket{2, 1, 1}, \qquad \ket{2, 1, -1}, \qquad \frac{\ket{2, 0, 0} + \ket{2, 1, 0}}{\sqrt 2}.
	\]
	Note that the degeneracy between $\ket{2, 1, \pm 1}$ is not lifted. Even a small $|E|$ causes the $n = 2$ states to align to
	\[
		\frac{\ket{2, 0, 0} + \ket{2, 1, 0}}{\sqrt 2}.
	\]
\end{exbox}

%lecture 20

\newpage

\section{Time-Dependent Perturbation Theory}
\label{sec:time_pt}

Consider a system with time-independent $H_0$ and time-dependent perturbation operator $\Delta (t)$. Some examples include:
\begin{itemize}
	\item An atom perturbed by electromagnetic radiation.
	\item A particle scattering.
\end{itemize}
Let $\{\ket n\}$ be an orthonormal basis of $\mathcal{H}$, with $H_0 \ket n = E_n \ket n$. Then,
\[
	\ket{\psi(t)} = \sum_n e^{-i E_n t/\hbar} a_n(t) \ket n. \tag{$\ast$}
\]
Here $a_n(t)$ depends on $t$ because of $\Delta(t)$.

As before, let $H_\lambda = H_0 + \lambda \Delta(t)$, and we will expand in  $\lambda$, trying to approximate $\lambda = 1$. Plugging $(\ast)$ in, we get
 \[
\sum_n (a_n E_n + i \hbar \dot a_n) e^{-i E_n t/\hbar} \ket n = \sum_n a_n (E_n + \lambda \Delta (t)) e^{-i E_n t/\hbar} \ket n.
\]
Contracting with $\bra k$, we get
\[
	i \hbar \dot a_k = \lambda \sum_n a_n(t) e^{i t (E_k - E_n)/\hbar} \braket{k|\Delta(t)|n},
\]
so we get
\[
	a_k(t) = a_k(t_0) + \frac{\lambda}{i \hbar} \int_{t_0}^{t} \sum_n a_n(t') e^{-i (E_n - E_k)t'/\hbar} \braket{k|\Delta(t')|n} \diff t'.
\]
This is exact. We will try approximate
\[
a_n(t) = a_n^{(0)}(t) + \lambda a_n^{(1)}(t) + \mathcal{O}(\lambda^2),
\]
and solve order-by-order in $\lambda$.

In the $0$'th term, we have
\[
a_n(t) = a_n^{(0)}(t_0),
\]
hence $a_n(t)$ is constant. To first order,
\[
	a_k^{(1)}(t) = \frac{1}{i \hbar} \int_{t_0}^t \diff t' \sum_n a_n^{(0)}(t') e^{i \omega_{kn} t'/\hbar} \braket{k|\Delta(t')|n},
\]
where $\omega_{kn} = E_k - E_n$. Let's simply start in state $\ket m$ at $t = t_0$. Then $a_n^{(0)}(t) = \delta_{mn}$. Hence to first order
\[
	a_k(t) = \delta_{km} + \frac{\lambda}{i \hbar} \int_{t_0}^t e^{i \omega_{km} t'/\hbar} \braket{k|\Delta(t')|m} \diff t'.
\]

\subsection{Applications}
\label{sub:app_time_pt}

\begin{exbox}[Forced Oscillator]
	Consider a harmonic oscillator perturbed by time-dependent force
	\[
	\Delta(t) = F_0 X e^{-t^2/\tau^2},
	\]
	where $F_0$ and $\tau$ are real parameters and $X$ is the position operator. The Hamiltonian is
	\[
	H_\lambda = \frac{P^2}{2m} + \frac{1}{2} m \omega^2 X^2 - \lambda F_0 X e^{-t^2/\tau^2}.
	\]
	Suppose $\ket{\psi(t = -\infty)} = \ket 0$. For $\ket k \neq \ket 0$, we compute
	\begin{align*}
		\lim_{t \to +\infty} a_k(t) &= - \frac{F_0}{i \hbar} \int_{-\infty}^{\infty} \diff t' e^{i k \omega t'} e^{-t'^2/\tau^2} \braket{k|X|0} \\
					    &= i F_0 \sqrt{\frac{\pi \hbar}{2 m \omega}} \tau e^{-k^2 \omega^2 \tau^2/4} \braket{k|A^{\dagger}|0}.
	\end{align*}
	The probability that the oscillator is in state $\ket 1$ after infinite time is
	\[
	\mathbb{P}(\ket 0 \to \ket 1) = \frac{\pi \hbar F_0^2}{2 m \omega} \tau^2 e^{-\omega^2 \tau^2/2}.
	\]
	Note that:
\begin{itemize}
	\item The maximum probability is achieved when $\tau = \sqrt 2 / \omega$, when we have resonance.
	\item If we turn on and off the interaction adiabatically, i.e. much slower than the frequency $\omega$ of the quantum harmonic oscillator, then there is no transition.
	\item The higher states are excited at higher orders in the perturbation theory.
\end{itemize}
\end{exbox}

\begin{exbox}[Monochromatic Perturbation]
	Consider the perturbation
	\[
	\Delta(t)=
	\begin{cases}
		0 & t < 0,\\
		\Delta e^{-i \omega t} + \Delta^{\dagger}e^{i\omega t} & t \geq 0,
	\end{cases}
	\]
	where $\Delta$ is a time-independent operator. If the state was $\ket m$ at $t < 0$, then for $t > 0$ and $\ket k \neq \ket m$,
	\begin{align*}
		a_k(t) &= \frac{\braket{k|\Delta|m}}{\hbar(\omega_{km} - \omega)}\bigl[e^{i(\omega_{km} - \omega)t} - 1 \bigr] + \frac{\braket{k|\Delta^{\dagger}|m}}{\hbar(\omega_{km} + \omega)} \bigl[e^{i(\omega_{km} + \omega)t} - 1 \bigr].
	\end{align*}
	At $t \to \infty$, the first term contributes only if $E_k - E_m \approx \hbar \omega$. This is called \emph{absorption}\index{absorption}. Conversely, the second term dominates only when $E_k - E_m = - \hbar \omega$, and is called \emph{stimulated emission}\index{stimulated emission}. In terms of probabilities,
	\[
	|a_k(t)|^2 =
	\begin{dcases}
		\frac{4}{\hbar^2} \frac{|\braket{k|\Delta|m}|^2}{(\omega_{km} - \omega)^2} \sin^2 \biggl( \frac{(\omega_{km} - \omega)t}{2} \biggr), \\
		\frac{4}{\hbar^2} \frac{|\braket{k|\Delta^{\dagger}|m}|^2}{(\omega_{km} + \omega)^2} \sin^2\biggl( \frac{(\omega_{km} + \omega)}{2} t \biggr).
	\end{dcases}
	\]
	We claim that as distributions,
	\[
	\frac{\sin^2(\omega_{km} t / 2)}{\omega_{km}^2} \approx \frac{\pi}{2} \delta(\omega_{km})t + \mathcal{O}(t^2).
	\]
	Define the `transition rate' from $\ket m$ to $\ket k$ by
	\[
	\Gamma(\ket m \to \ket k) = \lim_{t \to \infty} \partial_t|a_k(t)|^2.
	\]
	At late times,
	\[
	\Gamma_{m \to k} =
	\begin{cases}
		\frac{2 \pi}{\hbar} |\braket{k|\Delta|m}|^2 \delta(E_k - E_m - \hbar \omega), \\
		\frac{2 \pi}{\hbar} |\braket{k|\Delta^{\dagger}|m}|^2 \delta(E_k - E_m + \hbar \omega).
	\end{cases}
	\]
	This is called \emph{Fermi's golden rule}\index{Fermi's golden rule}.
\end{exbox}
%lecture 21
\begin{exbox}[Ionization]
	Strong radiation can ionize the atom, liberating an electron to the continuum state.

	Consider hydrogen's ground state 
	\[
		\braket{\mathbf{x}|1,0,0} = e^{-r/a} \frac{1}{\sqrt{\pi a^3}}.
	\]
	We will compute the transition to the plane wave:
	\[
		\braket{\mathbf{x}|\mathbf{k}} = e^{i \mathbf{k} \cdot \mathbf{x}} \frac{1}{(2 \pi \hbar)^{3/2}}.
	\]
	This neglects the Coulomb interaction in the final state. In the dipole approximation, the perturbation is
	\[
	\Delta(t) = e(\mathbf{E} \cdot \mathbf{X} e^{-i \omega t} + (\mathbf{E} \cdot \mathbf{X})^{\dagger} e^{i \omega t}) = 2e \mathbf{E} \cdot \mathbf{X} \cos (\omega t),
	\]
	for radiation of frequency $\omega$ and electric field $\mathbf{E} = |\mathbf{E}|^{1/2}$.

	The matrix element is
	\begin{align*}
		\braket{\mathbf{k} | X_3 | 1, 0, 0} &= \int \Diff3 x \frac{e^{-i \mathbf{k} \cdot \mathbf{x}} z e^{-r/a}}{(2 \pi \hbar)^{3/2}\sqrt{\pi a^3}} = \frac{4 \pi k_z}{i (2 \pi \hbar)^{3/2} \sqrt{\pi a^3}} \frac{8a^{5}}{(1 + |\mathbf{k}|^2 a^2)^3} \\
						    &\approx \frac{8 \sqrt 2 \cos \theta}{i \pi \hbar^{3/2} k^{5} a^{5/2}},
	\end{align*}
	using $k^2a^2 \gg 1$ in the dipole approximation (and $\cos \theta = \mathbf{\hat E} \cdot \mathbf{\hat k}$).

	The transition absorbs energy. So Fermi's golden rule gives
	\[
		\Gamma_{\ket{1,0,0} \to \ket k} = \frac{2\pi}{\hbar} e^2 E^2 |\braket{\mathbf{k}|X_3|1,0,0}^2| \delta(E_{\mathbf{k}} - E_{1, 0, 0} - \hbar \omega).
	\]
	We define the differential transition rate
	\[
	\diff \Gamma = \frac{2 \pi}{\hbar} e^2 E^2 \delta(E_{\mathbf{k}} - E_{1, 0, 0} - \hbar \omega) \hbar^3 k^2 \diff k \sin \theta \diff \theta \diff \phi.
	\]
	Note that the Dirac delta sets
	\[
		\hbar |\mathbf{k}| = \sqrt{2m(\hbar \omega - E_{1, 0, 0})} \approx \sqrt{2m \hbar \omega}.
	\]
	Putting everything together,
	\[
	\frac{\diff \Gamma}{\diff \Omega} = \frac{256 e^2 E^2 m_\elec \cos^2\theta}{\pi \hbar^3 k^3 a^{5}}.
	\]
\end{exbox}

\newpage

\section{Density and Information}
\label{sec:dense_info}

\subsection{The Density Operator}
\label{sub:den_op}

We have the following issues:
\begin{itemize}
	\item Often we don't know the exact quantum state.
	\item It is hard to discuss measurement because we need to quantize the environment.
\end{itemize}

To make progress, we define a \emph{density operator}\index{density operator}:
\[
	\rho = \sum_{\alpha} p_\alpha \ket{\psi_\alpha} \bra{\psi_\alpha},
\]
where $p_\alpha \in \mathbb{R}$ are classical probabilities to be in the state $\ket{\psi_\alpha} \in \mathcal{H}$. Note these need not be orthogonal.

If there exists $\ket x \in \mathcal{H}$ such that $\rho = \ket x \bra x$, then the system is in a \emph{pure state}\index{pure state} otherwise it is in a \emph{mixed state}\index{mixed state}.

More formally, any $\rho : \mathcal{H} \to \mathcal{H}$ satisfying:
\begin{enumerate}
	\item $\rho = \rho^{\dagger}$,
	\item $\braket{\phi | \rho | \phi} \geq 0$ for all $\ket \phi \in \mathcal{H}$, which we will write as $\rho \geq 0$,
	\item $\Tr \rho = 1$,
\end{enumerate}
is the density operator of some system.
\begin{proofbox}
	We only prove for $\dim \mathcal{H} < \infty$.

	Take a basis $\{ \ket n \}$ of eigenstates of $\rho$, such that $\rho \ket n = p_n \ket n$, and $p_n \in \mathbb{R}$ since $\rho = \rho^{\dagger}$.

	In this basis, we have
	\[
	\rho = \sum_n p_n \ket n \bra n,
	\]
	and we can normalize such that $\sum p_n = 1$.
\end{proofbox}
We can think of $\rho$ as a matrix, with $\rho_{mn} = \braket{m|\rho|n}$. Moreover all component of $\rho_{mn}$ are bounded:
\begin{itemize}
	\item The diagonal elements $\rho_{nn} \leq 1$ as $\Tr \rho = 1$.
	\item The off-diagonal elements are bounded by $\rho \geq 0$ and Cauchy-Schwarz. Let $\rho = A^{\dagger} A$. Then,
		\[
			|\rho_{mn}|^2 = |\braket{m|A^{\dagger}A|n}|^2 \leq \|A\ket m\| \|A \ket n\| = \rho_{mm} \rho_{nn} \leq \frac{1}{4}.
		\]
\end{itemize}
Moreover, a system is pure if and only if $\rho^2 = \rho$.
\begin{proofbox}
	If $\rho = \ket x \bra x$, then
	\[
	\rho^2 = \ket x \bra x \ket x \bra x = \ket x \bra x = \rho.
	\]
	On the other hand, if $\rho(1 - \rho) = 0$, then the eigenvalues of $\rho$ are either $0$ or $1$. Since $\Tr \rho = 1$, there is only one eigenvector equal to one.

	Then there exists a normalized eigenstate $\ket x \in \mathcal{H}$ such that $\rho = \ket x \bra x$.
\end{proofbox}
Equivalent conditions for purity are:
\begin{itemize}
	\item $\rho^{n} = \rho$. One way is obvious. Conversely, in a diagonal basis $p_\alpha^{n} = p_\alpha$, so $p_\alpha \in \{0, 1\}$, hence $p_\alpha^2 = p_\alpha$.
	\item $\Tr \rho^2 = 1$. One way is obvious. Conversely, $\sum p_\alpha^2 = 1$, so by Cauchy-Schwarz $p_\alpha \in \{0, 1\}$.
\end{itemize}
Moreover, $\det \rho \geq 0$.

\subsection{Quantum Mechanics with the Density Operator}
\label{sub:qm_den_op}

$\rho$ is specified by $(\dim \mathcal{H})^2 - 1$ parameters. For a system described by $\rho$, the expectation value of $Q = Q^{\dagger}$ is
\[
	\braket Q = \Tr_{\mathcal{H}} (\rho \cdot Q) = \sum_\alpha p_\alpha \Tr(\ket{\psi_\alpha} \bra{\psi_\alpha} Q) = \sum_\alpha p_\alpha \braket{\psi_\alpha|Q|\psi_\alpha}.
\]
In the Schr\"odinger picture,
\[
	\ket {\psi(t)} = U(t) \ket{\psi(0)}.
\]
Hence we have
\[
	\rho(t) = U(t) \rho(0) U^{\dagger}(t) \implies \frac{\partial}{\partial t}\rho(t) = - \frac{i}{\hbar} [H, \rho(t)].
\]
%lecture 22
In the Heisenberg picture $\rho_{H}(t) = \rho(0)$. Note that the expectation of an operator $Q$ does not depend on the picture we take:
\begin{align*}
	\braket Q_H &= \Tr[\rho_H Q_H(t)] = \Tr[\rho(0) U^{\dagger}(t) Q(0) U(t)] \\
		    &= \Tr[U(t) \rho(0) U^{\dagger}(t) Q(0)] = \Tr[\rho_S Q_S] = \braket Q_S.
\end{align*}
\begin{exbox}
	Let us take a two-state system with basis $\ket \uparrow, \ket \downarrow$. If the state is $\ket \uparrow$, then $\rho = \ket \uparrow \bra \uparrow$.

	If the classical probability to be in the state $\ket \uparrow$ or $\ket \downarrow$ is $1/2$, then
	\[
		\rho = \frac{1}{2} \ket \uparrow \bra \uparrow + \frac{1}{2} \ket \downarrow \bra \downarrow = \frac{1}{2} \mathbbm{1}_{\mathcal{H}} = \frac{1}{2}(\ket{\uparrow_x} \bra{\uparrow_x} + \ket{\downarrow_x} \bra{\downarrow_x}).
	\]
	$\rho$ gives the least information about the system. If it can be in $\ket \uparrow$ or $\ket{\uparrow_x}$ with probability $1/2$, then
	\begin{align*}
		\rho &= \frac{1}{2} ( \ket \uparrow \bra \uparrow + \ket{\uparrow_x} \bra{\uparrow_x}) \\
		     &= \frac{1}{2} \ket \uparrow \bra \uparrow + \frac{1}{4}(\ket \uparrow + \ket \downarrow)(\bra \uparrow + \bra \downarrow) \\
		     &= \frac{1}{4} \mathbbm{1}_{\mathcal{H}} + \frac{1}{2} \ket \uparrow \bra \uparrow + \frac{1}{4} \ket \uparrow \bra \downarrow + \frac{1}{4} \ket \downarrow \bra \uparrow \\
		     &=
		     \begin{pmatrix}
			     3/4 & 1/4 \\
			     1/4 & 1/4
		     \end{pmatrix}.
	\end{align*}
	More systematically, any Hermitian matrix can be written as
	\[
	\rho = \frac{1}{2}(\mathbbm{1}_{\mathcal{H}} + \mathbf{b} \cdot \bm{\sigma}) = \frac{1}{2}
	\begin{pmatrix}
		1 + b_z & b_x - i b_y \\
		b_x + i b_y & 1 - b_z
	\end{pmatrix}.
	\]
	Since $\Tr \sigma_i = 0$, the coefficient of $\mathbbm{1}_{\mathcal{H}}$ has to be $1/2$. Moreover $\rho \geq 0$ requires
	\[
	\det \rho = \frac{1}{4}(1 - \mathbf{b} \cdot \mathbf{b}) \geq 0 \implies |\mathbf{b}| \leq 1.
	\]
	This is called Bloch's sphere.
	\begin{itemize}
		\item If $|\mathbf{b}| = 1$, then the system is pure with spin along $\mathbf{b}$.
		\item If $|\mathbf{b}| < 1$, then the system is mixed as both eigenvalues are positive.
		\item For $\mathbf{b} \neq 0$, the spin is preferentially along $\mathbf{b}$.
		\item When $\mathbf{b} = 0$, $\rho = \frac{1}{2} \mathbbm{1}_{\mathcal{H}}$ is maximally ignorant about the state.
	\end{itemize}
\end{exbox}

\subsection{The No-Cloning Theorem}
\label{sub:nc_thm}

We cannot make an exact copy of a quantum system. Let's try to copy a normalized $\ket \psi \in \mathcal{H}_1$ to $\mathcal{H}_2 \sim \mathcal{H}_1$.

The initial state is $\ket \psi \otimes \ket e$ for $\ket e \in \mathcal{H}_2$, being some blank state.

Now we want a unitary operator $C^{\dagger} = C^{-1}$ such that
\[
C : \ket \psi \otimes \ket e \to e^{-i \alpha(\psi, e)} \ket \psi \otimes \ket \psi.
\]
We claim that such a $C$ does not exist. Indeed, if $C$ existed,
\begin{align*}
	\braket{\phi|\psi} &= (\bra \phi_1 \bra e_2) (\ket \phi_1 \ket e_2) = (\bra \phi_1 \bra e_2) C^{\dagger} C (\ket \phi_1 \ket e_2) \\
			   &= e^{i( \alpha(\phi, e) - \alpha(\phi, e))} \langle \phi, \psi\rangle^2.
\end{align*}
So we have $|\braket{\phi|\psi}|^2 = |\braket{\phi|\psi}|$, hence $|\braket{\phi|\psi}| = 1$ or $0$. But this cannot be true for all $\ket \psi$ and $\ket \phi \in \mathcal{H}_1$, so $C$ cannot exist.

This theorem shows that to measure a system we have to disturb it. We cannot copy it to avoid this problem.

\subsection{Von Neumann Entropy}
\label{sub:entropy}

The von Neumann entropy of a density matrix is
\[
S(\rho) = - \Tr (\rho \log \rho).
\]
It quantifies how much information $\rho$ has about the system. Note that the eigenvalues of $\rho$ are in $[0, 1]$, so $S(\rho) \geq 0$.

Moreover, $S(\rho) = 0$ if and only if $\rho$ is pure, and $S(\rho)$ is concave. If $\rho_i$ are density operators and $\kappa_i \geq 0$ are with $\sum \kappa_i = 1$, then
\[
S \biggl( \sum_i \kappa_i \rho_i \biggr) \geq \sum_i \kappa_i S(\rho_i).
\]
To maximize entropy subject to $\Tr \rho = 1$, we can use Legendre multipliers $\lambda$ and extremize $S(\rho) - \lambda(1 - \Tr \rho)$:
\begin{align*}
	0 &= - \Tr(\delta \rho \log \rho + \rho \rho^{-1} \rho - \lambda \delta \rho) \\
	0 &= \delta \lambda (\Tr \rho - 1) \\
	\implies \rho_{\mathrm{max}} &= \frac{1}{\dim \mathcal{H}} \mathbbm{1}_{\mathcal{H}}.
\end{align*}
If $\dim \mathcal{H} < \infty$, then the maximal entropy is
\[
S_{\mathrm{max}} = S(\rho_{\mathrm{max}}) = \log (\dim \mathcal{H}).
\]
All states are equally likely.

\subsection{Thermal Density Operator: Gibbs Distribution}
\label{sub:gibbs_dist}

The density operator description is essential in quantum statistical mechanics. Let's extremize $S$ at at a fixed average energy $\Tr(\rho H) = U$, for some constant $U$.

Again using Lagrange multipliers $\lambda$ and $\beta$, we solve
\[
	0 = \delta[S(\rho) - \lambda(\Tr(\rho) - 1) - \beta(\Tr(\rho H) - U)],
\]
which gives us the three equations:
\begin{align*}
	0 &= \Tr[ \delta \rho(\log \rho + 1 + \lambda + \beta H)], \\
	\Tr(\rho) &= 1,\\
	\Tr(\rho H) &= U.
\end{align*}

%lecture 23

The solution of the first equation is $\rho = e^{-\beta H} e^{-\lambda - 1}$. Note that $\beta$ and $\lambda$ are fixed by the two conditions $\Tr(\beta H) = U$ and $\Tr \rho = 1$ respectively, so we find that
\[
	\rho = \frac{e^{-\beta H}}{\Tr(e^{-\beta H})} = \frac{1}{Z[\beta]} \sum_n e^{-\beta E_n \ket n \bra n},
\]
with $Z[\beta] = \Tr[e^{-\beta H}]$ the \emph{thermal partition function}\index{thermal partition function}.

\subsection{Entanglement}
\label{sub:entanglement}

Consider the bipartite system $\mathcal{H} = \mathcal{H}_A \otimes \mathcal{H}_B$.

We are interested in $A$, while $B$ is the environment. A state $\ket \psi \in \mathcal{H}$ that can be written as $\ket \psi = \ket \phi_A \otimes \ket \chi_B$ is called a \emph{product state}\index{product state}. Otherwise, it is called an \emph{entangled state}\index{entangled state}.

\begin{exbox}
	If $\mathcal{H}_A$ and $\mathcal{H}_B$ are qubits, then
	\[
		\ket \psi = \ket \uparrow \ket \downarrow \qquad \text{ and } \qquad \ket{\tilde \psi} = \biggl( \frac{\ket \uparrow + \ket \downarrow}{\sqrt 2} \biggr) \ket \downarrow
	\]
	are product states, while
	\[
		\ket{\mathrm{EPR}} = \frac{\ket \uparrow \ket \downarrow - \ket \downarrow \ket \uparrow}{\sqrt 2}
	\]
	is entangled.
\end{exbox}

\subsection{Reduced Density Operator}
\label{sub:reduced_density}

The reduced density operator $\rho_A$ for the subsystem $A$ is
\[
	\rho_A = \Tr_{\mathcal{H}_B}(\rho_{AB}) = \sum_n \braket{\beta_n | \rho_{AB} | \beta_n}.
\]
$\rho_A$ is useful to compute observables that depend only on $A$.

\begin{exbox}
	If $Q = Q_A \otimes \mathbbm{1}_{B}$, then
	\[
		\braket Q = \Tr_{\mathcal{H}_A \otimes \mathcal{H}_B} (\rho_{AB} Q) = \Tr_{\mathcal{H}_A} (\rho_A Q_A).
	\]
	For example, if
	\[
	\ket \psi = \frac{1}{\sqrt 2}( \ket \uparrow + \ket \downarrow)_A \otimes \ket \downarrow_B \in \mathcal{H},
	\]
	then we have
	\[
	\rho_A = \Tr_{\mathcal{H}_B}(\rho_{AB}) = \frac{1}{2} (\ket \uparrow + \ket \downarrow)(\bra \uparrow + \bra \downarrow) = \frac{1}{2}
	\begin{pmatrix}
		1 & 1 \\ 1 & 1
	\end{pmatrix}
	= \rho_A^2,
	\]
	so $\rho_A$ is pure. For the entangled state
	\[
		\ket{\mathrm{EPR}} = \frac{1}{\sqrt 2}(\ket\uparrow_A \ket\downarrow_B - \ket\downarrow_A\ket\uparrow_B),
	\]
	we have
	\[
		\rho_A = \Tr_{\mathcal{H}_B}(\ket{\mathrm{EPR}} \bra{\mathrm{EPR}}) = \frac{1}{2}
		\begin{pmatrix}
			1 & 0 \\ 0 & 1
		\end{pmatrix}
		\neq \rho_A^2,
	\]
	so $\rho_A$ is mixed.
\end{exbox}

\subsection{Entanglement Entropy}
\label{sub:ent_entropy}

Entanglement entropy quantifies entanglement between two subsystems. It is defined as
\[
S_A(\rho_A) = - \Tr_{\mathcal{H}_A}(\rho_A \log \rho_A).
\]
It is just the von Neumann entropy of $\rho_A$.

Schmidt decomposition says that any $\ket \psi \in \mathcal{H}_A \otimes \mathcal{H}_B$ can be written as
\[
	\ket \psi = \sum_{i = 1}^{N} \alpha_i \ket{\phi_i} \otimes \ket{\chi_i},
\]
with $\{\ket{\phi_i}\}$ and $\{\ket{\chi_i}\}$ being orthonormal subsets in $\mathcal{H}_A$ and $\mathcal{H}_B$ respectively, $\alpha_i \in \mathbb{R}$ and $N = \min(\dim \mathcal{H}_A, \dim \mathcal{H}_B)$.

This is much simpler than writing
\[
	\ket \psi = \sum_{i, \alpha} c_{i, \alpha} \ket{e_i}_A \otimes \ket{f_\alpha}_B.
\]
Using this, $\rho_A$ or $\rho_B$ are diagonal:
\[
	\rho_A = \sum_k \sum_i \braket{\chi_k| \alpha_i \ket{\phi_i} \ket{\chi_i} \bra{\phi_i} \bra{\chi_i} a_i^{\ast} | \chi_k}_B = \sum_i |\alpha_i|^2 \ket{\phi_i} \bra{\phi_i}.
\]

Assume that $\rho_{AB}$ is pure. Then we claim that $S_A = S_B$. Indeed, using Schmidt decomposition,
\[
S_A = \sum_i |\alpha_i|^2 \log |\alpha_i|^2 = S_B.
\]
Since $S$ is basis independent, $S_A = S_B$.

Moreover, $\rho_A$ is pure if and only if $\rho_{AB}$ is a product state. Indeed, if $S_A = 0$, then $\alpha_i = \{1, 0, 0, \ldots\}$ so $\ket \psi = \alpha_1 \ket{\phi_1} \otimes \ket{\chi_1}$, and $\rho_{AB}$ is in a product state.

Conversely, $\rho_A$ is mixed if and only if $\rho_{AB}$ is entangled.

If $A$ and $B$ are entangled, we lose information about $A$ when tracing over $B$.

\subsection{Sub-additivity}
\label{sub:subadd}

Sub-additivity states that
\[
S_{A \cup B} \leq S_A + S_B.
\]
Hence we introduce the \emph{quantum mutual information}\index{quantum mutual information} $I$:
\[
I(A:B) = S_A + S_B - S_{A \cup B} \geq 0.
\]
The proof uses Jensen's inequality.

Strong sub-additivity says that
\[
S_{A \cup B \cup C} \leq S_{A \cup B} + S_{B \cup C} - S_B.
\]

\section{Measurement}
\label{sec:measurement}

\subsection{Decoherence}
\label{sub:decoherence}

What happens when we measure in quantum mechanics?

Assume that we are in a pure and separable state at $t = 0$, so $\rho(0) = \ket{\psi_0} \bra{\psi_0}$ for some $\ket{\psi_0} = \ket \phi \otimes \ket \chi \in \mathcal{H} = \mathcal{H}_A \otimes \mathcal{H}_B$.

Then $\rho_A(0) = \ket \phi \bra \phi = \rho_A^2$ is pure. The time evolution of $\rho$ is
\[
	\rho(t) = U_{AB}(t) \rho(0) U_{AB}^{\dagger}(t) = U_{AB}(t) \ket{\psi_0} \bra{\psi_0} U_{AB}^{\dagger}(t).
\]
If $U_{AB}(t) = U_A(t) \otimes U_B(t)$, then $\rho_A(t)$ remains pure. Conversely, for general $U_{AB} \neq U_A \otimes U_B$, we have
\[
	\rho_A(t) = \Tr_{\mathcal{H}_B}(U_{AB} \ket{\psi_0} \bra{\psi_0} U_{AB}^{\dagger}) = \sum_\beta M_\beta(t) \rho_0 M_\beta^{\dagger}(t),
\] 
where $M_\beta(t) = \braket{\beta|U_{AB}(t)|\chi}$ and $\{\ket \beta\}$ is a basis of $\mathcal{H}_B$, and with $M_\beta(t) : \mathcal{H}_A \to \mathcal{H}_A$ and $\sum_\beta M_\beta^{\dagger} M_\beta = \mathbbm{1}_{\mathcal{H}_A}$, using $\braket{\chi|\chi} = 1$.

$\rho_A(t)$ is in general mixed, so interaction leads to entanglement. This is \emph{decoherence}\index{decoherence}.

%lecture 24

\subsection{Decoherence and Measurement}
\label{sub:dec_meas}

Let $A = \spn \{ \ket \uparrow, \ket \downarrow \}$ and $B = \spn \{ \ket 0, \ket 1, \ket 2\}$. Here $A$ represents our state we want to measure, and $B$ is our measuring instrument.

An ideal measurement of $A$ will change $B$, but not $A$. Suppose our measurement is the operator $U$, which acts as
\begin{align*}
	U(\ket \uparrow \otimes \ket 0) &= \ket \uparrow ( \sqrt{1 - p} \ket 0 + \sqrt{p} \ket 1), \\
	U(\ket \downarrow \otimes \ket 0) &= \ket \downarrow(\sqrt{1 - p} \ket 0 + \sqrt p \ket 2),
\end{align*}
for some probability $0 \leq p \leq 1$. Then
\begin{align*}
	M_0 &= \braket{0|U|0} = \sqrt{1 - p} \mathbbm{1}_{\mathcal{H}_A}, \\
	M_1 &= \braket{1|U|1} = \sqrt p \ket \uparrow \bra \uparrow, \\
	M_2 &= \braket{2|U|0} = \sqrt p \ket \downarrow \bra \downarrow.
\end{align*}
$\rho_A$ evolves as
\[
\rho_A(0) =
\begin{pmatrix}
	\rho_{\uparrow \uparrow} & \rho_{\uparrow \downarrow} \\
	\rho_{\downarrow \uparrow} & \rho_{\downarrow \downarrow}
\end{pmatrix}
\to \sum_{\beta} M_\beta \rho_A(0) M_\beta^{\dagger} =
\begin{pmatrix}
	\rho_{\uparrow \uparrow} & (1 - p)\rho_{\uparrow \downarrow} \\
	(1-p) \rho_{\downarrow \uparrow} & \rho_{\downarrow \downarrow}
\end{pmatrix}.
\]
Suppose $U$ acts over time $\delta t$, and define the rate as $\Gamma = p/\delta t$. After a time of $t = N \delta t$, the off diagonal elements of $\rho_A$ become
\[
	(1 - p)^N = \biggl(1 - \frac{\Gamma t}{N} \biggr)^n \to e^{- \Gamma t}.
\]
This is called \emph{phase damping}\index{phase damping}: if $A$ is initially $\ket \psi = a \ket \uparrow + b \ket \downarrow$, then
\[
\lim_{t \to \infty} \rho_A(t) = \lim_{t \to \infty}
\begin{pmatrix}
	|a|^2 & e^{- \Gamma t} a \bar b \\
	\bar a b e^{-\Gamma t} & |b|^2
\end{pmatrix} =
\begin{pmatrix}
	|a|^2 & 0 \\ 0 & |b|^2
\end{pmatrix}.
\]
As $A$ entangles with $B$, we are unlikely to find $A$ in a superposition of states in the bases preferred by the apparatus.

\subsection{EPR Paradox}
\label{sub:epr_paradox}

A famous debate in physics is whether quantum mechanics is probabilistic because of some hidden aspects of reality.

Einstein, Podolsky and Rosen devised a thought experiment to argue that quantum mechanics is an incomplete description of reality.

In Bohr's version of the EPR paradox, we start with and electron-positron pair in the spin-0 state
 \[
	 \ket{\mathrm{EPR}} = \frac{1}{\sqrt 2} ( \ket{\uparrow \downarrow} - \ket{\downarrow \uparrow}) = \frac{1}{\sqrt 2}(\ket{\uparrow_{\mathbf{a}}\downarrow_{\mathbf{a}}} - \ket{\downarrow_{\mathbf{a}}\uparrow_{\mathbf{a}}}).
\]
The electron travels to Alice, and the positron to Bob.

Alice measures the spin along $\mathbf{a}$, If $A$ measures $\hbar/2$, then
\[
	\ket{\mathrm{EPR}} \to \ket{\uparrow_{\mathbf{a}} \downarrow_{\mathbf{a}}} = \ket{\uparrow_{\mathbf{a}}} \otimes \ket{\downarrow_{\mathbf{a}}}.
\]
Bob measures the spin along $\mathbf{b}$. Since Bob's state $\ket{\uparrow_{\mathbf{b}}}$ is
\[
	\ket{\uparrow_{\mathbf{b}}} = \cos \frac{\theta}{2} e^{-i \phi / 2} \ket{\uparrow_{\mathbf{a}}} + \sin \frac{\theta}{2} e^{i \phi/2} \ket{\downarrow_{\mathbf{a}}},
\]
the probability that Bob finds the positron in $\ket{\uparrow_{\mathbf{b}}}$ is
\[
	\mathbb{P}(B \text{ measures } \ket{\uparrow_{\mathbf{b}}}) = |\braket{\uparrow_{\mathbf{b}}|\downarrow_{\mathbf{a}}}|^2 = \sin^2 \frac{\theta}{2}.
\]
In particular, after Alice measures, Bob will never find $\hbar/2$ along $\mathbf{a}$.

Einstein found the causality of this paradoxical, as the electron and neutrino could have been galaxies apart, yet one measurement still affects the other.

As a rebuttal, even if Alice and Bob are spacelike separated, there is no notion of causality. Indeed the probability is completely symmetric.

Suppose we could find hidden variables $\mathbf{v} \in \mathbb{R}^{n}$ that could completely determine the spins. Let $p(\mathbf{v}) \Diff n \mathbf{v}$ be the probability for $\mathbf{v}$ to be in $\mathbf{v} + \Diff n \mathbf{v}$.

We now look at the spins of the electron along their respective directions. By conservation of angular momentum,
\[
s_{\mathrm{e}}(\mathbf{a}, \mathbf{v}) + s_{\mathrm{p}}(\mathbf{a}, \mathbf{v}) = 0.
\]
Hence looking at the average value,
\begin{align*}
	\langle s_{\mathrm{e}}(\mathbf{a}) s_{\mathrm{p}}(\mathbf{b})\rangle &= \int p(\mathbf{v}) s_{\mathrm{e}}(\mathbf{a}, \mathbf{v}) s_{\mathrm{p}}(\mathbf{b}, \mathbf{v}) \Diff n \mathbf{v} \\
									     &= - \int p(\mathbf{v}) s_{\mathrm{e}}(\mathbf{a}, \mathbf{v}) s_{\mathrm{e}}(\mathbf{b}, \mathbf{v}) \Diff n \mathbf{v}.
\end{align*}

\subsection{Bell's Inequality}
\label{sub:bell}

Notice that $s_{\mathrm{e}}(\mathbf{a}, \mathbf{v})^2 = \frac{\hbar^2}{4}$, for all $\mathbf{a}$.

Suppose Bob measures the spin along $\mathbf{b}'$ instead of $\mathbf{b}$. Using this fact, we have
\begin{align*}
	\langle s_{\mathrm{e}}(\mathbf{a}) s_{\mathrm{p}}(\mathbf{b}) \rangle - \langle s_{\mathrm{e}}(\mathbf{a}) &s_{\mathrm{p}}(\mathbf{b}') \rangle = \int p(\mathbf{v}) s_{\mathrm{e}}(\mathbf{a}, \mathbf{v}) \bigl[ s_{\mathrm{p}}(\mathbf{b}, \mathbf{v}) - s_{\mathrm{p}}(\mathbf{b}', \mathbf{v}) \bigr] \Diff n \mathbf{v} \\
																		       &= \int p(\mathbf{v}) s_{\mathrm{e}}(\mathbf{a}, \mathbf{v}) s_{\mathrm{p}}(\mathbf{b}, \mathbf{v}) \biggl[ 1 - \frac{4}{\hbar^2} s_{\mathrm{p}}(\mathbf{b}, \mathbf{v}) s_{\mathrm{p}}(\mathbf{b}', \mathbf{v}) \biggr] \Diff n \mathbf{v}.
\end{align*}

Notice that the quantity $[1 - 4 s_{\mathrm{p}}(\mathbf{b}, \mathbf{v}) s_{\mathrm{p}}(\mathbf{b}', \mathbf{v})/\hbar^2] \geq 0$, and the quantity $s_{\mathrm{e}}(\mathbf{a}, \mathbf{v})s_{\mathrm{p}}(\mathbf{b}, \mathbf{v})$ fluctuates between $\pm \hbar^2/4$.

Replacing this fluctuation by $\hbar^2/4$ gives Bell's inequality:
\begin{align*}
	|\langle s_{\mathrm{e}}(\mathbf{a}) s_{\mathrm{p}}(\mathbf{b}) \rangle - \langle s_{\mathrm{e}}(\mathbf{a}) s_{\mathrm{p}}(\mathbf{b}') \rangle| &\leq \int p(\mathbf{v}) \biggl[\frac{\hbar^2}{4} - s_{\mathrm{p}}(\mathbf{b}, \mathbf{v}) s_{\mathrm{p}}(\mathbf{b}', \mathbf{v}) \biggr] \Diff n \mathbf{v} \\
																			 &= \frac{\hbar^2}{4} - \langle s_{\mathrm{p}}(\mathbf{b}) s_{\mathrm{p}}(\mathbf{b}')\rangle.
\end{align*}
This is a bound on theories of hidden variables.

However, quantum mechanics violates Bell's inequalities. Note that
\begin{align*}
	(\mathbf{a} \cdot \mathbf{S}_{\mathrm{e}} \otimes \mathbbm{1}_{\mathrm{p}})(\mathbbm{1}_{\mathrm{e}} \otimes \mathbf{b} \cdot \mathbf{S}_{\mathrm{p}})\ket{\mathrm{EPR}} &= -(\mathbbm{1}_{\mathrm{e}} \otimes \mathbf{a} \cdot \mathbf{S}_{\mathrm{p}})(\mathbbm{1}_{\mathrm{e}} \otimes \mathbf{b} \cdot \mathbf{S}_{\mathrm{p}}) \ket{\mathrm{EPR}} \\
																						 &= - \mathbbm{1}_{\mathrm{e}} \otimes(\mathbf{a} \cdot \mathbf{S}_{\mathrm{p}} \, \mathbf{b} \cdot \mathbf{S}_{\mathrm{p}}) \ket{\mathrm{EPR}}.
\end{align*}
We now have that
\[
\mathbf{a} \cdot \mathbf{S} \, \mathbf{b} \cdot \mathbf{S} = \frac{\hbar^2}{4} \mathbf{a} \cdot \mathbf{b} \mathbbm{1}_{\mathcal{H}} + \frac{i \hbar}{2} (\mathbf{a} \times \mathbf{b}) \cdot \mathbf{S}.
\]
However, $\braket{\mathrm{EPR}|\mathbf{S}_{\mathrm{p}}|\mathrm{EPR}} = 0$. Hence,
\[
	\braket{\mathbf{a} \cdot \mathbf{S}_{\mathrm{p}} \, \mathbf{b} \cdot \mathbf{S}_{\mathrm{p}}} = \frac{\hbar^2}{4} \mathbf{a} \cdot \mathbf{b}.
\]

Now in Bell's inequality, the left hand side comes to
\[
|\langle\mathbf{a} \cdot \mathbf{S}_{\mathrm{p}} \, \mathbf{b} \cdot \mathbf{S}_{\mathrm{p}} \rangle - \langle \mathbf{a} \cdot \mathbf{S}_{\mathrm{p}} \, \mathbf{b}' \cdot \mathbf{S}_{\mathrm{p}}\rangle| = \frac{\hbar^2}{4}|\mathbf{a} \cdot (\mathbf{b} - \mathbf{b}')|,
\]
and the right hand side is
\[
\frac{\hbar^2}{4} - \langle\mathbf{b} \cdot \mathbf{S}_{\mathrm{p}} \, \mathbf{b}' \cdot \mathbf{S}_{\mathrm{p}} \rangle = \frac{\hbar^2}{4}(1 - \mathbf{b} \cdot \mathbf{b}').
\]
In particular, for $\mathbf{a} = \mathbf{\hat x}$, $\mathbf{b} = \mathbf{\hat y}$ and $\mathbf{b'} = (\mathbf{\hat x} + \mathbf{\hat y})/\sqrt 2$, Bell's inequality is violated.

Experimentally testing the CHSH inequality shows that it is indeed violated. Quantum mechanics wins.


%Let $p$ be a prime. Call a sequence $a_0,a_1,\ldots$ of integers \emph{interesting} if $p\mid a_0$ and \begin{itemize} \item there exists some integer $k$ such that for all positive integers $i$, we have $a_{i+1}=ka_i-a_{i-1}$. \item there exist integers $k$ and $l$ such that $p\nmid l$ and for all positive integers $i$ we have $a_{i+1}=ka_i+la_{i-1}$. \end{itemize} Let $\{a_i\}$ be an interesting sequence. Prove that the subsequence of $\{a_i\}$ containing the elements divisible by $p$ is also interesting.

%Around a circle are written $2024$ positive integers, such that each number divides the sum of its two neighbours and the smallest number is $1$. Prove that the largest number is at most \begin{itemize} \item $2^{2024}$. \item $2^{1600}$. \item $7^{506}$. \end{itemize}

\newpage

\printindex

\end{document}
