\documentclass[12pt]{article}

\usepackage{ishn}

\makeindex[intoc]

\begin{document}

\hypersetup{pageanchor=false}
\begin{titlepage}
	\begin{center}
		\vspace*{1em}
		\Huge
		\textbf{II Algebraic Geometry}

		\vspace{1em}
		\large
		Ishan Nath, Lent 2024

		\vspace{1.5em}

		\Large

		Based on Lectures by Prof. Mark Gross

		\vspace{1em}

		\large
		\today
	\end{center}
	
\end{titlepage}
\hypersetup{pageanchor=true}

\tableofcontents

\newpage

\section{Affine Varieties}
\label{sec:aff_var}

\subsection{Algebraic Sets}
\label{sub:alg_set}

Our basic setup is as follows: we begin by fixing a field $\mathbb{K}$.

\begin{definition}
	The \emph{affine $n$-space}\index{affine $n$-space} over $\mathbb{K}$ is
	\[
	\mathbb{A}^{n} = \mathbb{K}^n.
	\]
	Let $A = \mathbb{K}[x_1, \ldots, x_n]$, and $S \subseteq A$. Set
	\[
		Z(S) = \{(a_1, \ldots, a_n) \in \mathbb{A}^n \mid f(a_1, \ldots, a_n) =0 \quad \forall f \in S\}.
	\]
\end{definition}

\begin{proposition}
	\begin{enumerate}[\normalfont(a)]
		\item[]
		\item $Z(\{0\}) = \mathbb{A}^n$.
		\item $Z(A) = \emptyset$.
		\item $Z(S_1 \cdot S_2) = Z(S_1) \cup Z(S_2)$, where $S_1 \cdots S_2 = \{f \cdot g \mid f \in S_1, g \in S_2\}$.
		\item Let $I$ be an indexing set, and suppose for each $i \in I$, we are given $S_i \subseteq A$. Then
			\[
			\bigcap_{i \in I}Z(S_i) = Z \left( \bigcup_{i \in I}S_i \right).
			\]
	\end{enumerate}
\end{proposition}

\begin{proofbox}


	(a) and (b) are obvious.

	(c) If $p \in Z(S_1) \cup Z(S_2)$, then either $f(p) = 0$ for all $f \in S_1$, or $g(p) = 0$ for all $g \in S_2$. Thus $(f \cdot g)(p) = 0$ for all $f \in S_1, g \in S_2$, hence $p \in Z(S_1 \cdot S_2)$.

	Conversely, suppose $p \in Z(S_1 \cdot S_2)$, and $p \not \in Z(S_1)$. So there exists $f \in S_1$ with $f(p) \neq 0$. But $(f \cdot g)(p) = 0$ for all $g \in S_2$, and $f(p) \neq 0$. So $g(p) = 0$ for all $g \in S_2$, thus $p \in Z(S_2)$.

	(d) If $p \in Z(S_i)$ for all $i$, then $p \in Z(\bigcup S_i)$.

	Conversely, if $p \in Z(\bigcup S_i)$, then $p \in Z(S_i)$ for all $i$.
\end{proofbox}

This says that the sets of the form $Z(S)$ form the closed sets of a topology on $\mathbb{A}^n$.

\begin{definition}
	A subset of $\mathbb{A}^n$ is \emph{algebraic}\index{algebraic} if it is of the form $Z(S)$ for some $S \subseteq A$.

	A \emph{Zariski open subset}\index{Zariski open subset} of $\mathbb{A}^n$ is a set of the form
	\[
	\mathbb{A}^n \setminus Z(S),
	\]
	for some $S \subseteq A$. This forms the \emph{Zariski topology}\index{Zariski topology} on $\mathbb{A}^n$.
\end{definition}

\begin{exbox}
	\begin{enumerate}
		\item If $\mathbb{K} = \mathbb{C}$, the Zariski open or closed subsets are also open or closed in the ``usual'' topology.
		\item For any $\mathbb{K}$, consider $\mathbb{A}^1$, and $S \subseteq K[x]$ containing a non-zero element. Then $Z(S)$ is finite.

			So the Zariski closed sets are $\mathbb{A}^1$ and all finite sets, so this is equivalent to the cofinite topology.
	\end{enumerate}	
\end{exbox}

Recall that if $A$ is a commutative ring and $S \subseteq A$ is a subset, the ideal generated by $S$ is the ideal $\langle S \rangle \subseteq A$ given by
\[
	\langle S \rangle = \left\{ \sum_{i = 1}^q f_i g_i \bigm| q \geq 0, f_i \in S, g_i \in A\right\}.
\]
This is the smallest ideal of $A$ containing $S$.

\begin{lemma}
	Let $S \subseteq \mathbb{K}[x_1, \ldots, x_n]$ and $I = \langle S \rangle$. Then
	\[
	Z(S) = Z(I).
	\]
\end{lemma}

\begin{proofbox}
	If $p \in Z(S)$, let $f_1, \ldots, f_q \in S$ and $g_1, \ldots, g_q \in A$. Then
	\[
	\sum_{i = 1}^q (f_i g_i)(p) = \sum_{i = 1}^q f_i(p) g_i(p) = 0.
	\]
	Thus $Z(S) \subseteq Z(I)$.

	But conversely, since $S \subseteq I$, $Z(I) \subseteq Z(S)$. So $Z(S) = Z(I)$.
\end{proofbox}

%lecture 2

\begin{definition}
	Let $X \subseteq \mathbb{A}^n$ be a subset. Define
	\[
		I(X) = \{f \in A = \mathbb{K}[x_1, \ldots, x_n] \mid f(p) = 0 \quad \forall p \in X\}.
	\]
\end{definition}

\begin{remark}
	$I(X)$ is an ideal: if $f, g \in I(X)$, then $f + g \in I(X)$, and if $f \in A$, $g \in I(X)$, then $f \cdot g \in I(X)$.

	Moreover, if $S_1 \subseteq S_2 \subseteq A$, then $Z(S_2) \subseteq Z(S_1)$, and if $X_1 \subseteq X_2 \subseteq \mathbb{A}^n$, then $I(X_2) \subseteq I(X_1)$.
\end{remark}

An intuitive thing to consider is the relationship between an ideal $I$ and $I(Z(I))$.

\begin{exbox}
	Take $I = \langle x^2 \rangle \subseteq \mathbb{K}[x]$.

	Then $Z(I) = \{0\} \subseteq \mathbb{A}^1$, but $I(Z(I)) = I(\{0\}) = \langle x \rangle \neq I$.
\end{exbox}

\begin{definition}
	Let $I \subseteq A$ be an ideal in the commutative ring $A$. The \emph{radical}\index{radical} of $I$ is
	\[
		\sqrt I = \{f \in A \mid f^n \in I \text{ for some } n > 0\}.
	\]
\end{definition}

\begin{lemma}
	$\sqrt I$ is an ideal.
\end{lemma}

\begin{proofbox}
	Suppose $f, g \in \sqrt I$, say $f^{n_1}, g^{n_2} \in I$. Then,
	\[
		(f + g)^{n_1 + n_2} = \sum_{i = 1}^{n_1 + n_2} \binom{n_1 + n_2}{i} f^{i} g^{n_1 + n_2 - i}
	\]
	For each $i$, either $i \geq n_1$ or $n_1 + n_2 - i \geq n_2$. Therefore each term lies in $I$, hence $(f + g)^{n_1 + n_2} \in I$. Hence $f + g \in \sqrt I$.

	Now if $f \in \sqrt I$ and $g \in A$, then $f^n \in I$ for some $n$. So $(fg)^n = f^n g^n \in I$, so $fg \in \sqrt I$.
\end{proofbox}

\begin{proposition}
	\begin{enumerate}[\normalfont(a)]
		\item[]
		\item If $X \subseteq \mathbb{A}^n$ is algebraic, then
			\[
			Z(I(X)) = X.
			\]
		\item If $I \subseteq A$ is an ideal, then
			\[
			\sqrt I \subseteq I(Z(I)).
			\]
	\end{enumerate}
\end{proposition}

\begin{proofbox}
	
	\begin{enumerate}[(a)]
		\item Since $X$ is algebraic, $X = Z(I)$ for some ideal $I$. Certainly, $I \subseteq I(X)$, by definition of $Z$ and $I(X)$. Thus $Z(I(X)) \subseteq Z(I) = X$. But we clearly have $X \subseteq Z(I(X))$.
		\item If $f^n \in I$, then $f^n$ vanishes in $Z(I)$, and hence $f$ vanishes on $Z(I)$ also. So $f \in I(Z(I))$, hence $\sqrt I \subseteq I(Z(I))$.
	\end{enumerate}
\end{proofbox}

\begin{theorem}[Hilbert's Nullstellensatz]
	Let $\mathbb{K}$ be an algebraically closed field. Then
	\[
	\sqrt I = I(Z(I)).
	\]
\end{theorem}

\begin{exbox}
	Take $\mathbb{K} = \mathbb{R}$, and $I = \langle x^2 + y^2 + 1 \rangle \subseteq \mathbb{R}[x, y]$.

	But now $Z(I) = \emptyset$, however $I(Z(I)) = \mathbb{R}[x, y] \neq \sqrt I$.
\end{exbox}

This shows why we need an algebraically closed field: sometimes the zero set cannot properly capture the detail of the algebra, for example if the variety has no solutions.

\subsection{Irreducible Subsets}
\label{sub:irr_sub}

\begin{definition}
	Let $X$ be a topological space, and $Z \subseteq X$ a closed subset. We say $Z$ is \emph{irreducible}\index{irreducible subset} if $Z$ is non-empty, and whenever $Z = Z_1 \cup Z_2$ with $Z_1, Z_2$ closed, then either $Z = Z_1$ or $Z = Z_2$.
\end{definition}

\begin{remark}
	This is a bad notion in the Euclidean topology in $\mathbb{C}^n$. The only irreducible sets are points.
\end{remark}

\begin{exbox}
	$\mathbb{A}^1$ is irreducible as long as $\mathbb{K}$ is infinite.
\end{exbox}

\begin{definition}
	An (affine, algebraic) \emph{variety}\index{variety}\index{affine variety}\index{algebraic variety} in $\mathbb{A}^n$ is an irreducible algebraic set.
\end{definition}

We are now interesting in recognizing irreducible algebraic sets algebraically.

\begin{proposition}
	If $X_1, X_2 \subseteq \mathbb{A}^n$, then $I(X_1 \cup X_2) = I(X_1) \cap I(X_2)$.
\end{proposition}

\begin{proofbox}
	Since $X_1, X_2 \subseteq X_1 \cup X_2$, we have $I(X_1 \cup X_2) \subseteq I(X_1), I(X_2)$. So $I(X_1 \cup X_2) \subseteq I(X_1) \cap I(X_2)$.

	Conversely, if $f \in I(X_1) \cap I(X_2)$, then $f \in I(X_1 \cup X_2)$.
\end{proofbox}

Recall that an ideal $P \subseteq A$ is \emph{prime}\index{prime ideal} if $P \neq A$, and whenever $f \cdot g \in P$, then either $f \in P$ or $g \in P$.

\begin{lemma}
	Let $P \subseteq A$ be prime, and $I_1, \ldots, I_n \subseteq A$ be ideals. Suppose that $P \supseteq \bigcap I_i$. Then $P \supseteq I_i$ for some $i$. Moreover, if $P = \bigcap I_i$, then $P = I_i$ for some $i$.
\end{lemma}

\begin{exbox}
	Take $A = \mathbb{Z}$, and $P = \langle p\rangle$ for $p$ a prime number, and $I_i = \langle n_i \rangle$. Then,
	\[
	\bigcap_i I_i = \langle \lcm(n_1, \ldots, n_s)\rangle.
	\]
	Now note that
	\[
		P \supseteq \bigcap_i I_i \iff p \mid \lcm(n_1, \ldots, n_s) \iff p \mid n_i \text{ for some } i.
	\]
\end{exbox}

\begin{proofbox}
	Suppose $P \not \supseteq I_i$ for any $i$. Then we can find $x_i \in I_i$ such that $x_i \not \in P$. But now
	\[
	\prod_{i = 1}^n x_i \in \bigcap_{i = 1}^n I_i \subseteq P,
	\]
	so there exists $x_i \in P$, which gives a contradiction.

	If $P = \bigcap I_i$, then $P \subseteq I_i$ for each $i$. We know that $I_i \subseteq P$ for some $i$, hence $P = I_i$ for some $i$.
\end{proofbox}

Here's the main point.

\begin{proposition}
	Let $K$ be algebraically closed. Then an algebraic set $X \subseteq \mathbb{A}^n$ is irreducible if and only if $I(X) \subseteq A = \mathbb{K}[x_1, \ldots, x_n]$ is prime.
\end{proposition}

%lecture 3

\begin{proofbox}
	

	$\implies$ If $f \cdot g \in I(X)$, then $X \subseteq Z(f \cdot g) = Z(f) \cup Z(g)$. Thus
	\[
	X = (X \cap Z(f)) \cup (X \cap Z(g)),
	\]
	so if $X$ is irreducible, then without loss of generality, we can assume $X = X \cap Z(f)$, so $X \subseteq Z(f)$. Hence $f \in I(X)$.

	$\impliedby$ If $P \subseteq A = \mathbb{K}[x_1, \ldots, x_n]$ is prime, suppose $Z(P) = X_1 \cup X_2$ where $X_1, X_2$ are closed. Then
	\[
	I(x_1) \cap I(X_2) = I(X_1 \cup X_2) = I(Z(P)) = \sqrt P,
	\]
	by Hilbert's Nullstellensatz. But note $\sqrt P = P$: if $f^n \in P$, then $f \in P$ by the primality of $P$. Therefore, $I(X_1) \cap I(X_2) = P$. So by our lemma, either $P = I(X_1)$ or $P = I(X_2)$, so $Z(P) = X_1$ or $Z(P) = X_2$.
\end{proofbox}

We now have a one-to-one correspondence between prime ideals of $A$, and varieties in $\mathbb{A}^n$, given by our maps $Z$ and $I$.

\begin{proposition}
	Any algebraic set in $\mathbb{A}^n$ can be written as a finite union of varieties.
\end{proposition}

\begin{proofbox}
	Let $\mathcal{S}$ be the set of all algebraic sets in $\mathbb{A}^n$ which cannot be written as a finite union of varieties. If $\mathcal{S} \neq \emptyset$, then I claim it has a minimal element with respect to inclusion. Otherwise, there exists $X_1, X_2, X_3, \ldots \in \mathcal{S}$ with
	\[
	X_1 \supset X_2 \supset X_3 \supset \cdots,
	\]
	and $X_i \neq X_{i+1}$. Now note that
	\[
	I(X_1) \subset I(X_2) \subset I(X_3) \subset \cdots \subseteq A.
	\]
	But note that $A = \mathbb{K}[x_1, \ldots, x_n]$ is Noetherian by Hilbert's basis theorem, so this is a contradiction.

	Let $X \in \mathcal{S}$ be minimal. Now $X$ is not irreducible, as otherwise $X$ is itself a variety. Otherwise, we can write $X = X_1 \cup X_2$ with $X_1 \subset X, X_2 \subset X$ with $X_1, X_2$ algebraic. Thus $X_1, X_2 \not \in \mathcal{S}$, hence they can be written as a union of irreducible sets, so $X$ can also be written as a finite union of irreducibles, so $X \not \in \mathcal{S}$, contradiction.
\end{proofbox}

\begin{definition}
	If $X = X_1 \cup \cdots \cup X_n$ with $X, X_i$ algebraic, $X_i$ irreducible and $X_i \not \subseteq X_j$ for any $i \neq j$, then we say $X_1, \ldots, X_n$ are the \emph{irreducible components}\index{irreducible components} of $X$.
\end{definition}

\begin{exbox}
	\begin{enumerate}
		\item In $\mathbb{A}^2$, $A = \mathbb{K}[x_1, x_2]$. Then
			\[
			X = Z(x_1 x_2) = Z(x_1) \cup Z(x_2).
			\]
		\item More generally, $A = \mathbb{K}[x_1, \ldots, x_n]$ is a UFD. So for $0 \neq f \in A$, we write $f = \prod f_i^{a_i}$, with $f_i$ irreducible. Since $A$ is a UFD, $\langle f_i \rangle$ is prime. Hence $Z(f_i)$ is irreducible, so
			\[
			Z(f) = Z(f_1) \cup \cdots \cup Z(f_n)
			\]
			is the irreducible decomposition of $Z(f)$.
		\item $Z(x_2^2 - x_1^3 + x_1)$ is irreducible.
	\end{enumerate}
\end{exbox}

\subsection{Regular and Rational Functions}
\label{sub:reg_rat_fs}

In algebraic geometry, polynomial functions are natural. Let $X \subseteq \mathbb{A}^n$ be an algebraic set, and $f \in A = \mathbb{K}[x_1, \ldots, x_n]$. This gives a function $f : \mathbb{A}^n \to K$.

This naturally gives $f|_X : X \to \mathbb{K}$. Hence if $f, g \in A$ with $f|_X = g|_X$, then $f - g$ vanishes on $X$. So $f - g \in I(X)$. So it is natural to think of $A/I(X)$ as being the set of polynomial functions on $X$.

\begin{definition}
	Let $X \subseteq \mathbb{A}^n$ be an algebraic set. The \emph{coordinate ring}\index{coordinate ring} of $X$ is
	\[
	A(X) = A/I(X).
	\]
\end{definition}

\begin{definition}
	Let $X \subseteq \mathbb{A}^n$ be an algebraic set, and $U \subseteq X$ an open subset. A function $f : U \to \mathbb{K}$ is \emph{regular}\index{regular} if for all $p \in U$ there exists an open neighbourhood $V \subseteq U$ of $p$ and functions $g, h \in A(X)$ with $h(q) \neq 0$ for any $q \in V$, and $f = g/h$ on $V$.
\end{definition}

A regular function is locally a rational function, but different points may require different representations.

\begin{exbox}
	Any $f \in A(X)$ defines a regular function on $X$.
\end{exbox}

\begin{definition}
	We write
	\[
		\mathcal{O}_X(U) = \{f : U \to \mathbb{K} \mid f \text{ regular}\}.
	\]
\end{definition}

Note that $\mathcal{O}_X(U)$ is a ring, and it is also a vector space over $\mathbb{K}$. This makes it a $\mathbb{K}$-algebra.

\begin{definition}
	If $A, B$ are rings, then an $A$-algebra structure on $B$ is the data of a ring homomorphism $\phi : A \to B$. This turn $B$ into an $A$-module via
	\[
	a \cdot b = \phi(a) \cdot b.
	\]
	Hence $\mathbb{K} \to \mathcal{O}_X(u)$ is given by $a \in \mathbb{K} \mapsto$ the constant function with value $a$.
\end{definition}

%lecture 4

We have the following lemma:
\begin{lemma}\label{lem:rat_alg}
	For all $X$ algebraic, if $\mathbb{K}$ is algebraically closed, then
	\[
	\mathcal{O}_X(X) = A(X).
	\]
\end{lemma}
The proof will be given after Hilbert's Nullstellensatz.

Recall that, if $A$ is an integral domain, then the \emph{field of fractions}\index{field of fractions} of $A$ is
\[
	\{f/g \mid f, g \in A, g \neq 0\} / \sim,
\]
where we have
\[
\frac{f}{g} \sim \frac{f'}{g'} \iff fg' = f'g.
\]
This is a field, as can be checked:
\begin{align*}
	\frac fg + \frac{f'}{g'} &= \frac{fg' + f'g}{gg'}, & \frac fg \frac{f'}{g'} &= \frac{ff'}{gg'}, & \left(\frac{f}{g}\right)^{-1} \!\!\!&= \frac gf.
\end{align*}
If $X \subseteq \mathbb{A}^n$ is an affine variety, then $A(X) = A/I(X)$ is an integral domain, since $I(X)$ is a prime ideal.

\begin{definition}
	If $X \subseteq \mathbb{A}^n$ is a variety, its \emph{fraction field}\index{fraction field} is $K(X)$, the fraction field of $A(X)$. Elements of $K(X)$ are called \emph{rational functions}\index{rational function}.
\end{definition}

Note that $g/h \in K(X)$ induces a regular function on $X \setminus Z(h)$.

\subsection{Morphisms}
\label{sub:morph}

\begin{definition}
	A map $f : X \to Y$ between affine varieties is called a \emph{morphism}\index{morphism} if:
	\begin{enumerate}
		\item $f$ is continuous in the induced Zariski topologies on $X$ and $Y$ (recall $Z \subseteq X \subseteq \mathbb{A}^n$ is closed in $X$ if and only if it is closed in $\mathbb{A}^n$).
		\item For all $V \subseteq Y$ open and $\phi : V \to \mathbb{K}$ a regular function,
			\[
			\phi \circ f : f^{-1}(V) \to \mathbb{K}
			\]
			is a regular function on $f^{-1}(V)$.
	\end{enumerate}
\end{definition}

Let $f : X \to Y$ be a morphism. Then for any $\phi \in A(Y)$, we get that $\phi \circ f : X \to \mathbb{K}$ is a regular function. Assuming that $\mathbb{K}$ is algebraically closed, $\mathcal{O}_X(X) = A(X)$, so $\phi \circ f \in A(Y)$. This gives a map $f^{\#} : A(Y) \to A(X)$. This is a $\mathbb{K}$-algebra homomorphism, and we can check it is a ring homomorphism.

Moreover, we have
\[
f^{\#}(a \cdot \phi) = a \cdot f^{\#}(\phi),
\]
which gives a $\mathbb{K}$-algebra homomorphism.

From now on, we look only at $\mathbb{K}$ algebraically closed. Assuming this, we get the following.

\begin{theorem}
	There is a one-to-one correspondence between morphisms $f : X \to Y$ and $\mathbb{K}$-algebra homomorphisms $f^{\#} : A(Y) \to A(X)$.
\end{theorem}

\begin{proofbox}
	We have already constructed $f^{\#}$ from $f$. Suppose $X \subseteq \mathbb{A}^n$, $Y \subseteq \mathbb{A}^m$. Then
	\begin{align*}
		A(X) &= \frac{\mathbb{K}[x_1, \ldots, x_n]}{I(X)}, & A(Y) &= \frac{\mathbb{K}[y_1, \ldots, y_m]}{I(Y)}.
	\end{align*}
	Suppose we are given $f^{\#} : A(Y) \to A(X)$. Set $f_i = f^{\#}(\bar y_i)$, where $\bar y_i$ is the image of $y_i$ in $A(Y)$. Then we define $f : X \to \mathbb{A}^m$ by
	\[
	f(p) = (f_1(p), \ldots, f_m(p)).
	\]
	We claim that $f(X) \subseteq Y$. Indeed, let $g \in I(Y)$, and $p \in X$. We need to show that $g(f(p)) = 0$. Consider the map
	\[
		\mathbb{K}[y_1, \ldots, y_m] \to A(Y) \to A(X).
	\]
	Then we have
	\[
		g(y_1, \ldots, y_m) \mapsto g(\bar y_1, \ldots, \bar y_m) \mapsto g(f_1, \ldots, f_m).
	\]
	Here it is important for $f^{\#}$ to be a $\mathbb{K}$-algebra homomorphism. Since $g \in I(Y)$, we get that $g(f_1, \ldots, f_m)(p) = 0$, i.e. $g(f(p)) = 0$. So $f(p) \in Y$.

	Note that, if $\phi \in A(Y)$, we can write $\phi = g(\bar y_1, \ldots, \bar y_m)$ and $f^{\#}(\phi) = g(f_1, \ldots, f_m) = \phi \circ f$. Now we claim that $f$ is a morphism.

	First, we show $f$ is continuous, by showing $f^{-1}(Z)$ is closed for $Z \subseteq Y$ closed. Note that $I(Z) \supseteq I(Y)$, so
	\[
	\overline{I(Z)} = \frac{I(Z)}{I(Y)} \subseteq A(Y)
	\]
	is an ideal in $A(Y)$. Then we can define
	\[
		Z(f^{\#}(\overline{I(Z)})) = \{p \in X \mid \phi(p) = 0 \quad \forall \phi \in f^{\#}(\overline{I(Z)})\}.
	\]
	This is a closed subset of $X$, since it coincides with
	\[
	Z(\pi_X^{-1}(f^{\#}(\overline{I(Z)}))),
	\]
	where $\pi_X : \mathbb{K}[x_1, \ldots, x_n] \to A(X)$. But,
	\begin{align*}
		Z(f^{\#}(\overline{I(Z)})) &= \{p \in X \mid \psi \circ f = 0 \quad \forall\psi \in \overline{I(Z)}\} = \{p \in X \mid f(p) \in Z\} \\
					   &= f^{-1}(Z).
	\end{align*}
	So $f^{-1}(Z)$ is closed.
	%lecture 5
	Finally we show that $f$ takes regular functions to regular functions. Let $U \subseteq Y$ be an open subset, $\phi \in \mathcal{O}_Y(U)$. then we need to show that $\phi \circ f : f^{-1}(U) \to \mathbb{K}$ is regular.

	Let $p \in f^{-1}(U)$, and let $V \subseteq U$ be an open neighbourhood of $f(p)$ for which we can write $\phi = g/h$, for $g, h \in A(Y)$. Then
	\[
	\phi \circ f|_{f^{-1}(V)} = \frac{g \circ f}{h \circ f} = \frac{f^{\#}(g)}{f^{\#}(h)}.
	\]
	Now $f^{\#}(g), f^{\#}(h)$ lie in $A(X)$, and $f^{\#}(h) = h \circ f$ does not vanish on $f^{-1}(V)$, as $h$ does not vanish on $V$.
\end{proofbox}

We can check this gives a one-to-one correspondence. We know that $f^{\#} \mapsto f \mapsto f^{\#}$, and we can check that $f \mapsto f^{\#} \mapsto f$.

The moral is that a morphism $f : X \to Y$ is given by choosing polynomial functions $f_1, \ldots, f_m \in \mathbb{K}[x_1, \ldots, x_n]$ and defining $f$ by
\[
f(p) = (f_1(p), \ldots, f_m(p)).
\]

\begin{exbox}
	Take $f : \mathbb{A}^1 \to \mathbb{A}^2$ by $t \mapsto (t, t^2)$. The image of this map is $Y = Z(x_1^2 - x_2)$, and this defines a morphism $f : \mathbb{A}^1 \to Y$.

	Now the inverse map is
	\[
		f^{\#} : \frac{\mathbb{K}[x_1, x_2]}{(x_2 - x_1^2)} \to \mathbb{K}[t]
	\]
	by $f^{\#}(x) = t$, $f^{\#}(y) = t^2$. Then this is an isomorphism.
\end{exbox}

\begin{definition}
	Two affine varieties $X, Y$ are isomorphic if there exist morphisms $f : X \to Y$, $g : Y \to X$ such that $g \circ f = \id_X$, $f \circ g = \id_Y$.
\end{definition}

\begin{theorem}
	If $X, Y$ are affine varieties, then $X$ is isomorphic to $Y$ if and only if $A(X) \cong A(Y)$ as $\mathbb{K}$-algebras.
\end{theorem}

As seen above, $\mathbb{A}^1 \cong Z(X^2 - Y) \subseteq \mathbb{A}^2$.

\begin{remark}
	A $\mathbb{K}$-algebra $A$ is \emph{finitely generated} if there exists a surjective $\mathbb{K}$-algebra homomorphism $\mathbb{K}[x_1, \ldots, x_n] \to A$ with $x_i \mapsto a_i$. Hence every element of $A$ can be written as a polynomial in $a_1, \ldots, a_n$ with coefficients in $\mathbb{K}$.

	If $I$ is the kernel of this map, then
	\[
		A \cong \frac{\mathbb{K}[x_1, \ldots, x_n]}{I}.
	\]
	Suppose further that $A$ is an integral domain. Then $I$ is a prime ideal of $\mathbb{K}[x_1, \ldots, x_n]$, so $A = A(X)$ where $X = Z(I)$.
\end{remark}

\newpage

\section{Hilbert's Nullstellensatz}
\label{sec:hil_null}

Our goal in this section is to prove, if $\mathbb{K} = \overline{\mathbb{K}}$, then
\[
I(Z(I)) = \sqrt I.
\]

\begin{definition}
	Let $F / \mathbb{K}$ be a field extension. We say an element $z \in F$ is \emph{transcendental}\index{transcendental} over $\mathbb{K}$ if it is not algebraic, i.e. there is no $f \in K[x]$ with $f \neq 0$, $f(z) = 0$.

	Similarly, $z_1, \ldots, z_d \in F$ are \emph{algebraically independent}\index{algebraically independent} over $\mathbb{K}$ if there is no $f \in \mathbb{K}[x_1, \ldots, x_d]$ such that $f \neq 0$, $f(z_1, \ldots, z_d) = 0$.

	A \emph{transcendence basis}\index{transcendence basis} for $F / \mathbb{K}$ is a set $z_1, \ldots, z_d \in F$, which are algebraically independent over $\mathbb{K}$, and such that $F$ is algebraic over $\mathbb{K}[z_1, \ldots, z_d]$.
\end{definition}

\begin{exbox}
	If $X$ is a variety, then $K(X)$ is a field over $\mathbb{K}$, and it usually has many transcendentals. For example,
	\[
		K(\mathbb{A}^n) = \left\{ f/g \mid f, g \in \mathbb{K}[x_1, \ldots, x_n], g \neq 0\right\} = \mathbb{K}(x_1, \ldots, x_n).
	\]
	Then $x_1, \ldots, x_n$ form a transcendence basis.
\end{exbox}

\begin{definition}
	If $F / \mathbb{K}$ is a field extension, we say $F$ is \emph{finitely generated}\index{finitely generated} over $\mathbb{K}$ if $F = \mathbb{K}(z_1, \ldots, z_n)$ for some $z_1, \ldots, z_n \in F$.
\end{definition}

\begin{exbox}
	$K(X)/\mathbb{K}$ is finitely generated. If $X \subseteq \mathbb{A}^n$, then $K(X)$ is the fraction field of $A(X) = \mathbb{K}[x_1, \ldots, x_n]/I$, and hence $K(X)$ is generated by the images of $x_1, \ldots, x_n$.
\end{exbox}

\begin{proposition}
	Every finitely generated field extension $F/\mathbb{K}$ has a transcendence basis, and any two transcendence bases have the same cardinality.

	Moreover, if $F = \mathbb{K}(z_1, \ldots, z_N)$, then there is a transcendence basis $\{y_1, \ldots, y_n\} \subseteq \{z_1, \ldots, z_N\}$.
\end{proposition}

\begin{proofbox}
	Write $F = \mathbb{K}(z_1, \ldots, z_N)$. If these are algebraically independent, then $z_1, \ldots, z_N$ is a transcendence basis. Also if they are algebraic over $\mathbb{K}$, then the transcendence basis can be taken to be empty.
%lecture 6
	
	After reordering, assume $\{z_1, \ldots, z_d\}$ is a maximal subset of algebraically independent elements of $\{z_1, \ldots, z_N\}$. Then we will show $\{z_1, \ldots, z_d\}$ is a transcendence basis, i.e $F$ is algebraic over $\mathbb{K}(z_1, \ldots, z_d)$.

	It is enough to show $z_j$ is algebraic over $\mathbb{K}(z_1, \ldots, z_d)$ for any $j > d$. By assumption, $z_1, \ldots, z_d, z_j$ are not algebraically independent, so there exists $f_j \in \mathbb{K}[x_1, \ldots, x_d, x_j]$ such that $f_j(z_1, \ldots, z_d, z_j) = 0$.

	Then consider the polynomial $F_j(x) = f_j(z_1, \ldots, z_d, x)$. This is a polynomial in $\mathbb{K}(z_1, \ldots, z_d)[x]$. Plugging in $x = z_j$, $F_j(z_j) = f_j(z_1, \ldots, z_d, z_j) = 0$. Also $F \neq 0$, as otherwise $F_j(z) = f_j(z_1, \ldots, z_d, z)$ would be an algebraic relation for $\{z_1, \ldots, z_d\}$, for all $z \in \mathbb{K}(z_1, \ldots, z_d)$. Hence $\{z_1, \ldots, z_d\}$ is indeed a transcendence basis.

	Now suppose $z_1, \ldots, z_d$ and $w_1, \ldots, w_e$ are both transcendence bases. Suppose $d \leq e$. We will use the same idea as the Steinitz exchange lemma. First, as $w_1$ is algebraic over $\mathbb{K}(z_1, \ldots, z_d)$, there is a polynomial $f \in \mathbb{K}[x_1, \ldots, x_d, x_{d+1}]$ such that $f(z_1, \ldots, z_d, w_1) = 0$. This is obtained by clearing the denominators.

	Since $w_1$ is not algebraic, $f$ must involve at least some of $x_1, \ldots, x_d$. Thus we can suppose $z_1$ is algebraic over $\mathbb{K}(w_1, z_2, \ldots, z_d)$, hence $F$ is algebraic over $\mathbb{K}(w_1, z_2, \ldots, z_d)$.

	We now repeat this process. As $w_2$ is algebraic over $\mathbb{K}(w_1, z_2, \ldots, z_d)$, and not algebraic over $\mathbb{K}(w_1)$, we can find $0 \neq g \in \mathbb{K}[x_1, \ldots, x_{d+1}]$ such that $g(w_1, z_2, \ldots, z_d, w_2) = 0$, and furthermore $g$ involves one of $x_2, \ldots, x_d$. Suppose it involves $x_2$, then $z_2$ is algebraic over $\mathbb{K}(w_1, w_2, z_3, \ldots, z_d)$, and hence $F$ is algebraic over $\mathbb{K}(w_1, w_2, z_3, \ldots, z_d)$.

	Continuing, eventually we find $F$ is algebraic over $\mathbb{K}(w_1, \ldots, w_d)$. If $e > d$, this means $w_e$ is algebraic over $\mathbb{K}(w_1, \ldots, w_d)$, contradicting the fact $\{w_1, \ldots, w_e\}$ is a transcendence basis.
\end{proofbox}

\begin{lemma}
	Let $M$ be a finitely generated $A$-module, for $A$ a commutative ring. Let $I \subseteq A$ and $\phi : M \to M$ be an $A$-module homomorphism such that
	\[
		\phi(M) \subseteq I \cdot M = \{a \cdot m \mid a \in I, m \in M\}.
	\]
	Then there exists an equation
	\[
	\phi^n + a_1 \phi^{n-1} + \cdots + a_n = 0,
	\]
	with $a_i \in I$.
\end{lemma}

\begin{proofbox}
	Let $x_1, \ldots, x_n \in M$ be a set of generators for $M$. Then each $\phi(x_i) \in I \cdot M$, so we can write
	\[
	\phi(x_i) = \sum_{j = 1}^n a_{ij} \cdot x_j,
	\]
	with $a_{ij} \in I$. Hence we have
	\[
	\sum_{j = 1}^n (\delta_{ij} \phi - a_{ij}) x_j = 0,
	\]
	where $\delta_{ij}$ is the usual Kronecker delta. Writing this out as a matrix,
	\[
	\begin{pmatrix}
		\phi - a_{11} & -a_{12} & \cdots \\
		-a_{21} & \phi - a_{22} & \cdots \\
		\vdots & \vdots & \ddots
	\end{pmatrix}
	\begin{pmatrix}
		x_1 \\ x_2 \\ \vdots
	\end{pmatrix} = 0.
	\]
	Multiplying by the adjoint matrix, we get
	\[
	\det((\delta_{ij} \phi - a_{ij})) x_j = 0,
	\]
	for all $j$. But $\det((\delta_{ij}\phi - a_{ij}))$ is a degree $n$ polynomial in $\phi$ annihilating each $x_j$, hence it annihilates every element in $M$. Moreover the leading term in $\phi$ is $\phi^{n}$, and all the other coefficients are elements in $I$.
\end{proofbox}

\subsection{Integrality}
\label{sub:integ}

\begin{definition}
	Let $A \subseteq B$ be integral domain. An element $b \in B$ is \emph{integral}\index{integral} over $A$ if $f(b) = 0$ for a monic polynomial $f(x) \in A[x]$
\end{definition}

\begin{proposition}
	$b \in B$ is integral over $A$ if and only if there is a subring $C \subseteq B$ containing $A[b]$, with $C$ a finitely generated $A$-module.
\end{proposition}

\begin{proofbox}
	Suppose $b^n + a_1 b^{n-1} + \cdots + a_n = 0$. Then since $A[b]$ is generated as an $A$-module by $1, b, b^2, \ldots$, it is also generated by $1, b, \ldots, b^{n-1}$. In particular, it is finitely generated. Then we can just take $C = A[b]$.

	On the other hand, if $C$ is finitely generated, let $\phi : C \to C$ be the module homomorphism given by $\phi(x) = b \cdot x$. Applying the previous lemma to the finitely generated $A$-module $C$ with $I = A$, we get $\phi^n + a_1 \phi^{n-1} + \cdots + a_n \equiv 0$, or $b^n + a_1 b^{n-1} + \cdots + a_n = 0$, by plugging in $1$.
\end{proofbox}

%lecture 7

\begin{lemma}
	Let $A \subseteq B$ be an inclusion of integral domains, and assume the fraction field $K$ of $A$ is contained in $B$. If $b \in B$ is algebraic over $K$, then there exists $p \in A$ non-zero such that $pb$ is integral over $A$.
\end{lemma}

\begin{proofbox}
	Suppose $g \in K[X]$ with $g(b) = 0$, $g \neq 0$. By clearing denominators, we can assume that $g \in A[X]$. Suppose that
	 \[
	g(x) = a_N x^N + \cdots + a_0,
	\]
	for $a_N \neq 0$, $a_i \in A$. Then
	\[
		a_N^{N-1} g = (a_Nx)^N + a_{N-1}(a_N x)^{n-1} + a_{N-2} a_N (a_N x)^{N-2} + \cdots + a_0 a_N^{N-1}.
	\]
	This is a monic polynomial in $a_N x$. Substituting $x = b$, this gives a monic polynomial killing $a_N b$. So $a_N b$ is integral over $A$, and we take $p = a_N$.
\end{proofbox}

\begin{lemma}[Rational Root Theorem]
	Let $A$ be a UFD with fraction field $K$. If $\alpha \in K$ is integral over $A$, we have $\alpha \in A$.
\end{lemma}

\begin{proofbox}
	If $\alpha \in K$ is integral over $A$, write $\alpha = a/b$, with $a, b$ having no common factor. Say $g(\alpha) = 0$ for some monic polynomial $\alpha$, with
	\[
	g(x) = x^n + a_1 x^{n-1} + \cdots + a_n.
	\]
	Then we have
	\[
	\frac{a^n}{b^n} + a_1 \frac{a^{n-1}}{b^{n-1}} + \cdots + a_n = 0.
	\]
	Multiplying out,
	\[
	a^n + a_1 b a^{n-1} + \cdots + a_n b^n = 0
	\]
	in $A$. So $b \mid a$, showing that $b$ must be a unit in $A$. Thus $\alpha = a/b \in A$.
\end{proofbox}

\begin{lemma}
	Let $A \subseteq B$ be integral domains, and $S \subseteq B$ the set of all elements in $B$ integral over $A$. Then $S$ is a subring of $B$.
\end{lemma}

\begin{proofbox}
	If $b_1, b_2 \in S$, then $A[b_1]$ is a finitely generated $A$-module. Also, $b_2$ is integral over $A$, hence over $A[b_1]$. So $A[b_1][b_2] = A[b_1, b_2]$ is a finitely generated $A[b_1]$-module.

	From this, we can conclude that $A[b_1, b_2]$ is a finitely generated $A$-module. Since $A[b_1 \pm b_2], A[b_1 \cdot b_2] \subseteq A[b_1, b_2]$, we have $b_1 \pm b_2, b_1 \cdot b_2 \in S$.
\end{proofbox}

\begin{lemma}[Hilbert's Nullstellensatz, Version 0]
	Let $\mathbb{K}$ be an algebraically closed field, and $F/\mathbb{K}$ be a field extension which is finitely generated as a $\mathbb{K}$-module.

	Then $F = \mathbb{K}$.
\end{lemma}

\begin{proofbox}
	Suppose $\alpha \in F$ is algebraic over $\mathbb{K}$, with irreducible polynomial $f(x) \in \mathbb{K}[x]$. Then $f$ factors into linear factors over $\mathbb{K}$, as $\mathbb{K}$ is algebraic. So $f$ is linear, and hence is of the form $c(x - \alpha)$. Thus $\alpha \in \mathbb{K}$.

	Suppose we are given surjective map $\mathbb{K}[x_1, \ldots, x_d] \to F$ surjective, where $x_i \mapsto z_i \in F$. Then $z_1, \ldots, z_d$ generate $F$ as a field extension of $\mathbb{K}$. Assume $z_1, \ldots, z_e$ form a transcendence basis for $F / \mathbb{K}$.

	Note if $F \neq \mathbb{K}$, then we must have $e \geq 1$. Let $R = \mathbb{K}[z_1, \ldots, z_e] \subseteq F$. This is a polynomial ring, as $z_1, \ldots, z_e$ are algebraically independent. Then $w_1 = z_{e+1}, \ldots, w_{d-e} = z_d$ are algebraic over $L = \mathbb{K}(z_1, \ldots, z_e)$.

	Let $S \subseteq F$ be the set of elements of $F$ integral over $R$. Then $S$ is a subring of $F$. But now there exists $p_1, \ldots, p_{d-e} \in R$, with $t_i = p_i w_i$ integral over $R$. In particular, $t_i \in S$.

	Choose $f/g \in \mathbb{K}(z_1, \ldots, z_e) = L$, with $f, g \in R$, $f, g$ relatively prime, and $g$ is relatively prime to $p_1, \ldots, p_{d-e}$. Then $p_1^{n_1} \cdots p_{d-e}^{n_{d-e}} \cdot \frac{f}{g} \not \in \mathbb{K}[z_1, \ldots, z_e]$ for any $n_1, \ldots, n_{d - e} \geq 0$. Here, don't think of $f, g$ as polynomials, but rather elements of $R$.

	But since $z_1, \ldots, z_d$ generate $F$ as a $\mathbb{K}$-algebra, there exists $q \in \mathbb{K}[x_1, \ldots, x_d]$ such that
	\[
	\frac{f}{g} = q(z_1, \ldots, z_d) = q\left(z_1, \ldots, z_e, \frac{t_1}{p_1}, \ldots, \frac{t_{d-e}}{p_{d-e}} \right).
	\]
	Let $n_j$ be the highest power of $x_{e+j}$ appearing in $q$. Multiplying by $\prod p_j^{n_j}$ clears the denominators of the right hand side, so we have
	\[
	p_1^{n_1} \cdots p_{d-e}^{n_{d-e}} \frac{f}{g} = q'(z_1, \ldots, z_e, t_1, \ldots, t_{d-e}).
	\]
	The right hand side lies in $S$ as $z_1, \ldots, z_e \in S$, $t_1, \ldots, t_{d-e} \in S$, so the left hand side lies in $S$. But the left hand side lies in $\mathbb{K}(z_1, \ldots, z_e)$, and thus lies in $\mathbb{K}[z_1, \ldots, z_e]$, a contradiction.

	Hence $e = 0$, so $F$ is algebraic over $\mathbb{K}$, hence $F = \mathbb{K}$.
\end{proofbox}

%lecture 8

Now we can prove the ``actual'' Nullstellensatz.
\begin{theorem}[Nullstellensatz I]
	Let $\mathbb{K}$ be algebraically closed. Then any maximal ideal $m \subseteq \mathbb{K}[x_1, \ldots, x_n]$ is of the form
	\[
		b = \langle x_1 - a_1, \ldots, x_n - a_n \rangle
	\]
	for some $a_1, \ldots, a_n \in \mathbb{K}$.
\end{theorem}

\begin{proofbox}
	Note we have an isomorphism
	\[
		\frac{\mathbb{K}[x_1, \ldots, x_n]}{\langle x_1 - a_1, \ldots, x_n - a_n \rangle} \to \mathbb{K},
	\]
	by $x_i \mapsto a_i$. Note $m \subseteq A$ is a maximal ideal if and only if $A / m$ is a field. This shows that $m = \langle x_1 - a_1, \ldots, x_n - a_n\rangle$ is a maximal ideal.

	Conversely, let $m \subseteq \mathbb{K}[x_1, \ldots, x_n]$ be maximal. Then $\mathbb{K}[x_1, \ldots, x_n] / m = F$ is a field, which is generated as a $\mathbb{K}$-algebra by $x_1, \ldots, x_n$. Thus $F = \mathbb{K}$ by our previous lemma.

	We thus have an isomorphism
	\[
		\frac{\mathbb{K}[x_1, \ldots, x_n]}{m} \overset{\phi}{\to} \mathbb{K}.
	\]
	Let $a_i = \varphi(x_i)$. Then $\phi(x_i - a_i) = \phi(x_i) - a_i = 0$, so $x_i - a_i \in m$. Hence $\langle x_1 - a_1, \ldots, x_n - a_n \rangle \subseteq m$. Since the left hand ideal is maximal, we have equality.
\end{proofbox}

\begin{exbox}
	This is false if our field is not algebraically closed. For example, $\langle x^2 + 1 \rangle$ is a maximal ideal in $\mathbb{R}[x]$, but of course $\langle x^2 + 1 \rangle \neq \langle x - a \rangle$ for any $a \in \mathbb{R}$.
\end{exbox}

Here is another form.

\begin{theorem}[Nullstellensatz II]
	Let $\mathbb{K}$ be algebraically closed, and $I = \langle f_1, \ldots, f_r \rangle \subseteq \mathbb{K}[x_1, \ldots, x_n]$. Then either:
	\begin{enumerate}[\normalfont1.]
		\item $I = \mathbb{K}[x_1, \ldots, x_n]$, or
		\item $Z(I) \neq \emptyset$.
	\end{enumerate}
\end{theorem}

\begin{proofbox}
	Suppose $1 \not \in I$, i.e. we are not in the first case. Then there exists a maximal ideal $m \subseteq \mathbb{K}[x_1, \ldots, x_n]$, with $I \subseteq m$.

	But then $Z(m) \subseteq Z(I)$, and since $m = \langle x_1 - a_1, \ldots, x_n - a_n \rangle$, we have $Z(m) = \{(a_1, \ldots, a_n)\}$. So $Z(m) \neq \emptyset$, hence $Z(I) \neq \emptyset$.
\end{proofbox}

Here we actually go.

\begin{theorem}[Nullstellensatz III]
	Let $\mathbb{K}$ be algebraically closed, $I \subseteq \mathbb{K}[x_1, \ldots, x_n]$ an ideal. Then
	\[
	I(Z(I)) = \sqrt I.
	\]
\end{theorem}

\begin{proofbox}
	One direction we have already seen: $\sqrt I \subseteq I(Z(I))$.

	Let $g \in \mathbb{K}[x_1, \ldots, x_n]$. Define
	\[
	V_g = Z(z g(x_1, \ldots, x_n) - 1) \subseteq \mathbb{A}^{n+1},
	\]
	with coordinates $x_1, \ldots, x_n, z$. If we project $V_g$ via $(x_1, \ldots, x_n, z) \mapsto (x_1, \ldots, x_n)$, we get the set $D(g) = \mathbb{A}^n \setminus Z(g)$.

	Now suppose $g \in I(Z(I))$. Then $D(g) \cap Z(I) = \emptyset$. If $I = \langle f_1, \ldots, f_r \rangle$, consider $J = \langle f_1, \ldots, f_r, zg - 1\rangle \subseteq \mathbb{K}[x_1, \ldots, x_n, z]$. Then $Z(J) = \emptyset$, so $J = \mathbb{K}[x_1, \ldots, x_n, z]$ by our previous version of the Nullstellensatz. So we can write
	\[
	1 = \sum_{i = 1}^n h_i f_i + h (gz - 1),
	\]
	with $h_i, h \in \mathbb{K}[x_1, \ldots, x_n, z]$. Substitute $z = 1/g$, to get
	\[
	1 = \sum_{i = 1}^n h_i(x_1, \ldots, x_n, 1/g) f_i(x_1, \ldots, x_n).
	\]
	Multiplying by a high power of $g$ clears the denominators, giving
	\[
	g^N = \sum_{i = 1}^n h_i'(x_1, \ldots, x_n) f_i \in I.
	\]
	Thus $g^N \in I$, so $g \in \sqrt I$.
\end{proofbox}

Recall we need the proof of the lemma \ref{lem:rat_alg}. For this, we need the following.

\begin{lemma}
	Let $f, g : X \to \mathbb{K}$ be regular functions on $X$ an affine variety, and suppose there exists open $U \subseteq X$ non-empty with $f|_U = g|_U$.

	Then $f = g$.
\end{lemma}

\begin{proofbox}
	Consider the map $\phi = (f, g) : X \to \mathbb{A}^2$. This is a morphism. Let $\Delta = \{(a, a) \in \mathbb{A}^2 \mid a \in \mathbb{K}\}$. Then $\Delta = Z(x - y)$.

	Since $\phi$ is continuous, $\phi^{-1}(\Delta)$ is closed. But $U \leq \phi^{-1}(0)$, and $U$ is a dense subset of $X$, otherwise $X = \overline{U} \cup (X \setminus U)$ is a union of two proper closed subsets, violating irreducibility of $X$. Hence $U \subseteq \overline{U} = X \subseteq \phi^{-1}(0)$, so $\phi^{-1}(0) = X$.
\end{proofbox}

We are now ready to prove the proposition.

\begin{proofbox}
	We know $A(X) \subseteq \mathcal{O}_X(X)$. So let $f : X \to \mathbb{K}$ be a regular function, i.e. there exists an open cover $\{U_i\}$ of $X$ with $f$ given on $U_i$ by
	\[
	f|_{U_i} = \frac{g_i}{h_i},
	\]
	with $g_i, h_i \in A(X)$, and $h_i$ nowhere-vanishing in $U_i$. Then
	\[
		Z(\{h_i \mid i \in I\}) = \bigcap_i Z(h_i) \subseteq \bigcap_i \left( X \setminus U_i \right) = X \setminus \left( \bigcup_i U_i \right) = \emptyset.
	\]
	Thus $Z(\{h_i\}) = \emptyset$. We can now pull back to $\mathbb{K}[x_1, \ldots, x_n]$ and use Hilbert's second Nullstellensatz to get
	\[
	1 = \sum_i e_i h_i.
	\]
	Note on $U_i \cap U_j$, $\frac{g_i}{h_i} = \frac{g_j}{h_j}$, so $g_i h_j = g_j h_i$ on $U_i \cap U_j$, so by our previous lemma, $g_i h_j = g_j h_i$ on $X$. Hence $\frac{g_i}{h_i} = \frac{g_j}{h_j}$ on $K(X)$. Thus we have the equality in $K(X)$ 
	\[
	f = \sum_i (e_i h_i) f = \sum_i (e_i h_i) \frac{g_i}{h_i} = \sum e_i g_i \in A(X).
	\]
\end{proofbox}

\begin{remark}
	$U_i$ and $U_j$ always intersect, as they are dense sets: if not, $\overline{U_i}$ and $X \setminus U_i$ form a proper closed union of $X$.

	In essence open subsets of affine varieties are always dense, and this makes the Zariski topology interesting!
\end{remark}


\newpage

\section{Projective Varieties}
\label{sec:proj_var}

%lecture 9

\begin{definition}
	Let $\mathbb{K}$ be a field. We define
	\[
		\mathbb{P}^n = (\mathbb{K}^{n+1} \setminus \{(0, \ldots, 0\}) / \sim,
	\]
	where $(x_0, \ldots, x_n) \sim (\lambda x_0, \ldots, \lambda x_n)$ for any  $\lambda \in \mathbb{K}^\times$. Alternatively, this is the set of one-dimensional sub-vector spaces of $\mathbb{K}^{n+1}$.
\end{definition}

\begin{remark}
	If $\mathbb{K} = \mathbb{R}$, then $\mathbb{P}^n = S^n / \sim$, where $x \sim -x$.
\end{remark}

For arbitrary $\mathbb{K}$, we look at $\mathbb{P}^1$. For an arbitrary element $(x_0: x_1) \in \mathbb{P}^1$, if $x_1 \neq 0$, then we have
\[
	(x_0 : x_1) \sim \left( \frac{x_0}{x_1}, 1\right) \in \mathbb{A}^1,
\]
since there is a unique representative with the second coordinate 1. The missing points are of the form $(x_0 : 0) \sim (1 : 0)$. Thus we view $\mathbb{P}^1 = \mathbb{A}^1 \cup\{(1, 0)\}$, where we can view the point $(1, 0)$ as $\infty$. This is the Riemann sphere if $\mathbb{K} = \mathbb{C}$.

Now $\mathbb{P}^2$ consists of elements of the form $(x_0 : x_1 : x_2) \in \mathbb{P}^2$. If $x_2 \neq 0$, then
\[
	(x_0 : x_1 : x_2) \sim \left( \frac{x_0}{x_2}, \frac{x_1}{x_2}, 1 \right) \in \mathbb{A}^2.
\]
If $x_2 = 0$, we get a point $(x_0 : x_1: 0) \in \mathbb{P}^1$. So $\mathbb{P}^2 = \mathbb{A}^2 \cup \mathbb{P}^1$, where we view $\mathbb{P}^1$ as the line at infinity.

As we did for $\mathbb{A}^n$, we now look to define a topology via algebraic subsets of $\mathbb{P}^n$. But we cannot just define it as the zeros of a polynomial $f(x_0, \ldots, x_n)$, as then we may have two equivalent points not being in the same algebraic set.

\begin{definition}
	$f \in S = \mathbb{K}[x_0, \ldots, x_n]$ is \emph{homogeneous}\index{homogeneous polynomial} if every term of $f$ is the same degree, or equivalently 
	\[
	f(\lambda x_0, \ldots, \lambda x_n) = \lambda^d f(x_0, \ldots, x_n),
	\]
	for some $d \geq 0$, where $d$ is the degree.
\end{definition}

\begin{exbox}
	$x_0^3 + x_1x_2^2$ is homogeneous of degree 3, whereas $x_0^3 + x_1^2$ is not homogeneous.
\end{exbox}

\begin{definition}
	If $T \subseteq S$ is a set of homogeneous polynomials, define
	\[
		Z(T) = \{(a_0: \ldots: a_n) \in \mathbb{P}^n \mid f(a_0, \ldots, a_n) = 0 \quad \forall f \in T\}.
	\]
	An ideal $I \subseteq S$ is \emph{homogeneous}\index{homogeneous ideal} if $I$ is generated by homogeneous polynomials. For $I$ a homogeneous ideal, we define
	\[
		Z(I) = \{(a_0: \ldots: a_n) \in \mathbb{P}^n \mid f(a_0, \ldots, a_n) = 0 \quad \forall f \in I \text{ homogeneous}\}.
	\]
	A subset $\mathbb{P}^n$ is \emph{algebraic}\index{algebraic} if it is of the form $Z(T)$ for some $T$.
\end{definition}

\begin{exbox}
	Take $Z(a_0 x_0 + a_1 x_1 + a_2 x_2) \subseteq \mathbb{P}^2$, for $a_0, a_1, a_2 \in \mathbb{K}$. In $\mathbb{A}^2 \subset \mathbb{P}^2$ where $x_2 = 1$, we get the equation $a_0 x_0 + a_1 x_1 + a_2 = 0$.

	If $x_2 = 0$, we get the equation $a_0 x_0 + a_1 x_1 = 0$, which has the solution $(a_1:-a_0) \in \mathbb{P}^1$, assuming not both $a_0 = a_1 = 0$, as then we just have $x_2 = 0$, the line at infinity.
\end{exbox}

We can check that the algebraic set in $\mathbb{P}^n$ form the closed sets of a topology on $\mathbb{P}^n$. This is again the \emph{Zariski topology}\index{Zariski topology} on $\mathbb{P}^n$.

\begin{definition}
	A \emph{projective variety}\index{projective variety} is an irreducible closed subset of $\mathbb{P}^n$.

	Define $U_i \subseteq \mathbb{P}^n$ to be $U_i = \mathbb{P}^n \setminus Z(x_i)$. This is an open subset of $\mathbb{P}^n$, and moreover
	\[
	\bigcup_{i = 0}^n U_i = \mathbb{P}^n.
	\]
	We have a bijection $\phi_i : U_i \to \mathbb{A}^n$ by
	\[
		\phi_i(x_0 : \ldots : x_n) = \left( \frac{x_0}{x_i}, \ldots, \widehat{\frac{x_i}{x_i}}, \ldots, \frac{x_n}{x_i} \right).
	\]
	This is the \emph{standard open affine cover} of $\mathbb{P}^n$.
\end{definition}

\begin{proposition}
	With $U_i$ carrying the topology induced from $\mathbb{P}^n$ and $\mathbb{A}^n$ the Zariski topology, $\phi_i$ is a homeomorphism.
\end{proposition}

\begin{proofbox}
	Since $\phi_i$ is a bijection, it suffices to show $\phi_i$ identifies closed sets of $U_i$ with closed sets of $\mathbb{A}^n$. We take $i = 0$, $\phi = \phi_0$ and $U = U_0$.

	Then let $S = \mathbb{K}[x_0, \ldots, x_n]$, $S^h$ the set of homogeneous polynomials in $S$, and $A = \mathbb{K}[x_1, \ldots, x_n]$. Define maps $\alpha : S^h \to A$ and $\beta : A \to S^h$ by $\alpha(f(x_0, \ldots, x_n)) = f(1, x_1, \ldots, x_n)$, and if $g \in A$ is of degree $e$, define
	\[
	\beta(g) = x_0^e g \left( \frac{x_1}{x_0}, \ldots, \frac{x_n}{x_0} \right).
	\]
	This is a process known as \emph{homogenisation}\index{homogenisation}, for example if we take $x_2^2 - x_1^3 - x_1 + x_1x_2$, the homogenisation is
	\[
	x_0^3 \left( \frac{x_2^2}{x_0^2} - \frac{x_1^3}{x_0^3} - \frac{x_1}{x_0} + \frac{x_1x_2}{x_0^2}\right) = x_0x_2^2 - x_1^3 - x_0^2x_1 + x_0x_1x_2.
	\]
	If $Y \subseteq U$ is closed, $Y$ is the intersection $\bar Y \cap U$, where $\bar Y \subseteq \mathbb{P}^n$ is a closed subset, which we can take to be the closure of $Y$. Now $\bar Y = Z(T)$ for some $T \subseteq S^h$, and let $T' = \alpha(T)$. We will show
	\[
	\phi(Y) = Z(\alpha(T)).
	\]
	We can check that
	\begin{align*}
		f(a_0 : \ldots : a_n) = 0 &\iff f\left( 1, \frac{a_1}{a_0}, \ldots, \frac{a_n}{a_0}\right) = 0 \\
					  &\iff \alpha(f) \left( \frac{a_1}{a_0}, \ldots, \frac{a_n}{a_0} \right) = 0  \\
					  &\iff \alpha(f) \phi(a_0 : \ldots : a_n) = 0.
	\end{align*}
	We need to prove that if $W \subseteq \mathbb{A}^n$ is closed, then $\phi^{-1}(W) \subseteq U = U_0$ is closed. We have $W = Z(T')$ for some set $T' \subseteq A = \mathbb{K}[y_1, \ldots, y_n]$. We will show
	\[
	\phi^{-1}(W) = Z(\beta(T')) \cap U.
	\]
	Indeed, if $g \in T'$,
	\begin{align*}
		g(b_1, \ldots, b_n) = 0 &\iff \beta(g)(1, b_1, \ldots, b_n) = 0 \\
					&\iff \beta(g)(\phi^{-1}(b_1, \ldots, b_n)) = 0.
	\end{align*}
\end{proofbox}
%lecture 10

\begin{exbox}
	Take $f : \mathbb{P}^1 \to \mathbb{P}^3$, by
	\[
	f(u : t) = (u^3 : u^2 t : ut^2 : t^3).
	\]
	The image of this map is called the twisted cubic. Now we claim that this is a projective variety.

	Indeed, consider the homomorphism
	\[
		\phi : \mathbb{K}[x_0, \ldots, x_3] \to \mathbb{K}[u, t],
	\]
	by $x_0 \mapsto u^3$, $x_1 \mapsto u^2t$, $x_2 \mapsto ut^2$ and $x_3 \mapsto t^3$. Let $I = \ker \phi$. If $g \in I$, then $g$ vanishes on the image of the map $f$. Thus $\Img(f) \subseteq Z(I)$.

	Conversely, note that $x_0 x_3 - x_1 x_2$, $x_1^2 - x_0x_2$, $x_2^2 - x_1x_3 \in I$. Now let $p = (a_0 : a_1 : a_2: a_3) \in Z(I)$. Then we have four cases.

	If $a_0 \neq 0$, we can take $a_0 = 1$. Then $a_3 - a_1 a_2 = 0$, $a_1^2 - a_2 = 0$ and $a_2^2 - a_1a_3= 0$. Then $p = (1, a_1, a_1^2, a_1^3) = f(1 : a_1)$. So $p \in \Img(f)$.

	Similarly, we can check the cases when $a_1 \neq 0$, $a_2 \neq 0$ and $a_3 \neq 0$. The conclusion is that $p \in \Img(f)$ for all four cases, so $\Img f \supseteq Z(I)$. Hence $Z(I) = \Img f$. Thus the twisted cubic is an algebraic set.
\end{exbox}

Given $X \subseteq \mathbb{P}^n$ an algebraic set, define its \emph{ideal}\index{ideal of an algebraic set} $I(X)$ to be the ideal in $S = \mathbb{K}[x_0, \ldots, x_n]$, generated by homogeneous polynomials which vanish on $X$.

Then $X$ is irreducible if and only if $I(X)$ is prime. For the twisted cubic $X = \Img(f)$, we indeed have $I(X) = I = \Ker \phi$. But $\mathbb{K}[x_0, \ldots, x_3]/\Ker \phi$ is a subring of the integral domain $\mathbb{K}[u, t]$, hence is an integral domain, so $\Ker \phi$ is prime. Therefore $X$ is a projective variety.

\begin{definition}
	Let $X \subseteq \mathbb{P}^n$ be an affine variety. A \emph{regular function}\index{regular function} on $U \subseteq X$ open is a function $f : U \to \mathbb{K}$ such that, for every $p \in U$, there exists an open neighbourhood $V \subseteq U$ of $p$ and $g, h \in S$ homogeneous of the same degree with $h$ nowhere-vanishing on $V$, and with $f|_V = g/h$.
\end{definition}

\begin{definition}
	A \emph{quasi-affine variety}\index{quasi-affine variety} is an open subset of an affine variety.

	A \emph{quasi-projective variety}\index{quasi-projective variety} is an open subset of a projective variety.

	These types of varieties also have the same action of regular functions. A variety\index{variety} will henceforth refer to any of an affine, quasi-affine, projective or quasi-projective variety.
\end{definition}

\begin{definition}
	A morphism $\phi : X \to Y$ between varieties is a continuous function $\phi$ such that, for all $V \subseteq Y$ open, $f : V \to \mathbb{K}$ regular,
	\[
	f \circ \phi : \phi^{-1}(V) \to \mathbb{K}
	\]
	is regular.
\end{definition}

\begin{remark}
	If $X$ is projective, then in fact $\mathcal{O}_X(X) = \{X \to \mathbb{K} \text{ regular}\}$ is $\mathbb{K}$. Thus finding morphism from a projective variety becomes harder, and this is a lot of what algebraic geometry is about.
\end{remark}

\begin{exbox}
	Let $Q \subseteq \mathbb{P}^3$ be given by $Z(xy - zw)$. This is a \emph{quadric surface}\index{quadric surface}.

	For $(a : b) \in \mathbb{P}^1$, $Q$ contains the line
	\[
	ax = bz, \qquad by = aw.
	\]
	Indeed if $a \neq 0$, we can take $a = 1$, and the $xy - zw = (bz)y - z(by) = 0$. If $a = 0$, then $y = z = 0$ so $xy - zw = 0$. This gives a family of lines in $Q$ parametrized by $(a : b) \in \mathbb{P}^1$.

	We also have $ax = bw, by = az$ another family of lines.

	If we take a line from one family and a line from the other, they meet at one point. Indeed, $ax =  bz$, $by = aw$, $cx = dw$ and $dy = cz$ has a unique solution up to scaling: $(bd : ac : ad : bc)$.
%lecture 11

	This suggests we define a map $\Sigma : \mathbb{P}^1 \times \mathbb{P}^1 \to \mathbb{P}^3$ given by
	\[
	\Sigma((a:b), (c:d)) = (bd:ac:ad:bc).
	\]
	We claim that $\Sigma$ is a bijection with $Q = Z(xy - zw)$. First note that $(bd) \cdot (ac) - (ad)(bc) = 0$, so indeed $\Img \Sigma \subseteq Q$.

	Now we show that it is an injection. First suppose that $a, c \neq 0$. Then
	\[
	\Sigma((1:b), (1:d)) = (bd : 1 : d : b),
	\]
	which is injective on the set where $a, c \neq 0$. If $a = 0$, then
	\[
	\Sigma((0:b), (c:d)) = (bd : 0 : 0 : bc) = (d : 0 : 0 : c),
	\]
	which does not coincide with the previous point and recovers $(c:d)$. If $a = c = 0$, then
	\[
	\Sigma((0:1),(0:1)) = (1:0:0:0).
	\]
	If $a \neq 0, c = 0$, then we get
	\[
	\Sigma((a:b)(0:1)) = (b:0:a:0).
	\]
	So $\Sigma$ is injective. To prove it is surjective, suppose that $(a_0 : a_1 : a_2 : a_3) \in Q$, i.e. $a_0 a_1 - a_2a_3 = 0$. If $a_0 \neq 0$, we can take $a_0 = 1$, so $a_1 = a_2 a_3$. Hence
	\[
		(a_0 : a_1 : a_2 : a_3) = (1 : a_2 a_3 : a_2 : a_3) = \Sigma((a_2 : 1), (a_3 : 1)).
	\]
	A similar thing works in the case when $a_2, a_3$ or $a_4 \neq 0$.
\end{exbox}

\begin{remark}
	$\mathbb{P}^1 \times \mathbb{P}^1$ is not a priori a variety, but it can be given a variety structure by identifying it with $Q$, i.e. closed sets of $\mathbb{P}^1 \times \mathbb{P}^1$ are of the form $\Sigma^{-1}(Z)$ for $Z \subseteq Q$ closed. We can check that this is not the product topology on $\mathbb{P}^1 \times \mathbb{P}^1$.

	Regular functions on $U = \Sigma^{-1}(V)$ for $V \subseteq Q$ open are functions on $U$ of the form $\phi \circ \Sigma$ with $\phi : V \to \mathbb{K}$ regular.
\end{remark}

We can generalise this notion. The \emph{Segre embedding}\index{Segre embedding} is the map
\[
\Sigma : \mathbb{P}^n \times \mathbb{P}^m \to \mathbb{P}^{(n+1)(m+1)-1},
\]
where
\[
	\Sigma((x_0 : \cdots : x_n), (y_0 :  \cdots : y_n)) = (x_i y_j)_{0 \leq i \leq n,\, 0 \leq j \leq n}.
\]
Then we have the following:
\begin{theorem}
	$\Sigma$ is injective and its image is an algebraic variety.
\end{theorem}

Thus $\mathbb{P}^n \times \mathbb{P}^m$ acquires the structure of an algebraic variety. Another thing we can show is:
\begin{theorem}
	If $X \subseteq \mathbb{P}^n$, $Y \subseteq \mathbb{P}^n$ are projective varieties, then $\Sigma(X \times Y)$ is a projective variety.
\end{theorem}
The proofs are given in the attached handout. This allows us to think of $X \times Y$ as a projective variety.

We can also think of the geometry of $\mathbb{P}^n \times \mathbb{P}^m$ by thinking about bihomogeneous polynomials in
\[
	\mathbb{K}[x_0, \ldots, x_n, y_0, \ldots, y_m],
\]
i.e. polynomials $f$ satisfying
\[
f(\lambda x_0, \ldots, \lambda x_n, \mu y_0, \ldots, \mu y_n) = \lambda^d \mu^e f(x_0, \ldots, x_n, y_0, \ldots, y_m).
\]
We say that $f$ has \emph{bidegree}\index{bidegree} $(d, e)$. Now $f = 0$ makes sense as an equation in $\mathbb{P}^n \times \mathbb{P}^m$.

If $X$ and $Y$ are quasi-projective, i.e. $X \subseteq \bar X \subseteq \mathbb{P}^n$, $Y \subseteq \bar Y \subseteq \mathbb{P}^n$, then $X \times Y \subseteq \bar X \times \bar Y$ defines an open subset of $\bar X \times \bar Y$. This allows us to view $X \times Y$ as a quasi-projective variety.

\begin{exbox}[Blowup of $\mathbb{A}^n$]
	By the above, $\mathbb{A}^n \times \mathbb{P}^{n-1}$ is a quasi-projective variety, as $\mathbb{A}^n$ is an open subset of $\mathbb{P}^n$. Take coordinates $x_1, \ldots, x_n$ for $\mathbb{A}^n$, and $y_1, \ldots, y_n$ for $\mathbb{P}^{n-1}$. Then let
	\[
		X = Z (\{x_i y_j - x_j y_i \mid 1 \leq i < j \leq n\}) \subseteq \mathbb{A}^n \times \mathbb{P}^{n-1}.
	\]
	Let $\phi : X \to \mathbb{A}^n$ be given by
	\[
	\phi((x_1, \ldots, x_n)(y_1: \ldots: y_n)) = (x_1, \ldots, x_n),
	\]
	the projection onto $\mathbb{A}^n$. This is a morphism. We make a couple of observations:
	\begin{enumerate}
		\item If $p \in \mathbb{A}^n \setminus \{0\}$, then $\phi^{-1}(p)$ consists of one point. Indeed, let $p = (a_1, \ldots, a_n)$ with, say, $a_i \neq 0$. If
			\[
				((a_1, \ldots, a_n)(b_1 : \ldots: b_n)) \in \phi^{-1}(p),
			\]
			then for $j \neq i$, $a_i b_j - a_j b_i = 0$, so $b_j = a_jb_i/a_i$. So $b_1, \ldots, b_n$ are completely determined up to scaling. If we take $b_i = a_i$ for all $i$, then we see that
			\[
				\phi^{-1}(p) = \{((a_1, \ldots, a_n)(a_1 : \ldots : a_n))\}.
			\]
			Defining $\psi : \mathbb{A}^n \setminus \{0\} \to X \setminus \phi^{-1}(0)$ by $\psi(a_1, \ldots, a_n) = ((a_1, \ldots, a_n)(a_1 : \ldots: a_n))$, this map is an inverse to $\phi|_{X \setminus \phi^{-1}(0)} : X \setminus \phi^{-1}(0) \to \mathbb{A}^n \setminus \{0\}$.
		\item $\phi^{-1}(0) = \{0\} \times \mathbb{P}^{n-1}$.
		\item The points of $\phi^{-1}(0)$ are in one-to-one correspondence with lines through the origin in $\mathbb{A}^n$.
	\end{enumerate}
	For $n = 2$ we have the following picture: instead of taking $\mathbb{A}^2$, we somehow replace the origin with a copy of $\mathbb{P}^1$.
%lecture 12

	We prove the third statement. A line $L$ through the origin can be parametrized by $\ell : \mathbb{A}^1 \to \mathbb{A}^n$, such that $\ell(t) = (a_1 t, \ldots, a_n t)$ for some $a_1, \ldots, a_n$ not all $0$. For $t \neq 0$,
	\begin{align*}
		\phi^{-1}(a_1t, \ldots, a_nt) &= ((a_1 t, \ldots, a_nt)(a_1t:\ldots:a_nt))\\
					      &=((a_1t,\ldots,a_nt)(a_1:\ldots:a_n)).
	\end{align*}
	This is the lift of $L \setminus\{0\}$, which is given parametrically by
	\[
	t \mapsto ((a_1t,\ldots,a_nt)(a_1:\ldots:a_n)).
	\]
	This takes $\mathbb{A}^1\setminus\{0\}$ to $\phi^{-1}(\mathbb{A}^0\setminus\{0\})\subseteq X$. This extends to all of $\mathbb{A}^1$, and also $\overline{\phi^{-1}(L\setminus\{0\})}$ is the image of this parametrisation.

	Finally, we can show that $X$ is irreducible. Indeed $X = (X \setminus \phi^{-1}(0)) \cup \phi^{-1}(0)$. The first set we showed is homeomorphic to $\mathbb{A}^n \setminus \{0\}$, and hence is irreducible (an open subset of an irreducible space is irreducible). But every point in $\phi^{-1}(0)$ is in the closure of $X \setminus \phi^{-1}(0)$ by the proof of property 3, so $X \setminus \phi^{-1}(0)$ is dense in $X$.

	Now I claim if $U \subseteq X$ is a dense open set and $U$ is irreducible, then $X$ is irreducible. Indeed if $X = Z_1 \cup Z_2$ for $Z_1, Z_2$ closed, then $U = (Z_1 \cap U) \cup (Z_2 \cap U)$. These are closed in $U$ under the induced topology, so as $U$ is irreducible, we may assume $U = Z_1 \cap U$. So $U \subseteq Z_1$, hence $\bar U \subseteq Z_1$. But since $\bar U = X$, by the density of $U$ we have $X = Z_1$.

	Thus the blow-up of $X$ is irreducible.
\end{exbox}

The blow-up is a useful tool.

\begin{definition}
	If $Y \subseteq \mathbb{A}^n$ is a closed subvariety with $0 \in Y$, we define the \emph{blowing up}\index{blow-up} of $Y$ at $0$ to be $\hat Y = \overline{\phi^{-1}(Y \setminus \{0\})} \subseteq X$.
\end{definition}

\begin{exbox}
	Let $Y \subseteq \mathbb{A}^2$ be given by
	\[
	Y = Z(x_2^2 - (x_1^3 - x_1^2)).
	\]
	This has something interesting going on at the origin: it intersects it twice. The blow up lives in $X \subseteq \mathbb{A}^2 \times \mathbb{P}^1$, and is the zero set of $x_1y_2 - x_2y_1 = 0$.

	We work in two coordinate patches: $U_1 = \{y_1 \neq 0\}$, and $U_2 = \{y_2 \neq 0\}$. In $U_2$, we can set $y_2 = 1$ and the equation for $X$ becomes $x_1 = x_2 y_1$. Then
	\[
	\phi^{-1}(Y) \cap U_2 = Z(x_2^2 - (x_1^3+x_2^2),x_1-x_2y_1) \subseteq \mathbb{A}^2 \times \mathbb{A}^1.
	\]
	This is isomorphic to $Z(x_2^2 - (x_2^3y_1^3 + x_2^2y_1^2)) \subseteq \mathbb{A}^2$. Indeed, in terms of coordinate rings
	\begin{align*}
		\frac{\mathbb{K}[x_1, x_2, y_1]}{\langle x_2^2 - (x_1^3 - x_1^2), x_1 - y_1x_2\rangle} \cong \frac{\mathbb{K}[x_2, y_1]}{\langle x_2^2 - (x_2^3y_1^3 + x_2^2y_1^2)\rangle}.
	\end{align*}
	But note that the latter polynomial is $x_2^2(1 - x_2y_1^3 - y_1^2).$ Note that $\phi^{-1}(0) \cap U_2 = Z(x_2)$. The blow up $\hat Y \cap U_2 = \overline{\phi^{-1}(Y \setminus \{0\})} \cap U_2$ is now given by the equation $1 - x_2 y_1^2 - y_1^2$ in $\mathbb{A}^2$. In particular, we gain two new points $(x_2, y_1) = (0, \pm 1)$.

	For thoroughness, we also consider $\hat Y \cap U_1$, where $y_1 = 1$. Then $x_2 = x_1 y_2$, so we can eliminate $x_2$ from the equation to get $x_1^2y_2^2 - (x_1^3 + x_1^2) = x_1^2(y_2^2 - x_1 - 1)$. So $\hat Y \cap U_1$ has equation $y_2^2 - x_1 - 1 = 0$. This is the same as in the previous blow-up.
\end{exbox}

\subsection{Rational Maps}
\label{sub:rat_map}

Let $X, Y$ be varieties. Consider the equivalence relation on pairs $(U, f)$ where $U \subseteq X$ is open, and $f : U \to Y$ is a morphism. Then
\[
	(U, f) \sim (V, g) \text{ if } f|_{U \cap V} = g|_{U \cap V}.
\]
We can check that this is an equivalence relation.

\begin{definition}
	A \emph{rational map}\index{rational map} $f : X \dashrightarrow Y$ is an equivalence relation of a pair.
\end{definition}

\begin{exbox}
	If $X$ is affine and $q = f/g \in K(X)$, then we have a morphism $\phi : X \setminus Z(g) \to \mathbb{A}^1$. This defines a rational map to $\mathbb{A}^1$.
\end{exbox}

%lecture 13

\begin{definition}
	A \emph{birational map}\index{birational map} is a rational map $f : X \dashrightarrow Y$ with a rational inverse $g : Y \to X$, such that $f \circ g = \id_Y$ and $g \circ f = \id_X$ as rational maps.
\end{definition}

\begin{remark}
	We cannot always compose rational maps. Suppose we are given $f : X \dashrightarrow Y$, $g : Y \dashrightarrow Z$ with $f : U \to Y$ and $g : V \to Z$.

	If $f(U) \subseteq X \setminus V$, then we cannot compose. If this is not the case, then $f^{-1}(Y \setminus V) \subset U$ is a proper subset of $U$, and then $g \circ f : U \setminus f^{-1}(Y \setminus V) \to Z$ defines a rational map $g \circ f : X \dashrightarrow Z$.
\end{remark}

Note that the ability to compose may depend on the representative for $f, g$. One can show that if $f : X \dashrightarrow Y$ is a birational map, then there exists $U \subseteq X$, $V \subseteq Y$ such that $f$ is defined on $U$, $f(U) \subseteq V$, and $f : U \to V$ is an isomorphism.

\begin{definition}
	We say varieties $X, Y$ are \emph{birationally equivalent}\index{birationally equivalent} if there exists a birational map $f : X \dashrightarrow Y$. Equivalent, there exists $U \subseteq X$, $V \subseteq Y$, open subsets with $U \cong V$.
\end{definition}

\begin{exbox}
	Take $\varphi : X \to \mathbb{A}^n$, the blow-up of $\mathbb{A}^n$ at $0 \in \mathbb{A}^n$. This is a birational map since it induces an isomorphism $\varphi : \varphi^{-1}(\mathbb{A}^n \setminus \{0\}) \to \mathbb{A}^n \setminus \{0\}$.

	However $\varphi^{-1} : \mathbb{A}^n \to X$ is not a morphism, and is only defined on $\mathbb{A}^n \setminus \{0\}$.
\end{exbox}

\begin{remark}
	$f : X \dashrightarrow Y$ is a \emph{dominant} rational map, i.e. if $U \overset f \to Y$ is a representative for $f$, then $f(U)$ is dense in $Y$.
\end{remark}

\begin{definition}
	The \emph{function field}\index{function field} of a variety $X$ is
	\[
		K(X) = \{(U, f) \mid f : U \to \mathbb{K} \text{ is a regular function}\}/\sim,
	\]
	where $(U, f) \sim (V, g)$ if $f|_{U \cap V} = g|_{U \cap V}$. This is the field of fractions of $A(X)$ if $X$ is affine.

	If $f$ is dominant, we obtain a map $f^\# : K(Y) \to K(X)$ by $(V, \varphi) \mapsto (f^{-1}(V) \cap U, \varphi \circ f)$. Note that $f^{-1}(V) \cap U$ is non-empty, since $V \cap f(U) \neq \emptyset$ by density of $f(U)$.
\end{definition}

If $f : X \dashrightarrow Y$ is a birational map with birational inverse $g : Y \dashrightarrow X$, each are dominant since they induce isomorphisms between open subsets. Thus we get
\[
f^\# : K(Y) \to K(X), \qquad g^\# : K(X) \to K(Y)
\]
are inverse maps, so $K(X) \cong K(Y)$. In fact the converse is true: if $K(X) \cong K(Y)$, then $X$ and $Y$ are birational to each other.

\begin{exbox}
	Look at $0 \in Y \subseteq \mathbb{A}^n$, then $\hat Y \to Y$, the blow-up of $Y$ at $0$, is a birational morphism.
\end{exbox}

\newpage

\section{Tangent Spaces, Singularities and Dimension}
\label{sec:ts_s_d}

Recall that given an equation $f(x_1, \ldots, x_n) = 0$ in $\mathbb{R}^n$, where $X$ is the solution set and $p \in X$, the tangent space to $X$ is the orthogonal complement to $(\nabla f)(p)$, i.e. the tangent space to $X$ at $p$ is
\[
	T_p X = \left\{(v_1, \ldots, v_n) \in \mathbb{R}^n \mid \sum_{i = 1}^n v_i \frac{\partial f}{\partial x_i}(p) = 0 \right\}.
\]

This is a vector subspace of $\mathbb{R}^n$.

\begin{definition}
	If $X \subseteq \mathbb{A}^n$ is an affine variety with $I = I(X) = \langle f_1, \ldots, f_r \rangle$, $f_1, \ldots, f_r \in \mathbb{K}[x_1, \ldots, x_n]$ we define for $p \in X$, the \emph{tangent space}\index{tangent space} to $X$ at $p$ by
	\[
		T_p X = \left\{ (v_1, \ldots, v_n) \in \mathbb{K}^n \mid \sum_{i = 1}^n v_i \frac{\partial f_j}{\partial x_i}(p) = 0, 1 \leq j \leq r \right\}.
	\]
	The description is defined using the standard differentiation rules for polynomials.
\end{definition}

\begin{exbox}
	Set $I = \langle x_2^2 - x_1^3 \rangle \subset \mathbb{K}[x_1, x_2]$, and $X = Z(I)$. Let $p = (a_1, a_2)$. Then
	\[
		T_pX = \{(v_1, v_2) \in \mathbb{K}^2 \mid v_1(-3 a_1^2) + v_2(2 a_2) = 0 \}.
	\]
	Then we see that $\dim_{\mathbb{K}}T_p X = 1$, unless $p = (0, 0)$ in which case it is 2.
\end{exbox}

\begin{definition}
	Let $X \subseteq \mathbb{A}^n$ be an affine variety. Then the \emph{dimension}\index{dimension} of $X$ is
	\[
		\dim X = \min \{ \dim_{\mathbb{K}} T_p X \mid p \in X\}.
	\]
	We say $X$ is \emph{singular}\index{singular} at $p$ if $\dim_{\mathbb{K}} T_p X > \dim X$ in $X$.
\end{definition}

\begin{lemma}
	The set $\{ p \in X \mid \dim_{\mathbb{K}} T_p X \geq k\}$ is a closed subset of $X$, for all $k$.
\end{lemma}

\begin{proofbox}
	The key property is rank-nullity. Note that $T_p X$ is the null space of
	\[
		\begin{pmatrix}
			\partial f_1/ \partial x_1 & \cdots & \partial f_1/ \partial x_n \\
			\vdots & \ddots & \vdots \\
			\partial f_r / \partial x_1 & \cdots & \partial f_r / \partial x_n
		\end{pmatrix},
	\]
	where $I(X) = \langle f_1, \ldots, f_r \rangle$. But the dimension of the null space plus the rank of the matrix is $n$, so
	\[
	\dim T_p X \geq k \iff n - \rank \geq k \iff \rank \leq n - k.
	\]
%lecture 14
	If $A$ is an $r \times n$ matrix, then $\rank(A) \geq k + 1$ if and only if there is a $(k+1) \times (k+1)$ submatrix of $A$ whose determinant is non-zero. So $\rank J \leq n - k$ if and only if all $(n - k + 1) \times (n - k + 1)$ minors vanish.

	But these minors are simply polynomial equations. Thus the set
	\begin{align*}
		\{ p \in X \mid \dim T_pX \geq k\} &= Z(f_1, \ldots, f_r \\
	&\mid f_i \text{ a $(n-k+1) \times (n-k+1)$ minor of } J).
	\end{align*}
	Hence this set is closed.
\end{proofbox}

Recall that $p \in X$ is singular if $\dim_K T_p X \geq \dim X$, which is the infimum of $\dim T_p X$. This lemma tells us that the set of singular points is a proper closed subset.

\begin{exbox}
	Look at $y^2 - x^3 = 0$. Then the Jacobian matrix is $(2y, - 3x^2)$, which vanishes when $(x, y) = (0,0)$.

	Now consider the cone $x^2 + y^2 - z^2 = 0$. Then $J = (2x, 2y, -2z)$, vanishing at the origin.
\end{exbox}

Note we only care about where the Jacobian vanishes on the variety, not in the general space.

\subsection{Intrinsic Characterization of the Tangent Space}
\label{sub:int_char_tp}

Let $X$ be an affine variety. For $p \in X$, define $\phi_p : A(X) \to \mathbb{K}$ to be the $\mathbb{K}$-algebra homomorphism given by $\phi_p(f) = f(p)$.

\begin{definition}
	A \emph{derivation centred at} $p$ \index{derivation} is a map $D : A(X) \to \mathbb{K}$ such that:
	\begin{enumerate}[(i)]
		\item $D(f + g) = D(f) + D(g)$.
		\item $D(f \cdot g) = \phi_p(f) D(g) + D(f) \phi_p(g)$.
		\item $D(a) = 0$ for $a \in \mathbb{K}$.
	\end{enumerate}
	Denote by $\mathrm{Der}(A(X), p)$ to be the set of derivations centred at $p$. Note that $\mathrm{Der}(A(X), p)$ is a $\mathbb{K}$-vector space.
\end{definition}

\begin{theorem}
	$T_p X \cong \mathrm{Der}(A(X), p)$ as $\mathbb{K}$-vector spaces.
\end{theorem}

\begin{proofbox}
	Suppose $(v_1, \ldots, v_n) \in T_pX$, so if $I(X) = \langle f_1, \ldots, f_r\rangle$, then
	\[
	\sum_{i = 1}^n v_i \frac{\partial f_j}{\partial x_i}(p) = 0,
	\]
	for all $j$. Define $\mathbb{K}[x_1, \ldots, x_n] \to \mathbb{K}$ by
	\[
	f \mapsto \sum_{i= 1}^n v_i \frac{\partial f}{\partial x_i}(p).
	\]
	This vanishes on elements of $I(X)$, which are of the form $f = \sum g_j f_j$ for $g_j \in \mathbb{K}[x_1, \ldots, x_n]$. Then
	\begin{align*}
		f &\mapsto \sum_{i = 1}^n v_i \left( \sum_{j = 1}^r \left( \frac{\partial f_j}{\partial x_i} g_j + \frac{\partial g_j}{\partial x_i} f_j \right) (p) \right) \\
		  &= \sum_{i, j} \left( v_i \frac{\partial f_j}{\partial x_i} g_j(p) \right) = \sum_{j} g_j(p) \left( \sum_{i} v_i \frac{\partial f_j}{\partial x_i} (p) \right) = 0,
	\end{align*}
	since $f_j(p) = 0$ as $p \in X$. Thus we get a well-defined $\mathbb{K}$-linear map
	\[
		D_v : \frac{\mathbb{K}[x_1, \ldots, x_n]}{I(X)} = A(X) \to \mathbb{K}.
	\]
	We can check that this is a derivation. Now we want to generate tangent vectors from derivations.

	Given $D \in \mathrm{Der}(A(X), p)$, define $v_i = D(x_i)$. By repeated use of the Leibniz rule,
	\[
	D(f) = \sum_{i = 1}^n v_i \frac{\partial f}{\partial x_i}(p).
	\]
	For example, for $n = 2$,
	\begin{align*}
		D(x_1 x_2) &= D(x_1) \cdot x_2(p) + x_1(p) \cdot D(x_2) = v_1 x_2(p) + v_2 x_1(p) \\\
			   &= v_1 \frac{\partial (x_1 x_2)}{\partial x_1}(p) + v_2 \frac{\partial (x_1 x_2)}{\partial x_2}(p).
	\end{align*}
	Therefore we find
	\[
	D(f_j) = \sum_{i = 1}^n v_i \frac{\partial f_j}{\partial x_i}(p),
	\]
	but $f_j \in I(X)$, so $D(f_j) = 0$. Hence
	\[
	\sum_{i = 1}^n v_i \frac{\partial f_j}{\partial x_i}(p) = 0
	\]
	for all $j$, so $(v_1, \ldots, v_n) \in T_pX$.
\end{proofbox}

\begin{remark}
	Singular points and tangent spaces are intrinsic to affine varieties.
\end{remark}

\begin{definition}
	Let $X$ be a variety, and $p \in X$. We define the \emph{local ring}\index{local ring} to $X$ at $p$ to be
	\[
		\mathcal{O}_{X, p} = \{(U, f) \mid U \text{ is an open neighbourhood of } p, f: U \to \mathbb{K} \text{ regular}\} / \sim,
	\]
	where $(U, f) \sim (V, g)$ if $f|_{U \cap V} = g|_{U \cap V}\subseteq K(X)$, the field of functions.
\end{definition}

\begin{exbox}
	\begin{enumerate}
		\item If $X \subseteq \mathbb{A}^n$ is an affine variety,
			\[
				\mathcal{O}_{X, p} = \left\{ \frac{f}{g} \in K(X) \mid g(p) \neq 0, f, g \in A(X)\right\}.
			\]
			\item If $X \subseteq \mathbb{P}^n$ is projective, then
			\[
				\mathcal{O}_{X, p} = \left\{ \frac{f}{g} \mid f, g \in \frac{\mathbb{K}[x_1, \ldots, x_n]}{I(X)}, g(p) \neq 0, f, g \text{ hom, same degree}\right\},
			\]
			as a subset of $K(X)$.
	\end{enumerate}
\end{exbox}

\begin{remark}
	The definition of $\mathcal{O}_{X, p}$ makes it intrinsic, i.e. not dependent on the embedding. Moreover, $\mathcal{O}_{X, p}$ is a ring:
	\[
		(U, f) + (V, g) = (U \cap V, f|_{U \cap V} + g|_{U \cap V}),
	\]
	and multiplication defined similarly. We can define
	\[
		m_p = \{ (U, f) \in \mathcal{O}_{X, p} \mid f(p) = 0\}.
	\]
	This is an ideal, and every element of $\mathcal{O}_{X, p} \setminus m_p$ is invertible. Thus $m_p$ is the unique maximal ideal of $\mathcal{O}_{X, p}$.
\end{remark}

\begin{definition}
	A ring $A$ with a unique maximal ideal is called a \emph{local ring}\index{local ring}.
\end{definition}
%lecture 15

\begin{theorem}
	If $X \subseteq \mathbb{A}^n$ is an affine variety, then $T_p X \cong (m_p / m_p^2)^\ast$, where $V^\ast$ is the dual of the $\mathbb{K}$-vector space $V$.
\end{theorem}

\begin{proofbox}
	Note that there is an isomorphism
	\begin{align*}
		\mathcal{O}_{X, p} / m_p &\to \mathbb{K}, \\
		f &\mapsto f(p).
	\end{align*}
	This is surjective since constants are regular functions, and injective by the definition of $m_p$. Then we can define the $\mathbb{K}$-vector space structure on $m_p / m_p^2$ by identifying $\mathbb{K}$ with $\mathcal{O}_{X,p} / m_p$, and
	\[
		(f + m_p) \cdot (g + m_p^2) = (f \cdot g + m_p^2).
	\]
	We will show that $\mathrm{Der}(A(X), p) \subseteq (m_p/ m_p^2)^\ast$. Given $D \in \mathrm{Der}(A(X), p)$, we define $\phi_D : m_p / m_p^2 \to \mathbb{K}$ as follows: for $f, g \in A(X)$, $g(p) \neq 0$ and $f(p) = 0$, with
	\[
		\left( X \setminus Z(g), \frac{f}{g} \right) \in m_p \subseteq \mathcal{O}_{X, p},
	\]
	we set
	\[
	\phi_D \left( \frac{f}{g} \right) = D \left( \frac{f}{g} \right) = \frac{g(p) D(f) - f(p) D(g)}{g(p)^2} = \frac{D(f)}{g(p)},
	\]
	since $f(p) = 0$. Note that if $f_1/g_1, f_2/g_2 \in m_p$, then
	\[
	\phi_D \left( \frac{f_1 f_2}{g_1 g_2} \right) = \frac{f_1(p)}{g_1(p)} \cdot \phi_D \left( \frac{f_1}{g_1} \right) + \frac{f_1(p)}{g_1(p)} \phi_D \left( \frac{f_2}{g_2} \right) = 0.
	\]
	Thus $\phi_D (m_p^2) = 0$, so we obtain a well defined map $\phi_D : m_p/m_p^2 \to \mathbb{K}$. Conversely, if we are given $\phi : m_p /m_p^2 \to \mathbb{K}$, for $p = (a_1, \ldots, a_n) \in X \subseteq \mathbb{A}^n$, note that $x_i - a_i \in m_p$ for all $i$. Then define
	\[
	D_\phi(x_i - a_i) = \phi(x_i - a_i).
	\]
	This is sufficient to determine $D_\phi$ as before.
\end{proofbox}

\begin{exbox}
	Suppose that $X = \mathbb{A}^n$, and $p = 0$. Then
	\[
		\frac{m_p}{m_p^2} = \frac{(X_1, \ldots, X_n)}{(X_1, \ldots, X_n)^2}.
	\]
\end{exbox}

\begin{definition}
	If $X$ is any variety, the \emph{Zariski tangent space}\index{Zariski tangent space} to $X$ at $p$ is
	\[
	T_p X = (m_p/m_p^2)^\ast,
	\]
	where $m_p \subseteq \mathcal{O}_{X, p}$ is the maximal ideal.
\end{definition}

\begin{theorem}
	Any variety has an open cover by affine varieties.
\end{theorem}

Note if $X \subseteq \mathbb{P}^n$ is projective, then $\{U_i \cap X \mid 0 \leq i \leq n\}$ is a cover of $X$ by affine varieties.

\begin{proofbox}
	Consider the most general case, where $X$ is quasi-projective. Then each $U_i \cap X$ is quasi-affine, so it is enough to show that each quasi-affine variety is covered by affine varieties.

	Let $p \in X$. We will find an affine neighbourhood of $p$ in $X$. Then $\bar X \subseteq \mathbb{A}^n$, the closure, is an affine variety, and $Z = \bar X \setminus X$ is closed in $\bar X$. Choose $f \in I(Z)$ with $f(p) \neq 0$. Then $\langle f \rangle \subseteq I(X)$, so
	\[
	Z(f) \subseteq Z(I(Z)) = Z,
	\]
	so $p \in \bar X \setminus Z(f) \subseteq \bar X \setminus Z = X$.

	But $\bar X \setminus Z(f)$ can be identified with the closed subset of $\mathbb{A}^{n+1}$ given by $Z(I(\bar X), yf - 1)$, as in the first example sheet.
\end{proofbox}

\begin{remark}
	The definition of dimension and singular points goes through unchanged with the Zariski tangent space:
	\[
		\dim X = \inf\{ \dim T_p X \mid p \in X\},
	\]
	and $p \in X$ is singular if $\dim X < \dim T_p X$. By applying the above theorem, in fact the set of singular points of an arbitrary variety $X$ is closed in $X$. This also shows that the dimension, and singularity is intrinsic to $X$.
\end{remark}

We can alternatively define dimension in the Zariski tangent space as follows.

\begin{definition}
	if $F / \mathbb{K}$ is a finitely generated field extension, then the \emph{transcendence degree}\index{transcendence degree} of $F / \mathbb{K}$, written as $\mathrm{trdeg}_K F$, is the cardinality of a transcendence basis.
\end{definition}

\begin{definition}
	If $A$ is a ring, the \emph{Krull dimension}\index{Krull dimension} of $A$ is the largest $n$ such that there exists a chain of prime ideals
	\[
	P_0 \subset P_1 \subset \cdots \subset P_n \subseteq A.
	\]
\end{definition}

\begin{definition}
	If $X$ is a topological space, the \emph{Krull dimension} of $X$ is the largest $n$ such that there exists a chain of irreducible subsets
	\[
	Z_0 \subset Z_1 \subset \cdots \subset Z_n \subseteq X.
	\]
\end{definition}

%lecture 16

\begin{remark}
	If $\mathbb{K}$ is algebraically closed, then $\dim \mathbb{K}[x_1, \ldots, x_n]$ agrees with the Krull dimension of $\mathbb{A}^n$.

	If $X \subseteq \mathbb{A}^n$ is an affine variety, then $\dim A(X)$ is equal to the Krull dimension of $X$. We can check there is a one-to-one correspondence between prime ideals of $A(X)$ and irreducible closed subsets of $X$.
\end{remark}

\begin{theorem}
	If $X$ is a variety, then
	\[
		\dim X = \mathrm{trdeg}_{\mathbb{K}} K(X) = \text{Krull dimension of } X = \text{Krull dimension of } \mathcal{O}_{X, p},
	\]
	for any $p \in X$.
\end{theorem}

\begin{proofbox}
	This is by dimension theory. It is non-examinable.
\end{proofbox}

\begin{exbox}
	In the first example sheet, we showed that if
	\[
	X = Z(f) \subseteq \mathbb{A}^2,
	\]
	then the closed subsets of $X$ are $X$, and the finite subsets of $X$. Thus the Krull dimension of $X$ is 1.
\end{exbox}

\newpage

\section{Curves}
\label{sec:c}

\begin{definition}
	An (algebraic) \emph{curve}\index{curve} is a variety $C$ with $\dim C = 1$.
\end{definition}

\begin{definition}
	Let $C \subseteq \mathbb{P}^n$ be a projective non-singular curve. We define $\mathrm{Div} C$ to be the free abelian group generated by points of $C$. This is called the group of \emph{divisors}\index{divisor} of $C$.

	An element of $\mathrm{Div} C$ is of the form $\sum_{i= 1}^n a_i p_i$, for $a_i \in \mathbb{Z}$, $p_i \in C$.
\end{definition}

The point of this definition is as follows. Consider $C = \mathbb{P}^1$. An element of $K(C)$ is a ratio
\[
\frac{f(x_0, x_1)}{g(x_0, x_1)},
\]
where $f, g$ are homogeneous polynomials of the same degree. We can factor
\[
	\frac{f}{g} = \frac{\prod_i (b_i x_0 - a_i x_1)^{m_i}}{\prod_j (d_j x_0 - c_j x_1)^{n_j}},
\]
where $\sum m_i = \sum n_j = d$. Let $p_i = (a_i : b_i)$, and $q_j = (c_j : d_j)$. Then $f/g$ has a zero of order $m_i$ at $p_i$, and a pole of order $n_j$ at $q_j$. The divisors of zeroes and poles of $f/g$ is
\[
	\left( \frac{f}{g} \right) = \sum_{i} m_i p_i - \sum_j n_j q_j.
\]
We call a divisor $D \in \mathrm{Div} \, C$ \emph{principal}\index{principal divisor} if it is of the form $(f/g)$. Let $\mathrm{Prin}\, C \subseteq \mathrm{Div}\, C$ be the subgroup of principal divisors, and define the \emph{class group}\index{class group} of $C$, to be
\[
\mathrm{Cl}\, C = \frac{\mathrm{Div}\, C}{\mathrm{Prin}\, C}.
\]
We can see that $\mathrm{Cl}\, \mathbb{P}^1 = \mathbb{Z}$.

In order for this definition to be sensible, for any non-singular curve $f \in K(X)$, we want to define the order of $0$ of a pole at $p \in X$.

\begin{lemma}
	Let $A$ be a ring, $M$ a finitely generated $A$-module and $I \subset A$ an ideal such that $I M = M$. Then there exists $x \in A$ such that $x \equiv 1 \mod I$, and $x M = 0$.
\end{lemma}

\begin{proofbox}
	Recall if we have $\phi : M \to M$ an $A$-module homomorphism with $\phi(M) \subseteq IM$, then there exists $a_1, \ldots, a_n \in I$ such that
	\[
	\phi^n + a_1 \phi^{n-1} + \cdots + a_n = 0.
	\]
	Take $\phi$ to be the identity map. This means multiplication by $1 + a_1 + a_2 + \cdots + a_n$ is the zero homomorphism of $M$. Then taking this to be $x$, $x \equiv 1 \mod I$ and $x M = 0$.
\end{proofbox}

\begin{theorem}[Nakayama's Lemma]
	Let $A$ be a local ring with maximal ideal $m$. Let $I \subseteq m$ be an ideal. Then for finitely generated $M$, $I M = M$ implies $M = 0$.
\end{theorem}

\begin{proofbox}
	As before, there exists $x \in A$ with $x M = 0$ and $x \equiv 1 \mod I$, so $x \equiv 1 \mod m$. Thus $x \not \in m$. But this implies $x$ is invertible, otherwise $\langle x \rangle \neq A$, and hence $\langle x \rangle \subseteq m$.

	But then $M = x^{-1}( x M) = 0$.
\end{proofbox}

Note all $x \in A \setminus m$ for a local ring $A$ are invertible.

\begin{corollary}
	Let $A$ be a local ring with maximal ideal $m$, $M$ a finitely generated $A$-module, and $I \subseteq m$ an ideal. Then if $M = IM + N$ for a submodule $N \subseteq M$, we have $M = N$.
\end{corollary}

\begin{proofbox}
	Note that $M/N$ satisfies
	\[
	I \left( \frac{M}{N} \right)= \frac{IM + N}{N}.
	\]
	If $M = IM + N$, we get
	\[
	I \left(\frac{M}{N} \right) = \frac{M}{N} \implies \frac{M}{N} = 0.
	\]
\end{proofbox}

\begin{corollary}
	$A$ is local ring with $m$ its maximal ideal. Let $x_1, \ldots, x_n \in M$ be a set of elements of a finitely generated module $M$, such that the images $\bar x_1 ,\ldots, \bar x_n \in M/mM$ form a basis for $M/mM$ as an $A/m$-vector space. Then $x_1, \ldots, x_n$ generate $M$ as an $A$-module.
\end{corollary}

\begin{remark}
	$A/m$ is a field since $m$ is maximal. Further $M/mM$ is a vector space over $A/m$, since
	\[
		(a + m) \cdot (\alpha + m M) = a \alpha + m M,
	\]
	is well-defined.
\end{remark}

\begin{proofbox}
	Let $N \subseteq M$ be the submodule of $M$ generated by $x_1, \ldots, x_n$. Then the composition
	\[
	N \injto M \to M/mM
	\]
	is surjective, so $M = N + mM$. By the previous corollary, $M = N$.
\end{proofbox}

%lecture 17

\begin{corollary}
	Let $C \subseteq \mathbb{P}^n$ be a non-singular projective curve. Then
	\[
		\{(U, f) \mid f(p) = 0\} = m_p \subseteq \mathcal{O}_{C, p}
	\]
	is a principal ideal.
\end{corollary}

\begin{proofbox}
	We begin by proving $\mathcal{O}_{C, p}$ is Noetherian. Replace $C$ by an open affine neighbourhood of $p$ in $C$, say $C'$. This does not change $\mathcal{O}_{C, p}$. Then
	\[
		\mathcal{O}_{C', p} = \left\{ \frac{f}{g} \mid f, g \in A(C') = \frac{\mathbb{K}[x_1, \ldots, x_n]}{I(C')}, g(p) \neq 0 \right\} \subseteq K(C').
	\]
	If $J \subseteq \mathcal{O}_{C', p}$ is a ideal, then
	\[
		J = \left\{ \frac{f}{g} \mid f \in A(C') \cap J, g \in A(C'), g(p) \neq 0 \right\} \subseteq \mathcal{O}_{C', p}. \tag{$\ast$}
	\]
	Indeed, one way is because if $f/g \in J$, then $g(f/g) = f \in J$, so $f \in A(C') \cap J$. Conversely, if $f \in A(C') \cap J$, then $f/g = 1/g \cdot f \in J$.

	Now $\mathbb{K}[x_1, \ldots, x_n]$ is Noetherian by Hilbert's basis theorem, hence
	\[
		A(C') = \mathbb{K}[x_1, \ldots, x_n] / I(C')
	\]
	is Noetherian. Hence $A(C') \cap J$ is finitely generated, and by $(\ast)$, the set of generators of $A(C') \cap J$ generate $J$ as an ideal in $\mathcal{O}_{C', p}$. Since $C$ is non-singular of dimension $1$,
	\[
	1 = \dim T_p C = \dim(m_p/m_{p^2})^\ast.
	\]
	Also the map $\mathcal{O}_{C, p} / m_p \to \mathbb{K}$ by $f + m_p \mapsto f(p)$. Thus $m_p / m_p^2$ is a one-dimensional vector space over $\mathcal{O}_{C, p} / m_p$, hence by the previous corollary to Nakayama's lemma, $m_p$ is generated by the lift of a 1-element basis of $m_p/m_p^2$. Thus $m_p$ is principal (we need $m_p$ finitely generated here!).
\end{proofbox}

\begin{remark}
	Let $t \in m_p$ be a generator. Then we get a chain of ideals
	\[
	\mathcal{O}_{C, p} \supseteq m_p = (t) \supseteq (t^2) \supseteq (t^3) \supseteq \cdots.
	\]
	Note that if $(t^{k+1}) = (t^k)$, then $m_p \cdot (t^k) = (t^k)$. But then Nakayama's lemma tells us $(t^k) = 0$, but it cannot since $\mathcal{O}_{C, p}$ is an integral domain and $t \neq 0$.

	Also, consider
	\[
	I = \bigcap_{k = 1}^\infty (t^k).
	\]
	Then clearly $t \cdot I = I$, so $m_p \cdot I = I$, hence $I = 0$.
\end{remark}

\begin{corollary}
	If $f \in \mathcal{O}_{C, p} \setminus \{0\}$, there exists a unique $\nu \geq 0$ such that $f \in (t^\nu)$, $f \not \in (t^{\nu + 1})$.
\end{corollary}

\begin{definition}
	Define $\nu : \mathcal{O}_{C, p} \setminus \{0\} \to \mathbb{Z}$ by $\nu(f) = \nu$, as above.
\end{definition}

We can show that $\nu$ satisfies the following:
\begin{itemize}
	\item $\nu(f \cdot g) = \nu(f) + \nu(g)$.
	\item $\nu(f + g) \geq \min\{\nu(f), \nu(g)\}$ with equality if  $\nu(f) \neq \nu(g)$.
\end{itemize}
We can extend $\nu$ to a map
\[
	\nu : K(C) \setminus \{0\} = K(C)^\ast \to \mathbb{Z},
\]
by $\nu(f/g) = \nu(f) - \nu(g)$. $\nu$ is an example of a \emph{discrete valuation}\index{discrete valuation}. It essentially tells us the order of the zero of $f/g$ at $p$.

\begin{definition}
	Let $K$ be a field. A \emph{discrete valuation} on $K$ is a function $\nu : K^\times \to \mathbb{Z}$ such that:
	\begin{enumerate}[(i)]
		\item $\nu(f \cdot g) =\nu(f) + \nu(g)$.
		\item $\nu(f + g) \geq \min\{\nu(f), \nu(g)\}$ with equality if $\nu(f) \neq \nu(g)$.
	\end{enumerate}
	Given a discrete valuation, we define the corresponding \emph{discrete valuation ring}\index{discrete valuation ring} (DVR) by
	\[
		R = \{f \in K^\times \mid \nu(f) \geq 0\} \cup \{0\},
	\]
	a subring of $K$. Moreover, we can take $m = \{f \in K^\times \mid \nu(f) \geq 1\} \cup \{0\}$, which is the unique maximal ideal of $R$. If $f \in R \setminus m$, then $\nu(f) = 0$, so $\nu(f^{-1}) = 0$, and so $f^{-1} \in R$.
\end{definition}

\begin{exbox}
	\begin{enumerate}
		\item Take $R = \mathcal{O}_{C, p} \subseteq K = K(C)$. Then $\nu$ is the discrete valuation we defined.
		\item Let $p \in \mathbb{Z}$ be prime, and $K = \mathbb{Q}$. Then any rational number can be written as $\frac{a}{b} p^\nu$, with $(a, p) = (b, p) = 1$. Then define
			\[
			\nu_p \left( \frac{a}{b} p^\nu\right) = \nu.
			\]
			This is a discrete valuation, with discrete valuation ring
			\[
				\mathbb{Z}_{(p)} = \left\{ \frac{a}{b} \in \mathbb{Q} \mid p \nmid b \right\}.
			\]
			These are the $p$-adic valuation and $p$-adic integers, respectively.
		\item Take $K = \mathbb{K}(x)$, and $a \in \mathbb{K}$. Then
			\[
			\nu_a \left( \frac{f}{g} (x - a)^\nu \right) = \nu,
			\]
			where $f, g$ are relatively prime to $x-a$. Here the DVR is $\mathcal{O}_{\mathbb{A}^1, 0}$.
		\item Let $K = \mathbb{K}(X)$, and define
			\[
			\nu(f/g) = \deg g - \deg f.
			\]
			This is the ``order'' of the zero at $\infty$.
	\end{enumerate}
\end{exbox}

The setup is follows: let $C \subseteq \mathbb{P}^n$ be a projective non-singular curve. Then each point $p \in C$ gives a valuation $\nu_p : K(C)^\times \to \mathbb{Z}$, with DVR $\mathcal{O}_{C, p}$. For $f \in K(C)^\times$, we define the divisor of zeros and poles of $f$ to be
\[
	(f) = \sum_{p \in C} \nu_p(f) p.
\]
We need to check this is finite.
%lecture 18

Note $f$ is represented on some open subset $U \subseteq C$ by $g/h$, for homogeneous polynomials $g, h$. We shrink $U$ by removing $Z(g)$, $Z(h)$. Now if $p \in U$, $f = g/h \in \mathcal{O}_{C, p}$ is a regular function with $f(p) \neq 0$, so $\nu_p(f) = 0$. Thus the sum defining $(f)$ is a sum over points of $C \setminus U$, which is a finite set.

Here, we use the fact that $\dim C = 1$, so the irreducible sets are $C$ and singleton sets.

\begin{definition}
	The group of \emph{principal divisors}\index{principal divisor} on $C$ is
	\[
		\mathrm{Prin} C = \{(f) \mid f \in K(C) \setminus \{0\}\}.
	\]
	This is a subgroup, as $(fg) = (f) + (g)$, and $(f^{-1}) = (-f)$.

	The (divisor) \emph{class group}\index{class group} is
	\[
	\mathrm{Cl} C = \frac{\mathrm{Div} C}{\mathrm{Prin} C}.
	\]
	If $D, D' \in \mathrm{Div} C$ satisfy $D - D' = (f)$ for some $f \in K(C)^\times$, then we say $D$ is \emph{linearly equivalent}\index{linearly equivalent} to $D'$, and we write $D \sim D'$.
\end{definition}

Extending morphisms to projective space: let $C$ be a projective non-singular curve, and $\emptyset \neq U \subseteq C$ is an open subset, and $f_0, \ldots, f_n$ being regular functions on $U$ without a common zero.

Then we obtain a morphism $f : U \to \mathbb{P}^n$ by $p \mapsto (f_0(p): \ldots: f_n(p))$.

\begin{theorem}
	$f : U \to \mathbb{P}^n$ extends to a morphism $f : C \to \mathbb{P}^n$.
\end{theorem}

\begin{proofbox}
	Suppose either $f_i$ has a pole at $p \in C$, i.e. $\nu_p(f_i) < 0$, or all $f_i$'s are zero at $p$. Let
	\[
		m = \min \{\nu_p(f_i) \mid 0 \leq i \leq n\}.
	\]
	Let $t$ be a local parameter at $p$, i.e. a generator of the maximal ideal $m_p \subseteq \mathcal{O}_{C, p}$. So $\nu_p(t) = 1$. Then $\nu_p(t^{-m} f_i) = \nu_p(f_i) - m$, so $\nu_p(t^{-m} f_i) = 0$ for some $i$, and $\nu_p(t^{-m} f_j) \geq 0$. Thus $t^{-m} f_0, \ldots, t^{-m} f_p \in \mathcal{O}_{C, p}$ are regular functions which do not simultaneously vanish at $p$.

	Hence in some neighbourhood $V$ of $p$, we obtain a morphism $f_p : V \to \mathbb{P}^n$ by $q \mapsto ((t^{-m} f_0)(q), \ldots, (t^{-m} f_n)(q))$. This agrees with $f$ on the intersection by rescaling, so gluing gives a morphism.
\end{proofbox}

\begin{proposition}
	Let $f : X \to Y$ be a non-constant morphism between projective non-singular curves. Then:
	\begin{enumerate}[(i)]
		\item $f^{-1}(q)$ is a finite set, for all $q \in Y$.
		\item $f$ induces an inclusion $K(Y) \injto K(X)$ such that $[K(X):K(Y)]$ is finite. We call $[K(X): K(Y)]$ the \emph{degree}\index{degree} of $f$.
	\end{enumerate}
\end{proposition}

\begin{proofbox}
	
	\begin{enumerate}[(i)]
		\item $f^{-1}(q) \subseteq X$ is closed, and since $\dim X = 1$, either $f^{-1}(q)$ is finite, or $f^{-1}(q) = X$. The latter contradicts $f$ non-constant.
		\item If $\phi \in K(Y)$, $\phi$ defines a regular function on some open $U \subseteq Y$, i.e. $\phi : U \to \mathbb{K}$.

			Then $\phi \circ f$ makes sense, provided $f(X) \not \subseteq Y \setminus U$. But $f(X)$ is irreducible, so $f$ is constant if $f(X) \not \subseteq Y \setminus U$. Thus $\phi \circ f$ makes sense as a rational function on $X$. Thus $K(Y) \to K(X)$ exists, and is automatically an injection since both are fields.

			We omit the proof of finiteness (the idea is to look at the transcendence degrees; both are $1$).
	\end{enumerate}
\end{proofbox}

\begin{definition}
	Suppose $f : X \to Y$ is a non-constant morphism between projective non-singular curves. Let $p \in Y$, $m_p = (t) \subseteq \mathcal{O}_{Y, p}$, where $t$ is a local parameter.

	Let $q \in f^{-1}(p)$. Then $t \circ f \in \mathcal{O}_{X, q}$. Define
	\[
	e_q = \nu_q(t \circ f),
	\]
	the \emph{degree of ramification}\index{degree of ramification} of $f$ at $q$.
\end{definition}

\begin{theorem}
	Let $f : X \to Y$ be as above. Then for $p \in Y$,
	\[
	\sum_{q \in f^{-1}(p)} e_q = \deg f
	\]
	is the degree of $f$.
\end{theorem}

The proof is omitted, however the theorem is crucial.

\begin{exbox}
	\begin{enumerate}[1.]
		\item Suppose $\chr \mathbb{K} \neq 2$, and take $f : \mathbb{P}^1 \to \mathbb{P}^1$ by $(u : v) \mapsto (u^2 : v^2)$. Setting $v = 1$, this gives a morphism $\mathbb{A}^1 \to \mathbb{A}^1$ by $u \mapsto u^2$.

			If $p \in \mathbb{A}^1$, then $t = u - p$ is a local parameter at $p$, and $t \circ f = u^2 - p = (u - q)(u + q)$ where $q^2 = p$, so $e_q = e_{-q} = 1$, hence $\deg f = e_q + e_{-q} = 2$.

			If $p = 0$, then $f^{-1}(p) = \{0\}$, and $e_0 = \nu_0(u^2) = 2$.

			Looking as function fields, letting $K(\mathbb{P}^1) = \mathbb{K}(u)$, then this map is $\mathbb{K}(u) \to \mathbb{K}(u)$ by $u \mapsto u^2$.
		\item Look at $\chr \mathbb{K} = 2$, and $f : \mathbb{P}^1 \to \mathbb{P}^1$ by $(u : v) \mapsto (u^p : v^p)$. Setting $v = 1$, this is $u \mapsto u^p$.

			Here $f^{-1}(q) = \{r\}$, where $r^p = q$ is unique. Then $t = u - q$, and $t \circ f= u^p - q = (u - r)^p$.
	\end{enumerate}
\end{exbox}

Let $X$ be a projective non-singular curve, and $f \in K(X)^\times$. This gives a morphism $X \supseteq U \overset{(f, 1)} \to \mathbb{P}^1$, where $U$ is the open set on which $f$ is regular.

This extends to $f : C \to \mathbb{P}^1$, non-constant as long as $f \not \in \mathbb{K}$.

% lecture 19

We can either extend by $p \mapsto (f(p): 1)$ or  $(g(p) : h(p))$, but both can be ill-defined, so we need to use our theorem from last time somewhere. Then note that
\[
	(f) = \sum_{p \in f^{-1}((0 : 1))}e_p \cdot p - \sum_{q \in f^{-1}((1 : 0))} e_q \cdot q.
\]
Thus if we define
\[
\deg \sum_{p \in C} a_p \cdot p = \sum_{p \in C}a_p,
\]
then
\[
\deg (f) = \deg f - \deg f = 0.
\]
Thus every principal divisor is degree 0. So the homomorphism $\deg : \mathrm{Div} C \to \mathbb{Z}$ descends to $\deg : \mathrm{Cl} C \to \mathbb{Z}$, and this is surjective as $\deg p = 1$.

\subsection{Linear Systems}%
\label{sub:lin_sys}

Let $D \in \mathrm{Div} C$, so
\[
D = \sum n_i \cdot p_i.
\]
We say that $D$ is \emph{effective}\index{effective} if $n_i \geq 0$, for all $i$. Define
\[
	\mathcal{L} (D) = \{ f \in K(C)^{\times} \mid D + (f) \text{ is effective}\} \cup \{0\}.
\]

\begin{lemma}
	$\mathcal{L}(D)$ is a vector space.
\end{lemma}

\begin{proofbox}
	Note that $f \in \mathcal{L}(D) \implies cf \in \mathcal{L}(D)$ for $c \in K, c \neq 0$ since $(f) = (cf)$.

	If $f, g \in \mathcal{L}(D)$, where $f, g$ are non-zero and $f + g \neq 0$, then
	\begin{align*}
		(f + g) &= \sum_{p} \nu_p (f + g) p,
	\end{align*}
	and $\nu_p(f + g) \geq \min\{\nu_p(f), \nu_p(g)\}$. Thus if $D + (f)$, $D + (g)$ are effective, then so is $D + (f + g)$.
\end{proofbox}

\begin{theorem}
	$\mathcal{L}(D)$ is a finite-dimensional vector space, and $L(0) = \mathbb{K}$. Furthermore, $\dim_{\mathbb{K}} \mathcal{L}(D) \leq \deg D + 1$, for $\deg D \geq 0$.
\end{theorem}

\begin{proofbox}
	We prove this by induction on $\deg D$. If $\deg D < 0$, there are no effective divisors linearly equivalent to $D$ since $\deg (D + (f)) = \deg D < 0$, so $\mathcal{L}(D) = 0$.

	Suppose that $\deg D \geq 0$, and write
	\[
	D = \sum_{i = 1}^m n_i p_i.
	\]
	Pick $p \in C \setminus \{p_1, \ldots, p_m\}$. Consider the map
	\begin{align*}
		\lambda : \mathcal{L}(D) &\to \mathbb{K} \\
		f &\mapsto f(p),
	\end{align*}
	which makes sense since $\nu_p(f) \geq 0$ for $f \in \mathcal{L}(D)$, since otherwise the coefficient of $p$ in $D + (f)$ is negative.

	If $f \in \Ker \lambda$, then $f \in m_p \subseteq \mathcal{O}_{C, p}$, so $\nu_p(f) \geq 1$. Thus $f \in \mathcal{L}(D - p)$. Note also $\mathcal{L}(D - p) \subseteq \mathcal{L}(D)$, since if $D - p + (f)$ is effective, so is $D + (f)$. Thus $\mathcal{L}(D - p) = \Ker \lambda$, and
	\[
	\frac{\mathcal{L}(D)}{\mathcal{L}(D - p)} \subseteq \mathbb{K}.
	\]
	Thus $\dim_{\mathbb{K}} \mathcal{L}(D) \leq \dim \mathcal{L}(D - p) + 1$. Thus by induction, $\dim_{\mathbb{K}} \mathcal{L}(D) \leq \deg D + 1$.

	Thus $\dim \mathcal{L}(0) \leq 1$, but $\mathbb{K} \subseteq \mathcal{L}(D)$ since $0 + (c) = 0$, so $\dim \mathcal{L}(0) = 1$.
\end{proofbox}

\begin{remark}
	$\mathcal{L}(0) = \{f : C \to \mathbb{K} \text{ regular}\}$, and hence the regular functions on $C$ are constants.
\end{remark}

\begin{definition}
	Given a divisor $D$, we define the \emph{complete linear system}\index{complete linear system} associated to $D$ to be
	\begin{align*}
		|D| &= \{D' \in \mathrm{Div} C \mid D' \text{ effective}, D' \sim D\} &\\
		    &= \frac{L(\mathcal{D}) \setminus \{0\}}{\sim} &(f \sim \lambda f)\\
		    &= \mathbb{P}(\mathcal{L}(D)) &.
	\end{align*}
\end{definition}

\subsection{Morphisms to Projective Space}%
\label{sub:morph_proj}

Let $D$ be a divisor, $f_0, \ldots, f_n \in \mathcal{L}(D)$, with not all $f_i$ being 0. This gives a morphism $f : C \to \mathbb{P}^n$ by $p \mapsto (f_0(p): \ldots : f_n(p))$.

\begin{definition}
	Let $f : C \to \mathbb{P}^n$ be a morphism. Let $H \subseteq \mathbb{P}^n$ be a hyperplane, with $f(C) \not\subseteq H$.

	We define $f^\ast H \in \mathrm{Div} C$ as follows. Let $H = Z(\phi)$, with $\phi$ a linear homogeneous polynomial, and choose $\psi$ linear homogeneous so that $H' = Z(\psi)$ satisfies
	\[
	f^{-1}(H) \cap f^{-1}(H') = \emptyset.
	\]
	Define
	\[
	f^\ast H = \sum_{p \in f^{-1}(H)} \nu_p\left( \frac{\phi}{\psi} \circ f \right) p.
	\]
\end{definition}

Insert cool diagram.

\begin{remark}
	This is independent of the choice of $\psi$, as
	\[
	\frac{\phi}{\psi'} = \frac{\phi}{\psi} \cdot \frac{\psi}{\psi'},
	\]
	and the latter does not affect the coefficient of vanishing.
\end{remark}

Now let's relate this to morphisms. Let $f_0, \ldots, f_n \in \mathcal{L}(D)$ be such that:
\begin{enumerate}[(i)]
	\item the $f_i$ aren't all $0$,
	\item for all $p \in C$, there exists $a_0, \ldots, a_n \in \mathbb{K}$ such that the coefficient of $p$ in $D + (\sum a_i f_i)$ is 0.
\end{enumerate}
As above, we get a morphism $f : C \to \mathbb{P}^n$. Let $H \subseteq \mathbb{P}^n$ be given by an equation $\sum a_i x_i = 0$.

\begin{theorem}
	$f^\ast H = D + (\sum a_i f_i)$.
\end{theorem}

% lecture 20

\begin{proofbox}
	Let $p \in f^{-1}(H)$. Suppose the coefficient of $p$ in $D$ is 0. Let $\psi = \sum a_i x_i$. Let $b_0, \ldots, b_n$ be such that $p \not \in Z(\sum b_i x_i)$, then let $\psi = \sum b_i x_i$. Then the coefficient of $p$ in $f^\ast H$ is
	\[
	\nu_p \left( \frac{\phi}{\psi} \circ f \right).
	\]
	Necessarily, $f_0, \ldots, f_n$ do not have a pole at $p$, since otherwise $D + (f_i)$ has a negative coefficient for $p$. Thus, $f_0, \ldots, f_n$ are regular in a neighbourhood of $p$, so we can write $f = (f_0: \ldots : f_n)$ in this neighbourhood. Now
	\[
	\nu_p \left( \frac{\phi}{\psi} \circ f \right) = \nu_p \left( \frac{\sum a_i f_i}{\sum b_i f_i} \right) = \nu_p \left( \sum a_i f_i \right),
	\]
	since $\sum b_i f_i$ is non-vanishing and regular at $p$. But this is precisely the coefficient of $p$ in $D + (\sum a_i f_i)$.

	If $p$ appears in $D$ with coefficient $m$, then
	\[
	\nu_p \left( \sum_i b_i f_i \right) \geq -m,
	\]
	for any $b_0, \ldots, b_n \in \mathbb{K}$. There is also some choice of $b_0, \ldots, b_n$ with equality, by assumption.

	In a neighbourhood of $p$, the morphism $f$ is given by
	\[
	f = (t^m f_0 : \ldots : t^m f_n),
	\]
	where $t$ is a local parameter of $p$. Thus the coefficient of $p$ in $f^\ast H$ is
	\begin{align*}
		\nu_p \left( \frac{\sum a_i t^m f_i}{\sum b_i t^m f_i} \right) = \nu_p \left( \sum a_i t^m f_i \right) = m + \nu_p \left( \sum a_i f_i \right),
	\end{align*}
	which is the coefficient of $p$ in $D + (\sum a_i f_i)$. Thus
	\[
	f^\ast H = D + \left( \sum a_i f_i \right).
	\]
\end{proofbox}

The picture so far: we know $f_0, \ldots, f_n$ span a subspace $V \subseteq \mathcal{L}(D)$ .This gives a linear subspace
\[
	\mathcal{D} = \frac{V \setminus \{0\}}{\mathbb{K}^\ast} = \mathbb{P}(V) \subseteq |D| = \mathbb{P}(\mathcal{L}(D)).
\]
For a divisor $D = \sum a_i p_i$ with $a_i \neq 0$ and $p_i$ distinct, we define the \emph{support}\index{support} of $D$ to be
\[
	\mathrm{Supp}(D) = \{p_1, \ldots, p_n\}.
\]
We say $\mathcal{D}$ is \emph{base-point free}\index{base-point free} if for all $p \in C$, there exists $D' \in \mathcal{D}$ with $p \not \in \mathrm{Supp}D'$. This is the same as assumption (ii) in the above.

In this case, the theorem applies and we obtain $f : C \to \mathbb{P}^n$ with the property that $\mathcal{D} = \{f^\ast H \mid H \subseteq \mathbb{P}^n \text{ a hyperplane}\}$.

The converse is as follows. Suppose $f : C \to \mathbb{P}^n$ be a morphism. Set $D = f^\ast Z(x_0)$, assuming $f(C) \not \subseteq Z(x_0)$.

Let $f_1 \in K(C)$ be given by
\[
f_1 = \frac{x_1}{x_0} \circ f,
\]
which is a rational function on $C$ regular on $C \setminus f^{-1}(Z(x_0))$. Then $f = (f_0 : f_1 : \ldots : f_n)$ on $C \setminus f^{-1}(Z(x_0))$, and hence $f$ is induced by the linear system $\mathcal{D} \subseteq |D|$, $\mathcal{D} = \mathbb{P}(V)$ with $V$ spanned by $f_0, \ldots, f_n \in \mathcal{L}(D)$.

Also by the previous theorem, $f^\ast Z(\sum a_i x_i) = D + (\sum a_i f_i) \in \mathcal{D}$. Also $\mathcal{D}$ is base-point free, since given $p \in C$, we can find a hyperplane $H \subseteq \mathbb{P}^n$ with $f(p) \not \in H$, so $p \not \in \mathrm{Supp} f^\ast H$, while $f^\ast H \in \mathcal{D}$.

\begin{remark}
	If $f : C \injto \mathbb{P}^n$ is an embedding, then $f^\ast H$ can be viewed as `$H \cap C$ with multiplicity', and then $\mathcal{D} = \{ H \cap C \mid H \subseteq \mathbb{P}^n \text{ a hyperplane}\}$.

	We can also pull-back hypersurfaces $H \subseteq \mathbb{P}^n$ with $H = Z(\phi)$, where $\phi$ is a homogeneous polynomial of degree $d$, as follows. For $p \in f^{-1}(H)$, choose a homogeneous polynomial $\psi$ which doesn't vanish at $f(p)$, and take the coefficient of $p$ in $f^\ast H$ to be
	\[
	\nu_p \left( \frac{\phi}{\psi} \circ f \right).
	\]
\end{remark}

\begin{definition}
	Let $f : C \to \mathbb{P}^n$ be a morphism, $L \subseteq \mathbb{P}^n$ a hyperplane, $f(C) \not \subseteq L$. The \emph{degree}\index{degree} of $f$ is the degree of the divisor $f^\ast L$.
\end{definition}

This is well-defined since $f^\ast L$, $f^\ast L'$ are linearly equivalent, and linearly equivalent divisors have the same degree.

\begin{exbox}
	Let $f : C \injto \mathbb{P}^2$ identify $C$ with $Z(\phi)$, where $\phi$ has degree $d$. In this case, the degree of $f$ is $d$.

	To check this, we need to compare the coefficients in $f^\ast L$ with the multiplicity of zeroes of $\phi|_L$.
\end{exbox}

\begin{theorem}
	Let $f : C \to \mathbb{P}^n$ be a morphism, $H \subseteq \mathbb{P}^n$ a hypersurface with $f(C) \not \subseteq H$, and $H = Z(\phi)$, where $\deg \phi = e$. Then $\deg f^\ast H = (\deg f) \cdot e$.
\end{theorem}

\begin{proofbox}
	Choose some $x_i$ such that $f(C) \not \subseteq Z(x_i)$. Then $\phi/x_i^e$ is a rational function in $\mathbb{P}^n$, and
	\begin{align*}
		\left( \frac{\phi}{x_i^e} \circ f\right) &= \sum_{p \in f^{-1}(H)} \nu_p \left( \frac{\phi}{x_i^e} \circ f \right) p - \sum_{ p \in f^{-1}(L)} \nu_p \left( \frac{x_i^e}{\phi} \circ f \right) \\
							 &= f^\ast H - e f^\ast L.
	\end{align*}
	Since the degree of a principal divisor is $0$, we get $\deg f^\ast H = e \cdot \deg f^\ast L$.
\end{proofbox}

% lecture 21

\begin{remark}
	This is known as \emph{B\'ezout's theorem}. This is usually expressed as follows:

	Let $C, C' \subseteq \mathbb{P}^2$ be curves of degree $d$ and $e$ respectively. Then the number o points in $C \cap C'$, assuming $C \neq C'$, calculated with multiplicity is $d \cdot e$.
\end{remark}

For example if $C$ is non-singular, $f : C \injto \mathbb{P}^2$ an embedding, then $d = \deg f$, and $\deg f^\ast C' = d \cdot e$. So if $p = C \cap C'$, its multiplicity is the coefficient of $p$ in $f^\ast C'$. If $C$ is singular then we need a more subtle definition of multiplicity.

In general, given a divisor $D$ on a projective non-singular curve $C$, we would like to understand when $|D|$ induces an embedding $C$ in projective space. In other words, suppose $|D|$ is base-point free, i.e. for all $p \in C$, there exists $D' \in |D|$ with $p \not \in \mathrm{Supp}D'$.

Then by choosing $f_0, \ldots, f_n \in \mathcal{L}(D)$ spanning $\mathcal{L}(D)$, we obtain a morphism $f = (f_0, \ldots, f_n) : C \to \mathbb{P}^n$. When is this an embedding?

We can also use a sub-linear system $\mathcal{D} = \mathbb{P}(V) \subseteq |D| = \mathbb{P}(\mathcal{L}(D))$, and choose $f_0, \ldots, f_n \in V$ a spanning set.

\begin{theorem}
	Suppose a linear system $\mathcal{D} \subseteq |D|$ is base-point free. Then the induced morphism $f : C \to \mathbb{P}^n$ is an embedding, if and only if:
	\begin{enumerate}[\normalfont(i)]
		\item $\mathcal{D}$ separates points, i.e. for all $p, q \in C$ distinct, there exists a $D' \in \mathcal{D}$ such that $p \in \mathrm{Supp}D'$, and $q \not \in \mathrm{Supp}D'$.
		\item $\mathcal{D}$ separates tangent vectors, i.e. for all $p \in C$, there exists $D' \in \mathcal{D}$ such that the coefficient of $p$ in $D'$ is 1.
	\end{enumerate}
\end{theorem}

\begin{definition}
	We say a divisor $D$ is \emph{very ample}\index{very ample} if $D$ induces an embedding into some projective space.
\end{definition}

We can rewrite the above as follows:

\begin{theorem}
	$D$ is very ample if, for all $p, q \in C$, not necessarily distinct, we have
	\[
	\dim |D - p - q| = \dim |D| - 2.
	\]
\end{theorem}

\begin{proofbox}
	Recall that $\dim |D| = \dim \mathcal{L}(D) - 1$. For any $p \in C$, we have a map $\mathcal{L}(D) \to \mathbb{K}$, which is constructed as follows.

	Suppose the coefficient of $p$ in $D$ is $n$. Then if $f \in \mathcal{L}(D)$, then $\nu_p(t^n \cdot f) = n + \nu_p (f) \geq 0$ by definition of $\mathcal{L}(D)$, where $t$ is uniformizing.

	So $t^n f \in \mathcal{O}_{C, p}$, then we define
	\begin{align*}
		\mathrm{ev}_p : \mathcal{L}(D) &\to \mathbb{K} \\
		f &\mapsto (t^n \cdot f)(p).
	\end{align*}
	If $f \in \Ker(\mathrm{ev}_p)$, we have $\nu_p(t^n f) \geq 1$, so $\nu_p (f) > -n$. Hence the coefficient of $p$ in $D + (f)$ is at least one, so $(D - p) + (f)$ is effective, so $f \in \mathcal{L}(D - p)$.

	Conversely, if $f \in \mathcal{L}(D - p)$, then $(D - p) + (f)$ is effective, so $\nu_p(f) \geq -n + 1$, and $\nu_p(t^n \cdot f) \geq 1$, so $f \in \Ker(\mathrm{ev}_p)$. Therefore $\mathcal{L}(D - p) = \Ker(\mathrm{ev}_p)$.

	If $|D|$ is base-point free, then $\mathrm{ev}_p : \mathcal{L}(D) \to \mathbb{K}$ is surjective for all $p$, and conversely. So
	\[
	\dim |D - p| = \dim \mathcal{L}(D - p) - 1 = \dim \mathcal{L}(D) - 2 = \dim |D| - 2
	\]
	for all $p$, if and only if $|D|$ is base-point free. 

	Now $|D|$ separates point and tangent vectors if and only if $|D - p|$ is base-point free for all $p \in C$. Indeed, if $D' = |D - p|$ does not have $q$ in its support, then $D' + [$ separates $p$ and $q$ if $q \neq p$. If $p = q$, and $p \not \in \mathrm{Supp}D'$, then $D' + [$ has coefficient 1 for $p$.

	Now $\dim |D - p - q| = \dim |D - p| - 1$ if and only if $D - p|$ is base-point free, so $|D|$ is very ample and base-point free if 
	\[
	\dim |D - p - q| = \dim |D - p| - 1 = \dim |D| - 2,
	\]
	for all $p, q$.
\end{proofbox}

The moral is, if we can control $\dim \mathcal{L}(D)$, then we know a lot about embeddings.

\newpage

\section{Differentials and Riemann-Roch}%
\label{sec:diff_rr}


\begin{definition}
	Let $B$ be a ring, and $A \subseteq B$ a subring. We define
	\[
		\Omega_{B/A} = \frac{(\text{free $B$-module generated by symbols $\diff b$ for $b \in B$})}{\text{submodule $R$ of relations}},
	\]
	where $R$ is the submodule with generators:
	\begin{itemize}
		\item $\diff (b b') - b \diff b' - b' \diff b$, for all $b, b' \in B$.
		\item $\diff (b + b') - \diff b - \diff b'$, for all $b, b' \in B$.
		\item $\diff a$, for all $a \in A$.
	\end{itemize}
	
\end{definition}

% lecture 22

\begin{exbox}
	Consider $\Omega_{\mathbb{K}[x]/\mathbb{K}}$. Then for $f \in \mathbb{K}[x]$, we find
	\[
	\diff f = f'(x) \diff x.
	\]
	Thus $\Omega_{\mathbb{K}[x]/\mathbb{K}}$ is the free $\mathbb{K}[x]$-module with one generator $\diff x$.

	Similarly $\Omega_{\mathbb{K}(x)/\mathbb{K}}$ satisfies $\diff f = f'(x) \diff x$. Thus $\Omega_{\mathbb{K}(x)/\mathbb{K}}$ is the one-dimensional vector space over $\mathbb{K}(x)$ with basis $\diff x$.
\end{exbox}

\begin{proposition}
	If $L/K$ is a separable algebraic field extension. Then
	\[
	\Omega_{L/K} = 0.
	\]
\end{proposition}

Here separable means everything in $L$ is a solution to some irreducible polynomial equation $f(\alpha) = 0$ with $f(\alpha) \in K[x]$, and $f'(\alpha) \neq 0$.

\begin{proofbox}
	Given $\alpha \in L$, $f(x) \in K[x]$ with $f(\alpha) = 0$, $f'(\alpha) \neq 0$, then
	\[
	0 = f(\alpha) \implies 0 = d(f(\alpha)) = f'(\alpha) \diff \alpha,
	\]
	so $\diff \alpha = 0$ since $f'(\alpha) \neq 0$.
\end{proofbox}

\begin{lemma}
	Let $C$ be a curve, $p \in C$, and $t$ a local (uniformizing) parameter for $C$ at $p$. Then
	\[
	\Omega_{K(C)/\mathbb{K}} = K(C) \diff t.
	\]
\end{lemma}

\begin{proofbox}
	Since $t$ is a local parameter, it is not a constant functions, and thus defines a non-constant map $t : C \to \mathbb{P}^1$, inducing a finite field extension
	\[
	\mathbb{K}(\mathbb{P}^1) = \mathbb{K}(t) \to K(C).
	\]
	This extension is separable. The proof is omitted; for $\chr \mathbb{K} = 0$ it is immediate. For positive characteristic, the idea is that if the extension is not separable, then $\chr \mathbb{K} \mid e_q$ for all $q \in C$. However since $t$ is a local parameter at $p$, $e_p = 1$.

	If $\alpha \in K(C)$, there exists $f \in \mathbb{K}(t)[x]$ such that $f(\alpha) = 0$, $f'(\alpha) \neq 0$. Write
	\[
	f(x) = \sum_{i \geq 0}f_i(t) x^i,
	\]
	for $f_i(t) \in \mathbb{K}(t)$. Then,
	\begin{align*}
		0 &= \diff (f(\alpha)) = \diff \left( \sum_{i \geq 0} f_i(t) \alpha^i \right) \\
		  &= \left( \sum_{i \geq 0} f_i'(t) x^i \right) \diff t + \left( \sum_{ i \geq 1} i f_i(t) x_{i-1} \right) \diff \alpha,
	\end{align*}
	where $f'(\alpha) \neq 0$. So dividing, we can solve for $\diff \alpha$, getting
	\[
	\diff \alpha = g \diff t \in K(C) \diff t.
	\]
\end{proofbox}

\begin{definition}
	Let $C$ be a projective non-singular curve, and $\omega \in \Omega_{K(C)/\mathbb{K}}$, $p \in C$. We define $\nu_p(\omega)$ as follows: let $t \in \mathcal{O}_{C, p}$ be a local parameter, and write $\omega = f \diff t$, for $f \in K(C)$. Define
	\[
	\nu_p(\omega) = \nu_p(f).
	\]
	We also define
	\[
	\mathrm{div} (\omega) = \sum_{p \in C} \nu_p(\omega) \cdot p \in \mathrm{Div}C.
	\]
\end{definition}

We say that $\omega$ is \emph{regular}\index{regular} at $p$ if $\nu_p(\omega) \geq 0$.

To show this is a sensible definition we need a few lemmas.

\begin{lemma}
	\begin{enumerate}[\normalfont(i)]
		\item[]
		\item $f \in \mathcal{O}_{C, p} \implies \nu_p(\diff f) \geq 0$.
		\item If $t'$ is another local parameter at $p$, then $\nu_p(\diff t') = 0$ and $\nu_p(f \diff t') = \nu_p(f) + \nu_p(\diff t')$ is independent of $t$.
		\item If $f \in K(C)$ and $\nu_p(f) \neq 0$ in $\mathbb{K}$, then $\nu_p(\diff f) = \nu_p(f) - 1$.
	\end{enumerate}
\end{lemma}

\begin{proofbox}
	(i) We let $p \in C \subseteq \mathbb{P}^n$, $p \in C \cap U_i$, where $U_i = \mathbb{P}^n \setminus Z(x_i)$. Work on $U_1 \cap C, $ where rational functions are just ratios of polynomials. If $f = g/h$, $h(p) \neq 0$, we have
	\[
	\diff f = \frac{h \diff g - g \diff h}{h^2} = \sum \gamma_i \diff x_i,
	\]
	with $\gamma_i \in \mathcal{O}_{C, p}$. So,
	\begin{align*}
		\nu_p(\diff f) & \geq \min\{ \nu_p(\gamma_i \diff x_i) \mid 1 \leq i \leq n\} \\
			       &\geq \min \{ \nu_p(\diff x_i) \mid 1 \leq i \leq n \}.
	\end{align*}
	Thus $\nu_p(\diff f)$ is bounded below, independently of $f$.

	Choose $f \in \mathcal{O}_{C, p}$ such that $\nu_p(\diff f)$ is minimal, $t$ a local parameter at $p \in C$. Then $\nu_p(f - f(p)) \geq 1$, so we can write $f - f(p) = t f_1$, for some $f_1 \in \mathcal{O}_{C, p}$, so
	\[
	\diff f=  \diff(f - f(p)) = \diff (t f_1) = f_1 \diff t + t \diff f_1.
	\]
	If $\nu_p(\diff f) < 0$, then note $\nu_p(f_i \diff t) \geq 0$, and hence this implies
	\begin{align*}
		\nu_p(\diff f) &= \nu_p(\diff f - f_1 \diff t) = \nu_p(t \diff f_1) \\
			       &= \nu_p(t) + \nu_p(\diff f_1) = 1 + \nu_p(\diff f_1).
	\end{align*}
	So $\nu_p( \diff f_1) < \nu_p (\diff f)$, which contradicts the minimality of $\nu_p(\diff f)$. Thus $\nu_p(\diff f) \geq 0$.

	(ii) Any two local parameters are related by a unit, so we may write $t' = u \cdot t$, for $u$ a unit, $u \in \mathcal{O}_{C, p}^\ast$, the group of units in $\mathcal{O}_{C, p}$. Then,
	\[
	\diff t' = u \diff t + t \diff u,
	\]
	and note $\diff u = g \cdot \diff t$ for some $g$ with $\nu_p(g) \geq 0$, by the above. So
	\[
	\diff t' = (u + t g) \diff t,
	\]
	where $\nu_p(u + tg) = 0$, so $\nu_p(\diff t') = 0$ by definition.

	If $f \diff t = h \diff t' = h(u + tg) \diff t$, then note
	\[
	\nu_p(h(u + tg)) = \nu_p(h) + \nu_p(u + tg) = \nu_p(h).
	\]
	So this is independent of choice of local parameter.

	(iii) Suppose $f = t^n u$, where $n = \nu_p(f)$, $u \in \mathcal{O}_{C, p}^\ast$. Then
	\[
	\diff f = n t^{n-1} u \diff t + t^n \diff u.
	\]
	If $\chr \mathbb{K} \nmid n$, then
	\begin{align*}
		\nu_p(f) &\geq \min \{ \nu_p(n t^{n-1} u \diff t), t^n \diff u \} = \min\{n-1, n\} = n-1,
	\end{align*}
	and equality holds. Hence $\nu_p(\diff f) = \nu_p(f) - 1$.
\end{proofbox}

\begin{proposition}
	If $\omega \in \Omega_{K(C)/\mathbb{K}}$, then $\nu_p(\omega) = 0$ for all but a finite number of $p$.
\end{proposition}

The proof is omitted. Thus, $\mathrm{div}(\omega) \subseteq \mathrm{Div}(C)$.

% lecture 23

\begin{proposition}
	Let $\omega, \omega' \in \Omega_{K(C)/\mathbb{K}}$. Then $\mathrm{div}(\omega)$ and $\mathrm{div}(\omega')$ are linearly equivalent.
\end{proposition}

\begin{proofbox}
	For $t$ a local parameter at some point $p \in C$, we have $\omega = f \diff t$, $\omega\ = f' \diff t$, so
	\[
	\omega' = \frac{f'}{f} \cdot \omega,
	\]
	and so we get
	\[
	\mathrm{div}(\omega') = \mathrm{div}(\omega) + \left( \frac{f'}{f} \right).
	\]
\end{proofbox}

\begin{definition}
	The \emph{canonical class}\index{canonical class} of a proper, non-singular curve $C$ is the linear equivalence class of $\mathrm{div}(\omega)$ in $\mathrm{Cl} C$, for any $0 \neq \omega \in \Omega_{K(C)/\mathbb{K}}$.

	We write the canonical class as $K_C$.
\end{definition}

\begin{definition}
	The \emph{genus}\index{genus} of $C$ is $\dim_{\mathbb{K}} \mathcal{L}(K_C)$.
\end{definition}

If $\mathbb{K} = \mathbb{C}$, any one uses the Euclidean topology rather than the Zariski topology, then this is the usual notion of genus.

\begin{exbox}
	Consider $C = \mathbb{P}^1$. Then $K(C) = \mathbb{K}(t)$, where $t = x_0/x_1$.

	Note when $x_1 = 1$, $t - p_0$ is a local parameter for $C$ at $p_0 = (p_0 : 1) \in \mathbb{P}^1$. Thus $\diff t  = \diff (t - p_0)$, and $\nu_{p_0}(\diff (t - p_0)) = 0$. Thus $\nu_{p_0}(\diff t) = 0$, for all $p_0 \in \mathbb{P}^1 \setminus Z(x_1)$.

	At $t = \infty$, we look at $\mathbb{A}^1 = \mathbb{P}^1 \setminus Z(x_0)$, so $s = x_1/x_0$ is a local parameter at $q = (1 : 0)$. Note that $t = s^{-1}$, so
	\[
	\diff t = \diff(1/s) = \frac{\diff s}{s^2},
	\]
	so $\nu_q( \diff t) = -2$. Thus $K_C \sim -2 \cdot q$. Thus $\mathcal{L}(K_C) = \mathcal{L}(-2 q) = 0$, and so $g(C) = \dim \mathcal{L}(K_C) = 0$.
\end{exbox}

\begin{exbox}
	Consider the plane cubic, which in $\mathbb{A}^2$ is
	\[
	y^2 = (x - \lambda_1)(x-\lambda_2)(x-\lambda_3) = f(x),
	\]
	and in $\mathbb{P}^2$ is
	\[
	y^2 z = (x - \lambda_1z)(x - \lambda_2z)(x - \lambda_3z),
	\]
	where $\lambda_1, \lambda_2, \lambda_3 \in \mathbb{K}$ are distinct. Take a differential
	\[
	\omega = \frac{\diff x}{y}.
	\]
	Differentiating the equation for a plane cubic we get that
	\[
	2y \diff y = f'(x) \diff x,
	\]
	so
	\[
	\frac{2 \diff x}{f'(x)} = \frac{\diff x}{y}.
	\]
	In fact $\mathrm{div}(\omega) = 0$. The hardest part is checking the point at infinity, $q = (0 : 1 : 0)$. Thus $K_C \sim 0$, and $\mathcal{L}(K_C) = \mathcal{L}(0) = \mathbb{K}$, so $g(C) = 1$.
\end{exbox}

\subsection{Riemann-Roch}%
\label{sub:rr}

Write $\ell(D)$ for $\dim_{\mathbb{K}} \mathcal{L}(D)$ for $D \in \mathrm{Div}C$.

\begin{theorem}[Riemann-Roch Theorem]
	\[
	\ell(D) - \ell(K_C - D) = \deg D + 1 - g,
	\]
	where $g$ is the genus of $C$.
\end{theorem}

As a corollary, we get the following:
\begin{itemize}
	\item If $D = 0$, then $\ell(D) = 1$, so
	\[
	1 - \ell(K_C) = 0 + 1 - g,
	\]
	or $\ell(K_C) = g$, the definition of $g$.
	\item If $D = K_C$, then
		\[
		\ell(K_C) - \ell(0) = \deg K_C + 1 - g,
		\]
		so $\deg K_C = 2g - 2$.
	\item If $\deg D > 2g - 2$, then $\deg (K_C - D) = 2g - 2 - \deg D < 0$, thus $\ell(K_C - D) = 0$, and
		\[
		\ell(D) = \deg D + 1 - g.
		\]
	\item If $\deg D > 2g$, then for all $p, q \in C$,
		\[
		\ell(D - p - q) = \ell(D) - 2,
		\]
		by the above. Hence $|D|$ induces an embedding of $C$ in some $\mathbb{P}^n$.
\end{itemize}

\begin{remark}
	For $0 \leq \deg D \leq 2g - 2$, the behaviour of $\ell(D)$ can be complicated and unpredictable.
\end{remark}

\begin{exbox}
	If $C$ has genus 0, then every positive degree divisor induces an embedding. For example if $p \in C$, then $|p|$ is very ample, $\ell(p) = 2$, so we get an embedding of $C$ in $\mathbb{P}^1$. Thus $C \simeq \mathbb{P}^1$.
\end{exbox}

\begin{exbox}
	Take $g = 1$. If $\deg D = 3$, then $D$ is very ample, and $\ell(D) = 3 + 1 - 1 = 3$. So $|D|$ induces an embedding of $C$ in $\mathbb{P}^2$.

	Thus in particular $C$ is isomorphic to a curve of degree 3 in $\mathbb{P}^2$. We can show that $C \simeq Z(f)$ for some homogeneous polynomial of degree $3$.

	More specifically, fix $p_0 \in C$, and embedding $|3p_0|$. Let $D \in \mathrm{Div} C$ be degree 0. Then,
	\[
	\ell(D + p_0) - \ell(K_C - D - p_0) = \deg (D + p_0) + 1 - g,
	\]
	which simplifies to $\ell(D + p_0) = 1$. So there exists an effective divisor linearly equivalent to $D + p_0$, which necessarily must be $D + p_0 \sim p$. Thus $p - p_0 \sim D$.

	Moreover $p$ is unique, as if $p - p_0 \sim p' - p_0$, then $p \sim p'$, so if $p \neq p'$, $\dim |p| = \geq 1$, so $\ell(p) \geq 2$. But $\ell(p) = 1$ by Riemann-Roch.

	Hence every divisor class in $C$ of degree 0 can be represented uniquely by $p - p_0$, for some $p \in C$. So
	\begin{align*}
		C &\to \Ker(\deg : \mathrm{Cl} C \to \mathbb{Z}) \\
		p &\mapsto p - p_0
	\end{align*}
	is a bijection. This gives a group structure on $C$.
% lecture 24
	Hence we can say that $p + q = r$ for $p, q, r \in C$ if
	\[
		(p - p_0) + (q - p_0) \sim (r - p_0).
	\]
\end{exbox}

Let's talk about this group structure a bit more, with a geometric description. Consider $p, q \in C \overset{i}{\injto} \mathbb{P}^2$. Let $L$ be the line joining $p$ and $q$, tangent to $C$ at $p$ if $p = q$.

Then we can take the intersection $L \cap C$, which is formally $i^\ast L = p + q + s$. Now $p + q + s \sim 3 p_0$, or
\[
	(p - p_0) + (q - p_0) + (s - p_0) \sim 0.
\]
Next let $L'$ be the line joining $S$ with $p_0$, which intersects at $r$. Then $s + p_0 + r \sim 3 p_0$. So
\[
	(s - p_0) \sim -(r - p_0).
\]
Therefore $(p - p_0) + (q - p_0) \sim (r - p_0)$, so $p + q = r$.

In terms of geometric description, we need a diagram. First take $y^2 = (x - \lambda_1)(x-\lambda_2)(x-\lambda_3)$, taking $p_0 = (0 : 1 : 0)$.

The sum $p + q$ is first by taking the line through $p, q$ to intersect again at $s$, then taking the intersection of the vertical line through $s$ with the curve again to get $r = p + q$.

\begin{exbox}
	Let $C$ have genus $2$. Then
	\[
	\deg K_C = 2g - 2 = 2,
	\]
	so $\ell(K_C) = 2$. We claim that $|K_C|$ is base-point free, so induces a morphism $f : C \to \mathbb{P}^1$.
\end{exbox}

\begin{lemma}
	Let $C$ be a projective non-singular curve. If there exists $p, q \in C$, $p \neq q$, $p \sim q$, then $C \cong \mathbb{P}^1$.
\end{lemma}

\begin{proofbox}
	Consider the linear system $|p|$. Since $q \in |p|$, $\dim |p| \geq 1$, so $\ell(p) \geq 2$. But we have an upper bound $\dim \mathcal{L}(D) \leq \deg D + 1$. So $\ell(p) = 2$.

	Hence if $q, r \in C$, then $\dim \mathcal{L}(p - q - r) = 0$, since its degree is $-1$. Thus $|p|$ induces an embedding of $C$ into $\mathbb{P}^1$, so $C \cong \mathbb{P}^1$.
\end{proofbox}

We now return to our proof of the claim in the above example.

\begin{proofbox}
	If $|K_C|$ is not base-point free, then there exists $p \in C$ such that $\ell(K_C - p) = \ell(K_C) = 2$.

	However, since $\deg (K_C - p) = 1$, this says there exists $q, r \in |K_C - p|$,  $q \neq r$ with $q \sim r$. Hence $C \cong \mathbb{P}^1$.
\end{proofbox}

Thus if $g = 2$, we obtain a degree 2 morphism $f : C \to \mathbb{P}^1$, induced by $|K_C|$.

\begin{definition}
	A projective non-singular curve $C$ is \emph{hyperelliptic}\index{hyperelliptic} if there exists a degree 2 morphism $f : C \to \mathbb{P}^1$.

	Thus all genus 2 curves are hyperelliptic.
\end{definition}

\begin{theorem}
	Let $C$ be a projective non-singular curve of genus $g \geq 3$. Then either:
	\begin{itemize}
		\item $C$ is hyperelliptic, or
		\item $|K_C|$ induces an embedding $C \injto \mathbb{P}^{g-1}$.
	\end{itemize}
\end{theorem}

\begin{proofbox}
	$|K_C|$ induces an embedding in $\mathbb{P}^{\ell(K_C) - 1} = \mathbb{P}^{g - 1}$ if and only if, for all $p, q \in C$,
	\[
	\ell(K_C - p - q) = \ell(K_C) - 2 = g - 2.
	\]
	In any event,
	\[
	\ell(p + q) - \ell(K_C - p - q) = \deg(p + q) + 1 - g = 3 - g.
	\]
	Thus $|K_C|$ induces an embedding if and only if $\ell(p + q) = 1$ for all $p, q \in C$.

	Now suppose that $K_C$ does not induce an embedding. Then there exists $p, q \in C$ such that $\ell(p + q) > 1$. If $\ell(p + q) \geq 3$, then for any $r \in C$, $\ell(p + q - r) \geq 2$, so there exists $p_1, p_2 \in |p + q - r|$ distinct. But then $C \cong \mathbb{P}^1$ by our lemma.

	Thus $\ell(p + q) = 2$. Note similarly $\ell(p + q - r) = 1$, for all $r \in C$. Thus $|p + q|$ is base-point free and induces a degree 2 morphism $f : C \to \mathbb{P}^1$. So $C$ is hyperelliptic.
\end{proofbox}

\begin{theorem}[Riemann-Hurwitz Formula]
	Let $f : X \to Y$ be a non-constant morphism between projective non-singular curves, with $\chr \mathbb{K} = 0$ (or $K(Y) \subseteq K(X)$ is a separable field extension). Then:
	\[
	2 - 2 g(X) = (\deg f)(2 - 2 g(Y)) - \sum_{p \in X} (e_p - 1).
	\]
\end{theorem}

\begin{exbox}
	Take $X = C$ hyperelliptic, $Y = \mathbb{P}^1$, and $f : C \to \mathbb{P}^1$, with degree 2. Then
	\[
	2 - 2 g(C) = 2 (2 - 2 \cdot 0) - \sum_{p \in C}(e_p - 1).
	\]
	THus,
	\[
	\sum_{p \in C} (e_p - 1) = 2 g(C) + 2.
	\]
	Therefore, $2 g(C) + 2$ is the number of points $p \in C$ with $e_p > 1$, known as \emph{branch points}.
\end{exbox}


\newpage

\printindex

\end{document}
