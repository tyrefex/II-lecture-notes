\documentclass[12pt]{article}

\usepackage{ishn}

\makeindex[intoc]

\begin{document}

\hypersetup{pageanchor=false}
\begin{titlepage}
	\begin{center}
		\vspace*{1em}
		\Huge
		\textbf{II Algebraic Topology}

		\vspace{1em}
		\large
		Ishan Nath, Lent 2024

		\vspace{1.5em}

		\Large

		Based on Lectures by Prof. Oscar Randal-Williams

		\vspace{1em}

		\large
		\today
	\end{center}
	
\end{titlepage}
\hypersetup{pageanchor=true}

\tableofcontents

\newpage

\setcounter{section}{-1}

\section{Introduction}
\label{sec:intro}

A basic problem in topology is the following:

(Extension problem). If $X$ is a space and $A \subset X$ a subspace, and $f : A \to Y$ is a continuous function, is there a continuous $F : X \to Y$ such that $F|_A = f$?

In fact, the answer is no.

\begin{theorem}
	There is no continuous $F : D^n \to S^{n-1}$ such that
	\[
		S^{n-1} \overset{\text{incl}}{\injto} D^n \overset{F}{\to} S^{n-1}
	\]
	is the identity.
\end{theorem}

What does this have to do with algebra? In fact, we will reduce this problem to showing the corresponding problem in algebra.

\begin{theorem}
	There is no group homomorphism $F : \{0\} \to \mathbb{Z}$ such that
	\[
		\mathbb{Z} \to \{0\} \overset{F}{\to} \mathbb{Z}
	\]
	is the identity.
\end{theorem}

Of course, this is now trivial. One more thing we can show is:
\begin{theorem}
	$\mathbb{R}^n \cong \mathbb{R}^m \iff n = m$.
\end{theorem}
Once again, this will be proved by relating it to a problem in algebra.

We can do it the other way round as well.

\begin{theorem}[Fundamental Theorem of Algebra]
	Any non-constant polynomial in $\mathbb{C}$ has a root.
\end{theorem}

This can be proven by looking at $\mathbb{C}$ as a topological space.

\subsection{Recollections}
\label{sub:rec}

We say that a continuous function is called a \emph{map}\index{map}.

Here are a couple of useful lemmas.

\begin{lemma}[Gluing Lemma]
	Let $f : X \to Y$ be a function between topological spaces, and let $C, K \subset X$ be closed sets such that $X = C \cup K$.

	Then $f$ is continuous $\iff f|_C, f|_K$ is continuous.
\end{lemma}

\begin{lemma}[Lebesgue Number Lemma]
	Let $(X, d)$ be a metric space, and assume it is compact. For any open cover $\mathcal{U} = \{U_\alpha\}$, there is $\delta > 0$ such that each $B_\delta(x)$ is contained in some $U_\alpha$.
\end{lemma}

Here is a way of making reasonable spaces out of reasonable spaces.

\begin{definition}
	For a space $X$ and a map $f : S^{n-1} \to X$, the space obtained by attaching an $n$-cell to $X$ is
	\[
	X \cup_{f} D^n = (X \sqcup D^n) / (z \in S^{n-1} \subset D^{n-1} \sim f(z) \in X).
	\]
\end{definition}

\begin{definition}
	A (finite) \emph{cell complex}\index{cell complex} is a space $X$ obtained by the following:
	\begin{enumerate}[(i)]
		\item Start with a finite set $X^0$ with the discrete topology.
		\item If $X^{n-1}$ has been defined, form $X^n$ by attaching a finite collection of $n$-cells along some maps $\{f_n : S^{n-1} \to X^{n-1}\}$. This $X^n$ is called an \emph{$n$-skeleton}\index{$n$-skeleton}.
		\item Stop with $X = X^k$, where $k$ is called the \emph{dimension} of $X$.
	\end{enumerate}
\end{definition}

\newpage

\section{Homotopy}
\label{sec:homotopy}

In the following let $I = [0, 1]$.

\begin{definition}
	Let $f, g : X \to Y$ be maps. A \emph{homotopy}\index{homotopy} from $f$ to $g$ is a map $H : X \times I \to Y$ such that
	\[
	H(x, 0) = f(x), \qquad H(x, 1) = g(x).
	\]
	If such an $H$ exists, we say $f$ is \emph{homotopic} to $g$, and we write $f \simeq g$.

	If $A \subset X$ is a subspace, we say $H$ is a \emph{homotopy relative to} $A$ if $H(a, t) = H(a, 0)$ for all $t \in I$ and $a \in A$. We write $f \simeq g$ rel $A$.
\end{definition}

\begin{proposition}
	Being homotopic relative to $A$ is an equivalence relation on the set of maps from $X$ to $Y$.
\end{proposition}

\begin{proofbox}
	We show it is an equivalence relation.
	\begin{enumerate}[(i)]
		\item $f \simeq f$ via $H(x, t) = f(x)$.
		\item If $f \simeq g$ via $H$, let $H'(x, t) = H(x, 1-t)$. This is a homotopy from $g$ to $f$.
		\item If $f \simeq g$ via $H$, and $g \simeq h$ via $H'$, then let
			\[
			H''(x, t) =
			\begin{cases}
				H(x, 2t) & 0 \leq t \leq 1/2,\\
				H'(x, 2t - 1) & 1/2 \leq t \leq 1.
			\end{cases}
			\]
			This is continuous on $X \times [0, 1/2]$ and $X \times [1/2, 1]$, so by the gluing lemma it is continuous on $X \times I$.
	\end{enumerate}
\end{proofbox}

\begin{definition}
	A map $f : X \to Y$ is a \emph{homotopy equivalence}\index{homotopy equivalence} if there is a $g : Y \to X$ such that
	\[
	f \circ g \simeq \id_Y, \qquad g \circ f \simeq \id_X.
	\]
	Say $X$ is \emph{homotopy equivalent to} $Y$, written $X \simeq Y$, if a homotopy equivalence $f : X \to Y$ exists.
\end{definition}

\begin{exbox}
	Let $X = S^1$, and $Y = \mathbb{R}^2 \setminus \{0\}$.

	Let $i : X \to Y$ be the inclusion, and let $r : Y \to X$ by $x \mapsto x/|x|$ be the normalizing function. This is continuous.

	Now $r \circ i = \id_X$. On the other hand, $i \circ r : Y \to Y$ is not the identity. But
	\[
	H(x, t) = \frac{x}{t + |x|(1-t)}
	\]
	is a homotopy from $\id_Y$ to $i \circ r$.

	So $\mathbb{R}^2 \setminus \{0\} \simeq S^1$, but they are not homeomorphic.
\end{exbox}

\begin{definition}
	$X$ is called \emph{contractible}\index{contractible} if $X \simeq \{\ast\}$, the one-point space.
\end{definition}

\begin{lemma}
	Let $f_0, f_1 : X \to Y$ and $g_0, g_1 : Y \to Z$ be homotopic maps.

	Then $g_0 \circ f_0 \simeq g_1 \circ f_1$.
\end{lemma}

\begin{proofbox}
	Let's show that $g_0 \circ f_0 \simeq g_0 \circ f_1 \simeq g_1 \circ f_1$.

	For the first equivalence, if $H$ is a homotopy from $f_0$ to $f_1$, then $g_0 \circ H : X \times I \to Z$ is a homotopy of $g_0 \circ f_0$ to $g_0 \circ f_1$.

	Now if $G$ is a homotopy from $g_0$ to $g_1$, then $G \circ (f_1 \times \id_I) : X \times I \to Z$ is a homotopy of $g_0 \circ f_1$ to $g_1 \circ f_1$.
\end{proofbox}

%lecture 2

\begin{proposition}
	\begin{enumerate}[\normalfont(i)]
		\item[]
		\item $X \simeq X$.
		\item If $X \simeq Y$, then $Y \simeq X$.
		\item If $X \simeq Y$ and $Y \simeq Z$, then $X \simeq Z$.
	\end{enumerate}	
\end{proposition}

\begin{proofbox}
	
	\begin{enumerate}[(i)]
		\item Take $f = g = \id_X$, and the stationary homotopies.
		\item Given $f : X \to Y$ and $g : Y \to X$ with $f \circ g \simeq_H \id_Y$ and $g \circ f \simeq_G \id_X$, then this is the same data as $Y \simeq X$.
		\item Suppose we have maps
			\[
			\begin{tikzcd}
				X \arrow[r, bend left, "f"] & Y \arrow[r, bend left, "f'"] \arrow[l, bend left, "g"] & Z\arrow[l, bend left, "g'"]
			\end{tikzcd}
			\]
			with $f \circ g \simeq \id_Y$, $g \circ f \simeq \id_X$, $f' \circ g' \simeq \id_Z$, $g' \circ f' \simeq \id_Y$.

			Consider $f' \circ f : X \to Z$ and $g \circ g' : Z \to X$. We claim these generate the required functions. Indeed,
			\[
				(g \circ g') \circ (f' \circ f) = g \circ (g' \circ f') \circ f \simeq g \circ \id_Y \circ f = g \circ f \simeq \id_X,
			\]
			and the other composition is similar.
	\end{enumerate}
\end{proofbox}

\begin{definition}
	If $i : A \to X$ is the inclusion of a subspace, then
	\begin{enumerate}[(i)]
		\item a \emph{retraction}\index{retraction} is a $r : X \to A$ such that $r \circ i = \id_A$.
		\item a \emph{deformation retraction}\index{deformation retraction} is a retraction such that also $i \circ r \simeq \id_X$.
	\end{enumerate}
\end{definition}

\subsection{Paths}
\label{sub:path}

\begin{definition}
	For a space $X$ and points $x_0, x_1 \in X$, a \emph{path}\index{path} from $x_0$ to $x_1$ is a map $\upgamma : I = [0, 1] \to X$ such that $\upgamma(0) = x_0, \upgamma(1) = x_1$. If $x_0 = x_1$, we call $\upgamma$ a \emph{loop}\index{loop} based at $x_0$.

	If $\upgamma$ is a path from $x_0$ to $x_1$, and $\upgamma'$ is a path from $x_1$ to $x_2$, then we can form the \emph{concatenation}\index{concatenation} $\upgamma \cdot \upgamma' : I \to X$ via
	\[
		(\upgamma \cdot \upgamma')(t) =
		\begin{cases}
			\upgamma(2t) & 0 \leq t \leq 1/2,\\
			\upgamma'(2t) & 1/2 \leq t \leq 1.
		\end{cases}
	\]
	This is continuous via the gluing lemma, and is a path from $x_0$ to $x_2$.

	Also define the \emph{inverse}\index{inverse path} $\upgamma^{-1} : I \to X$ via
	\[
	\upg^{-1}(t) = \upg(1 - t).
	\]
	Define the \emph{constant path}\index{constant path} $c_{x_0} : I \to X$ via $c_{x_0}(t) = x_0$, and define an equivalence relation on $X$ via
	\[
		x_0 \sim x_1 \iff \text{there exists a path } \upg \text{ from } x_0 \text{ to } x_1.
	\]
\end{definition}

\begin{definition}
	The equivalence class of $\sim$ are called \emph{path components}\index{path components} of $X$. We say $X$ is \emph{path-connected}\index{path-connected} if there is only one equivalence class. Let
	\[
	\pi_0(X) = X/\sim.
	\]
\end{definition}

\begin{proposition}
	For a map $f : X \to Y$, there is a well-defined function
	\[
	\pi_0(f) : \pi_0(X) \to \pi_0(Y)
	\]
	given by $\pi_0(f)([x]) = [f(x)]$. Furthermore,
	\begin{enumerate}[\normalfont(i)]
		\item If $f \simeq g$, then $\pi_0(f) = \pi_0(g)$.
		\item For $A \overset{h}{\to} b \overset{k}{\to} C$, then $\pi_0(k \circ h) = \pi_0(k) \circ \pi_0(h)$.
		\item $\pi_0(\id_X) = \id_{\pi_0(X)}$.
	\end{enumerate}
\end{proposition}

\begin{proofbox}
	To see this is well defined, if $[x] = [x']$, then there is a path $\upg$ from $x$ to $x'$. Then $(f \circ \upg) : I \to Y$ is a path from $f(x)$ to $f(x')$, so $[f(x)] = [f(x')]$.

	Now properties (ii) and (iii) are immediate. For (i), let $H : X \times I \to Y$ be a homotopy from $f$ to $g$. Now
	\[
		H|_{\{x\} \times I} : \{x\} \times I \to Y
	\]
	is a path from $f(x)$ to $g(x)$, so $[f(x)] = [g(x)]$, hence $\pi_0(f)([x]) = \pi_0(g)([x])$.
\end{proofbox}

\begin{corollary}
	If $f : X \to Y$ is a homotopy equivalence, then $\pi_0(f)$ is a bijection.
\end{corollary}

\begin{proofbox}
	If $g : Y \to X$ is a homotopy inverse, then
	\[
	\pi_0(f) \circ \pi_0(g) = \pi_0(f \circ g) = \pi_0(\id_Y) = \id_{\pi_0(Y)},
	\]
	and similarly $\pi_0(g) \circ \pi_0(f) = \id_{\pi_0(X)}$, so $\pi_0(f)$ is a bijection.
\end{proofbox}

\begin{exbox}
	The space $\{-1, +1\}$ with the discrete topology is not contractible. This is because any path in this space is constant, so
	\[
		\pi_0(\{-1, +1\}) = \{-1, +1\},
	\]
	whereas $\pi_0(\{\ast\}) = \{\ast\}$ has cardinality 1.
\end{exbox}

\begin{exbox}
	The space $[-1, 1]$ does not retract onto $\{-1, +1\}$. Suppose it does, then
	\[
	\begin{tikzcd}
		\id : \{-1, +1\} \arrow[r, hookrightarrow, "\mathrm{inj}"] & \![-1, 1] \arrow[r, "r"] \arrow[d, "\pi_0", Rightarrow] & \{-1, +1\} \\
		\id : \{-1, +1\} \arrow[r, "\pi_0(\mathrm{inc})"] & \pi_0[-1, 1] \arrow[r, "\pi_0(r)"] & \{-1, +1\}.
	\end{tikzcd}
	\]
\end{exbox}

\begin{definition}
	Two paths $\upg, \upg' : I \to X$ both from $x_0$ to $x_1$ are called \emph{homotopic as paths}\index{homotopic as paths} if they are homotopic relative to $\{0, 1\} \subset I$ as in the previous lecture. We say $\upg \simeq \upg'$, relative to $x_0$ and $x_1$.
\end{definition}

\begin{lemma}
	If $\upg_0 \simeq \upg_1$ are paths from $x_0$ to $x_1$, and $\upg_0' \simeq \upg_1'$ as paths from $x_1$ to $x_2$, then $\upg_0 \cdot \upg_0' \simeq \upg_1 \cdot \upg_1'$ as paths from $x_0$ to $x_2$.
\end{lemma}

\begin{proofbox}
	Let $H$ be the homotopy from $\upg_0$ to $\upg_1$ rel. $x_0$ and $x_1$, and $H'$ be the homotopy from $\upg_0'$ to $\upg_1$ rel. $x_1$ and $x_2$. Then define
	\[
	H''(s, t) =
	\begin{cases}
		H(2s, t) & 0 \leq s \leq 1/2, \\
		H'(2s-1, t) & 1/2 \leq s \leq 1.
	\end{cases}
	\]
	This is a homotopy from $\upg_0 \cdot \upg_0'$ to $\upg_1 \cdot \upg_1'$ rel. $x_0$ and $x_2$.
\end{proofbox}

%lecture 3

\begin{proposition}
	Let $\upg_0$ be a path from $x_0$ to $x_1$, $\upg_1$ be a path from $x_1$ to $x_2$, and $\upg_2$ be a path from $x_2$ to $x_3$. Then,
	\begin{enumerate}[(i)]
		\item $(\upg_0 \cdot \upg_1) \cdot \upg_2 \simeq \upg_0 \cdot (\upg_1 \cdot \upg_2)$ rel. $x_0$ and $x_3$.
		\item $\upg_0 \cdot c_{x_1} \simeq \upg_0 \simeq c_{x_0} \cdot \upg_0$ rel. $x_0$ and $x_1$.
		\item $\upg_0 \cdot \upg_0^{-1} \simeq c_{x_0}$, rel. $x_0$ and $x_0$, and $\upg_0^{-1} \cdot \upg_0 \simeq c_{x_1}$ rel. $x_1$ and $x_1$.
	\end{enumerate}
\end{proposition}

\begin{proofbox}
	Look at a cute little diagram for all three cases. This motivates the following homotopy for the first case:
	\[
	H(s, t) =
	\begin{cases}
		\gamma_0(\frac{4s}{t+1}) & 0 \leq s \leq \frac{t + 1}{4},\\
		\gamma_1(4s-1-t) & \frac{t+1}{4} \leq s \leq \frac{t+2}{4},\\
		\gamma_2(1 - \frac{4(1-s)}{2 - t}) & \frac{t+2}{4} \leq s \leq 1.
	\end{cases}
	\]
	For the second case we can take
	\[
	H(s, t)=
	\begin{cases}
		\gamma_0(\frac{2s}{t + 1}) & 0 \leq s \leq \frac{t+1}{2}, \\
		x_1 & \frac{t+1}{2} \leq s \leq 1.
	\end{cases}
	\]
	For the third case, we have
	\[
	H(s, t) =
	\begin{cases}
		\gamma_0(2s) & 0 \leq s \leq \frac{1 - t}{2}, \\
		\gamma_0(1-t) & \frac{1 - t}{2} \leq s \leq \frac{1 + t}{2}, \\
		\gamma_0(2 - 2s) & \frac{1 + t}{2} \leq s \leq 1.
	\end{cases}
	\]
\end{proofbox}

\subsection{The Fundamental Group}
\label{sub:fun_g}

\begin{theorem}
	Let $X$ be a space, and $x_0 \in X$. Let $\pi_1(X, x_0)$ be the set of homotopy classes of loops in $X$, starting and ending at $x_0$. Then the rule
	\[
		[\upg] \cdot [\upg'] = [\upg \cdot \upg']
	\]
	makes $(\pi_1(X, x_0), \cdot, [c_{x_0}])$ into a group.
\end{theorem}

\begin{proofbox}
	The lemma from last lecture shows that this is well defined, and the previous proposition shows that this group exists.
\end{proofbox}

\begin{definition}
	A \emph{basic space}\index{basic space} is a space $X$ with a chosen point $x_0 \in X$, called the \emph{basic point}\index{basic point}. A \emph{map} $f : (X, x_0) \to (Y, y_0)$ of basic spaces is a map $X \overset{f}{\to} Y$ such that $f(x_0) = y_0$. A \emph{based homotopy}\index{based homotopy} is a homotopy relative to $\{x_0\} \in X$.
\end{definition}

\begin{proposition}
	To a based map $f : (X, x_0) \to (Y, y_0)$ there is associated a function $\pi_1(f) : \pi_1(X, x_0) \to \pi_1(Y, y_0)$ given by
	\[
		\pi_1(f) ([\upg]) = [f \circ \upg].
	\]
	It satisfies:
	\begin{enumerate}[\normalfont(i)]
		\item It is a group homomorphism.
		\item If $f$ is a based homotopic to $f'$, then $\pi_1(f) = \pi_1(f')$.
		\item If $(A, a) \overset{h}{\to} (B, b) \overset{k}{\to} (C, c)$ are based maps, then $\pi_1(k \circ h) = \pi_1(k) \circ \pi_1(h)$.
		\item $\pi_1(\id_X) = \id_{\pi_1(X, x_0)}$.
	\end{enumerate}
\end{proposition}

\begin{proofbox}
	The proposed function is well-defined, as if $\upg \simeq \upg'$ as paths, then $f \circ \upg \simeq f \circ \upg'$ as paths.

	(i) We verify the group axioms. Note that $f \circ c_{x_0} = c_{y_0}$, so $\pi_1(f)$ preserves the identity element, and
	\[
	f \circ (\upg \cdot \upg') = (f \circ \upg) \cdot (f \circ \upg'),
	\]
	so $\pi_1(f)$ is a homomorphism.

	Then (ii), (iii) and (iv) are elementary.
\end{proofbox}

From now on, we let $\pi_1(f) = f_\ast$.

\begin{proposition}
	Let $u$ be a path from $x_0$ to $x_1$ in $X$. It induces a homomorphism $u_{\#} : \pi_1(X, x_0) \to \pi_1(X, x_1)$ satisfying
	\[
		[\upg] \mapsto [u^{-1} \cdot \upg \cdot u].
	\]
	This satisfies:
	\begin{enumerate}[\normalfont(i)]
		\item If $u \simeq u'$ as paths, then $u_{\#} = u_{\#}'$.
		\item $(c_{x_0})_{\#} = \id_{\pi_1(X, x_0)}$.
		\item If $v$ is a path from $x_1$ to $x_2$, then $(u \cdot v)_{\#} = v_{\#} \circ u_{\#}$.
		\item If $f : X \to Y$ sends $x_0$ to $y_0$ and $x_1$ to $y_1$, then the following diagram commutes:
			\[
			\begin{tikzcd}
				\pi_1(X, x_0) \arrow[r, "f_\ast"] \arrow[d, "u_{\#}"] & \pi_1(Y, y_0) \arrow[d, "(f \circ u)_{\#}"] \\
				\pi_1(X, x_1) \arrow[r, "f_\ast"] & \pi_1(Y, y_1)
			\end{tikzcd}
			\]
		\item If $u$ is a path from $x_0$ to $x_0$, then $u_{\#}$ is conjugation by $[u] \in \pi_1(X, x_0)$.
	\end{enumerate}
\end{proposition}

\begin{proofbox}
	$u_{\#}$ is a group homorphism via
	\begin{align*}
		u_{\#}([\upg]) u_{\#}([\upg']) &= [u^{-1} \cdot \upg \cdot u] [u^{-1} \cdot \upg' \cdot u] = [u^{-1} \cdot \upg \cdot u \cdot u^{-1} \cdot \upg' \cdot u] \\
					       &= [u^{-1} \cdot \upg \cdot c_{x_1} \cdot \upg' \cdot u] \\
					       &= [u^{-1} \cdot \upg \cdot \upg' \cdot u] = u_{\#}([\upg] \cdot [\upg']).
	\end{align*}
	It is an isomorphism as $(u^{-1})_{\#}$ is an inverse.

	For (iv), note
	\begin{align*}
		((f \circ u)_{\#})(f_\ast[\upg]) &= (f \circ u)_{\#}([f \circ \upg]) = [(f \circ u)^{-1} \cdot f \circ \upg \cdot f \circ u] \\
					   &= [f \circ (u^{-1} \cdot \upg \cdot u)] = f_\ast([u^{-1} \cdot \upg \cdot u]) \\
					   &= (f_\ast \circ u_{\#})([\upg]).
	\end{align*}
\end{proofbox}

\begin{lemma}
	If $H : X \times I \to Y$ is a homotopy from $f$ to $g$ and $x_0 \in X$ is a base point, then $u = H(x_0, t) : I \to Y$ is a path from $f(x_0)$ to $g(x_0)$.

	Then the following is a commutative diagram:
	\[
	\begin{tikzcd}
		\pi_1(X, x_0) \arrow[r, "f_\ast"] \arrow[d, "g_\ast"] & \pi_1(Y, f(x_0)) \arrow[dl, "u_{\#}"] \\
		\pi_1(Y, g(x_0)) &
	\end{tikzcd}
\]
\end{lemma}

\begin{proofbox}
	Draw a cool looking diagram. Look at the map
	\[
	\begin{tikzcd}
		I \times I \arrow[r, "\upg \times \id"] & X \times I \arrow[r, "H"] & Y.
	\end{tikzcd}
	\]
	We want $g \circ \upg \simeq u^{-1} \cdot (f \circ \upg) \cdot u$ as loops. If we look at the square, the top edge in the square is homotopic to the concatenation of the other edges. Applying $H \circ (\upg\times \id_I)$ gives us the required homotopy.
\end{proofbox}

%lecture 4

\begin{theorem}
	If $f : X \to Y$ is a homotopy equivalence, and $x_0 \in X$, then
	\[
	f_\ast : \pi_1(X, x_0) \to \pi_1(Y, f(x_0))
	\]
	is an isomorphism.
\end{theorem}

\begin{proofbox}
	Let $g : Y \to X$ be a homotopy inverse, with $f \circ g \simeq_H \id_Y$, $g \circ f \simeq_{H'} \id_X$. Let $u' : I \to X$ be $u'(t) = H'(x_0, 1-t)$ be a path from $x_0$ to $g \circ f(x_0)$. Then,
	\[
		u'_{\#} = (g \circ f)_{\ast} : \pi_1(X, x_0) \overset{f_\ast}{\injto} \pi_1(Y, f(x_0)) \overset{g_\ast}{\surjto} \pi_1(X, g \circ f(x_0))
	\]
	gives an isomorphism, by the previous lemma on the maps $\id$ and $g \circ f$. We want the first map to be an isomorphism, hence it is enough to show the second map is injective. But now consider $u(t) = H(f(x_0), 1-t)$, and this gives us an isomorphism
	\[
		u_{\#} = (f \circ g)_{\ast} : \pi_1(Y, f(x_0)) \overset{g_\ast}{\injto} \pi_1(X, g \circ f(x_0)) \overset{f_\ast}{\surjto} \pi_1(Y, f \circ g \circ f(x_0)).
	\]
	Hence the first map is injective, and so $f_\ast = g_\ast^{-1} \circ u_{\#}'$ is an isomorphism.
\end{proofbox}

\begin{definition}
	A space $X$ is \emph{simply-connected}\index{simply-connected} if it is path connected and $\pi_1(X, x_0) = \{e\}$ for some (hence all) $x_0 \in X$.
\end{definition}

\begin{exbox}
	A contractible space is simply connected; $X \simeq \ast$, so $\pi_0(X)$ and $\pi_1(X, x_0)$ are trivial.
\end{exbox}

\begin{lemma}
	$X$ is simply connected if and only if, for all $x_0, x_1 \in X$, there is a unique homotopy class of paths from $x_0$ to $x_1$.
\end{lemma}

\begin{proofbox}
	Let $X$ be simply-connected, and $x_0, x_1 \in X$. As $X$ is path-connected, there exists a path from $x_0$ to $x_1$. If $\upg, \upg'$ are two such paths, then $\upg^{-1} \cdot \upg'$ is a loop based at $x_1$, so $[\upg^{-1} \cdot \upg] \in \pi_1(X, x_1) = \{e\}$, hence $\upg^{-1} \cdot \upg' \simeq c_{x_1}$, rel. $x_1$. Hence 
	\[
	\upg' \simeq \upg \cdot \upg^{-1} \cdot \upg' \simeq \upg \cdot c_{x_1} \simeq \upg,
	\]
	relative to the endpoints.

	Conversely, if $X$ has the stated property, then
	\begin{enumerate}[(i)]
		\item It is path connected, as there exists a path between any two points.
		\item Any loop based at $x_0$ is homotopic to $c_{x_0}$ as loops, by uniqueness.
	\end{enumerate}
	Hence $\pi_1(X, x_0) = \{e\}$.
\end{proofbox}

\newpage

\section{Covering Spaces}
\label{sec:cov_spac}

\begin{definition}
	A \emph{covering map}\index{covering map} $p : \tilde X \to X$ is a continuous map such that for any $x \in X$, there exists an open neighbourhood $U \ni x$ such that
	\[
	p^{-1}(U) = \bigsqcup_{\alpha \in I} V_\alpha,
	\]
	where
	\[
	p|_{V_\alpha} : V_{\alpha} \to U
	\]
	is a homeomorphism.
\end{definition}

\begin{exbox}
	\begin{enumerate}[(i)]
		\item A homomorphism is a covering map.
		\item If $p : \tilde X \to X$ and $q : \tilde Y \to Y$, then $p \times q : \tilde X \times \tilde Y \to X \times Y$ are covering maps.
	\end{enumerate}
\end{exbox}

\begin{exbox}
	Let $S^1 \subset \mathbb{C}$ be the unit complex numbers, and
	\begin{align*}
		p : \mathbb{R} & \to S^1 \\
		t &\mapsto e^{2\pi i t} = (\cos(2 \pi t), \sin(2 \pi t)).
	\end{align*}
	Let $U_{y > 0} = \{x + iy \in S^1 \mid y > 0\}$. Then,
	\[
	p^{-1}(U_{y > 0}) = \bigsqcup_{j \in \mathbb{Z}}(j, j + 1/2).
	\]
	Now we can verify that $p|_{(j, j+1/2)}$ is a homeomorphism, via
	\begin{align*}
		(j, j + 1/2) &\to U_{y>0} \\
		j + \frac{\arccos(x)}{2 \pi} &\mapsfrom x + iu.
	\end{align*}
	Similarly, $U_{y < 0}$, $U_{x > 0}$ and $U_{x < 0}$ satisfy these properties, so $p$ is a covering map.
\end{exbox}

\begin{exbox}
	Let $S^1$ be as before, and
	\begin{align*}
		p : S^1 &\to S^1 \\
		z &\mapsto z^n,
	\end{align*}
	for $n > 0$ an integer. Let $y \in S^1$, and consider $p^{-1}(y)$, the $n$'th roots of $y$. Choosing a root $\xi$ and letting $\zeta = e^{2 \pi i/n}$, then these roots are $\{\xi, \zeta \xi, \zeta^2 \xi, \ldots, \zeta^{n-1} \xi\}$.

	Then $S^1 - \{y\}$ is open, and $p^{-1}(S^1 - \{y\}) = S^1 - \{\xi, \zeta \xi, \ldots, \zeta^{n-1} \xi\}$. Now let
	\[
		V_0 = \left\{z \in S^1 \mid 0 < \arg \left( \frac{\zeta}{\xi} \right) < \frac{2 \pi}{n} \right\},
	\]
	and define $V_i = \zeta^i \cdot V_0$. Then each $x \neq y \in S^1$ has a unique $n$'th root in each $V_i$, so $p|_{V_i} : V_i \to S - \{y\}$ is a bijection, in fact a homeomorphism. Hence $p$ is a covering map.
\end{exbox}

\begin{exbox}
	Let $S^2 \subset \mathbb{R}^3$ be the unit vectors, and let
	\[
		\mathbb{RP}^2 = S^2/\{x \sim -x\}.
	\]
	We will define
	\begin{align*}
		p : S^2 &\to \mathbb{RP}^2 \\
		x &\mapsto [x]
	\end{align*}
	Let $V = \{(x, y, z) \in S^2 \mid z \neq 0\}$, and $U = p(V)$. Then $p^{-1}(U) = V$ is open, so $U$ is open in $\mathbb{RP}^2$. Now $p^{-1}(U) = V = V_{z > 0} \sqcup V_{z < 0}$.

	Then we claim that $p|_{V_{z > 0}} : V_{z > 0} \to U$ and $p|_{V_{z < 0}}$ are homeomorphisms.

	Indeed, to construct an inverse $g : U \to V_{z > 0}$, we use the definition of the product topology. Consider
	\begin{align*}
		t : V &\to V_{z > 0}\\
		(x, y, z) &\mapsto
		\begin{cases}
			(x, y, z) & z > 0,\\
			(-x,-y,-z) & z < 0.
		\end{cases}
	\end{align*}
	Note that $t$ descends to $T : U \to V_{z > 0}$ as a map of sets, hence it is a continuous map by the definition of the quotient topology. It is the inverse to $p|_{V_{z > 0}}$.

	Doing the same with the $x$ and $y$-coordinates will let us cover $\mathbb{RP}^2$.
\end{exbox}

%lecture 5

\begin{definition}
	Let $p ; \tilde X \to X$ be a covering map and $f : Y \to X$ be a map. A \emph{lift}\index{lift} of $f$ along $p$ is a map $\tilde f : Y \to \tilde X$ such that $p \circ \tilde f = f$.
	\[
	\begin{tikzcd}
		& \tilde X \arrow[d, "p"] \\
		Y \arrow[ur, dotted, "\tilde f"] \arrow[r, "f"] & X
	\end{tikzcd}
	\]
\end{definition}

\begin{lemma}[Uniqueness of Lifts]
	If $\tilde f_0$ and $\tilde f_1$ are lifts of $f : Y \to X$ along a covering map $p : \tilde X \to X$, then the set
	\[
		S = \{y \in Y \mid \tilde f_0(y) = \tilde f_1(y)\}
	\]
	is both an open and closed set. Hence if $Y$ is connected, then $S = \emptyset$ or $S = Y$.
\end{lemma}

\begin{proofbox}
	We first show $S$ is open. Let $s \in S$, and let $U \ni f(s)$ be an open neighbourhood which is evenly-covered, meaning
	\[
	p^{-1}(U) = \bigsqcup_{\alpha \in I} V_\alpha.
	\]
	But $\tilde f_0(s)$ and $\tilde f_1(s)$ agree, so they exist in the same $V_\alpha$. Then if we set $N = \tilde f_0^{-1}(V_\alpha) \cap \tilde f_1^{-1}(V_\alpha)$, we have
	\[
	p|_{V_\alpha} \circ \tilde f_0|_N = f|_N = p|_{V_\alpha} \circ \tilde f_1|_N,
	\]
	but $p|_{V_\alpha}$ is a homeomorphism, so $\tilde f_0|_N = \tilde f_1|_N$, hence $s \in N \subset S$, and so $S$ is open.

	Now we show that $S$ is closed. Let $y \in \bar S$, and $\tilde f_0(y) \neq \tilde f_1(y)$. Let $U \ni f(y)$ be an open neighbourhood that is evenly-covered. Then $\tilde f_0(y) \in V_\beta$ and $\tilde f_1(y) \in V_\gamma$, with $\beta \neq \gamma$.

	But now $\tilde f_0^{-1}(V_\beta) \cap \tilde f_1^{-1}(V_\gamma)$ is an open neighbourhood containing $y \in \bar S$, hence it must intersect in $S$. But then $V_\beta$ and $V_\gamma$ must intersect, which is a contradiction.
\end{proofbox}

\begin{theorem}[Homotopy Lifting Lemma]
	Let $p : \tilde X \to X$ be a covering space, $H : Y \times I \to X$ from $f_0$ to $f_1$ be a homotopy, and $\tilde f_0$ be a lift of $f_0$. Then there exists a unique homotopy $\tilde H : Y \times I \to \tilde X$, such that
	\begin{enumerate}[\normalfont(i)]
		\item $\tilde H(\cdot, 0) = \tilde f_0(\cdot)$.
		\item $p \circ \tilde H = H$.
	\end{enumerate}
\end{theorem}

\begin{proofbox}
	Let $\{U_\alpha\}$ be an open cover of $X$ by sets which are evenly-covered. So
	\[
	p^{-1}(U_\alpha) = \bigsqcup_{\beta \in I_\alpha} V_\beta,
	\]
	and $p|_{V_\beta}$ is a homeomorphism from $V_\beta$ to $U_\alpha$. Now $\{H^{-1}(U_\alpha)\}$ is an open cover of $Y \times I$, and for each $y_0 \in Y$, it gives an open cover of $\{y_0\} \times I$.

	By the Lebesgue number lemma, there is a $N = N(y_0)$ such that each path
	\[
		H|_{\{y_0\} \times [\frac{i}{N}, \frac{i+1}{N}]} : \{y_0\} \times \left[ \frac{i}{N}, \frac{i+1}{N}\right] \to X
	\]
	has image inside some $U_\alpha$. In fact, as $\{y_0\} \times I$ is compact, there is an open $W_{y_0} \ni y_0$ such that $H(W_{y_0} \times [\frac{i}{N}, \frac{i+1}{N}])$ lies in some $U_\alpha$.

	We can construct a lift $\tilde H|_{W_{y_0} \times I}$ as follows:
	\begin{enumerate}[(i)]
		\item Note $H|_{W_{y_0} \times [0, \frac{1}{N}]} : W_{y_0} \times [0, \frac{1}{N}] \to U_\alpha \in X$, and hence $\tilde f_0 |_{W_{y_0}} : W_{y_0} \to \tilde X$ with image in some $V_\beta$ lying on $U_\alpha$. Define
			\[
				\tilde H|_{W_{y_0} \times [0, \frac{1}{N}]} : W_{y_0} \times \left[0, \frac{1}{N} \right] \overset{H|}{\to} U_\alpha \overset{p|_{V_\beta}^{-1}}{\to} V_\beta \subseteq \tilde X.
			\]
		\item We proceed in the same way, lifting $H|_{W_{y_0} \times [\frac{1}{N}, \frac{2}{N}]}$ starting at $\tilde H|_{W_{y_0} \times \{\frac{1}{N}\}}$, and continue.
	\end{enumerate}
	At the end of this process, we get a $\tilde H|_{W_{y_0} \times I}$ lifting $H|_{W_{y_0} \times I}$ and extending $\tilde f_0$ at time $0$.

	We can do this for each $y_0 \in Y$, so it is enough to check that on $(W_{y_0} \times I) \cap (W_{y_1} \times I) = (W_{y_0} \times W_{y_1}) \times I$, the two lifts agree.

	For a $y_2 \in W_{y_0} \cap W_{y_1}$, the two choices gives lifts of $H|_{\{y_0\} \times I}$ which agree with $\tilde f_0(y_2)$ at time $0$. By the uniqueness lemma, these lifts must agree on the whole of $\{y_2\} \times I$. So they agree.

	Now uniqueness of the lifted homotopy is immediate from uniqueness of lifts.
\end{proofbox}

If $Y = \{\ast\}$, we get the following corollary.

\begin{corollary}[Path Lifting]
	If $p : \tilde X \to X$ is a covering map,  $\upg : I \to X$ a path from $x_0$ to $x_1$, and $\tilde x_0 \in \tilde X$ is such that $p(\tilde x_0) = x_0$, then there is a unique path $\tilde \upg : I \to \tilde X$ such that $\tilde \upg(0) = \tilde x_0$ and $p \circ \tilde \upg = \upg$.
\end{corollary}

\begin{corollary}
	Let $p : \tilde X \to X$ be a covering map, $\upg, \upg' : I \to X$ be paths from $x_0$ to $x_1$, and $\tilde \upg, \tilde \upg' : I \to \tilde X$ be their lifts starting at $\tilde x_0 \in p^{-1}(x_0)$.

	If $\upg \simeq \upg'$ as paths, then $\tilde\upg \simeq \tilde\upg'$ as paths, in particular $\tilde \upg(1) = \tilde \upg'(1)$.
\end{corollary}

\begin{proofbox}
	Let $H : I \times I \to X$ be a homotopy from $\upg$ to $\upg'$ relative to the endpoints.

	The homotopy lifting lemma gives a $\tilde H : I \times I \to \tilde X$. At $t = 0$, it must be $\tilde \upg$, as it is forced. The edges at $x_0$ is a lift of the constant function, hence must be $c_{\tilde x_0}$. Therefore the edge at $t = 1$ is a lift of $\upg'$ starting at $\tilde x_0$, so is $\tilde \upg'$. Finally the rightmost edge is a lift of the constant $c_{x_1}$, so must be $c_{\tilde x_1}$.

	Hence this $\tilde H$ is a homotopy of paths, as required.
\end{proofbox}

\begin{corollary}
	Let $p : \tilde X \to X$ be a covering map, and $X$ be path-connected. Then the sets $p^{-1}(x)$ are all in bijection with each other.
\end{corollary}

\begin{proofbox}
	Let $\upg : I \to X$ be a path from $x_0$ to $x_1$. Define
	\begin{align*}
		\upg_\ast : p^{-1}(x_0) &\to p^{-1}(x_1) \\
		y_0 &\mapsto \tilde \upg(1),
	\end{align*}
	where $\tilde \upg$ is the lift of $\upg$ starting at $y_0$. $(\upg^{-1})_\ast : p^{-1}(x_1) \to p^{-1}(x_0)$ is defined similarly.

	Then $(\upg^{-1})_\ast \circ \upg_\ast (y_0)$ is the end point of the lift of $\upg \cdot \upg^{-1}$ which starts at $y_0$, which is the end point of $c_{y_0}$ as these are homotopic, which is itself $y_0$.
\end{proofbox}

%lecture 6

\begin{lemma}
	Let $p : \tilde X \to X$ be a covering map, $x_0 \in X$ be a base point, and $\tilde x_0 \in p^{-1}(x_0) \in \tilde X$. Then $p_\ast : \pi_1(\tilde X, \tilde x_0) \to \pi_1(X, x_0)$ is an injective homomorphism.
\end{lemma}

\begin{proofbox}
	Let $\upg : I \to \tilde X$ be a loop based at $\tilde x_0$, and suppose that $p_\ast[\upg] = [c_{x_0}]$, so $p \circ \upg \simeq_H c_{x_0}$ as loops. Now lift $H$ to a homotopy $\tilde H$ starting at $\upg$: then $\tilde H$ is a homotopy of paths from $\upg$ to a lift of $c_{x_0}$, which must be $c_{\tilde x_0}$. Thus $[\upg] = [c_{\tilde x_0}] = e \in \pi_1(\tilde X, \tilde x_0)$.
\end{proofbox}

In the proof of the previous corollary, we constructed, for a path $\upg : I \to X$ from $x_0$ to $x_1$ a bijection $\upg_\ast : p^{-1}(x_0)\to p^{-1}(x_1)$. It only depended on the homotopy class of the path $\upg$.

This defines a (right) action of $\pi_1(X, x_0)$ on $p^{-1}(x_0)$, via $y_0 \cdot [\upg] = \tilde \upg(1)$, for $\tilde \upg(1)$ the lift of $\upg$ starting at $y_0$.

\begin{lemma}
	Let $p : \tilde X \to X$ be a covering map, with $X$ path-connected, and $x_0 \in X$. Then:
	\begin{enumerate}[\normalfont(i)]
		\item $\pi_1(X, x_0)$ acts transitively on $p^{-1}(x_0) \iff \tilde X$ is path-connected.
		\item The stabiliser of $y_0 \in p^{-1}(x_0)$ is $\Img(\pi_1(\tilde X , y_0) \overset{p_\ast}{\to} \pi_1(X, x_0)) \leq \pi_1(X, x_0)$.
		\item If $\tilde X$ is path-connected, then there is a bijection
			\[
			\frac{\pi_1(X, x_0)}{p_\ast \pi_1(\tilde X, y_0)} \leftrightarrow p^{-1}(x_0),
			\]
			induced by acting on $y_0 \in p^{-1}(x_0)$.
	\end{enumerate}
\end{lemma}

\begin{proofbox}
	Let $\tilde X$ be path-connected, and $y_0, z_0 \in p^{-1}(x_0)$. Let $\tilde \upg : I \to \tilde X$ be a path from $y_0$ to $z_0$, so $\upg = p \circ \tilde \upg : I \to X$ is a loop based at $x_0$.

	Then the lift of $\upg$ starting at $y_0$ is $\tilde \upg$, which ends at $z_0$, so $y_0 \cdot [\upg] = z_0$. Hence the action is transitive.

	Conversely, suppose that the action is transitive. Take $z_0 \in p^{-1}(x_0)$, and $y_0$ arbitrary. We show there is a path from $y_0$ to $z_0$. Choose a path from $p(y_0)$ to $p(z_0) = x_0$, then take a lift of it starting at $y_0$. It ends at $z_1 \in p^{-1}(x_0)$. Then as the action on $\pi_1(X, x_0)$ on $p^{-1}(x_0)$ is transitive, there exists a loop which starts at $z_1$ and ends at $z_0$.

	Taking the concatenation of these paths gives a path from $y_0$ to $z_0$. Then for any other $y_1$, taking the concatenation of the path from $z_0$ to $y_1$ gives a path from $y_0$ to $_1$.

	For (ii), suppose that $y_0 \cdot [\upg] = y_0$, i.e. the lift of $\upg$ starting at $y_0$ ends at $y_0$. Then $[\upg] = p_\ast[\tilde \upg]$, hence $\Stab_{\pi_1(X, x_0)}(p^{-1}(x_0)) \leq \Img(p_\ast)$. If $[\upg] = p_\ast[\upg']$, then $[\upg']$ is the lift of $\upg$ starting at $y_0$, but also it ends at $y_0$, so $y_0 \cdot[\upg] = y_0$.

	Then (iii) is orbit-stabiliser.
\end{proofbox}

\begin{definition}
	If $p : \tilde X \to X$ is a covering map, we say:
	\begin{itemize}
		\item It is $n$\emph{-sheeted}\index{$n$-sheeted} if $p^{-1}(x_0)$ has cardinality $n \in \mathbb{N} \cup \{\infty\}$.
		\item It is a \emph{universal cover}\index{universal cover} if $\tilde X$ is simply-connected.
	\end{itemize}
\end{definition}

\begin{corollary}
	If $p : \tilde X \to X$ is a universal cover, then each $\tilde x_0 \in p^{-1}(x_0)$ determines a bijection
	\begin{align*}
		\ell : \pi_1(X, x_0) &\to p^{-1}(x_0) \\
		[\upg] &\mapsto \tilde \upg(1),
	\end{align*}
	for $\tilde \upg$ the lift of $\upg$ starting at $\tilde x$.
\end{corollary}

This induces a group-law on $p^{-1}(x_0)$ via
\[
y_0 \ast z_0 = \ell(\ell^{-1}(y_0) \cdot \ell^{-1}(z_0)).
\]
Spelling this out, this is given by:
\begin{itemize}
	\item Choose a path $\tilde \upg : I \to \tilde X$ from $\tilde x_0$ to $z_0$.
	\item Let $\upg$ be the lift of $p \circ \tilde \upg$ starting at $y_0$.
	\item Then $y_0 \ast z_0 = \upg(1)$.
\end{itemize}

\subsection{Fundamental Group of \texorpdfstring{$S^1$}{S\^1}}
\label{sub:fun_cir}

\begin{theorem}
	Let $u : I \to S^1$ be $u(s) = e^{2 \pi i s}$, which is based at $1 \in S^1 \subset \mathbb{C}$. Then there is an isomorphism $\pi_1(S^1, 1) \cong (\mathbb{Z}, +, 0)$ which sends $u$ to $1 \in \mathbb{Z}$.
\end{theorem}

\begin{proofbox}
	We have $p : \mathbb{R} \to S^1$ by $t \mapsto e^{2 \pi i t}$ is a covering map. Note $\mathbb{R}$ is contractible so it is simply-connected. Hence this is a universal cover, so
	\[
	\pi_1(S^1, 1) \leftrightarrow p^{-1}(1) = \mathbb{Z} \subset \mathbb{R}
	\]
	is a bijection. To compute $\ell^{-1}(m)$, we can take $\tilde u_m : I \to \mathbb{R}$ by $t \mapsto mt$, so $u_m = p \circ \tilde u_m$ is a loop in $S^1$. Take $\tilde x_0 = 0 \in \mathbb{Z}$. So $n \ast m$ is the end point of the lift of $u_m$ starting at $n$, which is $t \mapsto n + mt$. This ends at $n + m$, hence our group operation is addition.
\end{proofbox}

\begin{theorem}
	The disc $D^2$ does not retract to its boundary $S^1$.
\end{theorem}

\begin{proofbox}
	Suppose $r : D^2 \to S^1$ is a retraction, with $i : S^1 \injto D^2$ the inclusion. Then $r \circ i = \id_{S^1}$. Now,
	\[
	\begin{tikzcd}
		\id : \pi_1(S^1, 1) \cong \mathbb{Z} \arrow[r, "i_\ast"] & \pi_1(D^2, 1) \cong \{0\} \arrow[r, "r_\ast"] & \pi_1(S^1, 1).
	\end{tikzcd}
	\]
	This is not possible, so $r$ does not exist.
\end{proofbox}

\begin{corollary}[Brouwer Fixed Point Theorem]
	Any map $f : D^2 \to D^2$ has a fixed point.
\end{corollary}

\begin{proofbox}
	Suppose not. Define $r : D^2 \to S^1$, by intersecting the ray starting at $f(x)$ and going through $x$ with $S^1$. Then this fixes the boundary, and would be continuous, but this would give a retraction $X$.
\end{proofbox}

%lecture 7

Let's look at another application of the fundamental group of a circle.

\begin{corollary}[Fundamental Theorem of Algebra]
	Any non-constant polynomial over $\mathbb{C}$ has a root in $\mathbb{C}$.
\end{corollary}

\begin{proofbox}
	Let $p(z) = z^n + a_1 z^{n-1} + \cdots + z_n$, a non-constant polynomial. Choose $r > |a_1| + \cdots + |a_n| + 1$. On the circle  $|z| = r$, we have
	\[
	|z^n| = |z^{n-1}|r > |z^{n-1}|(|a_1| + \cdots + |a_n|) > |a_1 z^{n-1} + \cdots + a_n|,
	\]
	which means for any $t \in [0, 1]$, the polynomial $p_t(z) = z^n + t(a_1 z^{n-1} + \cdots + a_n)$ does not have a root in the circle $|z| = r$. Consider the homotopy of loops in $S^1 \subset \mathbb{C}$ 
	\[
	F(s, t) = \frac{p_t(r e^{2 \pi i s})/p_t(r)}{|p_t(r e^{2 \pi i s})/p_t(r)|}.
	\]
	Note that $F(0, t) = 1 = F(1, t)$, so this is indeed a loop homotopy.

	At $t = 0$, this is the loop $s \mapsto e^{2 \pi i sn}$, which is $n \in \mathbb{Z} \sim \pi_1(S^1, 1)$.

	At $t = 1$, it is the loop $s \overset{f_r}{\mapsto} [p(r e^{2 \pi i s})/p(r)]/|p(re^{2\pi is})/p(r)|$, which also represents $n \in \mathbb{Z}$.

	Now suppose $p$ has no roots. Then $f_r$ is a continuous loops for all $r \in [0, \infty)$, and varying $r$ gives a homotopy from $f_r$ to $f_0$. But
	\[
	s \mapsto f_0(s) = \frac{p(0)/p(0)}{|p(0)/p(0)|} = 1,
	\]
	hence we must have $n = 0 \in \mathbb{Z}$, so $p$ was constant.
\end{proofbox}

\subsection{Universal Covers}
\label{sub:uncover}

Universal covers are really helpful in determining the fundamental group of an object. We now look to methodically construct universal covers.

Our first observation is as follows: if $p : \tilde X \to X$ is a universal cover, with $x \in X$ and $U \ni x$ an evenly-covered neighbourhood, let $p^{-1}(U) = \bigsqcup V_\tau$. Let $\upg : I \to U$ be a loop based at $x_0$ this lifts to
\[
\begin{tikzcd}[column sep = small]
	\tilde \upg : I \arrow[rr]\arrow[rd, "\upg"] & & V_\alpha \subseteq \tilde X \\
						     & U \arrow[ur, "p|_{V_\alpha^{-1}}"'] &
\end{tikzcd}
\]
This is homotopic to a constant loop in $\tilde X$, as $\tilde X$ is simply connected. Applying $p$, this shows $\upg \simeq c_{x_0}$ in $X$. Hence every $x \in X$ has a neighbourhood $U \ni x$ such that $\pi_1(U, x_1) \to \pi_1(X, x)$ is trivial.

Note not every space has this property, for example the Hawaiian earring, obtained by gluing circles of shrinking radius at the same point as arbitrarily small neighbourhoods with non-trivial fundamental group.

Our second observation is as follows: suppose $p : \tilde X \to X$ is a universal cover, and $x_0 = p(\tilde x_0)$ is a basepoint. Then there is a unique path $\alpha$ from $\tilde x_0$ to $y$, up to homotopy. Hence $y$ is the end point of the lift $p \circ \alpha$, starting at $\tilde x_0$.

Therefore, there is a bijection
\begin{align*}
	\tilde x &\leftrightarrow \{\text{homotopy classes of paths in $X$ starting at $x_0$}\}, \\
	y &\mapsto p \circ \alpha, \\
	\tilde \upg(1) &\mapsfrom [\upg].
\end{align*}

So we aim to build a universal cover by creating a topology on the homotopy classes.

\begin{theorem}
	Let $X$ be path connected, locally path connected, and semi-locally simply connected. Then it has a universal cover.
\end{theorem}

\begin{proofbox}
	This is non-examinable as it is super annoying.

	As a set, let $\tilde X$ be the set of homotopy classes of paths in $X$ starting at $x_0$. Then define
	\begin{align*}
		p : \tilde X &\to X, \\
		[\upg] &\mapsto \upg(1).
	\end{align*}
	We have a couple of aims.
	\begin{enumerate}[(i)]
		\item We want to make a topology on $\tilde X$.
		\item We need to show $p$ is continuous.
		\item Moreover, we need $p$ to be a covering map.
		\item Finally we show $\tilde X$ is indeed simply-connected.
	\end{enumerate}
	Consider the sets
	\begin{align*}
		\mathcal{U} = \{ U \subseteq X \mid U &\text{ open, path-connected and}\\
		&\pi_1(U, x) \to \pi_1(X, x) \text{ is trivial for all } x \in U\}.
	\end{align*}
	We will show this is a basis for the topology on $X$. Indeed, let $V \ni x$ be an open neighbourhood.
	\begin{enumerate}[(i)]
		\item Since $X$ is semi-locally simply-connected, $U' \ni X$ such that $\pi_1(U', x) \to \pi_1(X, x)$ is trivial.
		\item As $X$ is locally path-connected, we can find $V \cap U' = U \ni x$ which is path-connected.
		\item The map
			\[
			\begin{tikzcd}[column sep = small]
				\pi_1(U, x) \arrow[rr] \arrow[rd] & & \pi_1(X, x) \\
						       & \pi_1(U', x) \arrow[ur, "\mathrm{trivial}"']
			\end{tikzcd}
			\]
			is trivial.
		\item Let $y \in U$ be another point, and $u : I \to U$ be a path from $x \to y$. Then
			\[
			\begin{tikzcd}
				\pi_1(U, y) \arrow[r] \arrow[d, "u_{\#}"] & \pi_1(X, y) \\
				\pi_1(U, x) \arrow[r, "\mathrm{trivial}"] & \pi_1(X, x) \arrow[u, "u_{\#}"]
			\end{tikzcd}
			\]
			shows that the top map is trivial.
	\end{enumerate}
	Hence we have a basis. For $[\alpha] \in \tilde X$ and a $U \in \mathcal{U}$ such that $\alpha(1) \in U$, define
	\[
		([\alpha], U) = (\alpha, U) = \{[\beta] \in \tilde X \mid [\beta] = [\alpha \cdot \alpha'] \text{ for some path } \alpha' \in U\}.
	\]
	We claim these sets form a basis for a topology on $\tilde X$. Indeed, let $[\beta] \in (\alpha_0, U_0) \cap (\alpha_1, U_1)$. Hence there are $\alpha_0'$, $\alpha_1'$ with $[\alpha_0 \cdot \alpha_0'] = [\beta] = [\alpha_1 \cdot \alpha_1']$.

	Let $\beta(1) \in W \subseteq U_0 \cap U_1$ with $W \in \mathcal{U}$. We want to show that $(\beta, W) \subseteq (U_0, \alpha_0) \cap (U_1, \alpha_1)$. Indeed, if $[\upg] \in (\beta, W)$, then there is a path $\delta$ in $W$ with $[\upg] = [\beta \cdot \delta] = [\alpha_0 \cdot \alpha_1' \cdot \delta]$, and notice $\alpha_1' \cdot \delta \in U_0$, so $[\upg] \in (U_0, \alpha_0)$. Similarly, $[\upg] \in (U_1, \alpha_1)$.

	%lecture 8

	To see that $p$ is continuous, it is enough to check that $p^{-1}(U)$ is open for $U \in \mathcal{U}$. But indeed, if $[\alpha] \in p^{-1}(U)$, then $[\alpha] \in (\alpha, U) \subseteq p^{-1}(U)$, so $p^{-1}(U)$ is open.

	To see that $p$ is a covering map, we first need to show that each
	\[
	p|_{(\alpha, U)} : (\alpha, U) \to U
	\]
	is a homeomorphism. As $U$ is path-connected, for any $y \in U$ there is a path $\upg$ in $U$ from $\alpha(1)$ to $y$, and $p([\alpha \cdot \upg]) = y$, so the map is surjective. Also, if $[\beta], [\beta'] \in (\alpha, U)$ map to the same thing under $p|_{(\alpha, U)}$, then $\beta$ and $\beta'$ end at the same point. Therefore there exists paths $\upg, \upg'$ in $U$ such that $[\beta] = [\alpha \cdot \upg]$ and $[\beta'] = [\alpha \cdot \upg']$. So then
	\[
		[\beta'] = [\alpha \cdot \upg \cdot \upg^{-1} \cdot \upg'] = [\alpha \cdot \upg] = [\beta],
	\]
	as $[\upg^{-1} \cdot \upg']$ is a loop in $U$, hence is homotopic to a constant loop in $X$. Hence $p|_{(\alpha, U)}$ is a bijection, and continuous. It is also open, as $p((\beta, V)) = V$, so $p|_{(\alpha, U)}$ is a homeomorphism.

	Now we claim that $p^{-1}(U)$ can be partitioned into $(\alpha, U)$'s. We have seen that they cover, so it suffices to show that if two intersect, then they are equal. Let $[\upg] \in (\alpha, U) \cap (\beta, U)$, i.e. there are paths $\alpha', \beta' \in U$ such that $[\upg] = [\alpha \cdot \alpha'] = [\beta \cdot \beta']$. Let  $[\delta] \in (\alpha, U)$. Let $[\delta] \in (\alpha, U)$, then
	\[
		[\delta] = [\alpha \cdot \alpha''] = [\alpha \cdot \alpha' \cdot (\alpha')^{-1} \cdot \alpha''] = [\beta \cdot \beta' \cdot (\alpha')^{-1} \cdot \alpha''],
	\]
	hence $[\delta] \in (\beta, U)$. Thus $(\alpha, U) \subseteq (\beta, U)$, and the reverse inclusion obviously holds.

	Finally, we need to show that $\tilde X$ is simply-connected. Note that if $\upg : I \to X$ is a path, then its lift $\tilde \upg : I \to \tilde X$ starting at $[c_{x_0}]$ ends at $[\upg]$, because
	\[
		s \mapsto [t \mapsto \upg(st)] : I \to \tilde X
	\]
	is the lift. So if a loop $\upg$ in $X$ lifts to a loop in $\tilde X$ based at $[c_{x_0}]$, then $[\upg] = [c_{x_0}]$, i.e. $p_\ast \pi_1(\tilde X, [c_{x_0}]) = \{e\} \in \pi_1(X, x_0)$. But $p_\ast$ is injective, so $\pi_1(\tilde X, [c_{x_0}]) = \{e\}$.
\end{proofbox}

\subsection{The Galois Correspondence}
\label{sub:gal_cor}

If $p : \tilde X \to X$ is a covering map, where $\tilde X$ is path connected, then for $x_0 \in X$ and $\tilde x_0 \in p^{-1}(x_0)$, then
\[
p_\ast : \pi_1(\tilde X, \tilde x_0) \to \pi_1(X, x_0)
\]
is injective, giving a subgroup of $\pi_1(X, x_0)$. If $\tilde x_0' \in p^{-1}(x_0)$ is another basepoint, let $\upg$ be a path in $\tilde X$ from $\tilde x_0$ to $\tilde x_0'$, then $p \circ \upg$ is a loop based at $x_0$, and we have
\[
	[p \circ \upg]^{-1} p_\ast \pi_1(\tilde X, \tilde x_0) [p \circ \upg] = p_\ast \pi_1(\tilde X, \tilde x_0') \leq \pi_1(X, x_0).
\]
So fixing a based space $(X, x_0)$, we get a map
\[
\{\text{based path-connected covering maps } p : (\tilde X, \tilde x_0) \to (X, x_0)\} \overset{p \mapsto \Img(p_\ast)}{\to}\{\text{subgroup of } \pi_1(X, x_0)\}.
\]
Moreover if we loosen our restriction on being based, we get a map
\[
	\{\text{path-connected covering maps } p : \tilde X \to X\} \to \{\text{conjugacy classes of subgroups of }\pi_1(X, x_0)\}.
\]

\begin{proposition}[Surjectivity]
	Let $X$ be path-connected, locally path-connected, and semi-locally simply-connected. Then for any $H \in \pi_1(X, x_0)$, there is a $p : (\tilde X, \tilde x_0) \to (X, x_0)$ with $p_\ast \pi_1(\tilde X, \tilde x_0) = H$.
\end{proposition}

\begin{proofbox}
	Let $\bar X \overset{q}{\to} X$ be the universal cover we have constructed. Define $\sim_H$ on $\bar X$ by
	\[
		[\upg] \sim_H [\upg'] \iff \upg(1) = \upg'(1) \text{ and } [\upg \cdot (\upg')^{-1}] \in H \leq \pi_1(X, x_0).
	\]
	First, note that $\sim_H$ is an equivalence relation:
	\begin{enumerate}[(i)]
		\item $[\upg] \sim_H [\upg]$.
		\item If $[\upg] \sim_H [\upg']$, then $[\upg \cdot (\upg')^{-1}] \in H$, so $[\upg' \cdot (\upg)^{-1}] \in H$.
		\item If $[\upg] \sim_H [\upg']$ and $[\upg'] \sim_H [\upg'']$ with $\upg(1) = \upg'(1) = \upg''(1)$, then
			\[
				[\upg \cdot (\upg'')^{-1}] = [\upg \cdot (\upg')^{-1} \cdot \upg' \cdot (\upg'')^{-1}] = [\upg \cdot (\upg')^{-1}] [\upg' \cdot (\upg'')^{-1}] \in H.
			\]
			So $\sim_H$ is an equivalence relation. 
	\end{enumerate}
	Define $\bar X_h = \bar X/ \sim_H$, the quotient space, and $p_H : \tilde X_H \to X$ be the induced map. If $[\upg] \in (\alpha, U)$, $[\upg'] \in (\beta, U)$ satisfies $[\upg] \sim_H [\upg']$, then $(\alpha, U)$ and $(\beta, U)$ are identified by $\sim_H$, as $[\upg \cdot q] \sim_H [\upg' \cdot q]$ for any path $q \in U$.

	It remains to show that $(p_H)_\ast \pi_1(\bar X_h, [[c_{x_0}]]) = H \leq \pi_1(X, x_0)$. If $[\upg] \in H$, then the lift of $\upg$ to $\bar X$ starting at $[c_{x_0}]$ ends at $[\upg]$, so the lift to $\bar X_H$ ends at $[[\upg]] = [[c_{x_0}]]$, hence is a loop. So $H \leq (p_H)_\ast \pi_1(\bar X_H, [[c_{x_0}]])$.

	On the other hand, if $[\upg] \in (p_H)_ast \pi_1(\bar X_h, [[c_{x_0}]])$, then the lift $\bar \upg$ of $\upg$ to $\bar X$ starting at $[c_{x_0}]$ ends at $[\upg]$, so $[\upg] \sim_H [c_{x_0}]$, as it becomes a loop in $\bar X_H$ by assumption. So $[\upg] \in H$.
\end{proofbox}

%lecture 9

Now we just need to show injectivity.

\begin{proposition}[Based Uniqueness]
	Let $(X, x_0)$ satisfy the usual conditions for the existence of a universal cover.

	If $p_1 : (\tilde X_1, \tilde x_1) \to (X, x_0)$ and $p_2 : (\tilde X_2, \tilde x_2) \to (X, x_0)$ are path-connected covering spaces, then there exists a based homeomorphism $\tilde X_1, \tilde x_1) \to (\tilde X_2, \tilde x_2)$ such that $p_2 \circ h = h_1$, if and only if \[
	(p_1)_\ast \pi_1(\tilde X_1, \tilde x_1) = (p_2)_\ast \pi_1(\tilde X_2, \tilde x_2).
\]
\end{proposition}

\begin{proofbox}
	If $h$ exists, then $\Img((p_1)_\ast) = \Img((p_2)_\ast \circ h_\ast) = \Img((p_2)_\ast)$, as $h_\ast$ is an isomorphism.

	On the other hand, suppose the images are both $H \leq \pi_1(X, x_0)$. We will show that $\tilde X_1$ and $\tilde X_2$ are both homeomorphic to $\bar X_H$.

	Consider $r : \bar X \to \tilde X_1$, by $[\upg] \mapsto \tilde \upg(1)$, the end point of the lift $\tilde \upg$ of $\upg$ to $\tilde X_1$ starting at $\tilde x_1$.

	\begin{align*}
	r([\upg]) = r([\upg']) &\iff \tilde \upg\text{ and } \tilde \upg' \text{ both end at the same point of } \tilde X_1 \\
			       &\iff [\upg' \cdot \upg^{-1}] \in (p_1)_\ast \pi_1(\tilde X_1, \tilde x_1) = H \\
			       &\iff [\upg] \sim_H [\upg'].
	\end{align*}
	So $r$ descends to a map $q : (\bar X_H, [[c_{x_0}]])\to (\tilde X_1, \tilde x_1)$, a bijection. It is also an open map, as $\bar X_H$ and $\tilde X_1$ are both locally homeomorphic to $X$. So $q$ is a homeomorphism.
\end{proofbox}

\begin{corollary}[Unbased Uniqueness]
	Suppose the usual hypothesis on $X$.

	If $p_1 : \tilde X_1 \to X$ and $p_2 : \tilde X_2 \to X$ are path connected covering spaces, then there exists a homeomorphism $h : \tilde X_1 \to \tilde X_2$ such that $p_2 \circ h = p_1$ if and only if
	\[
		(p_1)_\ast \pi_1(\tilde X_1, \tilde x_1) \text{ and } (p_2)_\ast \pi_1(\tilde X_2, \tilde x_2) \text{ are conjugate in } \pi_1(X, x_0),
	\]
	for any $\tilde x_1 \in p_1^{-1}(x_0)$, $\tilde x_2 \in p_2^{-1}(x_0)$.
\end{corollary}

\begin{proofbox}
	If $h$ exists, choose $\tilde x_1 \in p_1^{-1}(x_0)$ and $\tilde x_2 = h(\tilde x_1)$. Then the previous proposition applies, and the groups obtained are equal.

	Conversely, suppose $[\upg] \in \pi_1(X, x_0)$ is such that
	\[
		[\upg]^{-1} (p_1)_\ast \pi_1(\tilde X_1, \tilde x_1) [\upg] = (p_2)_\ast \pi_1(\tilde X_2, \tilde x_2).
	\]
	Then lifting $\upg$ to $\tilde X_1$ starting at $\tilde x_1$, it ends at $\tilde x_1' \in p_1^{-1}(x_0)$. Then choosing $\tilde x_1'$ as our basepoint, the left hand side is exactly $(p_1)_\ast \pi_1(\tilde X_1, \tilde x_1')$, using the change of basepoint isomorphism. So our previous proposition gives a based homeomorphism $h : (\tilde X_1, \tilde x_1') \to (\tilde X_2, \tilde x_2)$, which is also an unbased homeomorphism.
\end{proofbox}

\newpage

\section{Free Groups and Presentations}
\label{sec:fg_pres}

Let $S = \{s_\alpha\}_{\alpha \in I}$ be a set, called the \emph{alphabet}\index{alphabet}. Then $S^{-1} = \{s_\alpha^{-1}\}_{\alpha \in I}$. We suppose that $S \cap S^{-1} = \emptyset$.

A \emph{word}\index{word} in the alphabet $S$ is a (possibly empty) finite sequence $(x_1, x_2, \ldots, x_n)$ of elements in $S \cup S^{-1}$. A word is called \emph{reduced}\index{reduced word} if it does not contain $(s_\alpha, s_\alpha^{-1})$ or $(s_\alpha^{-1}, s_\alpha)$ as a subword.

An elementary reduction of the word $(x_1, x_2, \ldots, x_i, s_\alpha, s_\alpha^{-1}, x_{i+1}, \ldots, x_n)$ (or $s_\alpha^{-1}, s_\alpha$) is $(x_1, x_2, \ldots, x_i, x_{i+1}, \ldots, x_n)$.

\begin{definition}
	The \emph{free group}\index{free group} on the alphabet $S$, denoted $F(S)$, is the set of reduced (possibly empty) words in this alphabet.

	The group operation is given by concatenation, and performing elementary reductions until the word is reduced (note this is not obviously a group, as we have a choice on how to do our reductions---we can alternatively define it as a subgroup of a permutation group).

	We have that $()(x_1, \ldots, x_n) = (x_1, \ldots, x_n) = (x_1, \ldots, x_n)()$, and
	\[
		(x_n^{-1}, \ldots, x_1^{-1})(x_1, \ldots, x_n) = () = (x_1, \ldots, x_n)(x_n^{-1}, \ldots, x_1^{-1}).
	\]
	Moreover there is a function $\iota : S \to F(S)$, given by sending $s$ to $(s)$.
\end{definition}

\begin{lemma}[Universal Property of Free Groups]
	For any group $H$, the function
	\[
		\{\text{homomorphisms } \varphi : F(S) \to H\} \overset{\circ \iota}{\to} \{\text{functions } \phi : S \to H\}
	\]
	is a bijection.
\end{lemma}

\begin{proofbox}
	Given $\phi : S \to H$, we want $\varphi : F(S) \to H$ such that $\varphi((s)) = \phi(s)$. Let, on a not necessarily reduced word $(s_{\alpha_1}^{\eps_1}, \ldots, s_{\alpha_n}^{\eps_n})$,
	\[
	\varphi((s_{\alpha_1}^{\eps_1}, \ldots, s_{\alpha_n}^{\eps_n})) = \phi(s_{\alpha_1})^{\eps_1} \cdots \phi(s_{\alpha_n})^{\eps_n} \in H.
	\]
	If the word contained $(s_\alpha, s_\alpha^{-1})$, then the result contains $\phi(s_\alpha) \phi(s_\alpha)^{-1} = e \in H$. So $\varphi$ is well-defined. As the group operation on $F(S)$ is given by concatenation, we see that $\varphi$ is a homomorphism.
\end{proofbox}

\begin{definition}
	Let $S$ be a set, and $R \subseteq F(S)$. Then
	\[
	\langle S \mid R \rangle = F(S) / \langle\langle R \rangle\rangle,
	\]
	with
	\[
		\langle\langle R \rangle\rangle = \{(r_1^{\eps_1})^{g_1} \cdots (r_n^{\eps_n})^{g_n} \mid r_i \in R, \eps_i \in \{\pm1\}, g \in F(S)\} \lhd F(S),
	\]
	where $h^g = g^{-1}hg$. Call this a presentation of the group $\langle S \mid R \rangle$. If $S$ and $R$ are finite, call it a \emph{finite presentation}\index{finite presentation}.
\end{definition}

\begin{lemma}[Universal Property of Group Presentations]
	For any group $H$, the function
	\begin{align*}
		\{\text{group homomorphisms }\psi : &\langle S \mid R \rangle \to H\} \\
		\to \{\text{functions } \phi : S \to H &\text{ such that } R \subseteq \Ker(\varphi : F(S) \to H)\}
	\end{align*}
	is a bijection, by
	\[
		\psi \mapsto (\phi : S \overset{\iota}{\to} F(S) \surjto \langle S \mid R \rangle \overset{\psi}{\to} H).
	\]
\end{lemma}

%lecture 10

\begin{proofbox}
	Suppose $\psi$, $\psi'$ determine functions $\phi = \phi' : S \to H$. Then
	\[
		\begin{tikzcd}
			F(S) \arrow[r, "\text{quot}"] & \langle S \mid R \rangle \arrow[r, bend left, "\psi"'] \arrow[r, bend right, "\psi'"'] & H
		\end{tikzcd}	
	\]
	are equal by the universal property of free groups. As quotient is onto, $\psi = \psi'$.

	Conversely, given $\phi: S \to H$ such that $\varphi(r) = e$ for all $r \in R$, consider
	\[
	\varphi : F(S) \to H.
	\]
	Now $R \subseteq \Ker \varphi$, so as $\Ker \varphi$ is normal, $\langle \langle R \rangle \rangle \leq \Ker \varphi$. Thus $\varphi$ descends to a homomorphism
	\[
	\psi : \langle S \mid R \rangle = \frac{F(S)}{\langle \langle R \rangle \rangle } \to H,
	\]
	as is required.
\end{proofbox}

\begin{exbox}
	If $G$ is a group, the function $\id : G \to G$ gives a homomorphism $f : F(G) \to G$, which is onto. Let $R = \Ker(\phi)$. Then
	\[
	\langle G \mid R \rangle = \frac{F(G)}{\langle \langle R \rangle \rangle} \cong G.
	\]
	Essentially, this is saying that a group is what you get when you take the elements, related by their relations.
\end{exbox}

\begin{exbox}
	Let $G = \langle a, b \mid a \rangle$ and $H = \langle t \mid \rangle$. Now consider $\phi : \{a, b\} \to H$ by $a \mapsto e$ and $b \mapsto t$. Then $\varphi(a) = e$, so we get a homomorphism
	 \[
	\psi : \langle a, b, \mid a \rangle \to \langle t \mid \rangle.
	\]
	Moreover, we can take $\phi : \{t\} \to G$ by $t \mapsto b$. Then $\psi' : \langle t \mid \rangle \to \langle a, b \mid a \rangle$. Since $\psi' \circ \psi([a]) = [e] = [a]$, and  $\psi \circ \psi([b]) = [b]$, as $[a]$ and $[b]$ generate we have $\psi' \circ \psi = \id$. Moreover, $\psi \circ \psi' = \id$, so indeed
	\[
	\langle a, b \mid a \rangle = \langle t \mid \rangle.
	\]
\end{exbox}

\begin{exbox}
	Let $G = \langle a, b \mid ab^{-3}, ba^{-2} \rangle$. Then $[a][b]^{-3} = e$, so $[a] = [b]^3$. Moreover $[b][a]^{-2} = e$, so $[b] = [a]^2$.

	Hence we get $[a] = [a]^6$, or $e = [a]^5$. Moreover $[b] = [a]^2$, so these relation show that every element is equal to one of $e, [a], [a]^2, [a]^3$ or $[a]^4$. We will show that the group has exactly five elements.

	Consider $\phi : \{a, b\} \to \mathbb{Z}/5$, such that $\phi(a) = 1$, $\phi(b) = 2$. Then $\varphi(ab^{-3}) = e - \varphi(ba^{-2})$, so we get a homomorphism $\psi : \langle a, b \mid ab^{-3}, ba^{-2} \rangle \to \mathbb{Z} / 5$, which is isomorphic.
\end{exbox}

\subsection{Free Products with Amalgamation}
\label{sub:fp}

Consider group homomorphisms
\[
\begin{tikzcd}
	G_1 & H \arrow[l, "i_1"'] \arrow[r, "i_2"] & G_2,
\end{tikzcd}
\]
and suppose $G_i = \langle S_i \mid R_i \rangle$ The \emph{free product}\index{free product} of $G_1$ and $G_2$ is
\[
G_1 \ast G_2 = \langle S_1 \sqcup S_2 \mid R_1 \sqcup R_2 \rangle.
\]
The functions
\[
S_i \to S_1 \sqcup S_2 \to F(S_1 \sqcup S_2) \to G_1 \ast G_2
\]
induce homomorphisms
\[
\begin{tikzcd}
	G_1 \arrow[r, "j_1"] & G_1 \ast G_2 & \arrow[l, "j_2"'] G_2.
\end{tikzcd}
\]
The \emph{free product with amalgamation}\index{free product with amalgamation} over $H$ is the quotient
\[
G_1 \ast_H G_2 = G_1 \ast G_2 / \langle \langle j_1 i_1(h) (j_2 i_2(h))^{-1} \mid h \in H \rangle \rangle
\]
Hence the square
\[
\begin{tikzcd}
	H \arrow[r, "i_1"]\arrow[d, "i_2"] & G_1 \arrow[d, "j_1"] \\
	G \arrow[r, "j_2"] & G_1 \ast_H G_2
\end{tikzcd}
\]
commutes.

\begin{lemma}[Universal Property of Free Products with Amalgamation]
	For any group $K$, the following is a bijection:
	\begin{align*}
		\{\text{group homomorphisms } &\phi : G_1 \ast_H G_2 \to K\} \\
		\to \{\text{group homomorphisms } \phi_1 &: G_1 \to K, \phi_2 : G_2 \to K \text{ such that } \phi_1 \circ i_1 = \phi_2 \circ i_2\}.
	\end{align*}
	This is by
	\[
		\phi \mapsto [G_i \overset{j_i}{\to} G_1 \ast_H G_2 \overset{\phi}{\to} K].
	\]
\end{lemma}

In essence the following diagram commutes.

\[
\begin{tikzcd}
	H \arrow[r, "i_1"]\arrow[d, "i_2"] & G_1 \arrow[d, "j_1"] \arrow[ddr, bend left, "\phi_1"] &\\
	G \arrow[r, "j_2"] \arrow[drr, bend right, "\phi_2"] & G_1 \ast_H G_2 \arrow[dr, "\phi"] & \\
			   & & K
\end{tikzcd}
\]

\newpage

\section{The Seifert-Van Kampen Theorem}
\label{sec:svk_thm}

Let $X$ be a space, $A, B \subseteq X$ be subspaces. Moreover let $x_0 \in A \cap B$. Then we get a commutative diagram
\[
\begin{tikzcd}
	\pi_1(A \cap B, x_0) \arrow[r] \arrow[d] & \pi_1(A, x_0) \arrow[d] \\
	\pi_1(B, x_0) \arrow[r] & \pi_1(X, x_0)
\end{tikzcd}
\]
The universal property says that this gives a homomorphism
\[
\pi_1(A, x_0) \ast_{\pi_1(A \cap B, x_0)} \pi_1(B, x_0) \to \pi_1(X, x_0).
\]

\begin{theorem}[Seifert-Van Kampen]
	Let $X$ be a space, $A, B \subseteq X$ be open subsets which cover $X$ and such that $A \cap B$ is path-connected.

	Then for any $x_0 \in A \cap B$, the induced map
	\[
	\pi_1(A, x_0) \ast_{\pi_1(A \cap B, x_0)} \pi_1(B, x_0) \to \pi_1(X, x_0)
	\]
	is an isomorphism.
\end{theorem}

\begin{exbox}
	Consider $S^n$ for $n \geq 2$. Let $A = S^n - \{(1, 0, \ldots, 0)\}$ and $B = S^n - \{(-1, 0, \ldots, 0)\}$. Then $A, B \cong \mathbb{R}^n$ by a projection, and these are contractible, so $A , B \simeq \{\ast\}$.

	Moreover, $A \cap B \cong S^{n-1} \times (-1, 1) \simeq S^{n-1}$, which is path-connected for $n \geq 2$. So
	\[
		\pi_1(S^n, \ast) \cong \{e\} \ast_{\pi_1(S^{n-1}, \ast)} \{e\} = \{e\}.
	\]
	So $S^n$ is simply-connected.
\end{exbox}

\begin{exbox}
	There exists a 2-sheeted covering map $p : S^n \to \mathbb{RP}^n$. For $n \geq 2$, $S^n$ is simply-connected, so this is a universal cover. So,
	\[
		\pi_1(\mathbb{RP}^n, \ast) \overset{\text{bij}}{\to} p^{-1}(\ast),
	\]
	which has two elements. So $\pi_1(\mathbb{RP}^n, \ast) \cong \mathbb{Z}/2$.
\end{exbox}

For the next example, we define, for $(X, x_0)$ and $(Y, y_0)$ based spaces,
\[
X \vee Y = (X \sqcup Y)/(x_0 \sim y_0).
\]

\begin{exbox}
	Let $S^1 \in \mathbb{C}$ have basepoint $1 \in S^1 \subseteq \mathbb{C}$ Then consider $S^{1} \vee S^1$, which is covered by $(S^1 - \{-1\}) \vee S^1 = U$, and $S_1 \vee (S^1 - \{-1\}) = V$. Then $U, V \simeq S^1$, and $U \cap V \simeq \{\ast\}$.

	So by Seifert-Van Kampen,
	\[
		\pi_1(S^1 \vee S^1) \simeq \langle a \mid \rangle \ast_{\{e\}} \langle b \mid \rangle = \langle a, b \mid \rangle.
	\]
\end{exbox}

%lecture 11

\begin{exbox}
	The function $\{a, b\} \to \mathbb{Z}/3$ given by $a , b \mapsto 1$ determines a homomorphism
	\[
	\varphi : \langle a, b \mid \rangle = \pi_1(S^1 \vee S^1, x_0) \to \mathbb{Z}/3,
	\]
	which is surjective. So
	\[
	K = \Ker(\varphi) \leq \pi_1(S^1 \vee S^1, x_0)
	\]
	is a subgroup of index 3. This corresponds to a covering space. What is it? It is a $p : \tilde X \to X$ with
	\[
	p^{-1}(x_0) \cong \frac{\pi_1(S^1 \vee S^1, x_0)}{K} \cong \mathbb{Z}/3.
	\]
	Our thing looks like this:
	\[
	\begin{tikzcd}[column sep = small, row sep = huge]
		& Ka \arrow[dr, bend left = 10, "a"] \arrow[dr, bend right = 10, "b"'] & \\
		K \arrow[ur, bend left = 10, "a"] \arrow[ur, bend right = 10, "b"'] & & Ka^2 \arrow[ll, bend left = 10, "a"] \arrow[ll, bend right = 10, "b"']
	\end{tikzcd}
	\]
	Now if we consider $a \mapsto 1$, $b \mapsto 0$, this gives $\varphi : \pi_1(S^1 \vee S^1, x_0) \to \mathbb{Z}/3$, with kernel $K$. We want the covering space corresponding to $K$.
	\[
	\begin{tikzcd}[column sep = small, row sep = huge]
		& Ka \arrow[dr, bend left = 10, "a"] \arrow[loop, out = 110, in = 70, looseness = 5, "b"] & \\
		K \arrow[ur, bend left = 10, "a"] \arrow[loop, out = 230, in = 190, looseness = 5, "b"] & & Ka^2 \arrow[ll, bend left = 10, "a"] \arrow[loop, out = 350, in = 310, looseness = 5, "b"]
	\end{tikzcd}
	\]
	The universal cover of $S^1 \vee S^1$ is a tree thing.
\end{exbox}

We now defer to the proof of Seifert-Van Kampen.

\begin{proofbox}
	We can assume that $A, B$ are path-connected.

	First, we show that $\phi$ is surjective. Let $\upg : I \to X$ be a loop. Then $\{\upg^{-1}(A), \upg^{-1}(B)\}$ is an open cover of $I$, so by the Lebesgue number lemma there is an $n \gg 0$ such that each $[i/n, (i+1)/n]$ is sent into $A$ or $B$ (or both).

	By concatenating interval which lie entirely in $A$ or $B$, we can write $\upg = \upg_1 \cdot \upg_2 \cdot \ldots \cdot \upg_k$, with each $\upg_i$ having endpoints in $A \cap B$. This is done by concatenating adjacent paths which lie in both $A$ or $B$.

	Now choose paths $u_i$ from $\upg_i(1)$ to $x_0$ in $A \cap B$. Then
	\[
	\upg \simeq (\upg_1 \cdot u_1) \cdot (u_1^{-1} \cdot \upg_2 \cdot u_2) \cdot \ldots \cdot (u_{n-1}^{-1} \cdot \upg_n).
	\]
	Now each of these is a loop based at $x_0$, lying in $A$ or in $B$. So $[\upg] \in \Img(\phi)$, as required.

	Now we show that $\phi$ is injective. Note that $\pi_1(A, x_0) \ast_{\pi_1(A \cap B, x_0)} \pi_1(B, x_0)$ has the following description: it is generated by,
	\begin{enumerate}[(i)]
		\item for $\upg : I \to A$ a loop in $A$, $[\upg]_A$,
		\item for $\upg : I \to B$ a loop in $B$, $[\upg]_B$,
	\end{enumerate}
	with relations:
	\begin{enumerate}[(i)]
		\item If $\upg \simeq \upg'$ in $A$, then $[\upg]_A = [\upg']_A$, and similarly for $B$.
		\item If $\upg : I \to A \cap B$, then $[\upg]_A = [\upg]_B$.
		\item $[\upg]_A [\upg']_A = [\upg \cdot \upg']_A$, and similarly for $B$.
	\end{enumerate}
	Suppose that
	\[
		\phi([\upg_1]_{A_{i1}} [\upg_2]_{A_{i_2}} \cdots [\upg_n]_{A_{i_n}}) = [c_{x_0}],
	\]
	so $\upg_1 \cdots \upg_n \simeq_H c_{x_0}$ in $X$. Take this $H : I \times I \to X$. We can subdivide $I \times I$ into squares of size $1/N$ with $n \mid N$ such that each square is sent into $A$ or $B$, or both.

	Choose paths $u_{ij}$ from $H(i/N, j/N)$ to $x_0$ such that, if it is a vertex of a box labelled $A$, the path is in $A$, and if it is a vertex of a box labelled $B$, the path is in $B$.

	Now there is an insane diagram to form a homotopy $G$, which has the property that it decomposes into rectangles with vertices sent to $x_0$, and each rectangle is in $A$ or in $B$. This shows that $[\upg_1]_{A_{i1}} [\upg_2]_{A_{i_2}} \cdots [\upg_n]_{A_{i_n}}$ can be transformed into the trivial word using the three kinds of relations described.
\end{proofbox}

\subsection{Attaching a Cell}
\label{sub:+_cell}

Let $f : (S^{n-1}, \ast) \to (X, x_0)$. Then let
\[
Y = X \cup_f D^n = \frac{X \sqcup D^n}{f(x) \sim x \in \partial D^n = S^{n-1}}.
\]

\begin{theorem}
	If $n \geq 3$, then
	\[
		\mathrm{inc}_\ast (X, x_0) \to \pi_1(Y, [x_0])
	\]
	is an isomorphism. If $n = 2$, then
	\[
		\mathrm{inc}_\ast (X, x_0) \to \pi_1(Y, [x_0])
	\]
	is the quotient by the normal subgroup generated by $[f] \in \pi_1(X, x_0)$.
\end{theorem}

%lecture 12

\begin{proofbox}
	Let $U = \mathrm{int}(D^n)$, and $V = X \cup_f (D^n \setminus \{0\})$. Then these give an open cover of $Y$. Choose a path $u$ in $U, V$ from $y_0$ to some $y_1 \in \mathrm{int}(D^n) = U$. If $n \geq 3$, then we know that $U \simeq \ast$, and $U \cap V \simeq S^{n-1} \times (0, 1)$ is simply-connected.

	By Seifert-Van Kampen, we have that
	\[
	\pi_1(V, y_1) = \pi_1(U, y_1) \ast_{\pi_1(U \cap V, y_1)} \pi_1(V, y_1) \cong \pi_1(Y, y_1)
	\]
	Now by a change of basepoint isomorphism, we get $\pi_1(Y, y_0) \cong \pi_1(V, y_0)$. But $V$ strongly deformation retracts to $X$, so this is $\pi_1(X, y_0)$.

	If instead $n = 2$, then $U \simeq \ast$, and $U \cap V \simeq S^1 \times (0, 1)$, so Seifer-Van Kampen says
	\[
		\{e\} \ast_{\mathbb{Z}} \pi_1(V, y_1) \cong  \pi_1(Y, y_1) \ast_{\pi_1(U \cap V, y_1)} \pi_1(V, y_1) \cong \pi_1(Y, y_1),
	\]
	or looking at where $1$ goes,
	\[
		\frac{\pi_1(V, y_1)}{\langle \langle u^{-1}_\# [f] \rangle \rangle} \cong \pi_1(Y, y_1).
	\]
	Using a change of basepoint,
	\[
		\frac{\pi_1(V, y_0)}{\langle \langle [f] \rangle \rangle} \cong \pi_1(Y, y_0).
	\]
\end{proofbox}

\begin{exbox}
	Let's look at the torus $T$ (insert picture from geometry). This is the standard definition, however it has a cell structure with:
	\begin{itemize}
		\item 0-cell $x_0$.
		\item 1-cells $a$, $b$.
		\item A two-cell.
	\end{itemize}
	The 1-skeleton is $T^1 = S^1 \vee S^1$, so $\pi_1(T^1, x_0) = \langle a, b \mid \rangle$, and so
	\[
	\pi_1(T, x_0) = \langle a, b \mid aba^{-1}b^{-1}\rangle.
	\]
	Quite cool! We can check that this is isomorphic to $\mathbb{Z} \times \mathbb{Z}$, since this is saying that $ab = ba$, i.e. the group is abelian.
\end{exbox}

\begin{corollary}
	For $G = \langle S \mid R \rangle$ with $S, R$ finite, there is a two-dimensional based cell-complex $(X, x_0)$ with
	\[
	\pi_1(X, x_0) = G.
	\]
\end{corollary}

\begin{proofbox}
	Let $Y$ be the wedge of $|S|$-many circles. Sending $s \in S$ to the $s$-th circle gives an isomorphism $\langle s \mid \rangle \cong \pi_1(Y, y_0)$.

	Each word $r \in R$ is an element of $\langle S \mid \rangle$, so gives a loop $[ \upg_r ] \in \pi_1(Y, y_0)$. Attaching 2-cells to $Y$ along $\{\upg_r\}_{r \in R}$ gives $(X, x_0)$ with
	\[
	\pi_1(X, x_0) = \frac{\langle S \mid \rangle}{\langle\langle r \in R \rangle \rangle} = \langle S \mid R \rangle.
	\]
\end{proofbox}

\subsection{Refining Seifert-Van Kampen}
\label{ref_svk}

\begin{definition}
	A subset $A \subseteq X$ is called a \emph{neighbourhood deformation retract}\index{neighbourhood deformation retract} (NDR) if there is an open neighbourhood $A \subseteq U \subseteq X$, and $U$ strongly deformation retracts to $A$.
\end{definition}

\begin{theorem}[Seifert-Van Kampen]
	Let $X$ be a space, $A, B \subseteq X$ closed subsets which cover $X$ and such that $A \cap B$ is path-connected, and is a NDR in both $A$ and $B$. Then
	\[
	\pi_1(A, x_0) \ast_{\pi_1(A \cap B, x_0)} \pi_1(B, x_0) \cong \pi_1(X, x_0).
	\]
\end{theorem}

\begin{proofbox}
	Let $A \cap B \subseteq U \subseteq A$, and $A \cap B \subseteq V \subseteq B$ for $U, V$ open, and strongly deformation retractable to $A \cap B$. Observe that $(A \cup V)^{c} = B \setminus V$, and $(B \cup U)^{c} = A \setminus U$ are closed. Hence $A \cup V$, $B \cup U$ give an open cover of $X$.

	The deformations of $U$ and $V$ to $A \cap B$ glue to give a deformation of $(A \cup V) \cap (B \cup U) = U \cup V$ to $A \cap B$, $A \cup V$ to $A$ and $B \cup U$ to $B$.

	We can now use Seifert-Van-Kampen on the open cover, and observe that
	\[
	\begin{tikzcd}
		\pi_1(B) \arrow[d, "\sim"] & \pi_1(A \cap B) \arrow[l] \arrow[r] \arrow[d, "\sim"] & \pi_1(A) \arrow[d, "\sim"] \\
		\pi_1(B \cup U) & \pi_1((A \cup V) \cap (B \cup U)) \arrow[l] \arrow[r] & \pi_1(A \cup V),
	\end{tikzcd}
	\]
	so the amalgamated free products are the same, and hence we get what we want.
\end{proofbox}

\subsection{Surfaces}
\label{sub:surf}

\begin{exbox}
	We are going to look at a torus with a ball cut out of it (the ball is actually a triangle).

	Then since it has genus one and has boundary a triangle, it is in some way homeomorphic to a triangle with a little handle.

	But the torus cut out strongly deformation retract to $S^1 \vee S^1$, so $\pi_1(X, x_0) = \langle a, b \mid \rangle$ with $[r] = aba^{-1}b^{-1}$.

	Now consider a whole lot of bumpy pizzas glued together. We will do this $g$ many times to get $F_g$. Applying Seifert-Van Kampen (the closed version) $g$ times gives
	\[
	\pi_1(F_g, x_0) = \langle a_1, b_1, a_2, b_2, \ldots, a_g, b_g \mid\rangle.
	\]
	The boundary is $r_1 r_2 \cdots r_g$, so attached a two cell along it we get $\Sigma_g$, to get that the fundamental group is
	\[
	\pi_1(\Sigma_g, x_0) = \langle a_1, b_1, \ldots, a_g, b_g \mid a_1b_1a_1^{-1}b_1^{-1} a_2 b_2 a_2^{-1} b_2^{-1} \cdots a_g b_g a_g^{-1} b_g^{-1} \rangle.
	\]
	Nice!
\end{exbox}

\begin{exbox}
	Recall we can find $\mathbb{RP}^2$ as a skeleton: it has one-skeleton $a$, and the two-cell is attached along $a \cdot a$, so
	\[
	\pi_1(\mathbb{RP}^2, \ast) \simeq \langle a \mid a^2\rangle = \mathbb{Z}/2.
	\]
	Now take $Y$, which will be $\mathbb{RP}^2$ with a disk cut out (again a triangle). This is homeomorphic to something with a triangle $r$ as a boundary and a non-orientable thing in the middle.

	Now $Y \simeq S^1$, so $\pi_1(Y, y_0) = \langle a \mid \rangle$, and $[r] = a^2$. Glue $n$ of these together to form $E_n$. Then Seifert-Van Kampen says
	\[
	\pi_1(E_n, y_n) = \langle a_1, \ldots, a_n \mid \rangle.
	\]
	The boundary is then $r_1 r_2 \cdots r_n$, so attaching a 2-cell along it we get a closed surface
	\[
	\pi_1(S_n, y_0) = \langle a_1, a_2, \ldots, a_n \mid a_1^2 a_2^2 \cdots a_n^2 \rangle.
	\]
\end{exbox}

%lecture 13

\newpage

\section{Homology}
\label{sec:hom}

\subsection{Simplicial Complexes}
\label{sub:simp_comp}

\begin{definition}
	A finite set of points $a_0, a_1, \ldots, a_n \in \mathbb{R}^m$ is \emph{affinely independent}\index{affinely independent} if:
	\begin{itemize}
		\item $\sum_{i = 1}^n t_i a_i = 0$ and
		\item $\sum_{i = 1}^n t_i = 0$
	\end{itemize}
	if and only if $(t_1, \ldots, t_n) = 0$.
\end{definition}

\begin{lemma}
	$a_0, \ldots, a_n \in \mathbb{R}^m$ are affinely independent if and only if $a_1 - a_0, \ldots, a_n - a_0$ are linearly independent.
\end{lemma}

\begin{proofbox}
	Let $a_0, \ldots, a_n$ be affinely independent. Suppose that
	\[
	\sum_{i = 0 }^n s_i(a_i - a_0) = 0.
	\]
	This means that
	\[
		\left( - \sum_{i = 1}^n s_i \right)a_0 + s_1 a_1 + \cdots + s_n a_n = 0,
	\]
	and note that the sum of coefficients is $0$, so we get $(s_1, \ldots, s_n) = 0$ by affine independence.

	We can do a similar thing for the converse.
\end{proofbox}

In particular, since we have at most $n$ linearly independent points in $\mathbb{R}^n$, we can have at most $n + 1$ affinely independent points.

\begin{definition}
	If $a_0, \ldots, a_n \in \mathbb{R}^m$ are affinely independent (AI), then they define an $n$-\emph{simplex}\index{simplex}
	\[
		\sigma = \langle a_0, \ldots, a_n \rangle = \left\{ \sum_{i = 1}^n t_i a_i \mid \sum_{i = 0}^n t_i = 1, t_i \geq 0 \right\},
	\]
	given by the convex hull of the points $a_0, \ldots, a_n$. These $a_i$ are called the \emph{vertices}\index{vertex} of $\sigma$, and we say they span $\sigma$.

	If $x \in \langle a_0, \ldots, a_n \rangle$, then $x$ can be written uniquely as
	\[
	x = \sum_{i = 0}^n t_i a_i,
	\]
	for real numbers $t_0, \ldots, t_n$. Call the $t_i$'s the \emph{barycentric coordinates}\index{barycentric coordinates} of $x$.

	A \emph{face}\index{face} of a simplex $\sigma = \langle a_0, \ldots, a_n \rangle$ is a simplex $\tau$ spanned by a subset of $\{a_0, \ldots, a_n\}$. We write $\tau \leq \sigma$. Write $\tau < \sigma$ if $\tau$ is a proper face.

	The \emph{boundary}\index{boundary} of a simplex $\sigma$, written $\partial \sigma$, is the union of all of its proper faces. The \emph{interior}\index{interior} of $\sigma$, $\mathring \sigma$ is $\sigma \setminus \partial \sigma$.
\end{definition}

\begin{lemma}
	Let $\sigma$ be a $p$-simplex in $\mathbb{R}^m$ and $\tau$ be a $p$-simplex in $\mathbb{R}^n$. Then $\sigma$ and $\tau$ are homeomorphic.
\end{lemma}

\begin{proofbox}
	Let $\sigma = \langle a_0, \ldots, a_p \rangle$ and $\tau = \langle b_0, \ldots, b_p \rangle$. Define $h : \sigma \to \tau$ by
	\[
	\sum_{i = 0}^p t_i a_i \mapsto \sum_{i = 0}^p t_i b_i.
	\]
	This is well-defined and a bijection, by uniqueness of barycentric coordinates. As the $a_i - a_0$ are linearly independent, $h$ extends to an affine map $\hat h : \mathbb{R}^m \to \mathbb{R}^n$, so $h$ is continuous. Similarly so is its inverse.
\end{proofbox}

\begin{definition}
	A \emph{geometric} (or Euclidean) \emph{simplicial complex}\index{simplicial complex} in $\mathbb{R}^n$ is a finite set $K$ of simplices in $\mathbb{R}^n$ such that:
	\begin{enumerate}[(i)]
		\item If $\sigma \in K$ and $\tau \leq \sigma$, then $\tau \in K$.
		\item If $\sigma, \tau \in K$, then either $\sigma \cap \tau = \emptyset$ or $\sigma \cap \tau$ is a face of both $\sigma$ and $\tau$.
	\end{enumerate}
\end{definition}

I.e. a simplicial complex is what we get when we are allowed to glue a bunch of simplices along their faces.

The \emph{dimension}\index{dimension} of a simplicial complex $K$ is the largest $p$ such that $K$ contains a $p$-simplex. The \emph{polyhedron}\index{polyhedron} of $K$ is the space
\[
|K| = \bigcup_{\sigma \in K}\sigma \subseteq \mathbb{R}^m.
\]

The $d$-\emph{skeleton}\index{$d$-skeleton} $K_{(d)}$ of $K$ is the sub-simplicial complex containing all simplices of $K$ of dimension at most $d$.

Observe that as we asserted $K$ was finite, and simplices are compact, we get that $|K|$ is compact and Hausdorff.

\begin{definition}
	A \emph{triangulation}\index{triangulation} of a space $X$ is a geometric simplicial complex $K$ and a homeomorphism $h : |K| \to X$.
\end{definition}

\begin{exbox}
	The \emph{standard} $n$-\emph{simplex} is $\Delta^n = \langle e_1, \ldots, e_{n+1}\rangle \subseteq \mathbb{R}^{n+1}$. It, along with its faces, defines a simplicial complex.

	The \emph{simplicial} $n-1$-\emph{sphere} is the simplicial complex given by the proper faces of $\Delta^n$. Its polyhedron is $\partial \Delta^n$.

In $\mathbb{R}^{n+1}$ consider the $2^{n+1}$ simplices given by $\langle \pm e_1, \pm e_2, \ldots, \pm e_{n+1} \rangle$. Let $K$ be given by these and all their faces. Define $h : |K| \subset \mathbb{R}^{n+1} \to S^n$ by $x \mapsto x/|x|$. This is continuous, and a bijection. As $|K|$ and $S^n$ are compact Hausdorff, this is a homeomorphism.
\end{exbox}

\begin{definition}
	Write $V_K$ for the set of vertices of $K$. A \emph{simplicial map}\index{simplicial map} $f$ from $K$ to $L$ is a function $f : V_K \to V_L$ such that if $\sigma = \langle a_0, \ldots, a_n \rangle \in K$, then $\{f(a_0), \ldots, f(a_n)\}$ spans a simplex of $L$, called $f(\sigma)$. We write $f : K \to L$.
\end{definition}

We write $\{f(a_0), \ldots, f(a_n)\}$ instead of saying $f(a_0), \ldots, f(a_n)$ is a simplex to account for the possibility that $f(a_i)$ are not affinely independent, otherwise we could not map a larger dimensional simplex to a smaller one.

\begin{exbox}
	The map $f : \Delta^1 \to \Delta^2$ by $(1, 0) \mapsto (1, 0, 0)$ and $(0, 1) \mapsto (0, 1, 0)$ is a simplicial map.

	Geometrically we are including an edge into the 2-simplex.

	The map $g : \Delta^2 \to \Delta^1$ given by $(1, 0, 0) \mapsto (1, 0)$, $(0, 1, 0) \mapsto (1, 0)$ and $(0, 0, 1) \mapsto (0, 1)$ is a simplicial map.

	Geometrically we are reducing a 2-simplex into an edge.
\end{exbox}

\begin{lemma}
	A map $f : K \to L$ of simplicial complexes induces a continuous map $|f| : |K| \to |L|$, and $|f \circ g| = |f| \circ |g|$.
\end{lemma}

%lecture 14

\begin{proofbox}
	For $\sigma \in K$, where $\sigma = \langle a_0, a_1, \ldots, a_p\rangle$, define $f_\sigma : \sigma \to |L|$ by
	\[
	\sum_{i=  0}^p t_i a_i \mapsto \sum_{i = 0}^p t_i f(a_i),
	\]
	which is linear in the $t_i$, so is continuous. If $\tau \leq \sigma$, then $f_\tau = f_\sigma|_\tau$, so $f_\sigma|_{\sigma \cap \sigma'} = f_{\sigma \cap \sigma'} = f_{\sigma'}|_{\sigma \cap \sigma'}$ for any other $\sigma' \in K$.

	Hence the maps $f_\sigma$ give rise to a continuous 
	\[
	|f| : |K| = \bigcup_{\sigma \in K} \sigma \to |L|,
	\]
	by the gluing lemma. The formula for $|f|$ shows that it behaves as claimed under composition.
\end{proofbox}

We can recover $f$ from $|f|$ and the distinct sets and the discrete sets $V_K \subseteq |K|$, $V_L \subseteq |L|$ i.e. a simplicial map is the same as a continuous map $|K| \to |L|$ which sends vertices to vertices, and is affine on each simplex.

\begin{definition}
	For a $x \in |K|$,
	\begin{enumerate}[(i)]
		\item The (open) \emph{star}\index{star} of $x$ is the union of the interiors of the simplices which contain $x$:
			\[
			\mathrm{St}_K(x) = \bigcup_{\substack{\sigma \in K\\ x \in \sigma}} \mathring \sigma \subseteq \mathbb{R}^n.
			\]
			The complement of $\mathrm{St}_K(x)$ is the union of all simplices which do not contain $x$, a polyhedron, so it is closed. Thus $\mathrm{St}_K(x)$ is open.
		\item The \emph{link}\index{link} of $(x)$, $\mathrm{Lk}_K(x)$, is the union of those simplices which do not contain $x$, but are faces of a simplex which does contain $x$.
	\end{enumerate}
\end{definition}

We can draw the stars and links for points in 2-simplices:

\subsection{Simplicial Approximation}
\label{sub:simp_app}

\begin{definition}
	Let $f : |K| \to |L|$ be a continuous map. A \emph{simplicial approximation}\index{simplicial approximation} to $f$ is a function $g : V_K \to V_L$ such that
	\[
	f(\mathrm{St}_K(v)) \subseteq \mathrm{St}_L(g(v)),
	\]
	for all $v \in V_K$.
\end{definition}

\begin{lemma}
	If $g$ is a simplicial approximation to a continuous map $f$, then $g$ is a simplicial map, and $f$ is homotopic to $|g|$. Furthermore, this homotopy may be taken relative to $\{x \in |K| \mid f(x) = |g|(x)\}$.
\end{lemma}

\begin{proofbox}
	To show that $g$ defines a simplicial map, for $\sigma \in K$ we must show that the images of $g$ of the vertices of $\sigma$ span a simplex in $L$.

	For $x \in \mathring \sigma$, then
	\[
	x \in \bigcap_{v \in V_\sigma} \mathrm{St}_K(v),
	\]
	so we have
	\[
	f(x) \in \bigcap_{v \in V_\sigma}f(\mathrm{St}_K(v)) \subseteq \bigcap_{v \in V_\sigma} \mathrm{St}_L(g(v)).
	\]
	If $\tau$ is the unique simplex of $L$ with $f(x) \in \mathring \tau$, then each $g(v)$ is a vertex of $\tau$. So $\{g(v)\}$ span a face of $\tau$, which is a simplex of $L$.

	Now we want to show that $f \simeq |g|$. If $|L| \subseteq \mathbb{R}^m$, then let $H : |K| \times I \to |L|$ be given by
	\[
		(x, t) \mapsto t f(x) + (1 - t)|g|(x).
	\]
	This is continuous, and so we need to show that it lies in $|L|$. Let $x \in \mathring \sigma \subseteq |K|$, and suppose $f(x) \in \mathring \tau \subseteq |L|$. If $\sigma = \langle a_0, \ldots, a_p\rangle$, then by the above, each $g(a_i)$ is a vertex of $\tau$. Then
	\[
	|g|(x) = \sum_{i = 0}^p t_i g(a_i) \in \tau,
	\]
	as it is a convex linear combination of vertices of $\tau$. As $f(x) \in \tau$, each of $t f(x) + (1-t) |g|(x)$ lies in $\tau$ as well.
\end{proofbox}

\begin{definition}
	The \emph{barycentre}\index{barycentre} of a simplex $\sigma = \langle a_0, a_1, \ldots, a_p \rangle$ is the point
	\[
	\hat \sigma = \frac{1}{p+1}(a_0 + a_1 + \cdots + a_p).
	\]
	The \emph{barycentric subdivision}\index{barycentric subdivision} of a simplicial complex $K$ is
	\[
		K' = \{ \langle \hat{\sigma_0}, \ldots, \hat{\sigma_p} \rangle \mid \sigma_i \in K \text{ and } \sigma_0 < \sigma_1 < \cdots < \sigma_p\}.
	\]
	With this definition, we define $K^{(r)} = (K^{(r-1)})'$.
\end{definition}

It is not obvious that this is a simplicial complex.

\begin{proposition}
	$K'$ is a simplicial complex, and $|K'| = |K|$.
\end{proposition}

\begin{proofbox}
	There are many things to show in this proof. First we need, if $\sigma_0 < \sigma_1 < \cdots < \sigma_p$, then the $\hat{\sigma_i}$ are affinely independent. Indeed, suppose that
	\[
		\sum_{i = 0}^p t_i \hat{\sigma_i} = 0 \text{ and } \sum_{i = 0}^p t_i = 0.
	\]
	Let $j = \max\{i \mid t_j \neq 0\}$. Then
	\[
		\hat \sigma_j = - \sum_{i = 0}^{j-1} \frac{t_i}{t_j} \hat{\sigma_i} \in \sigma_{j-1},
	\]
	so $\hat{\sigma_j}$ lies in a proper face of $\sigma_j$, which is not possible. Thus all $t_i$ must be $0$.

	Next we show that $K'$ is a simplicial complex, which again has many parts. Let $\langle \hat{\sigma_0}, \ldots, \hat{\sigma_p}\rangle \in K'$. A face is given by omitting some of $\hat{\sigma_j}$. But omitting some $\sigma_j$'s from $\sigma_0 < \sigma_1 < \cdots < \sigma_p$ still gives a strictly increasing chain of simplices of $K$.

Now let $\sigma' = \langle \hat{\sigma_0}, \ldots, \hat{\sigma_p}\rangle$ and $\tau' = \langle \hat{\tau_0}, \ldots, \hat{\tau_q} \rangle$, and consider $\sigma' \cap \tau'$. This is inside $\sigma_p \cap \tau_q$, which is a simplex $\delta$ of $K$. So there are simplices
\[
	\sigma'' = \langle \hat{\sigma_0} \cap \delta, \ldots, \hat{\sigma_p} \cap \delta \rangle, \qquad \tau'' = \langle \hat{\tau_0} \cap \delta, \ldots, \hat{\tau_q} \cap \delta\rangle
\]
of $K$. Now $\sigma' \cap \tau'$ = $\sigma'' \cap \tau''$. This reduces us to the case that $\sigma''$ and $\tau''$ are contained in a simplex $\delta$ of $K$.

If $\sigma''$ and $\tau''$ contain $\hat \delta$, let $\bar\sigma'', \bar\tau''$ be the faces of $\sigma'', \tau''$ opposite to $\hat{\delta}$. Then $\sigma'' \cap \tau''$ is spanned by $\hat \delta$, and $\bar\sigma'' \cap \bar\tau'' $. But $\bar\sigma'' \cap \bar\tau'' \subseteq \partial \delta$, which has smaller dimension than $\delta$, so we can suppose it is a simplex of $\partial \delta$ by induction on dimension.

If neither $\sigma''$ nor $\tau''$ contain $\hat \sigma$, then again $\sigma'' \cap \tau'' \subseteq \partial \delta$, so we can finish by induction on dimension.
%lecture 15

Now we show that $|K'| = |K|$. Note that $\langle \hat \sigma_0, \ldots, \hat \sigma_p\rangle \leq \sigma_p \subseteq |K|$, so we have $|K'| \subseteq |K|$. Conversely, if $x \in \sigma = \langle a_0, \ldots, a_p\rangle \subseteq |K|$, then we can write it as
\[
x = \sum_{i = 0}^p t_i a_i.
\]
We can reorder the $a_i$ so that $t_0 \geq t_1 \geq \cdots \geq t_p$, so
\begin{align*}
	x &= (t_0 - t_1) a_0 + 2(t_1 - t_2) \left( \frac{a_0 + a_1}{2} \right) + 3 (t_2 - t_1) \left( \frac{a_1 + a_2 + a_3}{3} \right) + \cdots \\
	  &= (t_0 - t_1) \widehat{\langle a_0\rangle} + 2 (t_1 - t-2) \widehat{\langle a_0, a_1 \rangle} + 3 (t_2 - t_3) \widehat{\langle a_0, a_1, a_2 \rangle} + \cdots \\
	  &\in \langle \widehat{\langle a_0 \rangle}, \widehat{\langle a_0, a_1 \rangle}, \ldots, \widehat{ \langle a_0, \ldots, a_p \rangle} \rangle \subseteq |K'|.
\end{align*}
\end{proofbox}

The vertices of $K'$ are in bijection with the simplices of $K$. Choose a function $K \to V_K$ which assigns to $\sigma$ some vertex of $\sigma$. Then we can take $g : V_{K'} \cong K \to V_K$.

If $\langle \hat \sigma_0, \ldots, \hat \sigma_p \rangle$ is a simplex of $K'$, then $\sigma_0 \leq \cdots \leq \sigma_p$, and $g(\hat \sigma_i)$ is some vertex of $\sigma_i$, so is a vertex of $\sigma_p$. Thus $\{g(\hat \sigma_i)\}$ spans a face of $\sigma$, so is a simplex of $K$. So $g$ is a simplicial map.

Also, if $\hat \sigma \in \tau' = \langle \hat \tau_0, \ldots, \hat \tau_p \rangle \in K'$, then $\hat \sigma \in \tau_p$, so $\sigma$ is a face of $\tau_p$. Thus $v_\sigma \in \sigma \subseteq \tau_p$. So
\[
\mathring \tau' \subseteq \mathring \tau_p \subseteq \mathrm{St}_K(v_\sigma).
\]
Thus $\mathrm{St}_{K'}(\hat \sigma) \subseteq \mathrm{St}_K(v_\sigma) = \mathrm{St}_K(g(\hat \sigma))$, so $g$ is a simplicial approximation to $\id : |K'| \to |K|$, so $|g| \simeq \id$.

\begin{definition}
	The \emph{mesh}\index{mesh} of $K$ is
	\[
		\mu(K) = \max\{|v_0 - v_1| \mid \langle v_0, v_1 \rangle \in K\}.
	\]
\end{definition}

\begin{lemma}
	Suppose $K$ has dimension at most $n$. Then $\mu(K^{(r)}) \leq (\frac{n}{n+1})^r \mu(K)$, so $\mu(K^{(r)}) \to 0$ as $r \to \infty$.
\end{lemma}

\begin{proofbox}
	It is enough to prove the case $r = 1$. Let $\langle \hat \tau, \hat \sigma \rangle \in K'$, so $\tau \leq \sigma$ in $K$. Then
	\[
		|\hat \tau - \hat \sigma| \leq \max\{ |v - \hat \sigma| \mid v \text{ a vertex of } \sigma\}.
	\]
	Let $\sigma = \langle v_0, \ldots, v_m \rangle$ with $m \leq n$, and reorder so that the maximum is attained at $v = v_0$. Then
	\begin{align*}
		|v_0 - \hat \sigma| &= \left| v_0 - \frac{1}{m+1} \sum_{i = 0}^m v_i \right| = \left| \frac{m+1}{m+1} v_0 - \frac{1}{m+1} \sum_{i = 0}^m v_i \right| \\
				    &= \frac{1}{m+1} \left| \sum_{i = 0}^m v_0 - v_i \right| \leq \frac{1}{m+1} \sum_{i = 1}^m |v_0 - v_i| \leq \frac{m}{m+1} \mu(K) \\
				    &\leq \frac{n}{n+1} \mu(K).
	\end{align*}
\end{proofbox}

\begin{theorem}[Simplicial Approximation Theorem]
	Let $f : |K| \to |L|$ be a continuous map. Then there is a $r \gg 0$ and a simplicial map $g : K^{(r)} \to L$ such that $g$ is a simplicial approximation to $f$.

	If $f$ is simplicial on some $|N| \subset |K|$, we can take $g|_{V_N} = f|_{V_N}$.
\end{theorem}

\begin{proofbox}
	The $\mathrm{St}_L(w)$, for $w \in V_L$, is an open cover of $|L|$, so $\{f^{-1} \mathrm{St}_K(w)\}$ is an open cover of $|K|$; let $\delta > 0$ be the Lebesgue number for this cover. Choose $r \gg 0$ such that $\mu(K^{(r)}) < \delta$. For each $v \in V_{K^{(r)}}$, we have
	\[
	\mathrm{St}_{K^{(r)}}(v) \subseteq B_{\mu(K^{(r)})}(v) \subseteq f^{-1} \mathrm{St}_L(w),
\]
	for some $w \in V_L$. Define $g : V_{K^{(r)}} \to V_L$ by $g(v) = w$. Then
	\[
	f(\mathrm{St}_{K^{(r)}}(v)) \subseteq \mathrm{St}_L(g(v)),
	\]
	so $g$ is a simplicial approximation to $f$. So $g$ is a simplicial map. The final step is by choosing $w$ carefully when $v \in V_N$.
\end{proofbox}

\begin{corollary}
	If $n < m$, then any map $f : S^n \to S^m$ is homotopic to a constant map.
\end{corollary}

\begin{proofbox}
	Spheres are polyhedra. Take $S^n = |K|$, and $S^m = |L|$, then $f$ is homotopic to $g$ for some $g : K^{(r)} \to L$. This cannot hit any $m$-simplex of $L$, as $K$ has dimension at most $n$. So $|g|$ must miss every point on the interior of some $m$-simplex it is not onto.

	Thus, it factors through $(S^{m} - \{\ast\}) \simeq \ast$, so is homotopic to a constant map.
\end{proofbox}

%lecture 16

\subsection{Simplicial Homology}
\label{sub:simp_hom}

\begin{definition}
	Let $K$ be a simplicial complex, and $\mathcal{O}_n(K)$ be the free abelian group with basis
	\[
		\{[v_0, v_1, \ldots, v_n] \mid v_i \text{ vertices of $K$ which span a simplex}\}.
	\]
	The $v_i$ are considered to be ordered, and could span a simplex of $\dim< n$, i.e. we could have repeats.

	Let $T_n(K) \leq \mathcal{O}_n(K)$ be the subgroup spanned by:
	\begin{enumerate}[(i)]
		\item $[v_0, \ldots, v_n]$ containing a repeat.
		\item $[v_0, \ldots, v_n] - \sgn(\sigma) [v_{\sigma(0)}, \ldots, v_{\sigma(n)}]$ for a permutation $\sigma$ of $\{0, 1, \ldots, n\}$.
	\end{enumerate}
	Then we can define
	\[
	C_n(K) = \frac{\mathcal{O}_n(K)}{T_n(K)},
	\]
	the quotient group.
\end{definition}

\begin{lemma}
	There is a non-canonical isomorphism $C_n(K) \cong \mathbb{Z}\{n\text{-simplices of } K\}$.
\end{lemma}

\begin{proofbox}
	Choose a total order $\prec$ of $V_K$. Then each $n$-simplex of $K$ defines a canonical ordered simplex $[\sigma] \in \mathcal{O}_n(K)$ by ordering its vertices such that $a_0 \prec a_1 \prec \cdots \prec a_n$. This gives a map $\phi : Z\{n\text{-simplices of } K\} \to \mathcal{O}_n(K)$, by $\sigma \mapsto [\sigma]$, and hence factors through $C_n(K)$.

	For each $[a_0, \ldots, a_n] \in \mathcal{O}_n(K)$, there is a unique permutation $\tau$ of $\{0, 1, \ldots, n\}$ such that $a_{\tau(0)} \prec \cdots \prec a_{\tau(n)}$, and let $\sgn[a_0, \ldots, a_n] = \sgn(\tau)$. Define $\rho : \mathcal{O}_n(K) \to \mathbb{Z}\{n\text{-simplices of } K\}$ by
	\[
		[a_0, \ldots, a_n] \mapsto
		\begin{cases}
			\sgn[a_0, \ldots, a_n] \langle a_0, \ldots, a_n \rangle & \text{no repeats}, \\
			0 & \text{repeats}.
		\end{cases}
	\]
	For this to descend to $C_n(K)$, we need $T_n(K)$ to be in the kernel. Clearly $[a_0, \ldots, a_n]$ with repeats are in the kernel, and 
	\begin{align*}
		\rho([v_0, \ldots, v_n] - \sgn(\sigma) [v_{\sigma(0)}, &\ldots, v_{\sigma(n)}]) = \sgn[v_0, \ldots, v_n] \langle v_0, \ldots, v_n \rangle \\
											       &- \sgn(\sigma) \sgn[v_{\sigma(0)}, \ldots, v_{\sigma(n)} ] \langle v_{\sigma(0)}, \ldots, v_{\sigma(n)}\rangle \\
											       &\qquad \qquad= 0.
	\end{align*}
	So we get $\rho' : C_n(K) \to \mathbb{Z}\{n\text{-simplices of } K\}$. Now $\rho' = \phi'(\sigma) = \sigma$. If $[a_0, \ldots, a_n]$ has no repeats, then
	\begin{align*}
		\phi' \circ \rho'([a_0, \ldots, a_n]) &= \phi'(\sgn[a_0, \ldots, a_n] \langle a_0, \ldots, a_n\rangle) \\
						      &= \sgn[a_0, \ldots, a_n] \sgn[a_{\tau(0)}, \ldots, a_{\tau(1)}] \\
						      &= [a_0, \ldots, a_n],
	\end{align*}
	modulo $T_n(K)$, so $\phi'$ and $\rho'$ are the same.
\end{proofbox}

Define a homomorphism
\begin{align*}
	d_n : \mathcal{O}_n(K) &\to \mathcal{O}_{n-1}(K) \\
	[v_0, \ldots, v_n] &\mapsto \sum_{i = 0}^n (-1)^i [v_0, \ldots, \hat v_i, \ldots, v_n].
\end{align*}

\begin{lemma}
	$d_n$ sends $T_n(K)$ to $T_{n-1}(K)$.
\end{lemma}

\begin{proofbox}
	Note
	\begin{align*}
		d_n([v_0, \ldots, v_n] - \sgn(\sigma) &[v_{\sigma(0)}, \ldots, v_{\sigma(n)}]) \\
		&= \sum_{i = 0}^n (-1)^{i}[v_0, \ldots, \hat v_i, \ldots, v_n] \\
											      &- \sum_{i = 0}^n (-1)^i \sgn(\sigma) [v_{\sigma(0)}, \ldots, \hat v_{\sigma(i)}, \ldots, v_{\sigma(n)}].
	\end{align*}
	We need to show that this is trivial in $\mathcal{O}_{n-1}(K) / T_{n-1}(K)$. Suppose first $\sigma=  (j, j+1)$, so $\sgn(\sigma) = -1$. Then
	\begin{align*}
		\sum_{i = 0}^n(-1)^i &\sgn(\sigma) [v_{\sigma(0)}, \ldots, \hat v_{\sigma(i)}, \ldots, v_{\sigma(n)}] \\
		&= \sum_{i = 0}^{n} (-1)^{j-1} [v_0, \ldots, \hat v_i, \ldots, v_{j+1}, v_j, v_{j+2}, \ldots, v_n] \\
														     & \qquad + (-1)^{j+1}[v_0, \ldots, v_{j-1}, v_{j}, v_{j+1}, \ldots, v_n] \\
														     &\qquad + (-1)^{j+1} [v_0, \ldots, v_{j-1}, v_{j+1}, v_{j+2}, v_j, \ldots, v_n] \\
														     &+ \sum_{i = j+2}^n (-1)^{j+1} [v_0, \ldots, v_{j+1}, v_j, v_{j+2}, \ldots, \hat v_i, \ldots, v_n].
	\end{align*}
	In the first sum
	\begin{align*}
		[v_0, \ldots, \hat v_i, \ldots, v_{j-1}, v_{j+1}, v_j, \ldots, v_{n}] &= - [v_0, \ldots, \hat v_i, \ldots, v_n] \mod T_{n-1}(K).
	\end{align*}
	In the second sum, we get a similar thing. So the right hand side is equal to
	\[
		\sum_{i = 0}^n(-1)^i [v_0, \ldots, \hat v_i, \ldots, v_n] \mod T_{n-1}(K),
	\]
	as required. Since any $\sigma$ is a product of $(j, j+1)$, we get the same for any $\sigma$.

	Now suppose $[v_0, \ldots, v_n]$ is such that $v_j = v_{j+1}$, with permutations. Hence writing out $d_{n}[v_0, \ldots, v_n]$, notice the summations are in $T_{n-1}(K)$ as they have a repeat, and the other two terms cancel.
\end{proofbox}

So $d_n$ induces a homomorphism
\[
d_n : C_n(K) \to C_{n-1}(K),
\]
given by the same function.

\begin{lemma}
	The composition $d_{n-1} \circ d_n : C_n (K) \to C_{n-2}(K)$ is zero.
\end{lemma}

\begin{proofbox}
	At the level of $\mathcal{O}_n(K)$, compute
	\begin{align*}
		d_{n-1} \circ d_n [v_0, \ldots, v_n] &= d_{n-1} \left( \sum_{i = 0}^n (-1)^i [v_0, \ldots, \hat v_i, \ldots, v_n ] \right) \\
						     &= \sum_{i = 0}^n (-1)^i \Biggl[ \sum_{k = 0}^{i-1}(-1)^k [v_0, \ldots, \hat v_k, \ldots, \hat v_i, \ldots, v_n] \\
						     &+ \sum_{k = i}^{n-1} (-1)^k [v_0, \ldots, \hat v_i, \ldots, \hat v_{k+1}, \ldots, v_n] \Biggr].
	\end{align*}
	Then the coefficient of $[v_0, \ldots, \hat v_a, \ldots, \hat v_b, \ldots, v_n]$ is $(-1)^a (-1)^b + (-1)^a (-1)^{b+1} = 0$. As $[v_0, \ldots, v_n]$ generates, we get $d_{n-1} \circ d_n =0$.
\end{proofbox}

%lecture 17

This means that
\[
\Img (d_n : C_n(K) \to C_{n-1}(K)) \subseteq \Ker (d_{n-1} : C_{n-1}(K) \to C_{n-2}(K)).
\]

\begin{definition}
	The \emph{n'th simplicial homology group} of $K$ \index{simplicial homology group} is
	\[
	H_n(K) = \frac{\Ker(d_n : C_n(K) \to C_{n-1}(K))}{\Img(d_{n+1} : C_{n+1}(K) \to C_n(K))}.
	\]
\end{definition}

\begin{exbox}
	Let $K$ be the union of all the proper faces of the standard $2$-simplex $\Delta^2 \subseteq \mathbb{R}^3$, i.e.
	\[
		K = \{\langle e_1 \rangle , \langle e_2 \rangle , \langle e_3 \rangle, \langle e_1, e_2 \rangle, \langle e_2, e_3\rangle, \langle e_3, e_1\rangle\}.
	\]
	Order the vertices as $e_1 \prec e_2 \prec e_3$. Then
	\begin{align*}
		C_0(K) &= \mathbb{Z}\{[e_1], [e_2], [e_3]\}, \\
		C_1(K) &= \mathbb{Z}\{[e_1, e_2], [e_2, e_3], [e_1, e_3]\},
	\end{align*}
	and $C_n(K) = 0$ for $n \geq 2$. Now note that $d_1 : C_1(K) \to C_0(K)$ is given by $d_1[e_i, e_j] = [e_j] - [e_i]$, so in matrix form it is
	\[
	\begin{pmatrix}
		-1 & 0 & -1 \\
		1 & -1 & 0 \\
		0 & 1 & 1
	\end{pmatrix}.
	\]
	Note that $\Img d_1 = \langle [e_i] - [e_j] \rangle_{\mathbb{Z}}$, so
	\[
		H_0(K) = \frac{\mathbb{Z}\{[e_1], [e_2], [e_3]\}}{\langle[e_i] - [e_j]\rangle} \cong \mathbb{Z}.
	\]
	We have that $\Ker d_1 = \mathbb{Z}\{[e_1, e_2] - [e_1, e_3] + [e_2, e_3]\} \cong \mathbb{Z}$, and $\Img d_2 = 0$, so again $H_1(K) \cong \mathbb{Z}$.
\end{exbox}

\begin{exbox}
	Now let $L$ be the standard $2$-simplex $\Delta^2 \subseteq \mathbb{R}^3$, i.e.
	\[
		L = \{\langle e_1 \rangle , \langle e_2 \rangle , \langle e_3 \rangle, \langle e_1, e_2 \rangle, \langle e_2, e_3\rangle, \langle e_3, e_1\rangle, \langle e_1, e_2, e_3 \rangle\}.
	\]
	Then,
	\begin{align*}
		C_0(L) &= \mathbb{Z}\{[e_1], [e_2], [e_3]\}, \\
		C_1(L) &= \mathbb{Z}\{[e_1, e_2], [e_2, e_3], [e_1, e_3]\}, \\
		C_2(L) &= \mathbb{Z}\{[e_1, e_2, e_3]\}.
	\end{align*}
	So,
	\[
	\begin{tikzcd}[column sep = small]
		0 \arrow[r, "d_4"] & 0 \arrow[r, "d_3"] & C_2(K) \cong \mathbb{Z} \arrow[r, "d_2"] & C_1(K) \cong \mathbb{Z}^3 \arrow[r, "d_1"] & C_0(K) \cong \mathbb{Z}^3 \arrow[r] & 0.
	\end{tikzcd}
	\]
	Now $d_2[e_1, e_2, e_3] = [e_1, e_2] - [e_1, e_3] + [e_1, e_2] \neq 0$, so $d_2$ is injective. Hence $H_2(L) = 0$. But also,
	\begin{align*}
		H_1(L) &= \frac{\Ker d_1}{\Img d_2} = 0, \\
		H_0(L) &= \mathbb{Z}.
	\end{align*}
\end{exbox}

\subsection{Some Homological Algebra}
\label{sub:hom_alg}

\begin{definition}
	A \emph{chain complex}\index{chain complex} is a sequence $C_0, C_1, C_2, \ldots$ of abelian groups, and homomorphisms $d_n : C_n \to C_{n-1}$ such that
	\[
	d_{n-1} \circ d_n = 0,
	\]
	for all $n$. We write this data as $C_\bullet$, and call $d_n$ the \emph{differentials}\index{differential} of $C_\bullet$. Then define
	\[
	H_n(C_\bullet) = \frac{\Ker d_n}{\Img d_{n+1}}.
	\]
	Write $Z_n(C_\bullet) = \Ker d_n$, and $B_n(C_\bullet) = \Img d_{n+1}$. We can think of $Z_n$ as the $n$-cycles of $C_\bullet$, and $B_n$ as the $n$-boundaries of $C_\bullet$.

	A \emph{chain map}\index{chain map} $f_\bullet : C_\bullet \to D_\bullet$ is a sequence of homomorphisms $f_n : C_n \to D_n$ such that
	\[
	\begin{tikzcd}
		C_{n+1} \arrow[r, "f_{n+1}"] \arrow[d, "d_{n+1}"] & D_{n+1} \arrow[d, "d_{n+1}"] \\
		C_n \arrow[r, "f_n"] & D_n
	\end{tikzcd}
	\]
	commutes, i.e. $f_n \circ d_{n+1} = d_{n+1} \circ f_{n+1}$.
\end{definition}

A \emph{chain homotopy}\index{chain homotopy} between $f_\bullet, g_\bullet : C_\bullet \to D_\bullet$ is a sequence of homomorphisms $h_n : C_n \to D_{n+1}$ such that
\[
g_n - f_n = d_{n+1} \circ h_n + h_{n-1} \circ d_n.
\]
Just accept this.

\begin{lemma}
	A chain map $f_\bullet : C_n \to D_n$ induces a homomorphism
	\[
	f_\ast : H_n(C_\bullet) \to H_n(D_\bullet)
	\]
	by $[x] \mapsto [f_n(x)]$. Furthermore, if $g_\bullet$ is chain homotopic to $f_\bullet$, then $g_\ast = f_\ast$.
\end{lemma}

\begin{proofbox}
	We need to show that $f_\ast$ is well-defined.
	\begin{enumerate}[(i)]
		\item Let $[x] \in H_n(C_\bullet)$, i.e. $x \in C_n$ and $d_n(x) =0$. Then
			\[
				d_n f_n (x) = f_{n-1} d_n(x) = 0,
			\]
			so $f_n (x) \in Z_n(D_\ast)$.
		\item If $[x] = [y] \in H_n(C_\bullet)$, then $x - y \in B_n(C_\ast)$, so say $x  - y = d_{n+1}(z)$. Then
			\[
			f_n(x) - f_n(y) = f_n d_{n+1}(z) = d_{n+1}f_{n+1}(z) \in B_n(D_\bullet),
			\]
			so $[f_n(x)] = [f_n(y)]$, and so $f_\ast$ is well-defined, and so a homomorphism.
	\end{enumerate}
	Now let $g_\bullet$ be chain homotopic to $f_\bullet$, i.e.
	\[
	g_n - f_n = d_{n+1} \circ h_n + h_{n-1} \circ d_n.
	\]
	Let $x \in Z_n(C_\bullet)$, so
	\[
	g_n(x) - f_n(x) = d_{n+1} h_n(x) + h_{n-1}d_n(x) \in B_n(D_\bullet),
	\]
	so $[g_n(x)] = [f_n(x)]$.
\end{proofbox}

%lecture 18

Just as we did for homotopies of maps between spaces, one can show:
\begin{enumerate}[(i)]
	\item Being chain homotopic defines an equivalence relation on the set of chain maps from $C_\bullet$ to $D_\bullet$. We write $f_\bullet \simeq g_\bullet$.
	\item If $a_\bullet : A_\bullet \to C_\bullet$ is a chain map and $f_\bullet \simeq g_\bullet : C_\bullet \to D_\bullet$, then $f_\bullet \circ a_\bullet \simeq g_\bullet \circ a_\bullet$, and similarly with postcomposing.
\end{enumerate}

\begin{lemma}
	If $f_\bullet : C_\bullet \to D_\bullet$ is a chain homotopy equivalence, then $f_\ast : H_n(C_\bullet) \to H_n(D_\bullet)$ is an isomorphism.
\end{lemma}

\begin{proofbox}
	Using a homotopy inverse $g_\bullet$, we have
	\begin{align*}
		f_\ast \circ g_\ast &= (f_\bullet \circ g_\bullet)_\ast = (\id_{D_\bullet})_\ast = \id_{H_n(D_\bullet)},
	\end{align*}
	and similarly $g_\ast \circ f_\ast = \id_{H_n(C_\bullet)}$.
\end{proofbox}

Bonus exercise: let
\[
	\mathbb{Z}[n] = (\to 0 \overset{d_{n+1}} \to \mathbb{Z} \overset{d_n} \to 0 \overset{d_{n-1}} \to 0 \cdots).
\]
Then describe the set of chain maps $\{\mathbb{Z}[n] \to C_\bullet\}$, up to chain homotopy.

\subsection{Elementary Calculations}
\label{sub:el_cal}

Return to the chain complexes $C_\bullet(K)$ associated to a simplicial complex $K$.

\begin{lemma}
	Let $f : K \to L$ be a simplicial map. Then the formula
	\begin{align*}
		f_n : C_n(K) &\to C_n(L) \\
		[a_0, \ldots, a_n] &\mapsto [f(a_0), \ldots, f(a_n)]
	\end{align*}
	is a well-defined homomorphism, and defines a chain map $f_\bullet : C_\bullet(K) \to C_\bullet(L)$, and hence gives us $f_\ast : H_n(K) \to H_n(L)$.
\end{lemma}

\begin{proofbox}
	To be well-defined, we need the given formula to send $T_n(K)$ to $T_n(L)$. It does. To be a chain map, we check
	\begin{align*}
		f_{n-1}d_n [a_0, \ldots, a_n] &= f_{n-1} \left( \sum_{i = 0}^n (-1)^i [a_0, \ldots, \hat a_i, \ldots, a_n] \right) \\
					      &= \sum_{i = 0}^n (-1)^i [f(a_0), \ldots, \hat{f(a_i)}, \ldots, f(a_n)] \\
					      &= d_n f_n[a_0, \ldots, a_n].
	\end{align*}
\end{proofbox}

\begin{definition}
	Say a simplicial complex $K$ is a \emph{cone}\index{cone} with a \emph{cone point}\index{cone point}  $v_0 \in V_K$, if every simplex of $K$ is a face of a simplex which has $v_0$ as a vertex.
\end{definition}

\begin{proposition}
	If $K$ is a cone with cone point $v_0$, then the inclusion $i : \{v_0\} \injto K$ induces a chain homotopy equivalence $i_\bullet : C_\bullet(\{v_0\}) \to C_\bullet(K)$, and so
	\[
	H_n(K) \cong
	\begin{cases}
		\mathbb{Z}\{[v_0]\} & n = 0, \\
		0 & \text{else}.
	\end{cases}
	\]
\end{proposition}

\begin{proofbox}
	The only map $r : V_K \to \{v_0\}$ is the simplicial map $r : K \to \{v_0\}$, and $r \circ i = \id_{\{v_0\}}$. We show that $i \circ r \simeq \id_{C_\bullet(K)}$. Define $h_n : \mathcal{O}_n(K) \to \mathcal{O}_{n+1}(K)$ by
	\[
		[a_0, \ldots, a_n] \mapsto [v_0, a_0, \ldots, a_n].
	\]
	This is valid as every simplex is part of a face which contains $v_0$. Moreover, this map sends $T_n(K)$ into $T_{n+1}(K)$, so it descends to $h_n : C_n(K) \to C_{n+1}(K)$. For $n > 0$, then
	\begin{align*}
		(h_{n-1} \circ d_n + d_{n+1} \circ h_n)[a_0, \ldots, a_n] &= \Biggl( \sum_{i = 0}^n (-1)^i [v_0, a_0, \ldots, \hat a_i, \ldots, a_n] \Biggr) \\
	+ \Biggl( [a_0, \ldots, & a_n] + \sum_{i = 0}^n (-1)^{i+1}[v_0, a_0, \ldots, \hat a_i, \ldots, a_n ]\Biggr) \\
									  &= [a_0, \ldots, a_n] = (\id - i_n \circ r_n) [a_0, \ldots, a_n],
	\end{align*}
	as the last term is $[v_0, \ldots, v_0] = 0$ as $n > 0$. For $n = 0$,
	\begin{align*}
		(h_{-1} \circ d_0 + d_1 \circ h_0) [a_0] &= d_1 [v_0, a_0] = [a_0] - [v_0] \\
							 &= (\id - i_0 \circ r_0) [a_0].
	\end{align*}
	So $h$ provides a chain homotopy from $\id_{C_\bullet(K)}$ to $i_\bullet \circ r_\bullet$.
\end{proofbox}

\begin{corollary}
	The standard $n$-simplex $\Delta^n \subseteq \mathbb{R}^{n+1}$ and all its faces is a cone, with any vertex as a cone point. So
	\[
	H_i(K) \cong
	\begin{cases}
		\mathbb{Z} & i = 0, \\
		0 & \text{else}.
	\end{cases}
	\]
\end{corollary}

\begin{corollary}
	Let $K$ be the union of all the proper faces of $\Delta^n \subseteq \mathbb{R}^n$. Then for $n \geq 2$, we have
	\[
		H_i(K) =
		\begin{cases}
			\mathbb{Z} & i = 0, \\
			\mathbb{Z} & i = n-1, \\
			0 & \text{else}.
		\end{cases}
	\]
\end{corollary}

\begin{proofbox}
	Note that $K$ is the $(n-1)$-skeleton of $L$, so $i : K \injto L$ gives us an isomorphism $C_i(K) \to C_i(L)$ for $i \leq n-1$. These chain complexes are
	\[
	\begin{tikzcd}
		C_0(L) & C_1(L) \arrow[l, "d_0^L"] & \cdots \arrow[l, "d_1^L"] & C_{n-1}(L) \arrow[l, "d_{n-1}^L"] & C_n(L) \arrow[l, "d_n^L"] & 0 \arrow[l] \\
		C_0(K) \arrow[u, equal] & C_1(K) \arrow[l, "d_0^L"] \arrow[u, equal] & \cdots \arrow[l, "d_1^K"] & C_{n-1}(K) \arrow[l, "d_{n-1}^K"] \arrow[u, equal] & 0 \arrow[l] \arrow[u] & 0 \arrow[l]
	\end{tikzcd}
	\]
	For $i \leq n-2$, $H_i(K) = H_i(L)$. For $i > n-1$, $H_i(K) = 0$ as there are no $i$-simplices. Hence
	\[
	H_{n-1}(K) = \frac{\Ker(d_{n-1}^K)}{\Img(d_n^K)} = \Ker(d_{n-1}^K) = \Ker(d_{n-1}^L).
	\]
	As $H_{n-1}(L) = 0$, $\Ker(d_{n-1}^L)/\Img(d_{n}^L) = 0$, so they are equal. As $H_n(L) = 0$, we see that $d_n^L : \mathbb{Z} = C_n(L) \to C_{n-1}(L)$ is injective, so $H_{n-1}(K) = \Ker(d_{n-1}^L) = \Img(d_n^L) \cong \mathbb{Z}$, as it is generated by $d_n^L [e_0, \ldots, e_n]$.
\end{proofbox}
%lecture 19

\begin{lemma}
	There is an isomorphism $H_0(K) \cong \mathbb{Z}\{\pi_0(|K|)\}$.
\end{lemma}


\begin{proofbox}
	Note we have a homomorphism $\phi : C_0(K) \to \mathbb{Z}\{\pi_0|K|\}$ by $[v] \mapsto$ the path component of $v \in |K|$.

	This is onto, as any path-component of $|K|$ contains a vertex.

	If $[v, w]$ is an ordered $1$-simplex, then $d_1[v, w] = [w] - [v]$. But $[v]$ and $[w]$ live in the same path-component, as the $1$-simplex $\langle v, w \rangle$ goes between them. So $\Img(d_1) \subseteq \Ker \phi$, hence we get an induced $\phi : H_0(K) \to \mathbb{Z}\{\pi_0|K|\}$.

	If $\phi([v]) = \phi([w])$, then choose a path $\gamma : I \to |K|$ from $v$ to $w$. By simplicial approximation, $I = \Delta^1$ can be subdivided so that there is is $g : (\Delta^1)^{(r)} \to K$, with $|g| \simeq \gamma$, i.e. there are $1$-simplices $[v, v_1], [v_1, v_2], \ldots, [v_n, w]$. Then
	\[
		[w] - [v] = d_1([v, v_1] + [v_1, v_2] + \cdots + [v_n, w]),
	\]
	hence $[v] - [w] \in H_0(K)$.
\end{proofbox}

\subsection{Mayer-Vietoris Theorem}
\label{sub:mvt}

\begin{definition}
	Say that a pair of homomorphisms
	\[
	\begin{tikzcd}
		A \arrow[r, "f"] & B \arrow[r, "g"] & C
	\end{tikzcd}
	\]
	is \emph{exact}\index{exact} if $\Img f = \Ker g$. More generally, a collection of homomorphisms
	 \[
	\begin{tikzcd}
		\cdots \arrow[r] & A_i \arrow[r] & A_{i+1} \arrow[r] & A_{i+2} \arrow[r] & A_{i+3} \arrow[r] & \cdots
	\end{tikzcd}
	\]
	is \emph{exact} if it is exact at each $A_j$, where $A_j$ has a homomorphism in and out.

	A \emph{short exact sequence}\index{short exact sequence} is an exact sequence
	\[
	\begin{tikzcd}
		0 \arrow[r] & A \arrow[r, "f"] & B \arrow[r, "g"] & C \arrow[r] & 0,
	\end{tikzcd}
	\]
	so $f$ is injective, $\Img f = \Ker g$ and $g$ is surjective. The chain maps $i_\bullet : A_\bullet \to B_\bullet$ and  $j_\bullet : B_\bullet \to C_\bullet$ form a \emph{short exact sequence} of chain complexes if each
	\[
	\begin{tikzcd}
		0 \arrow[r] & A_n \arrow[r, "i_n"] & B_n \arrow[r, "j_n"] & C_n \arrow[r] & 0
	\end{tikzcd}
	\]
	is a short exact sequence.
\end{definition}

\begin{theorem}[Zig-zag Lemma]
	If
	\[
	\begin{tikzcd}
		0 \arrow[r] & A_\bullet \arrow[r, "i_\bullet"] & B_\bullet \arrow[r, "j_\bullet"] & C_\bullet \arrow[r] & 0
	\end{tikzcd}
	\]
	is a short exact sequence of chain complexes, then there are natural homomorphisms $\partial_\ast : H_n(C_\bullet) \to H_{n-1}(A_\bullet)$ such that
	\begin{equation*}
	\begin{tikzcd}
		& \vdots \arrow[dl] & \\
		 H_n(A_\bullet) \arrow[r, "i_\ast"'] & H_n(B_\bullet) \arrow[r, "j_\ast"] & H_n(C_\bullet) \arrow[dll, "\partial_\ast"] \\
						    H_{n-1}(A_\bullet) \arrow[r, "i_\ast"'] & H_{n-1}(B_\bullet) \arrow[r, "j_\ast"] & H_{n-1}(C_\bullet) \arrow[dl] \\
						    & \vdots &
	\end{tikzcd}
	\end{equation*}
	is an exact sequence.
\end{theorem}

\begin{proofbox}
	We begin by constructing $\partial_\ast$, using the snake lemma. Look at this diagram:
	\begin{equation*}
	\begin{tikzcd}
		0 \arrow[r] & A_n \arrow[r, "i_n"] \arrow[d, "d_n"] & B_n \arrow[r, "j_n"] \arrow[d, "d_n"] & C_n \arrow[d, "d_n"] \arrow[r] & 0 \\
		0 \arrow[r] & A_{n-1} \arrow[r, "i_{n-1}"] & B_{n-1} \arrow[r, "j_{n-1}"] & C_{n-1} \arrow[r] & 0 \\
	\end{tikzcd}
	\end{equation*}
	Pretty cool right. We are now going to chase this diagram. Let $[x] \in H_n(C_\bullet)$. Then $d_n x = 0$. By exactness of the top row, we can choose $y \mapsto x$.

	Applying $d_n y$ gives us an element of $B_{n-1}$. But by commutativity, $j_{n-1}(d_n y) = 0$. But by exactness, this means there is $z \in A_{n-1}$ that maps to $d_{n} y$.

	What can we say about $z$? Well $i_{n-2} d_{n-1}(z) = d_{n-1} i_{n-1}(z) = d_{n-1} d_n (y) = 0$. But as $i_{n-2}$ is injective, this shows $d_{n-1}(z) = 0$. Hence $z$ is a cycle, and so we can let $\partial_\ast[x] = [z]$.

	Now we check $\partial_\ast$ is well-defined. Suppose $[x] = [x'] \in H_n(C_\bullet)$. Then $x - x' = d_{n+1}(a)$ for some $a \in C_{n+1}$. The same process for $x'$ gives a $y' \in B_n$, and as $j_{n+1}$ is surjective, we can write $a = j_{n+1}(b)$. Then
	\[
	j_n(y - y') = x- x' =  d_{n+1}j_{n+1}(b) = j_n d_{n+1}(b),
	\]
	so by exactness at $B_n$,
	\[
	y - y' = d_{n+1}(b) + i_n(c),
	\]
	for some $c \in A_n$. Now $z'$ is such that $i_{n-1}(z') = d_n(y')$. Then
	\begin{align*}
		i_{n-1}(z - z') &= d_n(y - y') = d_n(d_{n+1}(b) + i_n(c)) \\
				&= d_n i_n(c) = i_{n-1} d_n(c).
	\end{align*}
	The injectivity of $i_{n-1}$ shows that $z - z' \in d_n(c)$, so $[z] = [z'] \in H_{n-1}(A_\bullet)$.

	Now we show that $\partial_\ast$ is a homomorphism. Given $[x_1], [x_2] \in H_n(C_\bullet)$ with components $y_1, y_2, z_1, z_2$, we choose $y_1 + y_2$ to be the lift of $x_1 + x_2$. This gives $z_1 + z_2$ as the result, so
	\[
		\partial_\ast[x_1 + x_2] = [z_1 + z_2] = [z_1] + [z_2].
	\]
	This shows $\partial_\ast$ exists. Now we show the sequence we defined is exact. First we show it at $H_n(C_\bullet)$. Let $[x] \in \Img(j_\ast)$, so there exists $y \in B_n$ such that $j_n(y) = x$ and $y$ is a cycle.

	We can use this $y$ to calculate $\partial_\ast[x]$. As $y$ is a cycle, $d_n(y) = 0$, so $z$ is $0$, hence $\partial_\ast[x] = 0$.

	Suppose now that $\partial_\ast[x] = 0$. We calculate this by choosing $y \in B_n$ and taking the corresponding $z$. Then $z = d_n(t)$, as $[z] = 0$. Thus $j_n(y - i_n(t)) = x$, and
	\begin{align*}
		d_n(y - i_n(t)) = d_n y - d_n i_n(t) = i_n(z - z) = 0.
	\end{align*}
	So $j_\ast[y - i_n(t)] = [x]$, so $[x] \in \Img(j_\ast)$.

	We also need exactness at $H_n(B_\bullet)$. As $j_n \circ i_n = 0$, we have $\Img(i_\ast) \subseteq \Ker(j_\ast)$. Now suppose $j_\ast[y] = 0$. Then $j_n(y) = d_{n+1}(a)$ for some $a \in C_{n+1}$. Let $b \in B_{n+1}$ be such that $j_{n+1}(b) = a$. Then
	\begin{align*}
		j_n(y - d_{n+1}(b)) = d_{n+1}(a) - j_n d_{n+1}(b) = d_{n+1}(a) - d_{n+1} j_{n+1}(b) = 0,
	\end{align*}
	hence $y - d_{n+1}(b) = i_n(t)$. Since $d_n(y - d_{n+1}(b)) = 0$, we get
	\[
		[y] = [y - d_{n+1}(b)] = [i_n(t)] = i_\ast[t],
	\]
	as required. Finally we want exactness at $H_n(A_\bullet)$. Let $[z] = \partial_\ast[x]$. Then $i_n(z) = d_{n+1}(y),$ so $i_\ast[z] = 0$. Conversely let $[z]$ be such that $i_\ast[z] = 0$. Then $i_n(z) = d_{n+1}(y)$ for some $y$, so $[z] = \partial_\ast[j_{n+1}(y)]$.

	Note $d_{n+1} j_{n+1}(y) = j_n d_{n+1}(y) = j_n i_n(z) = 0$, so this works.
\end{proofbox}

% lecture 20

\begin{theorem}[Mayer-Vietoris Theorem]
	Let $K$ be a simplicial complex, $M, N$ be subcomplexes, and $L = M \cap N$. Suppose also that $M$ and $N$ cover $K$, i.e. every simplex of $K$ is in $M$ or $N$. Write
	\begin{equation*}
	\begin{tikzcd}
		L \arrow[r, "j", hook] \arrow[d, "i", hook] & N \arrow[d, "\ell", hook] \\
		M \arrow[r, "k", hook] & K  \\
	\end{tikzcd}
	\end{equation*}
	for the inclusion maps. There are natural homomorphisms $\partial_\ast : H_n(K) \to H_{n-1}(L)$ such that
	\begin{equation*}
	\begin{tikzcd}
			& \vdots \arrow[dl] & \\
		H_n(L) \arrow[r, "{(i_\ast, j_\ast)}"'] & H_n(M) \oplus H_n(N) \arrow[r, "k_\ast - l_\ast"] & H_n(K) \arrow[dll, "\partial_\ast"] \\
					    H_{n-1}(L) \arrow[r, "{(i_\ast, j_\ast)}"'] & H_{n-1}(M) \oplus H_{n-1}(N) \arrow[r, "k_\ast - l_\ast"] & H_{n-1}(K) \arrow[dl] \\
						    & \vdots &
	\end{tikzcd}
	\end{equation*}
	is a long exact sequence.
\end{theorem}

\begin{proofbox}
	We have a short exact sequence of chain complexes
	\begin{equation*}
		\begin{tikzcd}
			0  \arrow[r] & C_\bullet(L) \arrow[r, "{(i_\bullet, j_\bullet)}", hook] & C_\bullet(M) \oplus C_\bullet(N) \arrow[r, "k_\bullet - l_\bullet", two heads] & C_\bullet(K) \arrow[r] & 0
		\end{tikzcd}
	\end{equation*}
	This is because:
	\begin{itemize}
		\item $i_\bullet$ and $j_\bullet$ are both injective, so the first is injective.
		\item $k_\bullet$ and $l_\bullet$ are jointly surjective, as $M$ and $N$ cover $K$.
		\item Suppose $(x, y) \in C_n(M) \oplus C_n(N)$ is in $\Ker(k_n - l_n)$, i.e. $k_n(x) = l_n(y)$. So both sides are a linear combination of simplices in $M \cap N = L$, i.e. there exists $z \in C_n(L)$ such that $x = i_n(z), y = j_n(z)$.

			So $\Ker(k_n - l_n) \subseteq \Img ((i_n, j_n))$. The other direction is clear.
	\end{itemize}
	Then we apply the algebraic theorem.
\end{proofbox}

Here's how to think about the map $\partial_\ast$. Start with a cycle in $K$. Some of the simplices are in $M$, some are in $N$, some are in both; for these we make an arbitrary choice.

Suppose $[x] \in H_n(K)$. Then we may write $x = u + v$, where $u \in C_n(M)$, $v \in C_n(N)$. We then see that $x = (k_n - l_n)(u, -v)$.

The theorem says to look at the boundary of $(u, -v)$, i.e. $d_n(u, -v) = (d_n u, -d_n v)$. These are the same, since $u + v$ is a cycle, so $d_n u + d_n v = 0$. The map $\partial_\ast[x]$ is then just $[d_nu] \in C_{n-1}(L)$.

\subsection{Continuous Maps and Homotopy Invariance}%
\label{sub:cont_map_hom_inv}

\begin{definition}
	Simplicial maps $f, g : K \to L$ are \emph{contiguous}\index{contiguous} if for each $\sigma \in K$, $f(\sigma)$ and $g(\sigma)$ are faces of some simplex $\tau$ of $L$.
\end{definition}

\begin{lemma}
	If $f$ and $g$ are both simplicial approximations to $F : |K| \to |L|$, then $f$ and $g$ are contiguous.
\end{lemma}

\begin{proofbox}
	If $x \in \mathring \sigma \subset |K|$, and  $F(x) \in \mathring \tau \subset |L|$, then as in the proof of ``simplicial approximations are simplicial maps'', $f(\sigma)$ and $g(\sigma)$ are both faces of $\tau$.
\end{proofbox}

\begin{lemma}
	If $f, g : K \to L$ are contiguous, then $f_\bullet \simeq g_\bullet : C_\bullet(K) \to C_\bullet(L)$, and so $f_\ast = g_\ast$.
\end{lemma}

\begin{proofbox}
	Choose an ordering $\prec$ of $V_K$, and represent a basis of $C_n(K)$ by $[a_0, \ldots, a_n]$ with $a_0 \prec \cdots \prec a_n$. Define a homomorphism
	\begin{align*}
		h_n : C_n(K) &\to C_{n+1}(L) \\
		[a_0, \ldots, a_n] &\mapsto \sum_{i = 0}^n (-1)^i [f(a_0), \ldots, f(a_i), g(a_i), \ldots, g(a_n)].
	\end{align*}
	Direct calculation shows that $d_{n+1} \circ h_n + h_{n-1} \circ d_n = g_n - f_n$, so this is the chain homotopy.
\end{proofbox}

\begin{lemma}
	Let $K'$ be the barycentric subdivision, and choose $a : K = V_{K'} \to K$, any function sending $\hat \sigma$ to any vertex of $\sigma$.

	Then such an $a : V_{K'} \to V_K$ is a simplicial approximation to the identity map on $|K| = |K'|$. Furthermore, all simplicial approximation to the identity are of this form.
\end{lemma}

\begin{proofbox}
	We have seen the first part.

	For the second part, if $g : V_{K'} \to V_K$ is a simplicial approximation to the identity, then
	\[
		\mathring \sigma = \id(\mathrm{St}_{K'}(\hat \sigma)) = \mathrm{St}_K(g(\hat \sigma)).
	\]
	So, $g(\hat \sigma)$ must be a vertex of $\sigma$.
\end{proofbox}

\begin{proposition}
	If $a : K' \to K$ is a simplicial approximation to the identity, then $a_\ast : H_n(K') \to H_n(K)$ is an isomorphism, for all $n$.

	Furthermore, this isomorphism does not depend on $a$.
\end{proposition}

\begin{proofbox}
	First suppose $K$ is a simplex $\Delta^n \subset \mathbb{R}^{n+1}$. Then $K'$ is a cone, with cone point $\hat \sigma$. Also $K$ is a cone, with any vertex as a cone point.

	So, $a_\ast : H_n(K') \to H_n(K)$ is an isomorphism for $n > 0$, since both sides are 0, and $a_\ast : H_0(K') \to H_0(K)$ is an isomorphism too, as $a_\ast[\hat \sigma] = [a(\hat \sigma)]$.

	For the general case, we will do a double induction on the dimension of $K$, and the number of simplices of $K$.

	Let $\sigma \in K$ be a simplex of maximal dimension. Then $L = K - \{\sigma\}$ is a simplicial complex. Let  $S = \{\sigma \text { and its faces}\}$, $T = S \cap L$, the proper faces of $\sigma$.

	THen any simplicial approximation $a : K' \to K$ to the identity sends $L'$ into $L$, $S'$ into $S$ and $T'$ into $T$. Thus, we get a map of MV sequences:
	\begin{equation*}
		\begin{tikzcd}[column sep = tiny]
			H_n(T') \arrow[r] \arrow[d, "{(a|_{T'})_\ast}"] & H_n(S') \oplus H_n(L') \arrow[r] \arrow[d, "{(a|_{S'})_\ast \oplus (a|_{L'})_\ast}"] & H_n(K') \arrow[r, "\partial_\ast"] \arrow[d, "{a_\ast}"] & H_{n-1}(T') \arrow[r] \arrow[d, "{(a|_{T'})_\ast}"] & H_{n-1}(S') \oplus H_{n-1}(L') \\
H_n(T) \arrow[r] & H_n(S) \oplus H_n(L) \arrow[r] & H_n(K) \arrow[r, "\partial_\ast"] & H_{n-1}(T) \arrow[r] & H_{n-1}(S) \oplus H_{n-1}(L)
		\end{tikzcd}
	\end{equation*}
	We continue as follows: $T$ is the boundary of a simplex $\sigma$, so the dimension of $T$ is strictly less than $K$. Therefore these maps $H_n(T') \to H_n(T)$ is an isomorphism.

	Now, we may also suppose the other two non-central maps are isomorphisms. Firstly because $S$ is a simplex, and also since $L$ has strictly less simplices than $K$.

	Now, the outer four maps are isomorphisms. This implies the inner map is an isomorphism, by the apt five-lemma.
\end{proofbox}

%lecture 21

Call $\nu_K: H_n(K') \to H_n(K)$, and similarly $\nu_{K, r} : H_(K^{(r)}) \to H_n(K)$, and $\nu_{K, r, s} : H_n(K^{(r)}) \to H_n(K^{(s)})$, for the respective isomorphisms.

\begin{proposition}
	To each continuous $f : |K| \to |L|$, there is an associated homomorphism $f_\ast : H_n(K) \to H_n(L)$ given by $f_\ast = s_\ast \circ \nu_{K, r}^{-1}$, where $s : K^{(r)} \to L$ is a simplicial approximation to $f$. Then,
	\begin{enumerate}[\normalfont(i)]
		\item This does not depends on the choice of $s$ and $r$.
		\item If $g : |M| \to |K|$ is another map, then $(f \circ g)_\ast = f_\ast \circ g_\ast$.
	\end{enumerate}
\end{proposition}

\begin{proofbox}
	For (i), let $s : K^{(r)} \to L$, and $t : K^{(q)} \to L$ be simiplicial approximations to $f$, and suppose $r \geq q$. Let $a : K^{(r)} \to K^{(q)}$ be a simplicial approximation to $\id$.

	Now $t \circ a$, $s : K^{(r)} \to L$ are both simplicial approximations to $f$, so they are continguous, so $s_\ast = (t \circ a)_\ast = t_\ast \circ a_\ast$. But $a_\ast = \nu_{K, r, q}$, so
	\[
	s_\ast \circ \nu_{K, r}^{-1} = t_\ast \circ \nu_{K, r, q} \circ \nu_{K, r}^{-1} = t_\ast \circ \nu_{K, q}^{-1}.
	\]
	For (ii), let $s :  K^{(r)} \to L$ approximate $f$, and let $t : M^{(q)} \to K^{(r)}$ approximate $g$. Then $s \circ t$ approximates $f \circ g$, and
	\[
		(f \circ g)_\ast = (s \circ t)_\ast \circ \nu_{M, q}^{-1} = (s_\ast \circ \nu_{K, r}^{-1}) \circ (\nu_{K, r} \circ t_\ast \circ \nu_{M, q}^{-1}) = f_\ast \circ g_\ast.
	\]
\end{proofbox}

\begin{corollary}
	If $f : |K| \to |L|$ is a homeomorphism, then $f_\ast : H_n(K) \to H_n(L)$ is an isomorphism.
\end{corollary}

\begin{proofbox}
	We know
	\[
	\id = (f \circ f^{-1})_\ast = f_\ast \circ (f^{-1})_\ast,
	\]
	and
	\[
	\id = (f^{-1} \circ f)_\ast = (f^{-1})_\ast \circ f_\ast,
	\]
	so $f_\ast$ is invertible.
\end{proofbox}

Let's show uniqueness of homology, up to homotopy. We will get started with this lemma.

\begin{lemma}
	For a simplicial complex $L$ in $\mathbb{R}^m$, there is an $\varepsilon = \varepsilon(L) > 0$, such that if $f, g : |K| \to |K|$ satisfy $|f(x) - g(x)| < \epsilon$ for all $x \in |K|$, then $f_\ast = g_\ast : H_n(K) \to H_n(L)$.
\end{lemma}

\begin{proofbox}
	The $\{\mathrm{St}_L(w)\}$ give an open cover of $|L|$, so by the Lebesgue number lemma there exists $\varepsilon > 0$ such that $B_\varepsilon(x)$ lies in some $\mathrm{St}_L(w)$.

	Let $f, g : |K| \to |L|$ be as in the statement, using this $\varepsilon$. Then $\{f^{-1}(B_\varepsilon(y))\}$ is an open cover of $|K|$, so there is a $\delta > 0$ such that each $f(B_{\delta}(x)) \subseteq B_{\varepsilon}(y)$. Then also $g(B_{\delta}(x)) \subseteq B_{2 \epsilon}(y)$.

	Let $r \gg 0$ be such that $\mu(K^{(r)}) < \delta/2$. Then for each $v \in V_{K^{(r)}}$, the diameter of $\mathrm{St}_{K^{(r)}}(v)$ is less than $\delta$. So $f(\mathrm{St}_{K^{(r)}}(v)), g(\mathrm{St}_{K^{(r)}}(v))$ are in a common $B_{2 \varepsilon}(y)$, which is in a $\mathrm{St}_L(w)$.

	Let $s(v) = w$. Then $s$ is a simplicial approximation to both $f$ and $g$, so $f_\ast = s_\ast \circ \nu_{K, r}^{-1} = g_\ast$.
\end{proofbox}

\begin{theorem}
	If $f \simeq g : |K| \to |L|$, then $f_\ast = g_\ast$.
\end{theorem}

\begin{proofbox}
	Let $H : |K| \times I \to |L|$ be a homotopy between $f$ and $g$. As $K$ is compact, $|K| \times I$ is compact, so $H$ is uniformly continuous.

	For the $\varepsilon = \varepsilon(L) > 0$ from the lemma, there is a $\delta > 0$ such that $|s - t| < \delta \implies |H(x, s) - H(x, t)| < \varepsilon$, for all $x \in |K|$. Choose $0 = t_0 < t_1 < \cdots < t_k =1$ such that $|t_{i+1} - t_i| < \delta$.

	Then the $f_i(x) = H(x, t_i)$, with $f_0 = f$, $f_k = g$ are $\varepsilon$-close. So $(f_i)_\ast = (f_{i+1})_\ast$, and so $f_\ast = g_\ast$.
\end{proofbox}

\begin{definition}
	An \emph{h-triangulation}\index{h-triangulation} of a space $X$ is a simplicial complex $K$ and a homotopy equivalence $g : |K| \to X$. We define $H_n(X) = H_n(K)$.
\end{definition}

\begin{lemma}
	The homology of a h-triangulated space does not depend on the choice of h-triangulation.
\end{lemma}

\begin{proofbox}
	Let $\bar g : |\bar K| \to X$ be another h-triangulation. Then
	\[
		|\bar K| \overset{\bar g}{\to} X \to |K|.
	\]
	The composition is a homotopy equivalence, so induces an isomorphism $H_n(\bar K) \to H_n(K)$.
\end{proofbox}

If $g : |K| \to X$, $\bar g : |\bar K| \to X$ are h-triangulations with homotopy inverses $f, \bar f$, and $h : X \to \bar X$ is a map,
\begin{equation*}
	\begin{tikzcd}
		H_n(K) \arrow[r] \arrow[d, equals] & H_n(\bar X) \arrow[d, equals] \\
		H_n(K) \arrow[r, bend right, "(\bar f \circ h \circ g)_\ast"] & H_n(\bar K)
	\end{tikzcd}
\end{equation*}
defines induces maps on the homology, for any map of h-triangulable spaces.

% lecture 22

\begin{lemma}
	The sphere $S^{n-1}$ is triangulable, and for $n = 1 \geq 1$, we have
	\[
	H_i(S^{n-1}) =
	\begin{cases}
		\mathbb{Z} & i = 0, n-1, \\
		0 &\text{else}.
	\end{cases}
	\]
\end{lemma}

\begin{proofbox}
	$\partial D^n$ has polyhedron homeomorphic to $S^{n-1}$, and we have calculated its $H_i$.
\end{proofbox}

\begin{theorem}[Brouwer]
	\begin{enumerate}[\normalfont(i)]
		\item[]
		\item There is no retraction of $D^n$ to $\partial D^n = S^{n-1}$.
		\item Any $f : D^n \to D^n$ has a fixed point.
	\end{enumerate}
\end{theorem}

\begin{proofbox}
	(i) implies (ii) by the same shooting procedure as in the case $n = 2$.

	For (i), let $r : D^n \to S^{n-1}$ be a retraction, and consider $\id_{S^{n-1}} = r \circ i$. Then,
	\[
		\id = (r \circ i)_\ast : H_{n-1}(S^{n-1}) = \mathbb{Z} \overset{i_\ast}{\to} H_{n-1}(D^n) = 0 \overset{r_\ast}{\to} H_{n-1}(S^{n-1}) = \mathbb{Z}.
	\]
	This is impossible.
\end{proofbox}

We have seen a different triangulation of $S^n$, vie the simplicial complex $K$ in $\mathbb{R}^{n=1}$ with simplices $\langle \pm e_1, \ldots, \pm e_{n=1} \rangle$ and their faces. We must have
\[
H_i(K) =
\begin{cases}
	\mathbb{Z} & i = 0, n \\
	0 &\text{else},
\end{cases}
\]
by independence of triangulation.

\begin{lemma}
	The element
	\[
		x = \sum_{a \in \{\pm1\}^{n+1}} a_1 a_2 \cdots a_{n+1} [a_1 e_1, a_2e_2, \ldots, a_{n+1} e_{n+1}]
	\]
	is a cycle, and generates $H_n(K) \cong \mathbb{Z}$.
\end{lemma}

\begin{proofbox}
	When we apply $d_n$ to $x$, the simplex $[a_1 e_1, \ldots, \hat{a_i e_i}, \ldots, a_{n+1} e_{n+1}]$ shows up twice, corresponding to $a_i = +1$, $a_i = -1$. So the coefficients cancel out.

	It generates as it is clearly not divisible in $C_n(K)$, as the coefficients are $\pm1$. The reflection  $r : \mathbb{R}^{n+1} \to \mathbb{R}^{n+1}$ in the $i$'th coordinate is
	\[
		(r_i)_\ast(x) = \sum a_1 \cdots a_{n+1} [a_1e_1, \ldots, -a_i e_i, \ldots, a_{n+1} e_{n+1}] = -x.
	\]
	So $(r_i)_\ast$ is multiplication by $-1$. The antipodal map $a : S^n \to S^n$ is $r_1 r_2 \cdots r_{n+1}$, so acts as multiplication by $(-1)^{n+1}$.
\end{proofbox}

\begin{corollary}
	If $n$ is even, then $a : S^n \to S^n$ is not homotopic to $\id$.
\end{corollary}

\subsection{Homology of Surfaces}%
\label{sub:surf_hom}

\begin{exbox}
	Consider $X$, the torus with a triangle cut out of it. Now create a $3 \times 3$ grid on a torus, half each square to form triangles, and remove one triangle.

	Say $X = |K|$. Consider many copies of $K$, with boundary $L$, and now consider the cone on $L$.
\end{exbox}

First we find the homology of $K$. Consider $W$, the deformation retraction of $K$ onto the outer square, which is just the join of two triangles.

Apply Mayer-Vietoris to the two triangles independetly, to get
\[
	H_0(K) = \mathbb{Z}\{[1]\}, \qquad H_1(K) = \mathbb{Z}\{a, b\},
\]
where $a = [1, 2] + [2, 3] + [3, 1]$, and $b = [1, 4] + [4, 5] + [5, 6]$. Note $r = [5, 6] + [6, 9] + [9, 5] \sim a  + b - a - b = 0$.

Now we look to find the homology of $K_g$. We can decompose $K_g = K_{g-1} \cup K$, where $K_{g-1} \cap K = \Delta'$. Mayer-Vietoris gives
\begin{equation*}
	\begin{tikzcd}
		0 \arrow[r] & H_2(K_{g-1}) \oplus H_2(K) \arrow[r, hook, two heads] & H_2(K_g) \arrow[dll, "\partial_\ast"] \\
 	H_1(\Delta') = 0 \arrow[r] & H_1(K_{g-1}) \oplus H_1(K) \arrow[r, hook, two heads] & H_1(K_g) \arrow[dll, "\partial_\ast = 0"] \\
 H_0(\Delta') = \mathbb{Z} \arrow[r, hook] & H_0(K_{g-1}) \oplus H_0(K) \arrow[r] & H_0(K_g)
	\end{tikzcd}
\end{equation*}

By diagram chasing and exactness, we get that the second map is injective and surjective, that the second-to-last map is injective, the previous map is 0, and hence the map before is surjective. But the previous space is 0, so it is also injective.

We then see that $H_1(K_g) = \mathbb{Z}^{2g}$, and $H_2(K_g) = 0$, by induction. The only non-trivial thing in $H_0(K_g) = \mathbb{Z}$. This is done on the last line, by induction.

Note that $L = \partial K_g$ is the cycle $r_1 + r_2 + \cdots + r_g = 0$.

Now we want to find the homology of $K_g \cup CL$. Apply Mayer-Vietoris, using $K_g \cap CL = L$, a triangulation of $S^1$. Hence
\begin{equation*}
	\begin{tikzcd}
		0 \arrow[r] & 0 \oplus 0 \arrow[r] & H_2(K_g \cup CL) \arrow[dll, "\partial_\ast", hook, two heads] \\
 	H_1(L) = \mathbb{Z}\arrow[r, "0"'] & H_1(K_g) \oplus H_1(CL) = \mathbb{Z}^{2g} \arrow[r, hook, two heads] & H_1(K_g \cup CL) \arrow[dll, "\partial_\ast = 0"] \\
 H_0(L) = \mathbb{Z}\arrow[r, hook] & H_0(K_g) \oplus H_0(CL) = \mathbb{Z}^2 \arrow[r, two heads] & H_0(K_g \cup CL)
	\end{tikzcd}
\end{equation*}

This gives $H_0(K_g \cup CL) = \mathbb{Z}$, $H_1(K_g \cup CL) = \mathbb{Z}^{2g}$ and $H_2(K_g \cup CL) = \mathbb{Z}$.

% lecture 23

Let's do the same thing for non-orientable surfaces.

\begin{exbox}
	Consider the projective plane. First, construct it as square with boundaries joined in a particular way, and cut a triangle out of the middle. Say $Y$ is this space, with triangulation $|K|$. We also have $K_n$, $n$ of these $K$'s glued together, and moreover if $L = \partial K_n$, we can get a cone by gluing $L$, $CL$, then let $S_n = |K_n \cup CL|$.

	The subcomplex $|W| = |K|$, where $W$ is simply just a triangle, is such that $|K|$ deformation retracts to it. So $H_0(K) = \mathbb{Z}\{\text{vector}\}$, $H_1(K) = \mathbb{Z}\{a\}$ where $a = [1, 2] + [2, 3] + [3, 1]$, and $H_n(K) = 0$ for larger $n$.

	Note that the boundary $r = [4, 5] + [5, 6] + [6, 4]$ is homologous to $2a$.

	Now if $K_n = K_{n-1} \cup K$, where $K_{n-1} \cap K = \Delta^1$, then by Mayer-Vietoris, we get this
\begin{equation*}
	\begin{tikzcd}
		0 \arrow[r] & H_2(K_{n-1}) \oplus H_2(K) \arrow[r, hook, two heads] & H_2(K_n) \arrow[dll, "\partial_\ast"] \\
 	H_1(\Delta') = 0 \arrow[r] & H_1(K_{n-1}) \oplus H_1(K) \arrow[r, hook, two heads] & H_1(K_n) \arrow[dll, "\partial_\ast = 0"] \\
 H_0(\Delta^1) = \mathbb{Z} \arrow[r, hook] & H_0(K_{n-1}) \oplus H_0(K) \arrow[r] & H_0(K_n)
	\end{tikzcd}
\end{equation*}
	This gives $H_0(K_n) = \mathbb{Z}$, $H_1(K_n) = \mathbb{Z}^n$ and $H_2(K_n) = H_i(K_n) = 0$ for $i \geq 2$.

	The boundary of $K_n$ is the cycle $r_1 + r_2 + \cdots + r_n$, which is homologous to $2(a_1 + a_2 + \cdots + a_n)$.

	Finally we look at the Homology of $K_n \cup CL$. Using Mayer-Vietoris once again,
\begin{equation*}
	\begin{tikzcd}
		0 \arrow[r] & 0 \oplus 0 \arrow[r] & H_2(K_n \cup CL) \arrow[dll, "\partial_\ast = 0"] \\
 	H_1(L) = \mathbb{Z}\arrow[r, "(\ast)"', hook] & H_1(K_n) \oplus H_1(CL) = \mathbb{Z}^{n} \arrow[r, two heads] & H_1(K_n \cup CL) \arrow[dll, "\partial_\ast = 0"] \\
 H_0(L) = \mathbb{Z}\arrow[r, hook] & H_0(K_n) \oplus H_0(CL) = \mathbb{Z} \arrow[r, two heads] & H_0(K_n \cup CL)
	\end{tikzcd}
\end{equation*}
Note that $(\ast)$ sends the generator $r_1 + \cdots + r_n$ to $2(a_1 + \cdots + a_n)$, hence $H_0(K_n \cup CL) = \mathbb{Z}$, $H_1(K_n \cup CL) = \mathbb{Z}\{a_1, \ldots, a_n\} / \langle 2(a_1 + \cdots + a_n) \rangle \cong \mathbb{Z}/2 \oplus \mathbb{Z}^n$, and $H_2(K_n \cup CL) = H_i(K_n \cup CL) = 0$.
\end{exbox}

Hence none of the surfaces $S^2, \Sigma_{1, g}$ or $S_n$ are homotopy equivalent to each other, as they have different homotopy groups.

\begin{theorem}
	Every triangulable surface is homeomorphic to one of these.
\end{theorem}

\section{Rational Homology, Euler and Lefschetz Numbers}%
\label{sec:rational_hom}

\begin{definition}
	For a simplicial complex $K$, let $\mathcal{O}_n(K, \mathbb{Q})$ be the $\mathbb{Q}$-vector space with basis the ordered simplices of $K$.

	Define $T_n(K, \mathbb{Q})$ as usual, and so we get $C_n(K, \mathbb{Q})$. Define $d_n$ by the same formula, and define
	\[
	H_n(K ; \mathbb{Q}) = \frac{\Ker(d_n : C_n(K; \mathbb{Q}) \to C_{n-1}(K; \mathbb{Q}))}{\Img(d_{n+1}: C_{n+1}(K, \mathbb{Q}) \to C_n(K, \mathbb{Q})}.
	\]
\end{definition}

\begin{lemma}
	If $H_n(K) = \mathbb{Z}^r \oplus G$, where $G$ is a finite abelian group, then $H_n(K, \mathbb{Q}) = \mathbb{Q}^r$.
\end{lemma}

\begin{exbox}
	We have
	\[
	H_i(S^n; \mathbb{Q}) =
	\begin{cases}
		\mathbb{Q} & i = 0, n,\\
		0 &\text{else}.
	\end{cases}
	\]
	\[
	H_i(\Sigma_g; \mathbb{Q}) =
	\begin{cases}
		\mathbb{Q} & i = 0,\\
		\mathbb{Q}^{2g} & i = 1, \\
		\mathbb{Q} & i = 2,\\
		0 &\text{else}.
	\end{cases}
	\]
	\[
	H_i(S_n; \mathbb{Q}) =
	\begin{cases}
		\mathbb{Q} & i = 0,\\
		\mathbb{Q}^{n-1} & i = 1, \\
		0 &\text{else}.
	\end{cases}
	\]
\end{exbox}

\begin{definition}
	Let $X$ be a polyhedron, and $f : X \to X$ be a continuous map. The \emph{Lefschetz number}\index{Lefschetz number} of $f$ is
	\[
	L(f) = \sum_{i = 0}^\infty (-1)^i \Tr(f_\ast : H_i(X; \mathbb{Q}) \to H_i(X, \mathbb{Q})).
	\]
	The \emph{Euler characteristic}\index{Euler characteristic} is 
	\[
		\chi(X) = L(\id) = \sum_{i = 0}^\infty (-1)^i \dim_{\mathbb{Q}} H_i(X; \mathbb{Q}).
	\]
\end{definition}

\begin{exbox}
	We have
	\[
	\chi(S^n) = 1 + (-1)^n =
	\begin{cases}
		2 & n \text{ even},\\
		0 & n \text{ odd}.
	\end{cases}
	\]
	Similarly $\chi(\Sigma_g) = 2 - 2g$, and $\chi(S_n) = 2 - n$.
\end{exbox}

\begin{exbox}
	The antipodal map $a : S^n \to S^n$ induces the identity map of $H_0(S^n, \mathbb{Q})$, and multiplication by $(-1)^n$ on $H_n(S^n; \mathbb{Q})$. So,
	\[
	L(a) = 1 + (-1)^n (-1)^{n+1} = 0.
	\]
\end{exbox}

We build up to Lefschetz fixed point theorem.

\begin{lemma}
	Let $V$ be a finite-dimensional vector space, $W \leq V$ a subspace, and $A : V \to V$ a linear map such that $A(W) = W$.

	Let $B = A|_W : W \to W$, and $C : V/W \to V/W$ the map induced by $A$. Then,
	\[
	\Tr(A) = \Tr(B) + \Tr(C).
	\]
\end{lemma}

\begin{proofbox}
	Let $e_1, \ldots, e_r$ be a basis for $W$, and extend it to a basis of $V$. In this basis,
	\[
	A = 
	\begin{pmatrix}
		B & * \\
		0 & C
	\end{pmatrix}.
	\]
	So $\Tr(A) = \Tr(B) + \Tr(C)$.
\end{proofbox}

\begin{corollary}
	For a chain map $f_\bullet : C_\bullet(K; \mathbb{Q}) \to C_\bullet(K; \mathbb{Q})$, we have
	\[
	\sum_i (-1)^i \Tr(f_\ast : H_i(K; \mathbb{Q}) \to H_i(K, \mathbb{Q})) = \sum_i (-1)^i \Tr(f_i : C_i(K; \mathbb{Q}) \to C_i(K; \mathbb{Q})).
	\]
\end{corollary}


\begin{proofbox}
	Consider exact sequences
	\begin{equation*}
		\begin{tikzcd}
			0 \arrow[r] & {B_i(K; \mathbb{Q})} \arrow[r] & {Z_i(K; \mathbb{Q})} \arrow[r] & {H_i(K; \mathbb{Q})} \arrow[r] & 0 \\
0 \arrow[r] & {Z_i(K; \mathbb{Q})} \arrow[r] & {C_i(K; \mathbb{Q})} \arrow[r, "d_i"] & {B_{i-1}(K; \mathbb{Q})} \arrow[r] & 0
%		0 \arrow[r] & 0 \oplus 0 \arrow[r] & H_2(K_n \cup CL) \\
		\end{tikzcd}	
	\end{equation*}
	Let $f^C, f^Z, f^B, f^H$ be the induced maps on chains/cycles/boundaries/homologies. Then,
	\begin{align*}
		\sum (-1)^i \Tr(f_i^H) &= \sum (-1)^i [\Tr(f_i^Z) - \Tr(f_i^B)] \\
				       &= \sum (-1)^i [\Tr(f_i^C) - \Tr(f_{i-1}^B) - \Tr(f_i^B)] \\
				       &= \sum(-1)^i \Tr(f_i^C).
	\end{align*}
\end{proofbox}

\begin{corollary}
	$\chi(|K|) = \sum (-1)^i |\{\text{$i$-simplices of $K$}\}|$
\end{corollary}

% lecture 24

\begin{lemma}
	There is a chain map $s_\bullet : C_\bullet(K) \to C_\bullet(K')$ given by sending a simplex $\sigma$ to an appropriate linear combination of the simplices of $K'$, which compose $\sigma$.

	On $H_\ast$, it induces $\nu_K^{-1} : H_n(K) \to H_n(K')$.
\end{lemma}

\begin{proofbox}
	We start by $s_0([v_0]) = [\hat v_0]$. Suppose $s_0, \ldots, s_{n-1}$ have been defined, and satisfy $d_i \circ s_i = s_{i-1} \circ d_i$, for all $i < n$.

	Then, we define
	\[
		s_n(\sigma) = [\hat \sigma, s_{n-1} d_n(\sigma)],
	\]
	interpreted linearly in the second variable. For example,
	\[
		s_1([v_0, v_1]) = [\langle v_0, v_1\rangle, \langle v_1 \rangle - \langle v_0 \rangle] = [\hat{\langle v_0, v_1 \rangle}, \langle \hat v_1 \rangle] - [\hat{\langle v_0, v_1 \rangle}, \langle \hat v_0 \rangle].
	\]
	This can be drawn as splitting a line in to. We can calculate that
	\begin{align*}
		d_n s_n([v_0, \ldots, v_n]) &= d_n[ \hat{\langle v_0, \ldots, v_n \rangle}, s_{n-1} d_n [v_0, \ldots, v_n] ] \\
					    &= s_{n-1} d_n [v_0, \ldots, v_n] - [ \hat{\langle v_0, \ldots, v_n \rangle}, d_{n-1} s_{n-1} d_n [v_0, \ldots, v_n] ] \\
					    &= s_{n-1} d_n [v_0, \ldots, v_n],
	\end{align*}
	using induction and the fact $d_{n-1} \circ d_n = 0$. So $s_\bullet$ is a chain map. Let $\prec$ be an ordering of $V_K$, and $a : K = V_{K'} \to V_K$ such that $\hat \sigma$ is sent to the smallest vertex of $\sigma$, with respect to $\prec$. This is some simplicial approximation to $\id$.

	Now note $a_0 \circ s_0 = \id$, so suppose $a_i \circ s_i = \id$. If $[v_0, \ldots, v_n]$ are with $v_0 \prec \cdots \prec v_n$, then
	\begin{align*}
		a_n s_n ( [v_0, \ldots, v_n]) &= a_n ( [ \hat{\langle v_0, \ldots, v_n \rangle}, s_{n-1} d_n [v_0, \ldots, v_n]]) \\
					      &= [v_0, a_{n-1} s_{n-1} d_n [v_0, \ldots, v_n] = [v_{0}, d_n [v_0, \ldots, v_n]] \\
					      &= [v_0, \langle v_1, \ldots, v_n \rangle - \sum_i (-1)^i \langle v_0, \ldots, \hat v_i, \ldots, v_n \rangle ] \\
					      &= [v_0, \ldots, v_n].
	\end{align*}
	So $a_\ast \circ s_\ast = \id$, hence $\nu_k \circ s_\ast = \id$, and $s_\ast = \nu_K^{-1}$.
\end{proofbox}

\begin{theorem}[Lefschetz Fixed Point Theorem]
	Let $f : X \to X$ be a map of a polyhedron. If $L(f) \neq 0$, then $f$ has a fixed point.
\end{theorem}

\begin{proofbox}
	Suppose $f$ does not have a fixed point. Let $\delta = \inf(|x - f(x)|, x \in X)$, by letting $X = |K| \subseteq \mathbb{R}^N$. As $X$ is compact, $\delta > 0$.

	Let $K$ be a triangulation of $X$ with $\mu(K) < \delta/2$, and choose a simplicial approximation $g : K^{(r)} \to K$ to $f$. For a vertex $v \in V_{K^{(r)}}$, we have
	\[
		f(v) \in f(\mathrm{St}_{K^{(r)}}(v)) \subset \mathrm{St}_K(g(v)),
	\]
	so $|f(v) - g(v)| < \delta/2$. But $|f(v) - v| > \delta$, so $|g(v) - v| > \delta/2$. So if $v \in \sigma \in K$, then $g(v) \not \in \sigma$.

	The map $f_\ast$ is defined as $g_\ast \circ \nu_{K, r} = g_\ast \circ s_\ast^{(r)}$, where $s_\ast^{(r)}$ is the $r$-fold iteration of the map in the previous lemma. So,
	\begin{align*}
		L(f) &= \sum_i (-1)^i \Tr(f_\ast : H_i(X) \to H_i(X)) \\
		     &= \sum_i (-1)^i \Tr(g_i \circ s_i^{(r)} : C_i(K) \to C_i(K)),
	\end{align*}
	by the last lemma. But now if $\sigma \in K$ is an $i$-simplex, then $s_i^{(r)}(\sigma)$ is a sum of simplices inside $\sigma$, so $g_i s_i^{(r)}(\sigma)$ is a sum of simplicies not including $\sigma$. So the matrix for $g_i \circ s_i^{(r)}$ has zeroes on its diagonal, hence it has trace 0.
\end{proofbox}

\begin{exbox}
	If $X$ is a contractible polyhedron, then
	 \[
		 H_n(X) =
		 \begin{cases}
			 \mathbb{Z} & n = 0,\\
			 0 & \text{else}.
		 \end{cases}
	\]
	So any $f : X \to X$ has $L(f) = 1$, hence it has a fixed point.
\end{exbox}

\begin{exbox}
	Note that $S_1 = \mathbb{RP}^2$ has
	\[
	H_n(\mathbb{RP}^2) = 
	\begin{cases}
		\mathbb{Z} & n = 0,\\
		\mathbb{Z}/2 & n = 1,\\
		0 &\text{else}
	\end{cases} \implies H_n(\mathbb{RP}^2; \mathbb{Q}) =
	\begin{cases}
		\mathbb{Q} & n = 0, \\
		0 & \text{else}.
	\end{cases}
	\]
	So any $f : \mathbb{RP}^2 \to \mathbb{RP}^2$ has $L(f) = 1$, so has a fixed point.
\end{exbox}

\begin{exbox}
	Let $G$ be a topologial group, which is path-connected and non-trivial. Suppose it is a polyhedron. If $g \neq 1 \in G$, then $g \cdot : G \to G$ has no fixed points, so
	\[
	0 = L(g \cdot) = L(1 \cdot) = L(\id) = \chi(G).
	\]
\end{exbox}

\begin{exbox}
	Consider $S_3$, the connected sum of three $\mathbb{RP}^2$. Let $f : S_3 \to S_3$ be such that $f \circ f = \id$. We have:
	\[
	H_n(S_3; \mathbb{Q}) =
	\begin{cases}
		\mathbb{Q} & n = 0,\\
		\mathbb{Q}^2 & n = 1,\\
		0 & \text{else}.
	\end{cases}
	\]
	So $L(f) = 1 - \Tr(f_\ast : H_1(S_3; \mathbb{Q}) \to H_1(S_3; \mathbb{Q}))$.

	As $f \circ f = \id$, $f_\ast \circ f_\ast = \id$, so $f_\ast : H_1(S_3; \mathbb{Q}) \to H_1(S_3; \mathbb{Q})$ squares to the identity. Hence its minimal polynomial must divide $x^2 - 1$, so it is diagonalizable with eigenvalues $\pm 1$.

	So its trace is $2, 0$ or $-2$. But then $L(f) \in \{-1, 1, 3\}$, so non-zero, so $f$ has a fixed point.
\end{exbox}


\newpage

\printindex

\end{document}
