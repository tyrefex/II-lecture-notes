\documentclass[12pt]{article}

\usepackage{ishn}

\makeindex[intoc]

\begin{document}

\hypersetup{pageanchor=false}
\begin{titlepage}
	\begin{center}
		\vspace*{1em}
		\Huge
		\textbf{II Stochastic Financial Models}

		\vspace{1em}
		\large
		Ishan Nath, Michaelmas 2023

		\vspace{1.5em}

		\Large

		Based on Lectures by Dr. Michael Tehranchi

		\vspace{1em}

		\large
		\today
	\end{center}
	
\end{titlepage}
\hypersetup{pageanchor=true}

\tableofcontents

\newpage

\section{Introduction}
\label{sec:intro}

Probability and Measure is desirable; not required. We will talk about about $\sigma$-algebras later. Moreover sometimes Fubini's or the dominated convergence theorem is used, without proof.

In the following models, we will use the following assumptions. Note that these are clearly false in the real world:

\begin{itemize}
	\item No dividends.
	\item  Shares are infinitely divisible. This is a reasonable assumption for a large investor.
	\item No bid-ask spread, which is reasonable for liquid markets.
	\item No price impact of buying/selling, so our investing size is not too large.
	\item No transaction fees.
	\item No short-selling constraints.
\end{itemize}

In the following, we use the conventions:
\begin{itemize}
	\item We will have $d$ stocks (risky assets).
	\item $S^i_t$ is the price of asset $i$ at time $t$. For this course, $t \in \{0, 1\}$ or $\{0, 1, 2, \ldots\}$ or $[0,\infty]$. The first few lectures will be focused on the one-period model.
	\item $S_t = (S^1_t, \ldots, S^d_t)^T \in \mathbb{R}^d$ is the vector of asset prices.
	\item We will also assume there is a constant risk-free interest rate $r$.
\end{itemize}

\subsection{One-period Setup}
\label{sub:one_period_setup}

For one-period, we will assume that $S_0$ is not random, but $S_1$ is a random vector. Moreover will assume that there is no uncertainty in the distribution.

Our investor holds $\theta^i$ shares of asset $i$. Denote by $\theta = (\theta^1, \ldots, \theta^d)^T \in \mathbb{R}^d$ the vector of the total position of the investor.

Note that $\theta^i < 0$ means that the investor is shorting $|\theta^i|$ shares.

The investor will also put $\theta^0$ in the bank, gaining the risk-free interest rate on this investment.

Let $X_t$ denote the wealth, or liquidation value, of the portfolio at time $t$. At time $0$, this is simply
\[
X_0 = \theta^0 + \sum_{i=1}^{d} \theta^i S^i_0 = \theta^0 + \theta \cdot S_0.
\]
As time progresses, $\theta^0$ accrues interest, so
\[
X_1 = \theta^0 (1+r) + \theta \cdot S_1.
\]
Hence
\[
	X_1 = (1+r)X_0 + \theta \cdot (S_1 - (1+r) S_0). \tag{1.1}\label{eq:1.1}
\]
In general,
\[
	X_t = (1+r)X_{t-1} + \theta_t \cdot (S_t - (1+r)S_{t-1}). \tag{1.2}\label{eq:1.2}
\]

\newpage

\section{The mean-variance portfolio problem}
\label{sec:mv_problem}

This is a problem first proposed by Markowitz in 1952. There are several variations; sometimes we want to find solutions where $\theta^i \geq 0$ for all $i$, i.e. we are not allowed to short.

Given $X_0$, our goal is to find the portfolio $\theta \in \mathbb{R}^{d}$ to minimise $\Var(X_1)$, subject to $\mathbb{E}[X_1] \ge m$.

We will say that $\mathbb{E}[S_1] = \mu \in \mathbb{R}^{d}$, and $\Cov(S_1) = \mathbb{E}[(S_1 - \mu)(S_1 - \mu)^T] = V$. This covariance matrix is symmetric and non-negative definite.

For this problem assume that $\mu \neq (1+r)S_0$, and that $V$ is positive definite, hence $V^{-1}$ exists. Then, from (\ref{eq:1.1}) we get
\[
\mathbb{E}[X_1] = (1+r)X_0 + \theta \cdot (\mu - (1+r)S_0),
\]
\[
\Var(X_1) = \theta^{T} V \theta.
\]
Our problem is hence to minimize $\theta^T V \theta$ subject to $\theta^T (\mu - (1+r)S_0) \ge m - (1+r)X_0$.

We will not prove this, but note the following:

\begin{theorem}
	The optimal portfolio is
	\[
	\theta = \lambda V^{-1}(\mu - (1+r)S_0),
	\]
	where
	\[
	\lambda = \frac{(m-(1+r)X_0))^+}{(\mu - (1+r)S_0)^T V^{-1} (\mu - (1+r)S_0)}.
	\]
\end{theorem}

Here we use the convention $a^{+} = \max\{a, 0\}$.

%lecture 2

\begin{lemma}
	The minima of $\theta^{T}V\theta$ subject to $\theta^{T}a = b$ is
	\[
	\frac{b^2}{a^{T}V^{-1}a},
	\]
	where $V$ is positive definite, $a \in \mathbb{R}^{d}$, $a \neq 0$ and $b \in \mathbb{R}$ is given. The unique minimiser in $\theta = \lambda V^{-1} a$, where
	\[
	\lambda = \frac{b}{a^{T}V^{-1}a}.
	\]
\end{lemma}

\begin{proofbox}
	The first method will be taking the Lagrangian
	\[
	L(\theta, \lambda) = \frac{1}{2} \theta^{T} V\theta + \lambda(b - a^{T}\theta).
	\]
	We can minimize by setting the derivatives $0$, i.e.
	\[
	D_\theta L = V\theta - \lambda a = 0 \implies \theta = \lambda V^{-1} a.
	\]
	Pick $\lambda$ to be feasible: so
	\[
	\theta^{T} a = b \implies \lambda a^{T} V^{-1} a = b \implies \lambda = \frac{b}{a^{T}V^{-1}a}.
	\]
	This isn't very formal; let's work backwards instead. Define $\lambda = b/(a^{T}V^{-1}a)$, and suppose $\theta^{T}a = b$. Then we get
	\begin{align*}
		\theta^{T}V\theta &= \theta^{T}V\theta + 2\lambda(b - a^{T}\theta) \\
				  &= (\theta - \lambda V^{-1}a)^{T}V(\theta - \lambda V^{-1}a) + 2 \lambda b - \lambda^2 a^{T}V^{-1} a \\
				  &\geq 2 \lambda b - \lambda^2 a^{T} V^{-1} a = \frac{b^2}{a^{T}V^{-1}a}.
	\end{align*}
	Since $V$ is positive definite, we have equality if and only if $\theta = \lambda V^{-1}a$.

	Let's look at another proof. Note that
	\begin{align*}
		(\theta^{T}a)^2 &= ((V^{1/2}\theta)^{T}(V^{-1/2}a))^2 \\
				&\leq (\theta^{T}V\theta)(a^{T}V^{-1}a) \\
		\implies \theta^{T}V\theta &\geq \frac{(a^{T}\theta)^2}{a^{T}V^{-1}a} = \frac{b^2}{a^{T}V^{-1}a},
	\end{align*}
	when $a^{T}\theta = b$. We have equality in Cauchy-Schwarz if and only if $V^{1/2}\theta = \lambda V^{-1/2}a$ for some $\lambda \in \mathbb{R}$, or $\theta = \lambda V^{-1}a$.
\end{proofbox}

As an application, if $a = \mu - (1 + r)S_0$, and $b = \mathbb{E}[X_1] - (1+r)X_0$, then
\[
\Var(X_1) \geq \frac{(\mathbb{E}[X_1] - (1+r)X_0)^2}{(\mu - (1+r)S_0)^{T}V^{-1}(\mu - (1+r)S_0)},
\]
with equality if and only if $\theta = \lambda V^{-1}(\mu - (1+r)S_0)$, where
\[
\lambda = \frac{\mathbb{E}[X_1] - (1+r)X_0}{(\mu - (1+r)S_0)^{T}V^{-1}(\mu - (1+r)S_0)}.
\]

\begin{definition}
	The \emph{mean-variance efficient frontier}\index{mean-variance efficient frontier} is the lower boundary of the set of possible pairs $(\mathbb{E}[X_1], \Var(X_1))$, i.e. the points on the parabola.

	A portfolio is \emph{mean-variance efficient}\index{mean-variance efficient} if it is the optimal solution of some mean-variance portfolio problem.
\end{definition}

Recall that we wanted to minimize $\Var(X_1)$ subject to $\mathbb{E}[X_1] \geq m$. However we can figure out what do to by looking at the above. If $m \leq (1+r)X_0$ then we choose $\theta = 0$ (i.e. we don't invest in risky assets). If $m \geq (1+r)X_0$, we choose $\theta = \lambda V^{-1} (\mu - (1+r)S_0)$, where $\lambda$ is as before.

\begin{theorem}[Mutual fund theorem]\index{mutual fund theorem}
	$\theta$ is mean-variance efficient if and only if $\theta = \lambda V^{-1}(\mu - (1+r)S_0)$ for some $\lambda \geq 0$.
\end{theorem}

\begin{proofbox}
	If it is mean-variance efficient, we know it must have this form. Conversely, given $\lambda \geq 0$, let $m = (1+r)X_0 + \lambda(\mu - (1+r)S_0)^{T} V^{-1} (\mu - (1+r)S_0)$.  Then $\theta$ is the optimal solution to the mean-variance problem with target $m$.
\end{proofbox}

\subsection{Capital Asset Pricing Model}
\label{sub:capm}

This subsection will be an application of the mutual fund theorem.

\begin{theorem}[Linear regression coefficients]
	Let $X$ and $Y$ be square-integrable, with $\Var(X) > 0$. There are unique constants $a$ and $b$ such that
	\[
	Y = a + bX + Z,
	\]
	where $\mathbb{E}[Z] = 0$, and $\Cov(X, Z) = 0$.
\end{theorem}

\begin{proofbox}
	Suppose this formula holds, so $Z = Y - a - bX$. Then
	\[
	\mathbb{E}[Z] = \mathbb{E}[Y] - a - b \mathbb{E}[X],
	\]
	\[
	\Cov(X, Z) = \Cov(X, Y) - b \Var(X).
	\]
	These are two unknowns, so their unique solution to $\mathbb{E}[Z] = 0$ and $\Cov(X, Z) = 0$ is given by
	\[
	b = \frac{\Cov(X, Y)}{\Var(X)}, \qquad a = \mathbb{E}[Y] - b \mathbb{E}[X].
	\]
\end{proofbox}

\begin{definition}
	$\theta_{\mathrm{mar}} = V^{-1}(\mu - (1+r)S_0)$ is called the \emph{market portfolio}\index{market portfolio}.
\end{definition}

\begin{definition}
	Fix $X_0 > 0$ and $\theta \in \mathbb{R}^{d}$. The \emph{excess return of the portfolio}\index{excess return} $R^{\mathrm{ex}}$ is
	\[
	\frac{X_1}{X_0} - (1+r) = \frac{\theta^{T}}{X_0}(S_1 - (1+r)S_0).
	\]
	Then the excess return of the market is
	\[
		R^{\mathrm{ex}}_{\mathrm{mar}} = \frac{\theta^{T}_{\mathrm{mar}}}{X_0}(S_1 - (1+r)S_0).
	\]
\end{definition}

\begin{theorem}[$\alpha$ is zero]
	Fix $X_0 > 0$ and $\theta \in \mathbb{R}^{d}$. Let $\alpha, \beta$ be such that
	\[
		R^{\mathrm{ex}} = \alpha + \beta R^{\mathrm{ex}}_{\mathrm{mar}} + \eps,
	\]
	where $\mathbb{E}[\eps] = 0$ and $\Cov(\eps, R^{\mathrm{ex}}_{\mathrm{mar}}) = 0$. Then $\alpha = 0$.
\end{theorem}

%lecture 3

\begin{proofbox}
	This is just a calculation. We need to calculate
	\begin{align*}
	\Cov(R^{\mathrm{ex}}, R^{\mathrm{ex}}_{\mathrm{mar}}) &= \Cov \biggl( \frac{\theta^{T}}{X_0}(S_1 - (1+r)S_0), \frac{\theta^{T}_{\mathrm{mar}}}{X_0}(S_1-(1+r)S_0)\biggr) \\
											 &= \frac{\theta^{T}}{X_0}\frac{V \theta_{\mathrm{mar}}}{X_0} = \frac{\theta^{T}}{X_0^2} (\mu - (1+r)S_0) \\
											 &= \frac{\mathbb{E}[R^{\mathrm{ex}}]}{X_0}.
	\end{align*}
	Hence we get
	\[
	\Var(R^{\mathrm{ex}}_{\mathrm{max}}) = \frac{\mathbb{E}[R^{\mathrm{ex}}_{\mathrm{mar}}]}{X_0} = \Cov(R^{\mathrm{ex}}_{\mathrm{mar}}, R^{\mathrm{ex}}_{\mathrm{mar}}).
	\]
	Therefore
	\[
	\beta = \frac{\Cov(R^{\mathrm{ex}}, R^{\mathrm{ex}}_{\mathrm{mar}})}{\Var(R^{\mathrm{ex}}_{\mathrm{mar}})} = \frac{\mathbb{E}[R^{\mathrm{ex}}]}{\mathbb{E}[R^{\mathrm{ex}}_{\mathrm{mar}}]},
	\]
	and so
	\[
	\alpha = \mathbb{E}[R^{\mathrm{ex}}] - \beta \mathbb{E}[R^{\mathrm{ex}}_{\mathrm{mar}}] = 0.
	\]
\end{proofbox}

The assumptions in our model are as follows:
\begin{itemize}
	\item There are $K$ agents, where agent $k$ holds portfolio $\theta_k$.
	\item There are $n_i$ shares of asset $i$ in total. Let $n = (n_1, \ldots, n_d)^{T} > 0$.
	\item All investors are mean-variance efficient, and they agree on the mean and covariance of $S_1$.
	\item By mutual fund theorem $\theta_k = \lambda_k \theta_{\mathrm{mar}}$ for $\lambda_k \geq 0$ scalar.
	\item As supply equals demand,
		\[
		\sum_{k = 1}^{K}\theta_k = n = \Biggl( \sum_{k = 1}^{K} \lambda_k \Biggr) \theta_{\mathrm{mar}} = \Lambda \theta_{\mathrm{mar}}.
		\]
\end{itemize}

The implication is that $R^{\mathrm{ex}}_{\mathrm{mar}}$ is the excess return of a broad market index (such as S\&P 500 or FTSE), and that
\[
R^{\mathrm{ex}} = \beta R^{\mathrm{ex}}_{\mathrm{mar}} + \eps.
\]
This is a testable theory, and we can test it in the market and see that it often fails; we can find (negative) $\alpha$ in places. The reason is that some of the assumptions we have made are very strong: we assume a frictionless market without spreads and market impact, but we also assume that all other investors behave nicely.

\newpage

\section{Expected Utility Hypothesis}
\label{sec:utility}

So far, implicitly, an investor would prefer $X$ over $Y$ if either:
\begin{itemize}
	\item $\mathbb{E}[X] > \mathbb{E}[Y]$, $\Var(X) \leq \Var(Y)$, or
	\item $\mathbb{E}[X] \geq \mathbb{E}[Y]$, $\Var(X) < \Var(Y)$.
\end{itemize}

This is quite strong. Let's assume a different condition.

\begin{proposition}[Expected Utility Hypothesis]\index{expected utility hypothesis}\index{utility function}
	An agent prefers $X$ to $Y$ if and only if
	 \[
	\mathbb{E}[U(X)] \geq \mathbb{E}[U(Y)],
	\]
	for a given function $U$, called the agent's utility function.
\end{proposition}

Note that if $\hat U(x) = a + bU(x)$ for all $x$, where $a, b \in \mathbb{R}$ and $b > 0$, then the preferences induced by $\hat U$ and $U$ are the same. The von-Neumann-Morgenstern theorem characterizes the preferences equivalent to the expected utility hypothesis.

\begin{definition}
	An expected utility agent is \emph{indifferent}\index{indifferent} to $X$ and $Y$ if and only if $\mathbb{E}[U(X)] = \mathbb{E}[U(Y)]$.
\end{definition}

\subsection{Risk-aversion and Concavity}

For now, we will assume:
\begin{itemize}
	\item $U$ is strictly increasing,
		 \[
		U(x) > U(y) \iff x > y.
		\]
		The implication is, if $X \geq Y$ a.s. and $\mathbb{P}(X > Y) > 0$, then $\mathbb{E}[U(X)] > \mathbb{E}[U(Y)]$, so $X$ is preferred to $Y$.
	\item $U$ is concave,
		\[
		U(px + (1-p)y) > pU(x) + (1-p)U(y),
		\]
		for all $x \neq y$ and $0 < p < 1$.

		The implication is, if $X$ is not constant, then by Jensen's inequality
		\[
		\mathbb{E}[U(X)] < U(\mathbb{E}[X]).
		\]
		Hence $\mathbb{E}[X]$ is preferred to $X$.
\end{itemize}

%lecture 4

\subsection{Concave functions}
\label{sub:concave}

First we assume that $U$ is always increasing, strictly concave and twice differentiable.

The derivative $U'$ is called the \emph{marginal utility}\index{marginal utility}. This is positive as $U$ is increasing, but decreasing as $U$ is concave. This means that $U''$ is negative.

\begin{definition}
	The \emph{coefficient of absolute risk aversion}\index{absolute risk aversion} is
	\[
	\frac{-U''(x)}{U'(x)}.
	\]
	The \emph{coefficient of relative risk aversion}\index{relative risk aversion} is
	\[
	\frac{- x U''(x)}{U'(x)},
	\]
	when $x > 0$.
\end{definition}

\begin{exbox}
	Here are some important utility functions.
	\begin{enumerate}
		\item $U(x) = - e^{-\gamma x}$ is the CARA (constant absolute risk aversion) utility function. Indeed,
			\[
			U'(x) = \gamma e^{-\gamma x}, \qquad U''(x) = - \gamma^2 e^{-\gamma x},
			\]
			\[
			\implies \frac{U''(x)}{U'(x)} = \gamma
			\]
			is constant.
	\item $U(x) = x^{1-R}/(1-R)$, for $R > 0$, $R \neq 1$ is the CRRA (constant relative risk aversion) utility function. We can calculate that
		\[
		\frac{-x U''(x)}{U'(x)} = R.
		\]
		Note this function is not defined everywhere. To be fully define the functions, we can let
		\[
		U(x)=
		\begin{cases}
			\frac{x^{1-R}}{1-R} & x > 0,\\
			-\infty & x \leq 0.
		\end{cases}
		\]
		\item $U(x) = \log x$ is the CRRA utility function with $R = 1$.
		\item $U(x) = x$ is a utility function that is not strictly concave. This is the \emph{risk-neutral utility}.
	\end{enumerate}
\end{exbox}

Let's prove some basic facts about concave functions.

\begin{theorem}
	Concave functions are continuous, and differential concave functions have graphs that lie beneath their tangents.
\end{theorem}

To prove this we first prove a lemma:

\begin{lemma}
	Let $U$ be concave and $a < b < c$. Then,
	\[
	\frac{u(b) - u(a)}{b - a} \geq \frac{u(c) - u(b)}{c - b}.
	\]
\end{lemma}

\begin{proofbox}
	Note that
	\[
	b = \biggl( \frac{b-a}{c-a} \biggr) c + \biggl(\frac{c-b}{c-a} \biggr)a,
	\]
	and these coefficients are in $(0, 1)$. Hence applying the definition of concavity,
	\[
		u(b) \geq \biggl( \frac{b-a}{c-a} \biggr) u(c) + \biggl( \frac{c-b}{c-a} \biggr) u(a).
	\]
	Rearranging gives what we want.
\end{proofbox}

Now we can apply this lemma to prove continuity.

\begin{proofbox}
	Fix $x$, $0 < \eps < t$. Apply the lemma to
	\[
	x-t < x - \eps < x < x + \eps < x + t.
	\]
	Then we get that
	\[
	\frac{\eps}{t} (u(x) - u(x-t)) \geq u(x) - u(x-\eps) \geq u(x+\eps)-u(x) \geq \frac{\eps}{t} (u(x+t) - u(x)),
	\]
	which goes to $0$ as $\eps \to 0$.

	Moreover if $u$ is differentiable at $x$, then we have
	\[
	u(x) - u(x-t) \geq t u'(x) \geq u(x+t) - u(x),
	\]
	by taking $\eps \to 0$. This gives
	\[
	u(y) \leq u(x) + u'(x) (y-x).
	\]
	Note if $u$ is twice differentiable, we can use second-order Taylor's:
	\[
	u(y) = u(x) + u'(x) (y-x) + u''(z) \frac{(y-x)^2}{2},
	\]
	where $u''(z) \leq 0$.
\end{proofbox}

\begin{theorem}
	Increasing concave functions are unbounded to the left, i.e. $u(x) \to -\infty$ as $x \to -\infty$.
\end{theorem}

\begin{proofbox}
	Pick $x < a < b$. Then,
	\[
	u(x) \leq u(a) + \biggl( \frac{x-a}{b-a} \biggr) (u(b) - u(a)),
	\]
	which goes to $-\infty$ as $x \to -\infty$
\end{proofbox}

\subsection{Optimal Investment and Marginal Utility}
\label{sub:optimal_invest}

\begin{theorem}
	Consider the problem of maximizing $\mathbb{E}[U(X_1)]$ given $X_0$, where
	\[
		X_1 = (1+r)X_0 + \theta^{T}(S_1 - (1+r)S_0),
	\]
	where $U$ is increasing, concave and suitable. If $\theta^{\ast}$ is optimal, then
	\[
	S_0 = \frac{1}{1+r} \frac{\mathbb{E}[U'(X_1^{\ast})S_1]}{\mathbb{E}[U'(X_1^{\ast})]},
	\]
	where
	\[
	X_1^{\ast} = (1+r)X_0 + (\theta^{\ast})^{T}(S_1 - (1+r)S_0).
	\]
\end{theorem}

%lecture 5

\begin{proofbox}
	Let 
	\[
	f(\theta) = \mathbb{E}[U((1+r)X_0) + \theta^{T}(S_1-(1+r)S_0))].
	\]
	At the maximum,
	\[
	Df(\theta^{\ast}) = 0 = \mathbb{E}[U'(X_1^{\ast})(S_1-(1+r)S_0)].
	\]
	This rearranges to what we want.
\end{proofbox}

\newpage

\section{Pricing Contingent Claims}
\label{sec:price_cont}

For us, a contingent claim is a random variable (the payout is random). The payout for us is at time 1. We will introduce the following framework:
\begin{itemize}
	\item There are $d$ assets given (fundamental).
	\item We introduce a new contingent claim with payout $Y$.
	\item Often $Y = g(S_1)$, a function of the price of the stocks at time 1, but not necessarily.
\end{itemize}
We are interested in finding a good time $0$ price of the claim.

The most important example is a \emph{call option}\index{call option}: the right but not the obligation to buy some given asset at a fixed price $K$, often called the \emph{strike price}\index{strike price}.

We will assume that the actors are rational. That is, if at time 1 the asset has price greater than $K$, then the buyer will exercise the option and sell the asset on the market, making $S_1 - K$. If the asset has price less than $K$, then the buyer will not exercise the right. This gives a payout
\begin{align*}
	\text{payout} &= 
	\begin{cases}
		S_1 - K & S_1 > K,\\
		0 & S_1 \leq K,
	\end{cases}
	\\
		      &= (S_1-K)^{+} = g(S_1).
\end{align*}

\subsection{Utility Indifference Pricing}
\label{sub:u_price}

Fix $X_0$ to be an initial wealth, and $U$ an increasing concave utility function. Let
\[
	\mathcal{X} = \{X_1 = (1+r)X_0 + \theta^{T}(S_1 - (1+r)S_0) \mid \theta \in \mathbb{R}^{d} \}.
\]
Note that the investor would prefer to buy one claim with payout $Y$ and price $\pi$ if there exists $X_1^{\ast} = X^{\ast}$ such that
\[
\mathbb{E}[U(X^{\ast} + Y - (1+r)\pi)] > \mathbb{E}[U(X)],
\]
for all $X \in \mathcal{X}$.

We will assume $U$, the distribution of $S_1$, $X_0$, $Y$, the values $r, S_0$ and $\pi$ are such that every utility maximisation problem has an optimal solution.

\begin{definition}
	The (utility) \emph{indifference (reservation) price}\index{indifference price} of the claim with payout $Y$ is any $\pi$ such that
	\[
	\max_{X \in \mathcal{X}}\mathbb{E}[U(X+Y-(1+r)\pi)] = \max_{X \in X} \mathbb{E}[U(X)].
	\]
\end{definition}

Here are some properties of the indifference price. First, our assumptions lead to the following.

\begin{theorem}
	The indifference price exists and is unique.
\end{theorem}

From this, we let $\pi(Y)$ denote the indifference price of the claim with payout $Y$. Now we will relate indifference prices of different claims.

\begin{theorem}
	If $Y_0 \leq Y_1$ almost surely and $\mathbb{P}(Y_0 < Y_1) > 0$, then $\pi(Y_0) < \pi(Y_1)$.
\end{theorem}

Moreover, we can show that indifference prices are concave:
\begin{theorem}[Indifference prices are concave]
	\[
	\pi(pY_1 + (1-p)Y_0) \geq p \pi(Y_1) + (1-p)\pi(Y_0),
	\]
	for all payouts $Y_0, Y_1$ and all $0 < p < 1$.
\end{theorem}

\begin{definition}
	The \emph{marginal utility price}\index{marginal utility price} of $Y$ is
	\[
	\pi_0(Y) = \frac{\mathbb{E}[U'(X^{\ast})Y]}{(1+r)\mathbb{E}[U'(X^{\ast})]},
	\]
	where $X^{\ast}$ solves
	\[
	\mathbb{E}[U(X^{\ast})] = \max_{X \in \mathcal{X}} \mathbb{E}[U(X)].
	\]
\end{definition}

Now we have a nice little theorem.

\begin{theorem}
	$\pi(Y) \leq \pi_0(Y)$ for any claim.
\end{theorem}

In fact, we can find the following.

\begin{theorem}
	\[
	\lim_{\eps \to 0} \frac{\pi(\eps Y)}{\eps} = \pi_0(Y).
	\]
\end{theorem}

%lecture 6

To prove this, we introduce the following definition:
\begin{definition}
	We define
	\[
	V(Z) = \max_{X \in \mathcal{X}} \mathbb{E}[U(X+Z)].
	\]
	This is the \emph{indirect utility}\index{indirect utility} of the random payout $Z$.
\end{definition}

In this notation, $\pi$ is the indifference price of payout $Y$ if and only if
\[
V(Y - (1 + r)\pi) = V(0).
\]

\begin{lemma}
	If $Z_0 \leq Z_1$ almost surely, and $\mathbb{P}(Z_0 < Z_1) > 0$, then $V(Z_0) < V(Z_1)$.
\end{lemma}

\begin{proofbox}
	Let $X^i$ be such that
	\[
	V(Z_i) = \mathbb{E}[U(X^i + Z_i)],
	\]
	for $i = 0, 1$. Then
	\begin{align*}
		V(Z_1) &= \mathbb{E}[U(X^1 + Z_1)] \geq \mathbb{E}[U(X^0 + Z_1)] \text{ as $X^1$ is a maximizer}\\
		       &> \mathbb{E}[U(X^0 + Z_0)] \text{ as  } Z_1 \geq Z_0 \\
		       &= V(Z_0).
	\end{align*}
\end{proofbox}

\begin{lemma}
	The indirect utility satisfies
	\[
	V(pZ_1 + (1-p)Z_0) \geq p V(Z_1) + (1-p)Z_0,
	\]
	for all $0 < p < 1$.
\end{lemma}

\begin{proofbox}
	Let $X^i$ be as before. Note that
	\begin{align*}
		p X^1 + (1-p) X^0 &= p((1 + r)X_0 + \theta_1^{T}(S_1 - (1_r)S_0)) \\
				  &\qquad\qquad+ (1-p)((1+r)X_0 + \theta_0^{T}(S_1 - (1+r)S_0)) \\
		&= (1+r)X_0 + (p \theta_1 + (1-p)\theta_0)^{T}(S_1 - (1+r)S_0) \in \mathcal{X},
	\end{align*}
	so we have
	\begin{align*}
		p V(Z_1) + (1-p) V(Z_0) &= \mathbb{E}[p U(X^1 + Z_1) + (1-p)U(X^0 + Z_0)] \\
					&\leq \mathbb{E}[U(pX^1 + (1-p)X^0 + p Z_1 + (1-p) Z_0)] \\
					&\leq \max_{X \in \mathcal{X}} \mathbb{E}[U(X + p Z_1 + (1-p) Z_0)] \\
					&= V(pZ_1 + (1-p)Z_0).
	\end{align*}
\end{proofbox}

Now we can prove the existence and uniqueness of indifference prices.

\begin{proofbox}
	We show there is a unique solution to $V(Y - (1+r)\pi) = V(0)$. Note that
	\[
	V(0) = \mathbb{E}[U(X^{\ast})] \in (U(-\infty), U(\infty)) = (-\infty, U(\infty)).
	\]
	Now consider $x \mapsto V(Y + x)$. This is strictly increasing and concave, by our lemma. Hence it is continuous (by concavity), with range $(\infty, \lim V(Y+x))$.

	But by MCT (actually not entirely sure, Fatou and MCT require non-negativity),
	\[
	V(Y + x) \geq \mathbb{E}[U(X^\ast + Y + x)] \uparrow U(\infty).
	\]
	Hence there exists a unique $x$ such that $V(Y + x) = V(0)$, and the unique $\pi$ is $-x/(1+r)$.
\end{proofbox}

Now we can prove that indifference prices increase.

\begin{proofbox}
	Let $Y_0 \leq Y_1$ almost surely, and $\mathbb{P}(Y_0 < Y_1) > 0$. Then
	\begin{align*}
		V(Y_0 - (1+r)\pi(Y_0)) &= V(0) = V(Y_1 - (1+r) \pi(Y_1)) \\
				       & \geq V(Y_0 - (1+r)\pi(Y_1)),
	\end{align*}
	and so $-(1+r)\pi(Y_0) > -(1+r)\pi(Y_1)$ since $x \mapsto V(Y_0 + x)$ is strictly increasing. Hence $\pi(Y_1) > \pi(Y_0)$.
\end{proofbox}

Here is a proof of concavity.

\begin{proofbox}
	We are given $Y_0, Y_1$ and $0 < p < 1$. Then
	\begin{align*}
		V(pY_1 + (1&-p)Y_0 - (1+r)\pi(pY_1 + (1-p)Y_0)) = V(0) \\
				    &= V(Y_1 - (1+r)\pi(Y_1)) = V(Y_0 - (1+r)\pi(Y_0)) \\
							       &= p V(Y_1 - (1+r)\pi(Y_1)) + (1-p) V(Y_0 - (1+r)\pi(Y_0)) \\
							       &\leq V(pY_1 + (1-p)Y_0 - (1+r)(p \pi(Y_1) + (1-p) \pi(Y_0))),
	\end{align*}
	hence $\pi(pY_1 + (1-p)Y_0) \geq p\pi(Y_1) + (1-p)\pi(Y_0)$.
\end{proofbox}

Now we prove the marginal utility price is bigger than the indifference price.

\begin{proofbox}
	Let $X^{\ast}$ be such that $V(0) = \mathbb{E}[U(X^{\ast})]$. Let $X^1$ be such that $V(Y - (1+r)\pi(Y)) = \mathbb{E}[U(X^1 + Y - (1+r)\pi(Y_0))]$. Then,
	\begin{align*}
		V(0) &= V(Y - (1+r)\pi(Y)) = \mathbb{E}[U(X^1 + Y - (1+r)\pi(Y))] \\
		     &\leq \mathbb{E}[U(X^{\ast}) + U'(X^{\ast})(X^1 - X^{\ast} + Y - (1+r)\pi(Y))] \\
		     &= V(0) + \mathbb{E}[U'(X^{\ast})(X^1 - X^{\ast})] + \mathbb{E}[U'(X^{\ast})(Y - (1+r)\pi(Y))].
	\end{align*}
	Now we claim the bit in the middle is 0. Assuming this, we have
	\[
	\pi(Y) \leq \frac{1}{1 + r} \frac{\mathbb{E}[U'(X^{\ast})Y]}{\mathbb{E}[U'(X^{\ast})]} = \pi_0(Y).
	\]
	To prove this claim note that
	\begin{align*}
		\mathbb{E}[U'(X^{\ast})(X^{\ast} - X^1)] &= (\theta^{\ast} - \theta^1)^{T} \mathbb{E}[U'(X^{\ast}) (S_1 - (1+r)S_0)] = 0,
	\end{align*}
	by the first marginal utility pricing result.
\end{proofbox}

%lecture 7

We now prove the convergence of the indifference price to the marginal utility price.

\begin{proofbox}
	Fix $Y$, and let $\pi_t = \pi(tY)/t$, the average indifference price of $t$ shares. Let
	\[
	p= \sup_{t > 0} \pi_t = \lim_{t \to 0} \pi_t,
	\]
	from the example sheet, where we show $t \mapsto \pi_t$ is decreasing. This is done by using concavity of $\pi$, and the fact $\pi(0) = 0$. Let $X^{\ast} \in \mathcal{X}$ maximize $\mathbb{E}[U(X)]$, as before. Then
	\begin{align*}
		0 &= \frac{V(tY - (1+r)t\pi_t) - V(0)}{t} \\
		  &\geq \frac{1}{t} \mathbb{E}[U(X^{\ast} + t(Y - (1+r)\pi_t)) - U(X^{\ast})] \\
		  &\geq \mathbb{E}\biggl[ \frac{U(X^{\ast} + t(Y - (1+r)p)) - U(X^{\ast})}{t} \biggr] \\
		  &\to \mathbb{E}[U'(X^{\ast})(Y - (1+r)p)],
	\end{align*}
	hence we get
	\[
	p \geq \pi_0 = \frac{\mathbb{E}[U'(X^{\ast})Y]}{(1+r)\mathbb{E}[U'(X^{\ast})]}.
	\]
	But on the other hand,
	\[
	\pi_t \leq \pi_0 \implies p \leq \pi_0,
	\]
	hence $p = \pi_0$.
\end{proofbox}

\newpage

\section{Risk Neutral Measures}
\label{sec:risk_0}

\begin{definition}
	Consider two probability measures $\mathbb{P}$ and $\mathbb{Q}$, defined on a $\sigma$-algebra $\mathcal{F}$.

	Then $\mathbb{P}$ and $\mathbb{Q}$ are equivalent if and only if $\mathbb{P}(A) = 0 \iff \mathbb{Q}(A) = 0$.
\end{definition}

\begin{remark}
 	Analogously, $\mathbb{P} \sim \mathbb{Q}$ if and only if $\mathbb{P}(A) = 1 \iff \mathbb{Q}(A) = 1$, or $\mathbb{P}(A) > 0 \iff \mathbb{Q}(A) > 0$ or $\mathbb{P}(A) < 1 \iff \mathbb{Q}(A) < 1$.
\end{remark}

\begin{exbox}
	Take $\Omega = \{1, 2, 3\}$ and $\mathcal{F} = \mathcal{P}(\Omega)$. Let
	\[
	\mathbb{P}(1) = \frac{1}{2}, \qquad \mathbb{P}(2) = \frac{1}{2}, \qquad \mathbb{P}(3) = 0,
	\]
	\[
	\mathbb{Q}(1) = \frac{1}{3}, \qquad \mathbb{Q}(2) = \frac{2}{3}, \qquad \mathbb{Q}(3) = 0.
	\]
	Then $\mathbb{P} \sim \mathbb{Q}$.

	Moreover, let $Z$ be a random variable such that $\mathbb{P}(Z > 0) = 1$, and such that $\mathbb{E}[Z] = 1$. Define $\mathbb{Q}(A) = \mathbb{E}[Z \mathbbm{1}_{A}]$.

	Then $\mathbb{Q}$ is a probability measure on $(\Omega, \mathcal{F})$, and $\mathbb{P} \sim \mathbb{Q}$.

	Indeed, if  $\mathbb{P}(A) = 0$, then $\mathbb{E}[\mathbbm{1}_{A}] = 0$, and so $\mathbb{Q}(A) = \mathbb{E}[Z \mathbbm{1}_{A}] = 0$ by MCT.

	Conversely, if $\mathbb{Q}(A) = 0$, then $\mathbb{E}[Z \mathbbm{1}_{A}] = 0$, so $\mathbb{P}(Z \mathbbm{1}_{A} = 0) = 1$. But since $\mathbb{P}(Z = 0) = 0$, we must have $\mathbb{P}(\mathbbm{1}_{A} = 0) = 0$, hence $\mathbb{P}(A) = 0$.
\end{exbox}

In fact, this is the only such example.

\begin{theorem}[Radon-Nikodym]
	$\mathbb{P} \sim \mathbb{Q}$ if and only if there exists random variable $Z$ with respect to $\mathbb{P}$ such that $\mathbb{P}(Z > 0) = 1$, $\mathbb{E}[Z] = 1$ and
	\[
	\mathbb{Q}(A) = \mathbb{E}[Z \mathbbm{1}_{A}],
	\]
	for all $A \in \mathcal{F}$.
\end{theorem}
The proof of this is in part III advanced probability.

For such $\mathbb{P}, \mathbb{Q}$, we define
\[
\frac{\diff \mathbb{Q}}{\diff \mathbb{P}} = Z.
\]
This is the \emph{Radon-Nikodym derivative}\index{Radon-Nikodym derivative}.

In the first example, $Z(1) = 2/3$, $Z(2) = 4/3$ and $Z(3) = e^2 - \pi$ gives $\mathbb{Q}(A) = \mathbb{P}(Z \mathbbm{1}_{A})$.

\begin{remark}
	If $\mathbb{P} \sim \mathbb{Q}$ and $\frac{\diff \mathbb{Q}}{\diff \mathbb{P}} = Z$, then $\mathbb{E}_{\mathbb{Q}}[X] = \mathbb{E}_{\mathbb{P}}[Z X]$, for any $\mathbb{Q}$-integrable $X$.
\end{remark}

\begin{exbox}
	Let $X$ be defined on $(\Omega, \mathcal{F}, \mathbb{P})$ and $X \sim \Exp(\lambda)$ under $\mathbb{P}$. Then let
	\[
	Z = \frac{u}{\lambda} e^{(\lambda - u)x}, \qquad \mathbb{E}[Z] = \int_0^{\infty} \frac{u}{\lambda} e^{(\lambda - u)x} \lambda e^{-\lambda x} \diff x = \int_0^{\infty} u e^{-u x} \diff x = 1.
	\]
	So if we define $\mathbb{Q}$ by density $Z$, then we want to find the distribution of $X$ under $\mathbb{Q}$. Note that
	\begin{align*}
		\mathbb{E}_{\mathbb{Q}}[f(X)] &= \mathbb{E}_{\mathbb{P}}[Z f(X)] = \int_0^{\infty} \frac{u}{\lambda} e^{(\lambda - u)x} f(x) \lambda e^{-\lambda x} \diff x \\
					      &= \int_0^{\infty} u e^{-u x} f(x) \diff x,
	\end{align*}
	so $X \sim \Exp(u)$ under $\mathbb{Q}$.
\end{exbox}

\begin{definition}
	In a financial model, any measure $\mathbb{Q}$ equivalent to $\mathbb{P}$ such that
	\[
	S_0 = \frac{\mathbb{E}^{\mathbb{Q}}[S_1]}{1 + r},
	\]
	is called \emph{risk-neutral}\index{risk-neutral measure}.
\end{definition}

%lecture 8

\begin{remark}
	\begin{itemize}
		\item[]
		\item The definition of a risk-neutral measure does not depend on the agent's equality.
		\item It does depend on $\mathbb{P}$.
		\item Agents with equivalent beliefs (i.e. agree on the probability 0 events) agree on the risk-neutrality of a measure.
	\end{itemize}
\end{remark}

\begin{theorem}[Marginal Utility Pricing $Z$]
	Let $X^{\ast} \in \mathcal{X}$ maximize $\mathbb{E}[U(X)]$ over $X \in \mathcal{X}$, where $U$ is strictly increasing, differentiable and suitably nice. Let $\mathbb{Q}$ be defined by
	\[
	\frac{\diff \mathbb{Q}}{\diff \mathbb{P}} = \frac{U'(X^{\ast})}{\mathbb{E}[U'(X^{\ast})]}.
	\]
	Then $\mathbb{Q}$ is a risk-neutral measure.
\end{theorem}

We can also say that
\[
\frac{\diff \mathbb{Q}}{\diff \mathbb{P}} \propto U'(X^{\ast}).
\]

\begin{proofbox}
	Note that
	\[
	\frac{\mathbb{E}_{\mathbb{Q}}[S_1]}{1+r} = \frac{\mathbb{E}_{\mathbb{P}}[U'(X^{\ast})S_1]}{(1+r) \mathbb{E}_{\mathbb{P}}[U'(X^{\ast})]} = S_0,
	\]
	by the first marginal utility pricing theorem.
\end{proofbox}

\subsection{Arbitrage}
\label{sub:arbitrage}

We define the concept of arbitrage, which will be used in the fundamental theorem of asset pricing: there exists a risk-neutral measure if and only if there does not exist an arbitrage.

\begin{definition}
	An \emph{arbitrage}\index{arbitrage} $\phi \in \mathbb{R}^{d}$ is such that $\mathbb{P}(\phi^{T}(S_1 - (1+r)S_0) \geq 0) = 1$, and $\mathbb{P}(\phi^{T}(S_1 - (1+r)S_0) > 0) > 0$.
\end{definition}

\begin{remark}
	\begin{itemize}
		\item[]
		\item The definition of arbitrage does not depend on $U$.
		\item It depends on $\mathbb{P}$, but only through the events of measure $0$.
	\end{itemize}
\end{remark}

Given $X \in \mathcal{X}$, and an arbitrage $\phi$, let
\[
X' = X + \phi^{T}(S_1 - (1+r)S_0) \in \mathcal{X}.
\]
Then $U(X') \geq U(X)$ almost surely, and $\mathbb{P}(U(X') > U(X)) > 0$. Hence $\mathbb{E}[U(X')] > \mathbb{E}[U(X)]$, so $X'$ is preferred to $X$.

Hence there cannot be a maximiser of $\mathbb{E}[U(X)]$. Indeed, if $\phi$ is an arbitrage, then so is $n \phi$ for all $n \geq 0$.

Hence, if there exists a utility maximiser, then there exists a risk-neutral measure, and if there exists a utility maximiser, then there is no arbitrage.

\begin{theorem}[Fundamental Theorem of Asset Pricing]
	There exists a risk-neutral measure if and only if there does not exist an arbitrage.
\end{theorem}

\begin{proofbox}
	Suppose there exists a risk-neutral measure. To show there is no arbitrage, let $\phi \in \mathbb{R}^{d}$ be such that $\phi^{T}(S_1 - (1+r)S_0) \geq 0$ $\mathbb{P}$-almost surely. Then by equivalence, $\phi^{T}(S_1 - (1+r)S_0) \geq 0$ $\mathbb{Q}$-almost surely. But
	\[
	\mathbb{E}_{\mathbb{Q}}[\phi^{T}(S_1 - (1+r)S_0)] = \phi^{T}\mathbb{E}_{\mathbb{Q}}[S_1 - (1+r)S_0] = 0,
	\]
	by risk-neutrality. Therefore $\phi^{T}(S_1 - (1+r)S_0) = 0$ $\mathbb{Q}$-almost surely, and hence $\mathbb{P}$-almost surely. So $\phi$ is not an arbitrage.

	Now we show the backwards direction. Assume there is no arbitrage; then we will find a solution to the utility maximisation problem.

	Without loss of generality, assume $\mathbb{E}_{\mathbb{P}}[e^{\theta \cdot S_1}] < \infty$ for all $\theta \in \mathbb{R}^d$. Otherwise, replace $\mathbb{P}$ with $\mathbb{P}'$ where
	\[
	\frac{\diff \mathbb{P}'}{\diff \mathbb{P}} \propto e^{- \|S_1\|^2}.
	\]
	Let
	\[
		f(\theta) = \mathbb{E}[e^{-\theta^T Z}] = - \mathbb{E}[U(X)],
	\]
	where $U(x) = -e^{-x}$, $Z = S_1 - (1+r)S_0$ and $X = \theta^{T} Z$. Let $\theta_k$ be a minimizing sequence, so $f(\theta_k) \to \inf f(\theta)$.
	\begin{enumerate}
		\item If $(\theta_k)$ is bounded, by Bolzano-Weierstrass, there exists a convergent subsequence. So assume $\theta_k \to \theta^{\ast}$. As $f$ is continuous, $f(\theta_k) \to f(\theta^{\ast})$, so $\theta^{\ast}$ is a minimizer. Then
			\[
			\frac{\diff \mathbb{Q}}{\diff \mathbb{P}} \propto e^{-(\theta^{\ast})^{T} (S_1 - (1+r)S_0)}
			\]
			is risk-neutral.

			%lecture 9
		\item Now assume all minimising sequences are unbounded. Recall we assumed there was no arbitrage, meaning if $\phi^{T} Z \geq 0$ almost surely, then $\phi^{T} Z = 0$ almost surely.

			Now I claim we may assume $Z^1, \ldots, Z^d$ are linearly independent, meaning if $\phi^{T}Z = 0$ almost surely, then $\phi = 0$. If not, then we can restrict to a linearly independent submarket, then there is still no arbitrage in the submarket.

			Let $(\theta_k)$ be a minimising sequence with $\|\theta_k\| \to \infty$. Then let $\phi_k = \theta_k/\|\theta_k\|$, so by Bolzano-Weierstrass, we assume $\phi_k \to \phi_0$, where $\|\phi_0\| = 1$. Now we will show that $\phi_0^{T}Z \geq 0$ almost surely. Assuming this, no-arbitrage would imply that $\phi_0^{T}Z = 0$ almost surely, but by linear independence we have $\phi_0 = 0$, which contradicts $\|\phi_0\| = 1$.

			We will now prove this claim. Since $0$ is not a minimiser, $f(0) = 1 \geq f(\theta_k)$ for $k$ large enough. But
			\[
				f(\theta_k) = \mathbb{E}[e^{-\theta_k^{T}Z}] \geq \mathbb{E}(e^{-(\phi_k^{T}Z)\|\theta_k\|} \mathbbm{1}_{\{\phi_0^{T}Z < -\eps, \|Z\| < r\}}]
			\]
			for arbitrary $\eps > 0$, $r > 0$. Now note
			\[
			\phi_k^{T} Z = (\phi_k - \phi_0)^{T} Z + \phi_0^{T} Z \leq \|\phi_k - \phi_0\|r - \eps \leq - \frac{\eps}{2},
			\]
			by choosing $k$ large enough. Hence
			\[
			1 \geq f(\theta_k) \geq e^{\eps \|\theta_k\|/2} \mathbb{P}(\phi_0^{T} Z < -\eps, \|Z\| < r),
			\]
			for all $k \geq N$. Hence
			\[
			\mathbb{P}(\phi_0^{T} Z < -\eps, \|Z\| < r) \leq e^{-\eps/2 \|\theta_k\|} \to 0
			\]
			as $k \to \infty$. Hence taking $r \to \infty$ and $\eps \to 0$, $\mathbb{P}(\phi_0^{T} Z < 0) = 0$.
	\end{enumerate}
\end{proofbox}

\subsection{No-arbitrage Pricing of Contingent Claims}
\label{sub:no_arb_price}

\begin{theorem}
	Given a market with prices $S_0, S_1$ and interest $r$, assume there is no arbitrage. Introduce a claim with payout $Y$ and initial price $\pi$.

	There is no arbitrage in the augmented market if and only if there exists $\mathbb{Q}$ risk-neutral for the original market such that
	\[
	\pi = \frac{1}{1+r} \mathbb{E}_{\mathbb{Q}}[Y].
	\]
	Then the set of all possible no-arbitrage prices $\pi$ is an interval.
\end{theorem}

\begin{proofbox}
	The first part is the fundamental theorem of asset pricing.

	For the last part, we show the possible set of prices is convex. Let
	\[
	\pi_i = \frac{\mathbb{E}_{\mathbb{Q}_i}[Y]}{1 + r},
	\]
	for $i = 0, 1$ and risk-neutral measures $\mathbb{Q}_i$. Let $0 < p < 1$, then let $\mathbb{Q}_p$ have density
	\[
	p \frac{\diff \mathbb{Q}_1}{\diff \mathbb{P}} + (1-p) \frac{\diff \mathbb{Q}_0}{\diff \mathbb{P}} = \frac{\diff \mathbb{Q}_p}{\diff \mathbb{P}}.
	\]
	Note that
	\[
	\frac{\mathbb{E}_{\mathbb{Q}_p}[S_1]}{1 + r} = p \frac{\mathbb{E}_{\mathbb{Q}_1}[S_1]}{1 + r} + (1-p) \frac{\mathbb{E}_{\mathbb{Q}_0}[S_1]}{1+r} = S_0,
	\]
	so $\mathbb{Q}_p$ is risk-neutral. Then
	\[
		p \pi_1 + (1-p) \pi_0 = \frac{\mathbb{E}_{\mathbb{Q}_p}[Y]}{1 + r}
	\]
	is a no-arbitrage price of the claim.
\end{proofbox}

\begin{remark}
	The marginal utility price is a no-arbitrage price, since $U'(X_1^{\ast})$ is proportional to the density of a risk-neutral measure.
\end{remark}

\subsection{Attainable Claims}
\label{sub:att_claim}

\begin{definition}
	A claim is \emph{attainable}\index{attainable claim} if there exists $a \in \mathbb{R}$ and $b \in \mathbb{R}^d$ such that
	\[
	Y = a + b^{T} S_1 = (1+r)x  + b^{T}(S_1 - (1+r)S_0),
	\]
	where
	\[
	x = \frac{a}{1 + r} + b^{T}S_0.
	\]
\end{definition}

For an attainable claim, the indifference price is $x$, for all $U$ and $X_0$. Hence the marginal utility price is the same for all $U$ and $X_0$.

\begin{theorem}[Uniqueness of No-Arbitrage Price]
	If $Y$ is an attainable claim, then there is a unique no-arbitrage price.

	Conversely, if $Y$ has a unique no-arbitrage price, then it is attainable.
\end{theorem}

\begin{proofbox}
	For the first part, suppose $Y = a + b^{T}S_1$. Then
	\[
	\frac{1}{1 + r} \mathbb{E}_{\mathbb{Q}}[Y] = \frac{a}{1 + r} + b^{T}S_0
	\]
	for all risk-neutral measures $\mathbb{Q}$.

	The second part is in the example sheets.
\end{proofbox}

%lecture 10

We look at some examples of attainable claims.

\begin{exbox}
	Let's look at the forward contract, which is the obligation to buy the stock at time $1$ for a fixed price $K$.

	Then the payout is $Y = S_1 - K$. Therefore the unique no arbitrage price is
	\[
	S_0 - \frac{k}{1 + r}.
	\]
	In reality, we should account for dividends, which are incorporated in the time $0$ price, but not the time $1$ price.

	Moreover, in real markets, $K$ is set to that the time 0 price is $\pi = 0$, i.e. $K = (1+r)S_0$.
\end{exbox}

\begin{exbox}
	Let's look at the one-period binomial model, with $-1 < a < b$, $0 < p < 1$ and interest rate $r$ given. Then
	\[
	S_1 =
	\begin{cases}
		S_0(1+b) &\text{with probability } p, \\
		S_0(1+a) &\text{with probability } 1-p.
	\end{cases}
	\]
	Consider a claim with payout $Y = g(S_1)$. This is attainable, since if we let
	\[
	X_1 = (1+r) X_0 + \theta(S_1 - (1+r)S_0) = g(S_1),
	\]
	then we must have
	\begin{align*}
		(1 + r)X_0 + \theta S_0(b - r) = g(S_0(1+b)),\\
		(1+r)X_0 + \theta S_0(a - r) = g(S_0(1+a)).
	\end{align*}
	Hence we can find
	\[
	\theta = \frac{g(S_0(1+b)) - g(S_0(1+a))}{S_0(b-a)},
	\]
	and also $X_0$. There is only no arbitrage in the model if $a < r < b$. Let $\mathbb{Q}$ be a risk-neutral probability with
	\[
	\mathbb{Q}(S_1 = (1+b)S_0) = q = 1 - \mathbb{Q}(S_1 - (1+a)S_0).
	\]
	Then we have
	\[
	q S_0(1+b) + (1-q)S_0 (1+a) = \mathbb{E}_{\mathbb{Q}}[S_1] = (1+r)S_0,
	\]
	hence we get
	\[
	q = \frac{r - a}{b - a}.
	\]
	Note that $a < r < b \iff 0 < q < 1$. Now to find $X_0$, note that
	\[
	\mathbb{E}_{\mathbb{Q}}[g(S_1)] = \mathbb{E}_{\mathbb{Q}}[(1+r)S_0 + \theta(S_1 - (1+r)S_0)] = (1+r)X_0,
	\]
	hence the no-arbitrage price is
	\[
	\pi = \frac{1}{1 + r} \mathbb{E}_{\mathbb{Q}}[g(S_1)] = \frac{q g(S_0(1+b)) + (1-q) g(S_0(1+a))}{1 + r}.
	\]
\end{exbox}

\newpage

\section{Multi-Period Models}
\label{sec:mult_models}

The challenge for multi-period models is to keep track of the information that is revealed at each time period. To do this, we use a filtered probability space: we will let $\mathcal{F}_0, \mathcal{F}_1, \ldots = (\mathcal{F}_n)$ be the $\sigma$-algebras of information available at time $n$. We assume that $\mathcal{F}_n \subseteq \mathcal{F}_{n+1}$.

To informally motivate why we consider $\sigma$-algebras, we let $\mathcal{G}$ be some set of information. Then we can define $\mathbb{P}(A \mid \mathcal{G})$, which is the conditional probability of $A$ happening given $\mathcal{G}$.

We say that $A$ is $\mathcal{G}$-measurable if $\mathbb{P}(A \mid \mathcal{G}) \in \{0, 1\}$.

\begin{exbox}
	Let $\Omega = \{HH, HT, TH, TT\}$ be the outcome of two coin flips, and $\mathcal{G}$ be the information available after the first flip.

	Then $A = \{HH, HT\}$ is $\mathcal{G}$-measurable, because we know whether it happened or not after the first flip, whereas $B = \{HH\}$ is not $\mathcal{G}$-measurable. If the first flip is heads, we still have to know about the second flip.
\end{exbox}

We can then define our $\sigma$-algebra to be the collection of measurable events (we assume it is a $\sigma$-algebra).

\subsection{Measurability}
\label{sub:meas}

Consider a probability space $(\Omega, \mathcal{F}, \mathbb{P})$ and $\mathcal{G} \subseteq \mathcal{F}$ a sub-$\sigma$-algebra.

\begin{definition}
	A $\in \mathcal{F}$ is $\mathcal{G}$-measurable $iff A \in \mathcal{G}$.

	A random variable $X$ is $\mathcal{G}$-measurable if and only if $\{X \leq x\}$ is $\mathcal{G}$-measurable for all $x \in \mathbb{R}$.
\end{definition}

\begin{remark}
	If $X$ is $\mathcal{G}$-measurable, then given $\mathcal{G}$ we know the value of $X$.
\end{remark}

\begin{exbox}
	Let $\Omega$ be our collection of two coin flips again, and $\mathcal{G} = \{\emptyset, \Omega, \{HH, HT\}, \{TT, TH\}\}$. Define
	\[
	X(\omega) =
	\begin{cases}
		1 & \omega \in \{HH, HT\}, \\
		0 &\text{else}.
	\end{cases}
	\]
	Then $X$ is $\mathcal{G}$-measurable. However,
	\[
	Y(\omega) =
	\begin{cases}
		2 & \omega = HH,\\
		1 & \omega \in \{HT, TH\},\\
		0 & \omega = TT,
	\end{cases}
	\]
	then $Y$ is not $\mathcal{G}$-measurable.
\end{exbox}

\begin{definition}
	The $\sigma$-algebra generated by $X$, denoted $\sigma(X)$, is the smallest $\sigma$-algebra containing $\{X \leq x\}$ for all $x \in \mathbb{R}$.
\end{definition}

\begin{theorem}
	$Y$ is $\sigma(X)$-measurable if and only if $Y = f(X)$ for a measurable function $f : \mathbb{R} \to \mathbb{R}$.
\end{theorem}


\newpage

\printindex

\end{document}
