\documentclass[12pt]{article}

\usepackage{ishn}

\makeindex[intoc]

\begin{document}

\hypersetup{pageanchor=false}
\begin{titlepage}
	\begin{center}
		\vspace*{1em}
		\Huge
		\textbf{II Galois Theory}

		\vspace{1em}
		\large
		Ishan Nath, Michaelmas 2023

		\vspace{1.5em}

		\Large

		Based on Lectures by Prof. Tom Fisher

		\vspace{1em}

		\large
		\today
	\end{center}
	
\end{titlepage}
\hypersetup{pageanchor=true}

\tableofcontents

\newpage

%lecture 1

\section{Field Extensions}
\label{sec:field_ext}

A \emph{field}\index{field} $K$ is a ring (commutative, with a 1 and $0_K \neq 1_K$) in which every non-zero element has an inverse under multiplication. The \emph{characteristic} of $K$ is the least positive integer $p$ (necessarily prime) such that $p \cdot 1_K = 0_K$. If no such $p$ exists, $\chr(K) = 0$.

Then $K$ contains a smallest subfield, either $\mathbb{F}_p$ if $\chr(K) = p \neq 0$, or $\mathbb{Q}$ if $\chr(K) = 0$.

\begin{lemma}\label{lemma:1.1}
	Let $K$ be a field, $0 \neq f \in K[X]$. Then $f$ has at most $\deg(f)$ roots in $K$.
\end{lemma}

\begin{proofbox}
	We induct on $\deg(f) = n$. If $f$ has degree 0, i.e. is constant, then it has no roots as $f \neq 0$. Otherwise, let $\deg(f) > 0$, and assume $f$ has a root $\alpha$ otherwise the result is trivial.

	Then $f(X) = (X - \alpha)g(X)$, where $g \in K[X]$, by polynomial division. Now $\deg(g) = n-1$, so $g$ has at most $n-1$ roots. Then $f$ has at most $n$ roots; the roots of $g$, plus possibly $\alpha$ if it is not a root of $g$. Hence $f$ has at most $n$ roots, as desired.
\end{proofbox}

\begin{definition}
	Let $L$ be a field and $K \subset L$ a subfield. Then $L$ is an \emph{extension}\index{extension}\index{field extension} of $K$, written $L/K$.
\end{definition}

Note that $\chr(K) = \chr(L)$.

\begin{exbox}[Field extension]
	\begin{enumerate}[(i)]
		\item $\mathbb{C}/\mathbb{R}$, $\mathbb{Q}[\sqrt 2]/\mathbb{Q}$, $\mathbb{R}/\mathbb{Q}$.
		\item Let $K$ be a field, and $f \in K[X]$ an irreducible polynomial. As $K[X]$ is an ED, it is a PID. Now I claim $(f)$ is a maximal ideal, as if $(f) \subset (g) \subset K[X]$, then $g \mid f$, a contradiction.

			So $L = K[X]/(f)$ is a field, and moreover it is a field extension of $K$. If we let $\alpha = X + (f) \in L$, then $\alpha$ is a root of $f$ in $L$.
	\end{enumerate}
\end{exbox}

Let $L/K$ be a field extension. Then addition in $L$ and multiplication by elements in $K$ make $L$ into a $K$-vector space.

\begin{definition}
	Let $L/K$ be a field extension. We say $L/K$ is \emph{finite}\index{finite field extension} if $L$ is finite dimensional as a $K$-vector space, in which case we write $[L:K] = \dim_K L$, which we call the \emph{degree}\index{degree} of the extension. If not, we say $L/K$ is an \emph{infinite extension}\index{infinite field extension}, and write $[L:K] = \infty$.
\end{definition}

\begin{exbox}[Degree of extensions]
	\begin{enumerate}[(i)]
		\item $[\mathbb{C}: \mathbb{R}] = 2$, $[\mathbb{Q}[\sqrt 2] : \mathbb{Q}] = 2$, $[\mathbb{R} : \mathbb{Q}] = \infty$.
		\item If $L = K[X]/(f)$ for $f$ irreducible, then $[L:K] = \deg(f)$. Indeed, if $\alpha \in L$ is as before, then $1, \alpha, \ldots, \alpha^{n-1}$ is a basis of $L$.
	\end{enumerate}
\end{exbox}

Let $K, L$ be field s and $\phi : K \to L$ a ring homomorphism. Then $\ker(\phi)$ is an ideal of $K$, but as $K$ is a field this means either $\ker(\phi) = \{0\}$ or $K$.

However as $\phi$ is a ring homomorphism, $1_K \mapsto 1_L$, so $1_K \not\in \ker(\phi)$. Hence $\ker(\phi) = \{0\}$, and so $\phi$ is injective.

We call such $\phi$ an \emph{embedding}\index{embedding} of $K$ in $L$. We may use $\phi$ to identify $K$ as a subfield of $L$, i.e. as an extension $L/K$.

\subsection{Finite Fields}
\label{sub:finite_fields}

Taking $K = \mathbb{F}_2$, and $f = X^2 + X + 1 \in \mathbb{F}_2[X]$, then $L = K[X]/(f)$ gives a field with 4 elements. We will see how to construct all finite feels in this subsection.

\begin{proposition}
	Let $K$ be a finite field of characteristic $p$. Then $|K| = p^{n}$, where $n = [K: \mathbb{F}_p]$.
\end{proposition}

\begin{proofbox}
	As $[K: \mathbb{F}_p] = n$, $K \cong \mathbb{F}_p^n$ as an $\mathbb{F}_p$-vector space. Hence $|K| = p^{n}$.
\end{proofbox}

Later we will show there is exactly one field of order $p^{n}$ for each such prime power.

The multiplicative group of a field is the set $K^{\times} = K \setminus \{0\}$, which is abelian under multiplication.

\begin{proposition}
	If $K$ is a field, then any finite subgroup $G \subset K^{\times}$ is cyclic. In particular, if $K$ is finite, then $K^{\times}$ is cyclic.
\end{proposition}

\begin{proofbox}
	By structure theorem and as $G$ is abelian, $G$ is a product of cyclic groups
	\[
	G \cong C_{d_1} \times C_{d_2} \times \cdots \times C_{d_t},
	\]
	where $1 \neq d_1 \mid d_2 \mid \cdots \mid d_t$. Moreover, if $t > 1$, then let $p | d_1$. We have that $C_p \times C_p$ is a subgroup of $G$, hence $X^{p}-1$ has at least $p^2$ roots. However this contradicts lemma \ref{lemma:1.1}.
\end{proofbox}

% lectures 2

\begin{proposition}
	Let $R$ be a ring of characteristic $p$. Then
	\begin{align*}
		\phi_p : R &\to R \\
		x &\to x^{p}
	\end{align*}
	is a homomorphism from $R$ to itself (the \emph{Frobenius endomorphism}\index{Frobenius endomorphism}).
\end{proposition}

\begin{proofbox}
We need to show $\phi_p(1) = 1$, $\phi_p(xy) = \phi_p(x) \phi_p(y)$, and $\phi_p(x+y) = \phi_p(x) + \phi_p(y)$. The first two properties follow easily, and the last follows from binomial expansion:
	\[
		\phi_p(x+y) = x^{p} + \sum_{i = 1}^{p-1} \binom{p}{i} x^{i} y^{p-i} + y^{p} = x^{p} + y^{p} = \phi_p(x) + \phi_p(y)
	.\]
\end{proofbox}

This can give a proof to Fermat's little theorem using induction.

\begin{theorem}[Tower law]\index{tower law}
	Let $M/L$ and $L/K$ be field extension. Then $M/K$ is finite if and only if both $M/L$ and $L/K$ are finite. In this case,
	\[
		[M:K] = [M:L][L:K].
	\]
	
\end{theorem}

\begin{proofbox}
	If $M/K$ is finite, then any $K$-basis for $M$ spans $M$ as an $L$-vector space, and $L$ is a $K$-vector subspace of $M$.

	Now suppose that $M/L$ and $L/K$ are finite, say $v_1, \ldots, v_n$ is a $K$-basis for $L$, and $w_1, \ldots, w_m$ is a $L$-basis for $M$.

	We claim that $\{v_iw_j \mid 1 \leq i \leq n, 1 \leq j \leq m\}$ is a $K$-basis for $M$:
	\begin{enumerate}[(i)]
		\item If $x \in M$, then $x = \sum_j \lambda_j w_j$ for some $\lambda_j \in L$, and $\lambda_j = \sum_i \mu_{ij} v_i$ for some $\mu_{ij} \in K$.

			Then $x = \sum_{i,j} \mu_{ij} v_i w_j$, showing these are spanning.
		\item Now suppose $\sum_{i,j} \mu_{ij} v_i w_j = 0$ for some $\mu_{ij} \in K$. Then
			\[
			\sum_j \Biggl( \sum_i \mu_{ij} v_i \Biggr) w_j = 0.
			\]
			As $w_1, \ldots, w_m$ are linearly independent over $L$, $\sum_i \mu_{ij} v_i = 0$ for all $j$. As $v_1, \ldots, v_n$ are linearly independent over $K$, $\mu_{ij} = 0$ for all $i, j$. Hence these elements are linearly independent.
	\end{enumerate}
\end{proofbox}

\begin{definition}
	Let $L/K$ be a field extension. Let $\alpha_1, \ldots, \alpha_n \in L$. Then
	\[
		K[\alpha_1, \ldots, \alpha_n] = \{f(\alpha_1, \ldots, \alpha_n) \mid f \in K[X_1,\ldots, X_n]\}.
	\]
	This is the smallest subring of $L$ to contain $K$ and $\alpha_1, \ldots, \alpha_n$. Moreover let
	\[
		K(\alpha_1, \ldots, \alpha_n) = \biggl\{ \frac{f(\alpha_1, \ldots, \alpha_n)}{g(\alpha_1, \ldots, \alpha_n)} \mid f, g \in K[X_1, \ldots, X_n], g(\alpha_1, \ldots, \alpha_n) \neq 0\}.
	\]
	This is the smallest subfield of $L$ to contain $K$ and $\alpha_1, \ldots, \alpha_n$. It is also the field of fractions of $K[\alpha_1, \ldots, \alpha_n]$.
\end{definition}

\begin{exbox}
	Note that $\mathbb{Q}(\sqrt 2) = \{a + b\sqrt 2 \mid a, b \in \mathbb{Q}\} = \mathbb{Q}[\sqrt 2] \subset \mathbb{R}$.

	Moreover $\mathbb{Q}(\sqrt 2) = \mathbb{Q}(1 + \sqrt 2) = \mathbb{Q}(17/(3 - \sqrt 2))$.
\end{exbox}

Moreover $K[\alpha_1, \ldots, \alpha_n]$ and $K(\alpha_1, \ldots, \alpha_n)$ are the intersections of the subrings (resp. subfields) of $L$ that contain $K$ and $\alpha_1, \ldots, \alpha_n$.

We can check that
\[
K(\alpha_1, \ldots, \alpha_n) = K(\alpha_1, \ldots, \alpha_{n-1})(\alpha_n) = K(\alpha_1)(\alpha_2,\ldots,\alpha_n).
\]

\begin{definition}
	A field extension $L/K$ is a \emph{simple extension} if $L = K(\alpha)$ for some $\alpha \in L$.
\end{definition}

\newpage

\section{Algebraic Elements and Extensions}%
\label{sec:algebraic_elements_and_extensions}

Let $L/K$ be a field extension and $\alpha \in L$. Then there is a unique ring homomorphism $\phi : K[X] \to L$ such that $\phi(c) = c$ and $c \in K$, and $\phi(X) = \alpha$.

Indeed, we can let $\phi(\sum c_i X^{i}) = \sum c_i \alpha^{i}$, for $c_i \in K$. In this case $\phi$ is the ``evaluation at $\alpha$'' map.

As $K[X]$ is a PID, $\ker(\phi) = (f)$ for some $f \in K[X]$.

\begin{definition}
	$\alpha$ is \emph{algebraic}\index{algebraic} over $K$ if $f \neq 0$. In this case $f$ is irreducible and unique up to multiplication by elements of $K^{\times}$. We scale $f$ so that it is monic, and call it the \emph{minimal polynomial} of $\alpha$ over $K$.
\end{definition}

By the isomorphism theorem for rings,
\[
	\frac{K[X]}{(f)} \cong K[\alpha],
\]
and as $(f)$ is a maximal ideal, $K[\alpha]$ is a field. So $K(\alpha) = K[\alpha]$.

Moreover $[K(\alpha):K] = \deg f$, since if $\deg f = n$, then $1, \alpha, \alpha^2, \ldots, \alpha^{n-1}$ is a $K$-basis for $K(\alpha)$.

\begin{exbox}
	\begin{itemize}
		\item If $x \in K$, then $m_{x, K} = T - x$.
		\item If $p$ is prime, $d \geq 1$, then $T^{d} - p \in \mathbb{Q}[T]$ is irreducible (e.g Eisenstein's and Gauss' Lemma). So it is the minimum polynomial of $\sqrt[d]{p} = x \in \mathbb{R}$ over $\mathbb{Q}$.
		\item If $z = e^{2 \pi i/p}$, then it is the root of $g(T) = T^{p-1} + \cdots + T + 1 \in \mathbb{Q}[T]$. This is irreducible by looking at $g(T+1)$ and applying Eisenstein's. Hence $g$ is the minimal polynomial of $z$ over $\mathbb{Q}$.
	\end{itemize}	
\end{exbox}

Let $\alpha \in L$ be algebraic over $K$ with minimum polynomial $f$. Let $0 \neq \beta \in K(\alpha)$, and say $\beta = g(\alpha)$ for some $g \in K[X]$. Since $f$ is irreducible and $\beta \neq 0$, we see that $f$ and $g$ are coprime

Running Euclid's algorithm gives $r, s \in K[X]$ such that $r(X)f(X) + s(X)g(X) = 1$. Hence plugging in $\alpha$,
\[
s(\alpha)g(\alpha) = r(\alpha)f(\alpha) + s(\alpha)g(\alpha) = 1,
\]
so the inverse of $\beta$ is simply $s(\alpha)$.

%lecture 3 should begin here!!!

\begin{definition}
	Let $L/K$ be a field extension.
	\begin{enumerate}[(i)]
		\item $\alpha \in L$ is \emph{transcendental}\index{transcendental} over $K$ if it is not algebraic. In this case $K[\alpha] \cong K[X]$ and $K(\alpha) \cong K(X)$. Since $K[X]$ is not a field and $1, X, X^2, \ldots$ are linearly independent over $K$, $K[\alpha] \neq K(\alpha)$ and $[K(\alpha) : K] = \infty$.
		\item $L/K$ is \emph{algebraic}\index{algebraic extension} if every $\alpha \in L$ is algebraic over $K$.
	\end{enumerate}
\end{definition}

\begin{remark}
	\begin{enumerate}[(i)]
		\item[]
		\item We have
			\[
				[K(\alpha):K] < \infty \iff \alpha \text{ is algebraic over } K.
			\]
		\item If $[L:K] < \infty$, then for any $\alpha \in L$ we certainly have $[K(\alpha) : K] < \infty$. So any finite extension is algebraic. Note that the converse is not true: take $K = \mathbb{Q}$ and $L = \bigcup \mathbb{Q}(\!\sqrt[2^n]{2})$. This is a union of nested sequences of fields, and so is a field.

			Then $[L:K] = \infty$ as $[\mathbb{Q}(\!\sqrt[2^n]{2}): \mathbb{Q}] = 2^{n}$ is unbounded, but every $\alpha \in L$ belongs to a finite extension of $\mathbb{Q}$, and so is algebraic over $\mathbb{Q}$.
		\item Classically $\alpha \in \mathbb{C}$ is called algebraic or transcendental, if it is algebraic or transcendental over $\mathbb{Q}$. However note every $\alpha \in \mathbb{C}$ is transcendental over $\mathbb{R}$.
	\end{enumerate}
\end{remark}


\subsection{Algebraic Numbers in \texorpdfstring{$\mathbb{R}$}{R} and \texorpdfstring{$\mathbb{C}$}{C}}%
\label{sub:algebraic_numbers_in_R_and_C}

Traditionally, $x \in \mathbb{C}$ is algebraic if it is algebraic over $\mathbb{Q}$, otherwise it is transcendental. We have seen that $\overline{\mathbb{Q}}$ is a subfield of $\mathbb{C}$, and is not equal to $\mathbb{C}$, as $\mathbb{Q}[T]$ is countable, so the number of possible roots is countable. However, even though `most' numbers are transcendental, it is harder to write one explicitly, or show a specific number isn't algebraic.

Liouville first provided a transcendental number,\index{Liouville's number}
\[
\sum_{n \geq 1} \frac{1}{10^{n!}}
.\]
He did this by showing that algebraic numbers can't be approximated very well by rationals, while his number is.

Later, Hermite and Lindemann showed that $e$ and $\pi$ are transcendental, and in the 20th century, the following theorem was proven:

\begin{theorem}[Gelfond-Schneider theorem]\index{Gelfond-Schneider theorem}
	If $x, y$ are algebraic, where $x \neq 0, 1$, then $x^{y}$ is algebraic if and only if $y \in \mathbb{Q}$.
\end{theorem}

This gives results such as $\sqrt{2}^{\sqrt 3}$ and $e^{\pi} = (-1)^{-i/2}$ are transcendental.

\begin{exbox}
	Some examples of calculating degrees of extensions.
	\begin{enumerate}
		\item Let $f(X) = X^{d}-n$, $d \geq 2, n \neq 0$. Suppose there exists a prime $p$ such that when we write $n = p^{e}m$ with $p \nmid m$, then $(d, e) = 1$.
	
		We claim that $f$ is irreducible in $\mathbb{Q}[X]$. Equivalently, we show that $[\mathbb{Q}(\alpha):\mathbb{Q}] = d$, where $\alpha = \sqrt[d]n$. By Euclid's algorithm, there exist $r, s \in \mathbb{Z}$ such that $rd + se = 1$.
	
		Then $p^{dr}n^{s} = p^{dr}(p^{e}m)^{s} = p m^{s}$. We put $\beta = p^{r}\alpha^{s}$, so that $\beta^{d} = p m^{s}$. Then $\beta$ is a root of $g(X) = X^{d} - p m^{s}$, which is irreducible in $\mathbb{Z}[X]$ by Eisenstein's criterion. Hence it is irreducible in $\mathbb{Q}[X]$ by Gauss' lemma, so $[\mathbb{Q}(\beta): \mathbb{Q}] = d $.
	
		But $\mathbb{Q}(\beta) \subset \mathbb{Q}(\alpha)$ and $[\mathbb{Q}(\alpha) : \mathbb{Q}] \leq \deg f = d$, so $[\mathbb{Q}(\alpha) : \mathbb{Q}] = d$, as required.
	\item Let $p$ be an odd prime, and $\zeta_p = e^{2\pi i/p}$, and $\alpha = 2 \cos(2\pi/p) = \zeta_p + \zeta_p^{-1}$. Let's compute $[\mathbb{Q}(\alpha): \mathbb{Q}]$. Note that $\zeta_p$ is a root of
		\[
		f(X) = \frac{X^{p}-1}{X-1} = X^{p-1} + \cdots + X^2 + X + 1.
		\]
		This is irreducible by Eisenstein's criterion applied to $f(X+1)$, so $[\mathbb{Q}(\zeta_p): \mathbb{Q}] = p-1$. $\zeta_p$ is a root of
		\[
			g(X) = (X - \zeta_p)(X-\zeta_p^{-1}) = X^2 - \alpha X + 1 \in \mathbb{Q}(\alpha)[X].
		\]
		Hence $[\mathbb{Q}(\zeta_p) : \mathbb{Q}(\alpha)] \leq \deg g = 2$. But $\alpha \in \mathbb{R}$ and $\zeta_p \not \in \mathbb{R}$, hence $[\mathbb{Q}(\zeta_p) : \mathbb{Q}(\alpha)] = 2$. By the tower law, we thus have $[\mathbb{Q}(\alpha) : \mathbb{Q}] = (p-1)/2$.
	\item Suppose $m, n$ and $mn$ are not perfect squares. Let $\alpha = \sqrt m + \sqrt n$. We will compute $[\mathbb{Q}(\alpha): \mathbb{Q}]$. Clearly $\mathbb{Q}(\alpha) \subset \mathbb{Q}(\sqrt m, \sqrt n)$. Conversely, we have
		\[
		m = (\alpha - \sqrt n)^2 = \alpha^2 - 2 \alpha \sqrt n + n \implies \sqrt n = \frac{\alpha^2 - m + n}{2\alpha} \in \mathbb{Q}(\alpha).
		\]
		Similarly $\sqrt m \in \mathbb{Q}(\alpha)$, so $\mathbb{Q}(\alpha) = \mathbb{Q}(\sqrt m, \sqrt n)$. Now $[\mathbb{Q}(\sqrt n) : \mathbb{Q}] = 2$, and we are just left to find $[\mathbb{Q}(\sqrt m, \sqrt n) : \mathbb{Q}(\sqrt n)]$. Note it is less than $2$, as $\sqrt m$ is a root of $X^2 - m$.

		Suppose it is 1, so $\sqrt m \in \mathbb{Q}(\sqrt n)$. Then $\sqrt m = r + s \sqrt n$ for some $r, s \in \mathbb{Q}$. Squaring both sides,
		\[
		m = r^2 + 2rs \sqrt n + s^2 n.
		\]
		Since $\sqrt n \not \in \mathbb{Q}$, we must have $rs = 0$. If $r = 0$, then $mn$ is a square. But if $s = 0$, then $m$ is a square, hence we have a contradiction.

		Therefore the degree is 2, and by tower law, $[\mathbb{Q}(\alpha) : \mathbb{Q}] = 4$.
	\end{enumerate}
\end{exbox}

%lecture 4 here

\begin{lemma}\label{le:alg_iff}
	Let $L/k$ be a field extension and $\alpha_1, \ldots, \alpha_n \ni L$. Then
	\[
		\alpha_1, \ldots, \alpha_n \text{ algebraic over } K \iff [K(\alpha_1, \ldots, \alpha_n) : K] < \infty.
	\]
\end{lemma}

\begin{proofbox}
	The case $n = 1$ was a remark in the previous lecture.

	The forwards direction is done by induction on $n$, using the tower law.

	The reverse direction is clear since $K(\alpha_i) \subseteq K(\alpha_1, \ldots, \alpha_n)$.
\end{proofbox}

\begin{corollary}
	Let $L/K$ be any field extension. Then the set
	\[
		\{\alpha \in L \mid \alpha \text{ is algebraic over } K\}
	\]
	is a subfield of $L$.
\end{corollary}

\begin{proofbox}
	If $\alpha, \beta$ are algebraic over $K$, then by the previous lemma $K(\alpha, \beta)$ is a finite extension of $K$. Let $\gamma$ be one of $\alpha \pm \beta$, or $\alpha \beta$, or (if $\alpha \neq 0$) $\alpha^{-1}$.

	Then $\gamma \in K(\alpha, \beta)$, so $K(\alpha, \beta, \gamma) = K(\alpha, \beta)$ is a finite extension of $K$, hence $\gamma$ is algebraic.
\end{proofbox}

\begin{exbox}
	Taking $K = \mathbb{Q}$ and $L = \mathbb{C}$, we see that $\overline{\mathbb{Q}}$, which are the algebraic numbers, is a field.

	Since $\overline{\mathbb{Q}} \supset \mathbb{Q}(\sqrt[d]2)$, and $[\mathbb{Q}(\sqrt[d]2): \mathbb{Q}]=d$ for all $d$, we see that $[\overline{\mathbb{Q}} : \mathbb{Q}] = \infty$.
\end{exbox}

\begin{proposition}
	Let $M/L/K$ be field extensions.

	Then $M/K$ is algebraic if and only if $M/L$ and $L/K$ are algebraic.
\end{proposition}

\begin{proofbox}
	For the forwards direction, every element of $M$ is algebraic over $K$, hence algebraic over $L$, so $M/L$ is algebraic. Moreover as $L \subset M$, $L$ is algebraic over $K$.

	For the other direction, let $\alpha \in M$. We must show that $\alpha$ is algebraic over $K$. Since $M/L$ is algebraic, $\alpha$ is a root of some polynomial
	\[
		f(X) = c_n X^{n} + \cdots + c_1 X + c_0 \in L[X].
	\]
	Let $L_0 = K(c_0, c_1, \ldots, c_n)$. Each $c_i \in L$, hence is algebraic. Lemma (\ref{le:alg_iff}) implies $[L_0:K] < \infty$. But $f$ has coefficients in $L_0$, so $[L_0(\alpha) : L_0] \leq \deg f < \infty$.

	By the tower law, $[L_0(\alpha) : K] < \infty$, hence $\alpha$ is algebraic over $K$.
\end{proofbox}

\newpage

\section{Ruler and Compass Constructions}%
\label{sec:ruler_and_compass_constructions}

We use our results on field extensions to show that certain classical problems concerning ruler and compass constructions cannot be solved.

Let $S \subset \mathbb{R}^2$ be a finite set of points. In plane geometry constructions, we are given three basic operations:
\begin{enumerate}[1.]
	\item Draw a straight line through any 2 distinct points in $S$.
	\item Draw a circle with centre at any point in $S$, and radius the distance between two points in $S$.
	\item Enlarge $S$ by adjoining any point of intersection of two distinct lines or circles.
\end{enumerate}

Say $(x, y) \in \mathbb{R}^2$ is \emph{constructible} from $S$ if we can enlarge $S$ to contain $(x, y)$ by a finite sequence of the above operations.\index{constructible}

Say $x \in \mathbb{R}$ is constructible if $(x, 0)$ is constructible from $(0, 0)$ and $(1, 0)$. It is easy to show every $x \in \mathbb{Q}$ and $\sqrt 2$ is constructible.

We will relate this notion to the following algebraic notion:

\begin{definition} 
	Let $K \subset \mathbb{R}$ be a subfield. Say $K$ is \emph{constructible}\index{constructible field} if there exists $n \geq 0$ and fields $\mathbb{Q} = F_0 \subset F_1 \subset \cdots \subset F_n \subset \mathbb{R}$ such that:
	\begin{enumerate}[(i)]
		\item $K \subset F_n$,
		\item $[F_i : F_{i-1}] = 2$.
	\end{enumerate}
\end{definition}

\begin{remark}
	By the tower law, $K / \mathbb{Q}$ is a finite extension and $[K : \mathbb{Q}]$ is a power of $2$.
\end{remark}

\begin{theorem}
	If $x \in \mathbb{R}$ is constructible, then $K = \mathbb{Q}(x)$ is constructible.
\end{theorem}

\begin{proofbox}
%	We induct on $k \geq 1$. We prove that if $(x, y) \in \mathbb{R}^2$ can be constructed in $k$ ruler and compass constructions, then $\mathbb{Q}(x, y)$ is a constructible extension of $\mathbb{Q}$. Assume we have $\mathbb{Q} = F_0 \subset \ldots \subset F_n$ satisfying the last two conditions, and such that the coordinates of all points obtained after $k-1$ constructions lie in $F_n$.
%	Elementary analytic geometry tells us:
%	\begin{enumerate}[1.]
%		\item The intersection point of two lines has coordinates which are rational functions of the coordinates of $(P_i, Q_i)$, with rational coefficients. So if the $k$'th construction is of this type, then the intersection point is in $F_n$.
%		\item The intersection points of a circle and a line/circle can be written as $(a \pm b \sqrt{e}, c \pm d \sqrt{e})$, where $a, \ldots, e$ are rational functions of the coordinates of $(P_i, Q_i)$. So the two newly constructed points satisfy
%			\[
%				x, y \in F_n(\sqrt e)
%			,\]
%			which is a constructible extension on $\mathbb{Q}$.
%	\end{enumerate}
	Suppose $S \subset \mathbb{R}^2$ is a finite set of points, all of whose coordinates belong to a constructible field $K$.

	It suffices to show that if we adjoin $(x, y) \in \mathbb{R}^2$ to $S$, then $K(x, y)$ is also constructible.

	The lines and circles in (i) and (ii) have equations of the form
	\[
	ax + by = c, \qquad (x - a)^2 + (y-b)^2 = c,
	\]
	with $a, b, c \in K$. If $(x, y)$ is the intersection of two such lines or circles, then
	\[
	x = r + s\sqrt v, y = t + u \sqrt v
	\]
	for $r, s, t, u, v \in K$. Hence $(x, y) \in K(\sqrt v) \subset F_n(\sqrt v)$, Since $[F_n(\sqrt v) : F_n] | 2$, it follows that $K(x, y)$ is constructible.
\end{proofbox}

\begin{remark}
	It is not hard to show that the converse is true, that is, if $\mathbb{Q}(x)$ is a constructible extension of $\mathbb{Q}$, then $x$ is constructible by ruler and compass.
\end{remark}

\begin{corollary}
	If $x \in \mathbb{R}$ is constructible, then $x$ is algebraic over $\mathbb{Q}$, and $[\mathbb{Q}(x) : \mathbb{Q}]$ is a power of 2.
\end{corollary}

This resolves the following classical problems:

\begin{enumerate}[1.]
	\item 'Squaring the circle' - constructing a square whose area is that of a given circle; this is equivalent to constructing $\sqrt{\pi}$. But since $\pi$ is transcendental, therefore $\sqrt \pi$ is transcendental, so such a square isn't constructible
	\item `Duplicating the cube' - constructing a cube with volume twice that of a given cube; this is equivalent to constructing $\sqrt[3]{2}$. But $[\mathbb{Q}(\sqrt[3]{2}) : \mathbb{Q}] = 3$, so $\mathbb{Q}(\sqrt[3]{2})$ (and thus $\sqrt[3]{2}$) isn't constructible.
	\item `Trisecting the angle' - let's say we are trying to trisect $2\pi/3$, which is certainly constructible. So if we can trisect $2\pi/3$, the angle $2\pi/9 = \theta$ is constructible, implying the real numbers $\sin \theta$, $\cos \theta$ are constructible. Using the triple angle formula,
		\[
		\cos 3 \theta = 4 \cos^3 \theta - 3 \cos \theta
		,\]
		which implies $\cos \theta$ is a root of $8X^3 - 6X + 1$, and $2 \cos \theta$ is a root of $X^3 - 3X + 1$, which is irreducible as it has no rational roots. So $[\mathbb{Q}(\cos \theta) : \mathbb{Q}] = 3$, implying $\cos \theta$ is not constructible.
\end{enumerate}

In fact, we will later generalise this result:
\begin{theorem}[Gauss]
	A regular polygon is constructible if and only if $n$ is the product of a power of $2$ and distinct primes of the form $2^{2^{k}} + 1$.
\end{theorem}

%lecture 5 and on

\newpage

\section{Splitting Fields}%
\label{sec:splitting_fields}

Pick a field $K$, and $f \in K[T]$ be non-constant. We want to find an extension $L/K$, as small as possible, such that $f$ has a root, or an extension in which $f$ can be written as a product of linear polynomials in $L[T]$.

For example, if $K \subset \mathbb{C}$, the Fundamental Theorem of Algebra says we can factor a monic $f \in \mathbb{C}[T]$ as a product of linear factors
\[
	f = \prod (T - x_i)
,\]
$x_i \in \mathbb{C}$. However the best $L$ would be $K(x_1, \ldots, x_n)$.

\begin{definition}
	Let $L/K$ and $M/K$ be extensions of $K$. A $K$\emph{-homomorphism} from $L$ to $M$ is a field homomorphism $L \to M$ which is the identity on $K$. It is also called a $K$-embedding.\index{$K$-homomorphism}
\end{definition}

\begin{theorem}
	Let $L = K(\alpha)$, where $\alpha$ is algebraic over $K$ with minimal polynomial $f$. Let $M/K$ be any field extension. Then there is a bijection
	\begin{align*}
		\{K\text{\normalfont{-homomorphisms }} L \to M\} &\leftrightarrow \{\text{\normalfont{roots of }} f \text{\normalfont{ in }} M\}, \\
		\tau &\mapsto \tau(\alpha).
	\end{align*}
	In particular, the number of $K$-homomorphisms is less than $\deg f$.
\end{theorem}

\begin{proofbox}
	Write
	\[
	f = \sum_{i = 0}^{d} c_i X^{i},
	\]
	where $c_i \in K$. Let $\tau : L \to M$ be a $K$-homomorphism. Then,
	 \[
	f(\tau(\alpha)) = \sum_{i} c_i \tau(\alpha)^i = \tau \Biggl( \sum_{i} c_i \alpha^{i} \Biggr) = \tau(f(\alpha)) = 0,
	\]
	hence $\tau(\alpha)$ is a root of $f$. But as $L = K(\alpha)$, any $K$-homomorphism is determined uniquely by $\tau(\alpha)$, so this map is injective.

	Now we saw earlier that the evaluation at $\alpha$ gives an isomorphism
	\[
		\frac{K[X]}{(f)} \overset{\phi}{\to} L, \qquad X + (f) \mapsto \alpha.
	\]
	Now let $\beta \in M$ be a root of $f$. Since $f$ is irreducible, it is the minimal polynomial for $\beta$ over $K$. Then evaluation at $\beta$ gives a ring homomorphism
	\[
		\frac{K[X]}{(f)} \overset{\psi}{\to} M, \qquad X + (f) \mapsto \beta.
	\]
	Since $\phi, \psi$ are $K$-homomorphisms and $\phi$ is an isomorphism, it follows that
	\[
	\tau = \psi \circ \phi^{-1} : L \to M
	\]
	is a $K$-homomorphism with $\tau(\alpha) = \beta$.
\end{proofbox}

\begin{exbox}
	There are exactly 2 $\mathbb{Q}$-homomorphisms $\mathbb{Q}(\sqrt 2) \to \mathbb{R}$, given by
	\[
	a + b \sqrt 2 \mapsto a + b \sqrt 2,
	\]
	\[
	a + b \sqrt 2 \mapsto a - b \sqrt 2.
	\]
\end{exbox}

Note for future proofs we will need a slight variant of the above theorem. The proof is the exact same, but the generality will be useful in inductive proofs.

\begin{definition}
	Let $L/K$, $M/K'$ be field extensions. Let $\sigma : K \to K'$ be a field homomorphism. If $\tau : L \to M$ is a homomorphism such that $\tau(x) = \sigma(x)$ for all $x \in K$, we say $\tau$ is a $\sigma$\emph{-homomorphism}\index{$\sigma$-homomorphism} from $L$ to $L'$.

	We also say $\tau$ extends $\sigma$, or that $\sigma$ is the restriction of $\tau$ to $K$, and write $\sigma = \tau|_{K}$.
\end{definition}

Taking $K' = K$ and $\sigma = \id$, then we recover the definition of a $K$-homomorphism.

\begin{theorem}\label{thm:num_sig_homs}
	Let $L = K(\alpha)$ where $\alpha$ is algebraic over $K$, with minimum polynomial $f$. Let $\sigma : K \to K'$ be a field embedding, and $M/K'$ be any field extension. Then there is a bijection
	\begin{align*}
		\{\sigma\text{\normalfont{-homomorphisms }} L \to M\} &\leftrightarrow \{\text{\normalfont{roots of }} \sigma f \text{\normalfont{ in }} M\}, \\
		\tau &\mapsto \tau(\alpha).
	\end{align*}
	In particular, the number of $\sigma$-homomorphisms is at most $\deg f$.
\end{theorem}

\begin{exbox}
	Let $K = \mathbb{Q}(\sqrt 2)$, and $L = \mathbb{Q}(\alpha)$ where $\alpha = \sqrt{1+\sqrt 2}$.

	Then there are exactly two $K$-embeddings $L \to \mathbb{R}$ given by $\alpha \mapsto \pm \sqrt{1+\sqrt 2}$.

	However if $\sigma : K \to K$ by $a + b\sqrt 2 \mapsto a - b \sqrt 2$, then there are no $\sigma$-embedding $L \to \mathbb{R}$, since $1 - \sqrt 2 < 0$.
\end{exbox}

\begin{definition}
	Let $K$ be a field. Let $0 \neq f \in K[X]$. An extension $L/K$ is a \emph{splitting field}\index{splitting field} of $f$ over $K$ if:
	\begin{enumerate}[(i)]
		\item $f$ splits into linear factors over $L$.
		\item $L = K(\alpha_1, \ldots, \alpha_n)$, where $a_i$ are the roots of $f$.
	\end{enumerate}
\end{definition}

The second condition is equivalent to saying that $f$ does not split into linear factors over any subfield of $L$ containing $K$. Moreover it implies $[L:K] < \infty$.

%lecture 6

\begin{theorem}
	let $0 \neq f \in K[X]$. Then there exists a splitting field for $f$ over $K$.
\end{theorem}

\begin{proofbox}
	The proof is by induction on $\deg f$. If $\deg f \leq 1$, then $L = K$.

	Now assume that every polynomial of degree $< \deg f$ has a splitting field. Let $g$ be an irreducible factor of $f$. Let
	\[
		K_1 = \frac{K[X]}{(g)}, \qquad \alpha_1 = X + (g) \in K_1.
	\]
	Then $f(\alpha_1) = 0$, so $f(X) = (X - \alpha_1)f_1(X)$ for some $f_1 \in K_1[X]$ with $\deg f_1 < \deg f$. By the induction hypothesis, there exists a splitting field $L$ for $f_1$ over $K_1$. Say that $L = K_1(\alpha_2, \ldots, \alpha_n)$ where $\alpha_2, \ldots, \alpha_n$ are the roots of $f_1$ in $L$.

	We claim that $L$ is a splitting field for $f$ over $K$. Since $f_1$ splits in $L$, so does $f(X) = (X-\alpha_1)f_1(X)$.

	Moreover $L = K_1(\alpha_2, \ldots, \alpha_n) = K(\alpha_1, \ldots, \alpha_n)$ and $\alpha_1, \ldots, \alpha_n$ are the roots of $f$, hence $L$ satisfies both conditions.
\end{proofbox}

\begin{theorem}[Uniqueness of Splitting Fields]
	Let $0 \neq f \in K[X]$, and let $L$ be a splitting field of $f$ over $K$. Let $\sigma : K \to M$ be any field embedding such that $\sigma f \in M[X]$ splits. Then,
	\begin{enumerate}[\normalfont(i)]
		\item There exists a $\sigma$-embedding $\tau : L \to M$.
		\item If $M$ is a splitting field for $\sigma f$ over $\sigma K$, then any $\tau$ as above is an isomorphism.
	\end{enumerate}
\end{theorem}

In particular, any two splitting fields for $f$ over $K$ are $K$-isomorphic.

\begin{proofbox}
	\begin{enumerate}[(i)]
		\item We will prove this by induction on $n = [L:K]$. If $n = 1$ then $L = K$ and there is nothing to prove. So suppose $n > 1$ and let $g \in K[X]$ be an irreducible factor of $f$, of degree at least 1.

			Let $\alpha \in L$ be a root of $g$, and let $\beta \in M$ be a root of $\sigma g$. Hence $\sigma$ extends to an embedding $\sigma_1 : K(\alpha) \to M$, where $\alpha \mapsto \beta$. Also $[L:K(\alpha)] < [L:K]$.

			As $L$ is a splitting field of $f$ over $K(\alpha)$, and $\sigma_1 f = \sigma f$ splits in $M$, by the induction hypothesis $\sigma_1$ extends to an embedding $\tau : L \to M$.
		\item Pick any $\tau : L \to M$ as in (i). Let $\alpha_1, \ldots, \alpha_n$ be the roots of $f$ in $L$. Then the roots of $\sigma f$ in $M$ are $\tau \alpha_1, \ldots, \tau \alpha_n$. So if $M$ is a splitting field for $\sigma f$ over $\sigma K$, then
			\begin{align*}
				M &= \sigma K(\tau \alpha_1, \ldots, \tau \alpha_n) \\
				  &= \tau (K(\alpha_1, \ldots, \alpha_n)) \\
				  &= \tau(L).
			\end{align*}
			Hence $\tau$ is surjective, so $\tau$ is an isomorphism.
	\end{enumerate}
	For the final statement, suppose $L/K$ and $M/K$ are splitting fields for $f$ over $K$, and let $\sigma : K \to M$ be the inclusion map. Then combining (i) and (ii), we get a $K$-isomorphism $L \to M$.
\end{proofbox}

\begin{exbox}
	If $K \subset \mathbb{C}$, then by the fundamental theorem of algebra, our splitting field for $f$ over $K$ is the subfield $K(\alpha_1, \ldots, \alpha_n) \subset \mathbb{C}$ where $\alpha_1, \ldots, \alpha_n \in \mathbb{C}$ are the roots of $f$.
	\begin{enumerate}[(i)]
		\item If $f(X) = X^3 - 2 \in \mathbb{Q}[X]$, then
			\[
				f(X) = (X - \sqrt[3]2)(X-\omega\sqrt[3]2)(X-\omega^2\sqrt[3]2),
			\]
			so $\mathbb{Q}(\omega, \sqrt[3]2)$ is a splitting field for $f$ over $\mathbb{Q}$. Now $[\mathbb{Q}(\omega): \mathbb{Q}] = 2$ and $[\mathbb{Q}(\sqrt[3]2): \mathbb{Q}] = 3$, so as $[\mathbb{Q}(\omega, \sqrt[3]2) : \mathbb{Q}] \leq 6$ and is divisible by both 2 and 3, it must be $6$.
		\item Let $p$ be an odd prime, and
			\begin{align*}
				f(X) &= \frac{X^{p}-1}{X-1} = X^{p-1} + \cdots + X^2 + X + 1 \in \mathbb{Q}[X]\\
				     &= \prod_{r = 1}^{p-1}(X - \zeta_p^r),
			\end{align*}
			where $\zeta_p = e^{2\pi i/p}$. Then $f$ has splitting field $\mathbb{Q}(\zeta_p, \zeta_p^2, \ldots, \zeta_p^{p-1}) = \mathbb{Q}(\zeta_p)$. So in this case, the splitting field is obtained by just adjoining one root.
		\item Let $f(X) = X^3 - 2 \in \mathbb{F}_7[X]$. Then $f$ is irreducible as it has no roots. Let
			\[
				L = \frac{\mathbb{F}_7[X]}{(f)},
			\]
			so $L = \mathbb{F}_7(\alpha)$ with $\alpha^3 = 2$. But then as $2^3 = 4^3 = 1$, we can factor
			\[
			f(X) = (X-\alpha)(X-2\alpha)(X-4\alpha).
			\]
			So $L = \mathbb{F}_7(\alpha)$ is a splitting field for $f$ over $\mathbb{F}_7$.
	\end{enumerate}
\end{exbox}

%lecture 7

\begin{definition}
	A field $K$ is \emph{algebraically closed}\index{algebraically closed} if every nonconstant polynomial in $K[X]$ has a root in $K$, equivalently every irreducible polynomial in $K[X]$ is linear.
\end{definition}

From the fundamental theorem of algebra, $\mathbb{C}$ is algebraically closed.

\begin{lemma}
	Let $K$ be a field. The following are equivalent:
	\begin{enumerate}[\normalfont(i)]
		\item $K$ is algebraically closed.
		\item If $L/K$ is a field extension and $\alpha \in L$ is algebraic over $K$, then $\alpha \in K$.
		\item If $L/K$ is algebraic then $L = K$.
		\item If $L/K$ is finite then $L = K$.
	\end{enumerate}
\end{lemma}

\begin{proofbox}
	(i) $\implies$ (ii) as the minimal polynomial of $\alpha$ over $K$ is irreducible hence linear. So $\alpha \in K$.

	Note that (ii) $\implies$ (iii) $\implies$ (iv) is clear. Now to show that (iv) $\implies$ (i), let $f \in K[X]$ be an irreducible polynomial.

	Then $L = K[X]/(f)$ is a finite extension of $K$, with $[L:K] = \deg f$. But by (iv) we have $L = K$, so $f$ is linear, and hence every irreducible polynomial is linear, so if $\alpha$ is algebraic then $\alpha \in K$.
\end{proofbox}

\begin{definition}
	If $L/K$ is algebraic and $L$ is algebraically closed, then we say that $L$ is an \emph{algebraic closure}\index{algebraic closure} of $K$.
\end{definition}

\begin{lemma}\label{lem:lin_split_alg_clo}
	Let $L/K$ be an algebraic extension such that every polynomial in $K[X]$ splits into linear factors over $L$. Then $L$ is algebraically closed, and hence an algebraic closure of $K$.
\end{lemma}

\begin{proofbox}
	If $L$ is not algebraically closed, then there must exist $M/L$ algebraic with $[M:L] > 1$. Both $M/L$ and $L/K$ are algebraic, hence $M/K$ is algebraic.

	Pick any $\alpha \in M$. Let $f$ be the minimal polynomial for $\alpha$ over $K$. By our assumption, $f$ splits over $L$, so $\alpha \in L$. Hence $M = L$.
\end{proofbox}

Later we will show that every field $K$ has an algebraic closure.

\begin{theorem}
	Suppose that:
	\begin{enumerate}[\normalfont(i)]
		\item $K \subset \mathbb{C}$, or
		\item $K$ is countable.
	\end{enumerate}
	Then $K$ has an algebraic closure.
\end{theorem}

\begin{proofbox}
	\begin{enumerate}[(i)]
		\item If $K \subset \mathbb{C}$, then let
			\[
				L = \{\alpha \in \mathbb{C} \mid \alpha \text{ is algebraic over }K\}.
			\]
			Then $L$ is a field, as we have proved, and $L/K$ is clearly algebraic. If $f \in K[X]$, then we can write $f(X) = \prod (X - \alpha_i)$ for some $\alpha_i \in \mathbb{C}$.

			By the definition of $L$, all $\alpha_i$ are in $L$, so $f$ splits into linear factors over $L$. Hence $L$ is algebraically closed by our previous lemma, so $L$ is the algebraic closure of $K$.
		\item If $K$ is countable, then so is $K[X]$. Enumerate the monic irreducible polynomials $f_1, f_2, \ldots$. Let $L_0 = K$ and for each $i \geq 1$ let $L_i$ be a splitting field for $f_i$ over $L_{i-1}$. Then,
			\[
			L_0 \subset L_1 \subset L_2 \subset \cdots
			\]
			Then $L = \bigcup L_n$ is a field, with $L/K$ algebraic and every polynomial in $K[X]$ splits over $L$. Hence $L$ is algebraically closed, so $L$ is an algebraic closure of $K$.
	\end{enumerate}
\end{proofbox}

\begin{remark}
	Taking $K = \mathbb{Q}$ in the proof of the above, we see that $\overline{\mathbb{Q}} \subset \mathbb{C}$ is algebraically closed.
\end{remark}

\newpage

\section{Symmetric Polynomials}
\label{sec:sym_polys}

\subsection{Motivation}
\label{sub:sym_mot}

Suppose we wish to find the roots of a cubic polynomial $f(X) = X^3 + aX^2 + bX + c \in \mathbb{Q}[X]$.

After substituting $X - \frac{a}{3}$ for $X$, we may assume $a = 0$. Writing
\[
f(X) = (X-\alpha)(X-\beta)(X-\gamma)
\]
and comparing coefficients, we get
\begin{align*}
	\alpha + \beta + \gamma &= 0,\\
	\alpha\beta + \beta\gamma+\gamma\alpha &= b,\\
	\alpha\beta\gamma &= -c.
\end{align*}
Let $\omega = e^{2\pi i/3}$, and write
\[
	\alpha = \frac{1}{3} [(\alpha+\beta+\gamma)+(\alpha+\omega\beta+\omega^2\gamma)+(\alpha+\omega^2\beta+\omega\gamma)] = \frac{1}{3}(u+v),
\]
where $u = \alpha + \omega\beta + \omega^2\gamma$, $v = \alpha + \omega^2\beta+\omega\gamma$. Then $u^3 + v^3$ and $uv$ are unchanged under permuting $\alpha,\beta,\gamma$.

After some calculations, we find that
\begin{align*}
	u^3 + v^3 &= -27c,\\
	uv &= -3b.
\end{align*}
Hence $u^3$ and $v^3$ are the roots of
\[
X^2 + 27cX - 27b^3 = 0.
\]
Solving this quadratic and taking cube roots gives a formula for the roots of a cubic, usually called Cardano's formula.

\subsection{Formalising}
\label{sub:sym_form}

Let $S_n$ be the symmetric group on $n$ letters.
\begin{definition}
	Let $R$ be a ring. A polynomial $f \in R[X_1, \ldots, X_n]$ is \emph{symmetric}\index{symmetric polynomial} if
	\[
	f(X_{\sigma(1)}, \ldots, X_{\sigma(n)}) = f(X_1, \ldots, X_n),
	\]
	for all $\sigma \in S_n$. If $f$ and $g$ are symmetric, then so are $f+g$ and $fg$. Hence the symmetric polynomials form a subring of $R[X_1, \ldots, X_n]$.
\end{definition}

\begin{definition}
	The \emph{elementary symmetric functions}\index{elementary symmetric function} are the polynomials $s_1, \ldots, s_n \in \mathbb{Z}[X_1, \ldots, X_n]$ such that
	\[
	\prod_{i = 1}^{n} (T+X_i) = T^{n} + s_1 T^{n-1} + \cdots + s_{n-1} T + s_n.
	\]
\end{definition}

For example, for $n = 3$,
\begin{align*}
	s_1 &= X_1+X_2+X_3,\\
	s_2 &= X_1X_2+X_2X_3+X_3X_1,\\
	s_3 &= X_1X_2X_3.
\end{align*}
In general,
\[
s_r = \sum_{1\leq i_1<\cdots<i_r\leq n} X_{i_1}X_{i_2}\cdots X_{i_r}.
\]
\begin{theorem}[Symmetric function theorem]
	\begin{enumerate}[\normalfont(i)]
		\item[]
		\item Every symmetric polynomial over $R$ can be expressed as a polynomial in the elementary symmetric functions, with coefficients in $R$.
		\item There are no nontrivial relations between the $s_r$.
	\end{enumerate}
\end{theorem}

This says, if we consider the ring homomorphism
\begin{align*}
	R[Y_1, \ldots, Y_n] &\overset{\theta}{\to} R[X_1, \ldots, X_n],\\
	Y_i &\mapsto s_i,
\end{align*}
then $\Img(\theta)$ is the symmetric polynomials, and $\theta$ is injective.

%lecture 8

\begin{proofbox}
	Note we can write any $f \in R[X_1, \ldots, X_n]$ as $f = \sum f_d$, where $f_d$ is homogeneous of degree $d$. Clearly $f$ is symmetric if and only if all $f_d$ are symmetric.

	So for the proof of this theorem, it suffices to consider only symmetric and homogeneous polynomials. Define the \emph{lexographic ordering} of monomials such that
	\[
	X_1^{i_1}X_2^{i_2}\cdots X_n^{i_n} > X_1^{j_1} X_2^{j_2} \cdot X_n^{j_n}
	\]
	if $i_1 = j_1, i_2 = j_2, \ldots, i_{r-1} = j_{r-1}, i_r > j_r$ for some $1 \leq r \leq n$. Then this is a total order.

	Let $X_1^{i_1}X_2^{i_2}\cdots X_n^{i_n}$ be the largest monomial to appear in $f$ with non-zero coefficient, say $c$. Then $X_{\sigma(1)}^{i_1}X_{\sigma(2)}^{i_2}\cdots X_{\sigma(n)}^{i_n}$ also appears in $f$ for all $\sigma \in S_n$. Hence we must have $i_1 \ge i_2 \ge \cdots \ge i_n$.

	Write
	\[
	X_1^{i_1}X_2^{i_2}\cdots X_n^{i_n} = X_1^{i_1-i_2}(X_1X_2)^{i_2-i_3}\cdots (X_1X_2\cdots X_n)^{i_n}.
	\]
	Let $g = S_1^{i_1-i_2}S_2^{i_2-i_3}\cdots S_n^{i_n}$. Then $f$ and $g$ are both homogeneous of degree $d$ and have the same largest monomial. So $f - cg$ is either $0$, or it is a symmetric homogeneous polynomial of degree $d$ whose leading monomial is smaller than that of $f$.

	As there are only finitely many monomials of degree $d$, this process eventually terminates.

	To prove part (ii), we will induct on $n$. Write $s_{r,n}$ instead of $s_r$ to indicate the number of variables involved. Suppose we have $G \in R[Y_1, \ldots, Y_n]$ with $G(s_{1,n}, s_{2,n}, \ldots, s_{n,n}) = 0$. We must prove that $G = 0$.

	The case for $n = 1$ is clear. Write $G = Y_n^{k} H$ with $Y_n \nmid H$, $k \geq 0$. Since $s_{n,n} = X_1X_2 \ldots X_n$, it is not a zero divisor in $R[X_1, \ldots, X_n]$, we have $H(s_{1,n}, \ldots, s_{n,n}) = 0$.

	So we may assume that $G$, if non-zero, is not divisible by $Y_n$. Replacing $X_n$ by $0$ gives
	\[
	s_{r,n}(X_1,\ldots, X_{n-1},0)=
	\begin{cases}
		s_{r,n-1}(X_1,\ldots,X_{n-1}) &r<n,\\
		0 &r=n.
	\end{cases}
	\]
	Therefore, $G(s_{1,n-1},s_{2,n-1},\ldots,s_{n-1,n-1},0)$. By the induction hypothesis, we get $G(Y_1, \ldots, Y_{n-1},0) = 0$. Hence $Y_n \mid G$, so $G = 0$.
	
\end{proofbox}

\begin{exbox}
	Consider the symmetric polynomial
	\[
	f = \sum_{i \neq j} X_i^2 X_j.
	\]
	The leading term is $X_1^2X_2=X_1(X_1X_2)$. Hence looking at
	\[
	s_1s_2 = \sum_i \sum_{j<k} X_i X_j X_k = \sum_{i\neq j} X_i^2X_j + 3\sum_{i<j<k} X_iX_jX_k.
	\]
	Hence we can subtract $3s_3$ to get $f = s_1s_2 - 3s_3$.
\end{exbox}

\begin{exbox}
	Let $f(X) = \prod (X - \alpha_i)$ be a monic polynomial with roots $\alpha_1, \ldots, \alpha_n$. The \emph{discriminant}\index{discriminant} of $f$ is
	\[
	\Disc(f) = \prod_{i<j}(\alpha_i - \alpha_j)^2.
	\]
	By the symmetric function theorem, we can write $\Disc(f)$ as a polynomial in the coefficients of $f$. For example, if $n = 2$, then
	\[
	f(X) = X^2 + bx + c = (X-\alpha_1)(X-\alpha_2).
	\]
	Then the discriminant is
	\[
	\Disc(f) = (\alpha_1-\alpha_2)^2 = (\alpha_1+\alpha_2)^2 - 4\alpha_1\alpha_2 = b^2 - 4c.
	\]
	It is clear from the definition that
	\[
		\Disc(f) = 0 \iff f \text{ has repeated roots}.
	\]
\end{exbox}

\newpage

\section{Normal and Separable Extensions}
\label{sec:norm_sep_ext}

\begin{definition}
	An extension $L/K$ is \emph{normal}\index{normal} if it is algebraic and the minimal polynomial of every $\alpha \in L$ splits into linear factors over $L$.

	Equivalently, if $f \in K[X]$ is irreducible and has a root in $L$, then it splits into linear factors.
\end{definition}

We can think of these as trade unions.

\begin{theorem}\label{thm:norm_split}
	Let $[L:K] < \infty$. Then,
	\[
		L/K \text{ is normal} \iff L\text{ is a splitting field for some } f \in K[X].
	\]
\end{theorem}

\begin{proofbox}
	As $[L:K] < \infty$, we can write $L = K(\alpha_1, \ldots, \alpha_n)$. Let $f_i$ be the minimal polynomial of $\alpha_i$ over $K$. Then,
	\begin{align*}
		L/K \text{ normal} &\implies f_i \text{ splits into linear factors over } L \\
				   &\implies L \text{ is a splitting field for } \prod f_i.
	\end{align*}
	For the other way round, let $L$ be the splitting field of $f \in K[X]$ over $K$. Let $\alpha \in L$ have minimal polynomial $g$ over $K$.

	Let $M/L$ be a splitting field for $g$. We must show that whenever $\beta \in M$ is a root of $g$, then in fact $\beta \in L$. We know that $L = L(\alpha)$ is a splitting field for $f$ over $K(\alpha)$, and $L(\beta)$ is a splitting field for $f$ over $K(\beta)$.

	But as $\alpha$ and $\beta$ have the same minimal polynomial $g$ over $K$, $K(\alpha)$ and $K(\beta)$ are $K$-isomorphic.

	By the uniqueness of splitting fields, $L(\alpha)$ and $L(\beta)$ are $K$-isomorphic, so $[L:K] = [L(\beta):K]$. By the tower law, this means $L(\beta) = L$, and so $\beta \in L$.
\end{proofbox}

%lecture 9

\subsection{Separability}
\label{sub:separability}

Over $\mathbb{R}$ or $\mathbb{C}$, we know from calculus that a polynomial $f$ has a repeated root $\alpha$ if and only if
\[
f(\alpha) = f'(\alpha) = 0.
\]
To work over arbitrary fields, we proceed purely algebraically.

\begin{definition}
	The formal derivative of
	\[
		f = \sum_{i = 0}^{d} c_i X^i \in K[X]
	\]
	is
	\[
	f' = \sum_{i = 1}^{d} i c_i X^{i-1}.
	\]
\end{definition}

This definition follows the Leibniz properties.

\begin{lemma}
	Let $f \in K[X]$ and $\alpha \in K$ a root of $f$. Then $\alpha$ is a simple root if and only if $f'(\alpha) = 0$.
\end{lemma}

\begin{proofbox}
	Write $f(X) = (X-\alpha)g(X)$ for some $g \in K[X]$. Then
	\begin{align*}
		\alpha \text{ is a simple root of } f &\iff X - \alpha \text{ is not a factor of } g \\
						      &\iff g(\alpha) \neq 0.
	\end{align*}
	But 
	\[
	f'(X) = (X-\alpha)g'(X) + g(X) \implies f'(\alpha) = g(\alpha).
	\]
\end{proofbox}

By the $\gcd$ of polynomials $f, g \in K[X]$, not both zero, we mean the unique monic polynomial $\gcd(f, g)$ which generates the ideal $(f, g) \subset K[X]$.

This is the unique monic polynomial which divides both $f$ and $g$, and can be written as $af + bg$ for some $a, b \in K[X]$.

We can compute $\gcd(f, g)$, together with $a, b$ using Euclid's algorithm.

\begin{lemma}
	Let $f, g \in K[X]$ and let $L/K$ be any field extension. Then $\gcd(f, g)$ is the same computed in $K[X]$ and in $L[X]$.
\end{lemma}

\begin{proofbox}
	Running Euclid's algorithm in $f, g \in K[X]$ gives the same answer whether we work in $K[X]$ or $L[X]$.
\end{proofbox}

\begin{definition}
	An irreducible polynomial $f \in K[X]$ is \emph{separable}\index{separable polynomial} if it splits into distinct linear factors in a splitting field.

	The convention in this course is that we use the same definition for any $0 \neq f \in K[X]$.

	Anything which is not separable is called \emph{inseparable}\index{inseparable}.
\end{definition}

\begin{lemma}
	Let $0 \neq f \in K[X]$. Then
	\[
		f \text{ is separable} \iff \gcd(f, f') = 1.
	\]
\end{lemma}

\begin{proofbox}
	Let $L$ be a splitting field of $f$, for $f$ separable. Then, then
	\begin{align*}
		f \text{ separable} &\iff f \text{ and } f' \text{ have no common roots in } L \\
				    &\iff \gcd(f, f') = 1 \text{ in } L[X] \\
				    &\iff \gcd(f, f') = 1 \text{ in } K[X].
	\end{align*}
\end{proofbox}

\begin{theorem}
	Let $f \in K[X]$ be irreducible. Then $f$ is separable unless $\chr(K) = p > 0$ and $f(X) = g(X^p)$ for some $g \in K[X]$.
\end{theorem}

\begin{proofbox}
	Assume $f$ is monic. Since $f$ is irreducible, $\gcd(f, f') = 1$ or $f$. If $f' \neq 0$, then since $\deg f' < \deg f$, we have $\gcd(f, f') \neq f$, hence $\gcd(f, f') = 1$ and $f$ is separable.

	Now suppose that $f' = 0$. If
	\[
	f = \sum_{i = 0}^{d} c_i X^i \implies f' = \sum_{i = 1}^{d} ic_i X^{i-1},
	\]
	so $f' = 0 \implies i c_i = 0$. If $\chr(K) = 0$, then this implies $c_i = 0$ for all $1 \leq i \leq d$. So $f$ is constant, hence irreducible.

	If $\chr(K) = p > 0$, then we still get $c_i = 0$ for all $i$ with $p \nmid i$. Hence $f(X) = g(X^p)$ for some $g \in K[X]$.
\end{proofbox}

\begin{definition}
	Let $L/K$ be a field extension.
	\begin{enumerate}[(a)]
		\item $\alpha \in L$ is separable over $K$ if it is algebraic over $K$ and its minimal polynomial is separable.
		\item $L/K$ is separable if every $\alpha \in L$ is separable. In particular, $L/K$ is algebraic.
	\end{enumerate}
\end{definition}

\begin{theorem}[Primitive Element]\label{thm:prim_elt}
	If $L/K$ is finite and separable, then $L = K(\theta)$ for some $\theta \in L$.
\end{theorem}

\begin{proofbox}
	Write $L = K(\alpha_1, \ldots, \alpha_n)$ for some $\alpha_i \in L$. We must show that $L = K(\theta)$ for some $\theta \in L$.

	It suffices to prove the case $n = 2$, since the general case follows by induction on $n$. Write $L = K(\alpha, \beta)$ and let $f$ be the minimal polynomial of $\alpha$ over $K$, and $g$ the minimal polynomial of $\beta$ over $K$.

	Let $M$ be the splitting field for $fg$ over $L$. Then if we write
	\[
	f(X) = \prod_{i = 1}^{r} (X-\alpha_i), \qquad g(X) = \prod_{i = 1}^{s} (X - \beta_i),
	\]
	for $\alpha_i, \beta_j \in M$ and $\alpha_1 = \alpha$, $\beta_1 = \beta$, then
	\[
		L/K \text{ separable} \implies \beta \text{ separable over } K \implies \beta_1, \ldots, \beta_s \text{ distinct}.
	\]
	We pick some $c \in K$ and let $\theta = \alpha + c\beta$. Let
	\[
		F(X) = f(\theta - cX) \in K(\theta)[X].
	\]
	Then,
	\[
	F(\beta) = f(\theta - c\beta) = f(\alpha) = 0,
	\]
	and also $g(\beta) = 0$, so if $\beta_2, \ldots, \beta_s$ are not roots of $F$, then
	\begin{align*}
		&\gcd(F,g) = X - \beta \text{ in } M[X]\\
		\implies &\gcd(F,g) = X - \beta \text{ in } K(\theta)[X] \\
		\implies&\beta \in K(\theta).
	\end{align*}
	But then $\alpha = \theta - c \beta \in K(\theta)$, so $K(\alpha, \beta) \subset K(\theta)$. But clearly $K(\theta) \subset K(\alpha,\beta)$, so $K(\alpha, \beta) = K(\theta)$.

	We are done unless $F(\beta_j) = 0$ for some $2 \leq j \leq s$. But then $f(\theta - c \beta_j) = 0$, so
	\[
	\alpha + c \beta = \alpha_i + c \beta_j
	\]
	for some $1 \leq i \leq r$, $2 \leq j \leq s$. If $|K| = \infty$, then since $\beta \notin \{\beta_2, \ldots, \beta_s\}$, we can pick $c \in K$ such that this never happens.

	If $|K| < \infty$, then $|L| < \infty$ and hence $L^{\times}$ is cyclic generated by $\theta$, so $L = K(\theta)$.
\end{proofbox}

\begin{exbox}
	We saw in a previous example that $\mathbb{Q}(\sqrt 2, \sqrt 3) = \mathbb{Q}(\sqrt 2 + \sqrt 3)$.
\end{exbox}

%lecture 10

\begin{remark}
	The above theorems show that if $[K: \mathbb{Q}] < \infty$, then $K = \mathbb{Q}(\alpha)$ for some $\alpha \in K$.
\end{remark}

Our aim is to show that if $L/K$ is a field extension, and $\alpha_1, \ldots, \alpha_n \in L$, then
\[
	\alpha_1, \ldots, \alpha_n \text{ separable over } K \iff K(\alpha_1, \ldots, \alpha_n) / K \text{ is separable}.
\]

For $L/K$, $M/K$ field extensions, we write $\Hom_K(L,M)$ to denote the number of $K$-embeddings from $L$ into $M$.

\begin{lemma}
	Let $[L:K] < \infty$. Suppose $L = K(\alpha)$, and $f$ is the minimal polynomial of $\alpha$ over $K$. Let $M/K$ be any field extension. Then
	\[
		|\!\Hom_K(L,M)| \leq [L:K],
	\]
	with equality if and only if $f$ splits into distinct linear factors over $M$.
\end{lemma}

\begin{proofbox}
	We know from before that
	\[
		|\!\Hom_K(L,M)| = |\{\text{roots of } f \text{ in } M\}| \leq \deg f = [L:K],
	\]
	with equality if and only if $f$ splits into distinct linear factors over $M$.
\end{proofbox}

\begin{theorem}\label{thm:num_of_homs}
	Let $[L:K] < \infty$. Write $L = K(\alpha_1, \ldots, \alpha_n)$ and let $f_i$ be the minimal polynomial of $\alpha_i$ over $K$. Let $M/K$ be any field extension. Then,
	\[
		|\!\Hom_K(L,M)| \leq [L:K],
	\]
	with equality if and only if $f_i$ splits into distinct linear factors over $M$.
\end{theorem}

We can easily extend this theorem as follows:

\begin{theorem}
	Let $\sigma : K \to M$ be an embedding. Then
	\[
		|\{\sigma\text{-homomorphisms } L \to M\}| \leq [L:K],
	\]
	with equality if and only if each $\sigma(f_i)$ splits into distinct linear factors over $M$.
\end{theorem}

We will use this variant in the induction argument.

\begin{proofbox}
	We will prove this by induction on $n$. The case $n = 1$ has already been proven, so suppose $n > 1$.

	Let $K_1 = K(\alpha_1)$. Then from the above lemma,
	 \[
		 |\!\Hom_K(K_1, M)| \leq [K_1:K]. \tag{1}
	\]
	Let $\sigma \in \Hom_K(K_1,M)$. Then by the induction hypothesis, the number of $\sigma$-homomorphisms
	\[
		|\{\sigma\text{-homomorphisms } L = K_1(\alpha_2, \ldots, \alpha_n) \to M\}| \leq [L:K_1]. \tag{2}
	\]
	Hence the number of homomorphisms is
	\[
		|\!\Hom_K(L,M)| \le [L:K_1][K_1:K] = [L:K].
	\]
	Note if equality holds then equality holds in both inequalities above. (1) gives that $f_1$ splits into distinct linear factors over $M$. But reordering $\alpha_i$ gives the same conclusion for all the $f_i$.

	Conversely, if each $f_i$ splits into distinct linear factors, then we get equality in (1).

	Moreover, for $2 \leq i \leq n$ the minimal polynomial of $\alpha_i$ over $K_1$ divides the minimal polynomial of $\alpha_i$ over $K$, hence $f_i$ splits into distinct linear factors over $M$.

	By the induction hypothesis, this gives equality everywhere.
\end{proofbox}

\begin{corollary}
	Let $[L:K] < \infty$, and write $L = K(\alpha_1, \ldots, \alpha_n)$. Let $f_i$ be the minimal polynomial of $\alpha_i$ over $K$. Let $M/K$ be any field extension in which $\prod f_i$ splits as a product of linear factors. Then the following are equivalent:
	\begin{enumerate}[\normalfont(i)]
		\item $L/K$ is separable.
		\item Each $\alpha_i$ is separable over $K$.
		\item Each $f_i$ splits into distinct linear factors over $M$.
		\item $|\!\Hom_K(L,M)|=[L:K]$.
	\end{enumerate}
\end{corollary}

\begin{proofbox}
	(i) $\implies$ (ii) $\implies$ (iii) by definition, and (iii) $\implies$ (iv) as we have just shown.

	To show (iv) $\implies$ (i), let $\beta \in L$. Then applying this theorem to $L = K(\alpha_1, \ldots, \alpha_n, \beta)$ shows that $\beta$ is separable over $K$.
\end{proofbox}

(iv) is a useful characterisation of separable extensions.

\begin{exbox}
	Let $K$ be any field. The polynomial $T^{n} - Y \in K[Y,T]$ is irreducible:

	It suffices to consider factorizations of the form $f(T)(g(T)+Yh(T))$, where $f, g, h \in K[T]$. Since $K[Y]$ is a UFD with field of fractions $K(Y)$, it follows by Gauss' lemma that
	\[
		T^{n} - Y \in K(Y)[T]
	\]
	is irreducible. The field extension $K(X)/K(X^{n})$ is generated by $X$ which is a root of $T^{n} - X^{n} \in K(X^{n})[T]$. Putting $Y = X^{n}$ in this shows it is irreducible, and $[K(X):K(X^{n})] = n$.

	Now take $K = \mathbb{F}_p$ and $n = p$. We claim that $\mathbb{F}_p(X) / \mathbb{F}_p(X^{p})$ is an inseparable extension of degree $p$. Indeed, the minimal polynomial of $X$ over $\mathbb{F}_p(X^{p})$ is
	\[
		f(T) = T^{p} - X^{p} \in \mathbb{F}_p(X^{p})[T],
	\]
	which is inseparable since
	\[
	f(T) = (T-X)^{p}.
	\]
\end{exbox}

%lecture 11

\newpage

\section{Galois Extensions}

\begin{definition}
	An \emph{automorphism}\index{automorphism} of a field $L$ is a bijective homomorphism $\sigma : L \to L$. We write $\Aut(K)$ for the group of automorphisms of $L$ under composition.
\end{definition}

Note if $\sigma$ is a homomorphism, then so is $\sigma^{-1}$.

\begin{definition}
	Let $L/K$ be a field extension. A $K$\emph{-automorphism}\index{$K$-automorphism} of $L$ is an automorphism $\sigma \in \Aut(L)$, whose restriction to $K$ is the identity map.

	The $K$-automorphisms of $L$ form a subgroup $\Aut(L/K) < \Aut(L)$.
\end{definition}

\begin{remark}
	\begin{enumerate}[(i)]
		\item[]
		\item $\Aut(\mathbb{Q})$ and $\Aut(\mathbb{F}_p)$ are both trivial, as $\Aut(L) = \Aut(L/K)$, where $K$ is the prime subfield of $L$.
		\item If $[L:K]< \infty$, then any $K$-embedding $L \to L$ is surjective, i.e. $\Hom_K(L,L) = \Aut(L/K)$.
	\end{enumerate}
\end{remark}

\begin{lemma}
	Let $L/K$ be a finite extension. Then
	\[
		|\!\Aut(L/K)| \leq [L:K].
	\]
\end{lemma}

\begin{proofbox}
	Take $M = L$ in theorem (\ref{thm:num_of_homs}).
\end{proofbox}

\begin{definition}
	If $S \subset \Aut(L)$ is any subset, we define the \emph{fixed field}\index{fixed field} of $S$ to be
	\[
		L^S = \{x \in L \mid \sigma(x) = x \text{ for all } \sigma \in S\}.
	\]
	This is a subfield of $L$.
\end{definition}

\begin{definition}
	A field extension $L/K$ is \emph{Galois}\index{Galois extension} if it is algebraic and
	\[
	K = L^{\Aut(L/K)}.
	\]
\end{definition}

In essence, the automorphisms which fix $K$ don't fix anything else.

\begin{exbox}
	\begin{enumerate}[(i)]
		\item $\Aut(\mathbb{C}/\mathbb{R}) = \{1, \tau\}$, where $\tau$ is complex conjugation.

			If $z \in \mathbb{C}$, then
			\[
			z \in \mathbb{R} \iff \tau(z) = z.
			\]
			Hence $\mathbb{C}/\mathbb{R}$ is Galois.
		\item Let $L = \mathbb{Q}(\sqrt 2)$, $f(X) = X^2 - 2$. Then
			\[
				\Aut(L/\mathbb{Q}) \leftrightarrow \{\text{roots of } f \text{ in } L\},
			\]
			hence $\Aut(L/\mathbb{Q}) = \{1, \tau\}$, where
			\[
			\tau(a + b \sqrt 2) = a - b \sqrt 2,
			\]
			for $a, b \in \mathbb{Q}$. Then $L^{\tau} = \mathbb{Q}$, so $L/\mathbb{Q}$ is Galois.
		\item Let $L = \mathbb{Q}(\sqrt[3]2)$, and $f(X) = x^3 - 2$. Then $\Aut(L/\mathbb{Q})$ corresponds with the number of roots of $f$ in $L$.

			But since $L \subset \mathbb{R}$, we have $\Aut(L/\mathbb{Q}) = 1$, so $L/\mathbb{Q}$ is not Galois.
		\item Let $K/\mathbb{F}_p$ be a finite extension, and let $\phi : K \to K$ be $x \mapsto x^{p}$.

			Then $\phi \in \Aut(K/\mathbb{F}_p)$, so $K^{\phi} = \{x \in K \mid \phi(x) = x\} = \{\text{roots of } X^p - x \text{ in } K\} \supset \mathbb{F}_p$. So $K / \mathbb{F}_p$ is Galois.
	\end{enumerate}
\end{exbox}

\begin{theorem}\label{thm:class_gal_ext}
	Let $[L:K]< \infty$, and $G = \Aut(L/K)$. The following are equivalent:
	\begin{enumerate}[\normalfont(i)]
		\item $L/K$ is Galois.
		\item $L/K$ is normal and separable.
		\item $L$ is the splitting field of a separable polynomial over $K$.
		\item $|G| = [L:K]$.
	\end{enumerate}
\end{theorem}

\begin{proofbox}
	To show (i) $\implies$ (ii), let $\alpha \in L$ and $\{\sigma(\alpha) \mid \sigma \in G\} = \{\alpha_1, \ldots, \alpha_m\}$, with $\alpha_1, \ldots, \alpha_m$ distinct. Then let
	\[
	f(X) = \prod_{i = 1}^{m}(X - \alpha_i).
	\]
	We let $\sigma \in G$ act on $L[X]$ by
	\[
	\sigma\Biggl( \sum c_i X^i \biggr) = \sum \sigma(c_i) X^i.
	\]
	Since $G$ permutes the $\alpha_i$, we have $\sigma f = f$ for all $\sigma \in G$.

	Hence if $L/K$ Galois, then $f \in K[X]$. So let $g$ be the minimal polynomial of $\alpha$ over $K$. Since $f(\alpha) = 0$, we have $g | f$, but as $g(\sigma(\alpha)) = \sigma(g(\alpha)) = 0$ for all $\sigma \in G$, every root of $f$ is a root of $g$.

	By construction, $f$ is separable and monic, so $f = g$. Hence the minimal polynomial of $\alpha$ over $K$ splits into distinct linear factors over $L$. Since $\alpha \in L$ was arbitrary, this shows that $L/K$ is normal and separable.

	For (ii) $\implies$ (iii), theorem \ref{thm:norm_split} says that $L$ is the splitting field of some $f \in K[X]$. Hence write
	\[
	f = \prod_{i = 1}^{m} f_i^{e_i},
	\]
	where the $f_i \in K[X]$ are distinct irreducibles, and $e_i \geq i$. Then, $L/K$ separable implies each $f_i$ is separable. Moreover, as $\gcd(f_i, f_j) = 1$ in $K[X]$, then $\gcd(f_i, f_j) = 1$ in $L[X]$, so
	\[
	g = \prod_{i = 1}^{m} f_i
	\]
	is separable, and $L$ is a splitting field for $g$ over $K$.

	To show (iii) $\implies$ (iv), let $L$ be the splitting field of a separable polynomial $f \in K[X]$. Then $L = K(\alpha_1, \ldots, \alpha_n)$ where $\alpha_1, \ldots, \alpha_n$ are the roots of $f$.

	The minimal polynomial $f_i$ of each $\alpha_i$ divides $f$, and so splits into distinct linear factors over $L$.

	Taking $M = L$ in theorem \ref{thm:num_of_homs} gives $|\!\Aut(L/K)| = [L:K]$.

	Finally, to show (iv) $\implies$ (i), note that
	\[
	G \leq \Aut(L/L^G) \leq \Aut(L/K) = G,
	\]
	hence $G = \Aut(L/L^G)$. Therefore,
	\[
		[L:K] = |G| = |\!\Aut(L/L^G)| \leq [L:L^G],
	\]
	which gives
	\[
		[L:L^G][L^G:K] \leq [L:L^G],
	\]
	so $L^G = K$.
\end{proofbox}

\begin{definition}
	If $L/K$ is a Galois extension, then we write $\Gal(L/K)$ for $\Aut(L/K)$. This is called the \emph{Galois group}\index{Galois group} of $L$ over $K$.
\end{definition}

%lecture 12

\begin{remark}
	We saw in this proof that if $L/K$ is Galois and $\alpha \in L$, then the minimal polynomial of $\alpha$ over $K$ is
	\[
	\prod_{i = 1}^{m} (X - \alpha_i),
	\]
	where $\alpha_1, \ldots, \alpha_m$ are the distinct elements of $\{\sigma(\alpha) \mid \sigma \in G\}$.
\end{remark}

\begin{theorem}[Fundamental Theorem of Galois Theory]
	Let $L/K$ be a finite Galois extension, $G = \Gal(L/K)$.
	\begin{enumerate}[\normalfont(i)]
		\item Let $F$ be an intermediate field, $K \subset F \subset L$. Then $L/F$ is Galois and $\Gal(L/F)$ is a subgroup of $G$.
		\item There is an inclusion reversing bijection from the intermediate fields to subgroups of $G$, by
			\begin{align*}
				F &\mapsto \Gal(L/F) \\
				L^H &\mapsfrom H.
			\end{align*}
		\item Let $F$ be an intermediate field. Then,
			\begin{align*}
				F/K \text{ Galois} &\iff \sigma F = F \text{ for all } \sigma \in G \\
						   &\iff H = \Gal(L/F) \text{ is a normal subgroup of } G.
			\end{align*}
			In this case, the restriction map
			\begin{align*}
				G &\to \Gal(F/K) \\
				\sigma &\mapsto \sigma|_F
			\end{align*}
			is surjective with kernel $H$, and so
			\[
			\Gal(F/K) \cong G/H.
			\]
	\end{enumerate}
\end{theorem}

\begin{proofbox}
	(i). By theorem \ref{thm:class_gal_ext}, $L$ is the splitting field of some separable polynomial $f \in K[X]$.

	Then $L$ is the splitting field of $f$ over $F$, so $L/F$ is Galois, and $\Gal(L/F)$ is a subgroup of $\Gal(L/K)$ since any automorphism of $L$ acting as the identity on $F$ also acts as the identity on $K$.

	(ii) To show we have a bijection, we need to check that both compositions are the identity.

	First, note that $F = L^{\Gal(L/F)}$. This holds as $L/F$ is Galois, so by our previous theorem we are done.

	Then it suffices to prove that $\Gal(L/L^H) = H$. We certainly have $H \le \Gal(L/L^H)$, so it suffices to show that
	\[
	|\!\Gal(L/L^H)| \leq |H|.
	\]
	Let $F = L^H$. As $L/F$ is finite and separable, the theorem of the primitive element tells us that $L = F(\alpha)$ for some $\alpha \in L$.

	Then $\alpha$ is a root of
	\[
	f(X) = \prod_{\sigma \in H} (X - \sigma(\alpha)),
	\]
	which has coefficients in $L^H = F$. Therefore,
	\[
		|\!\Gal(L/L^H)| = [L:L^H] = [F(\alpha):F] \leq \deg f = |H|,
	\]
	as required. Now if $F_1 \subset F_2$, then $\Gal(L/F_2) \subset \Gal(L/F_1)$, so the bijection reverses inclusions.

	(iii). We first show that $F/K$ is Galois if and only if $\sigma F = F$ for all $\sigma \in G$.

	$\implies$ For the forwards direction, let $\alpha \in F$ have minimal polynomial $f$ over $K$. Then for any $\sigma \in G$, $\sigma(\alpha)$ is a root of $f$.

	Since $F/K$ is normal, we have $\sigma(\alpha) \in F$. Hence $\sigma(F) \subset F$. As $[\sigma F : K] = [F : K]$, it follows that $\sigma F = F$.

	$\impliedby$ Now for the backwards direction. Assume that $\sigma F = F$. Let $\alpha \in F$, then its minimum polynomial over $K$ is
	\[
	f(X) = \prod_{i = 1}^{m} (X - \alpha_i),
	\]
	where $\{\alpha_1, \ldots, \alpha_m\}$ are the distinct elements of $\{\sigma(\alpha) \mid \sigma \in G\}$. The assumption that $\sigma(F) = F$ tells us that $\alpha_1, \ldots, \alpha_m \in F$, hence $F/K$ is normal.

	But also as $L/K$ is Galois, $L/K$ is separable, hence $F/K$ is separable. So $F/K$ is normal and separable, hence it is Galois.

	Now suppose $H \le G$ corresponds to $F = L^H$. Then for $\sigma \in G$, we have
	\begin{align*}
		L^{\sigma H \sigma^{-1}} &= \{x \in L \mid \sigma \tau \sigma^{-1}(x) = x \text{ for all } \tau \in H\} \\
					 &= \{x \in L \mid \tau \sigma^{-1}(x) = \sigma^{-1}(x) \text{ for all } \tau \in H\} \\
					 &= \{x \in L \mid \sigma^{-1}(x) \in L^H = F\} \\
					 &= \sigma F,
	\end{align*}
	hence as $\sigma F = F$ for all $\sigma \in G$, $L^{\sigma H \sigma^{-1}} = L^{H}$ for all $\sigma \in G$, hence $\sigma H \sigma^{-1} = H$ for all $\sigma \in G$, so $H \le G$ is a normal subgroup.

	Now for the third part. Consider the restriction map $\mathrm{res}$. Then
	\begin{align*}
		\Ker(\mathrm{res}) &= \{\sigma \in \Gal(L/K) \mid \sigma(x) = x \text{ for all } x \in F\} \\
				   &= \Gal(L/F) = H,
	\end{align*}
	so $G/H \cong \Img(\mathrm{res}) \leq \Gal(L/K)$. But,
	\[
		|G/H| = \frac{|G|}{|H|} = \frac{[L:K]}{[L:H]} = [F:K]= |\!\Gal(F/K)|,
	\]
	so $\mathrm{res}$ is surjective, and $\Gal(F/K) = G/H$.
\end{proofbox}

\begin{exbox}
	Let $K = \mathbb{Q}(\sqrt 2, \sqrt 3)$. Then $K/\mathbb{Q}$ is the splitting field of the polynomial $(X^2-2)(X^2-3)$. So $K/\mathbb{Q}$ is normal, and hence Galois.

	Now if $\sigma \in \Gal(K/\mathbb{Q})$, then it is uniquely determined by $\sigma(\sqrt 2)$ and $\sigma(\sqrt 3)$. Since $\sigma(\sqrt 2) = \pm \sqrt 2$ and $\sigma(\sqrt 3) = \pm \sqrt 3$, we have $|\!\Gal(K/\mathbb{Q})| \leq 4$.

	But we saw before that $[K: \mathbb{Q}] = 4$, hence $|\!\Gal(K/\mathbb{Q})| = 4$, so letting
	\begin{align*}
		\sigma : \sqrt 2 & \mapsto \sqrt 2, & \sqrt 3 &\mapsto - \sqrt 3,\\
		\tau : \sqrt 2 &\mapsto -\sqrt 2, & \sqrt 3 &\mapsto \sqrt 3,
	\end{align*}
	then $\sigma^2 = \tau^2 = \id$, and $\sigma \tau = \tau \sigma$, so $\Gal(K/\mathbb{Q}) \cong C_2 \times C_2$.
	\[
	\begin{tikzcd}
		& \{1\} \arrow[dl, dash] \arrow[dr, dash] \arrow[d, dash]& \\
		\langle \sigma \rangle\arrow[dr, dash] & \langle \tau \rangle \arrow[d, dash] & \langle \sigma \tau \rangle \arrow[dl, dash] \\
						       & C_2 \times C_2 &
	\end{tikzcd}
	\begin{tikzcd}[column sep = small]
		& \mathbb{Q}(\sqrt 2, \sqrt 3) \arrow[dl, dash] \arrow[d, dash] \arrow[dr, dash] & \\
		\mathbb{Q}(\sqrt 2)\arrow[dr, dash] & \mathbb{Q}(\sqrt 3) \arrow[d, dash] & \mathbb{Q}(\sqrt 6) \arrow[dl, dash] \\
						    & \mathbb{Q} &
	\end{tikzcd}
	\]
\end{exbox}

%lecture 13

\begin{exbox}
	Let $K = \mathbb{Q}(\alpha)$, where $\alpha = \sqrt{2 + \sqrt 2}$. Then $(\alpha^2 - 2)^2 = 2$, so $\alpha$ is a root of $f(X) = X^4 - 4 X^2 + 2$.

	This is irreducible in $\mathbb{Z}[X]$ by Eisenstein's criterion with $p = 2$, hence is irreducible in $\mathbb{Q}[X]$ by Gauss' lemma, so $[K : \mathbb{Q}] = 4$.

	Now $(2 + \sqrt 2)(2 - \sqrt 2) = 2$, so $f$ has roots $\pm a$ and $\pm \sqrt 2/\alpha$. Hence $K$ is a splitting field for $f$ over $\mathbb{Q}$, so $K/\mathbb{Q}$ is both normal and Galois.

	So $\sigma \in \Gal(K/\mathbb{Q})$ is uniquely determined by $\sigma(\alpha)$, but $\sigma(\alpha) = \{ \pm \alpha, \pm \sqrt 2/\alpha\}$ and $|\!\Gal(K/\mathbb{Q})| = [K: \mathbb{Q}] = 4$, so all possibilities must occur.

	Fix $\sigma \in \Gal(K/\mathbb{Q})$ with $\sigma(\alpha) = \sqrt 2/\alpha$. Then $\sigma(\alpha^2) = 2/\alpha^2$, hence $\sigma(2 + \sqrt 2) = 2 - \sqrt 2$, and $\sigma(\sqrt 2) = - \sqrt 2$.

	Hence $\sigma^2(\alpha) = \sigma(\sqrt 2/\alpha) = - \sqrt 2/(\sqrt 2 / \alpha) = - \alpha$, so $\sigma^2 \neq \id$, but $\sigma^{4} = \id$.

	Hence $\Gal(K/\mathbb{Q}) \cong C_4$.
	\[
	\begin{tikzcd}
		\{\id\}\arrow[d,dash] \\
		C_2 \cong \langle \sigma^2 \rangle \arrow[d, dash] \\
		C_4
	\end{tikzcd}\qquad \qquad
	\begin{tikzcd}
		K = \mathbb{Q}(\alpha) \arrow[d, dash] \\
		\mathbb{Q}(\sqrt 2) \arrow[d, dash] \\
		\mathbb{Q}
	\end{tikzcd}
	\]
\end{exbox}

\begin{definition}
	Let $L_1, L_2$ be subfields of a field $M$. The \emph{composite}\index{composite} $L_1L_2$ is the smallest subfield of $M$ to contain both $L_1$ and $L_2$.
\end{definition}

This exists, as the intersection of any collections of subfield is a subfield.

\begin{theorem}\label{thm:composite_gal}
	Let $[M:K] < \infty$, and let $L_1$, $L_2$ be intermediate field.

	\begin{enumerate}[\normalfont(i)]
		\item If $L_1/K$ is Galois, then $L_1L_2/L_2$ is Galois and there is an injective group homomorphism
			\[
			\Gal(L_1L_2/L_2) \injto \Gal(L_1/K).
			\]
			This is surjective if and only if $L_1 \cap L_2 = K$.
		\item If $L_1/K$ and $L_2/K$ are both Galois, then $L_1L_2/K$ is Galois, and there is an injective group homomorphism
			\[
			\Gal(L_1L_2/K) \injto \Gal(L_1/K) \times \Gal(L_2/K).
			\]
			This is surjective if and only if $L_1 \cap L_2 = K$.
	\end{enumerate}
\end{theorem}

\begin{proofbox}
	Let's draw a diagram.
	\[
	\begin{tikzcd}[column sep = small]
		& M \arrow[d, dash] & \\
		& L_1L_2 \arrow[dl, dash] \arrow[dr, dash] & \\
		L_1 \arrow[dr, dash] & & L_2 \arrow[dl, dash] \\
				     & K &
	\end{tikzcd}
	\]
	(i) First $L_1/K$ is Galois implies $L_1$ is the splitting field of some separable polynomial $f \in K[X]$. Then $L_1L_2$ is the splitting field for $f$ over $L_2$, so $L_1L_2/L_2$ is Galois.

	Now if $\sigma \in \Gal(L_1L_2/L_2)$, then $\sigma|_K = \id$, and since $L_1/K$ is Galois, we have $\sigma(L_1) = L_1$.

	We consider the group homomorphism
	\begin{align*}
		\Gal(L_1L_2/L_2) &\to \Gal(L_1/K) \\
		\sigma &\mapsto \sigma|_{L_1}.
	\end{align*}
	It is injective, since if $\sigma|_{L_1} = \id$, then $\sigma$ acts trivially on both $L_1$ and $L_2$, hence on $L_1L_2$.

	Now suppose $L_1 \cap L_2 = K$. Then $L_1/K$ is finite and separable, so $L_1 = K(\alpha)$ for some $\alpha \in L_1$. Let $f \in K[X]$ be the minimal polynomial of $\alpha$ over $K$.

	Suppose $f = gh$ for some $g, h \in L_2[X]$. Then as $f$ splits into linear factors over $L_1$, $g$ and $h$ have coefficients in $L_1 \cap L_2 = K$, which is a contradiction to $f$ irreducible over $K$. So $f$ is irreducible in $L_2[X]$, and
	\[
		[L_1 : K] = \deg f = [L_1 L_2: L_2].
	\]
	So the map $\mathrm{res}$ is an isomorphism. Conversely, $\Img(\mathrm{res}) \le \Gal(L_1/L_1 \cap L_2) \leq \Gal(L_1/K)$, so if $\mathrm{res}$ is surjective, then $L_1 \cap L_2 = K$.

	(ii) Note if $L_i/K$ is Galois, then $L_i$ is a splitting field of some separable polynomial $f_i \in K[X]$.

	Then $L_1L_2$ is the splitting field of the separable polynomial $\lcm(f_1, f_2)$, so $L_1L_2/K$ is Galois.

	If $\sigma \in \Gal(L_1L_2/K)$, then $\sigma|_K = \id$ and since $L_i/K$ is normal, $\sigma(K_i) = K_i$. Now consider the group homomorphism
	\begin{align*}
		\Gal(L_1L_2/K) &\to \Gal(L_1/K) \times \Gal(L_2/K) \\
		\sigma &\mapsto (\sigma|_{L_1}, \sigma_{L_2})
	\end{align*}
	This is injective, since if both $\sigma|_{L_1} = \id$ and $\sigma|_{L_2} = \id$, then $\sigma$ acts trivially on $L_1L_2$.

	Moreover,
	\begin{align*}
		\text{map is surjective} & \iff [L_1L_2:K] = [L_1:K] [L_2:L] \\
					 &\iff [L_1L_2:L_2][L_2:K] = [L_1:K][L_2:K] \\
					 &\iff [L_1L_2:L_2] = [L_1:K] \\
					 &\iff L_1 \cap L_2 = K.
	\end{align*}
\end{proofbox}

\begin{theorem}
	Let $L/K$ be finite and separable. Then there exists finite extension $M/L$ such that:
	\begin{enumerate}[\normalfont(i)]
		\item $M/K$ is Galois.
		\item If $L \subset M' \subset M$ and $M'/K$ is Galois, then $M' = M$.
	\end{enumerate}
	We say $M/K$ is a \emph{Galois closure}\index{Galois closure} of $L/K$.
\end{theorem}

\begin{proofbox}
	By the theorem of the primitive element, $L = K(\alpha)$ for some $\alpha \in L$. Let $f$ be the minimal polynomial of $\alpha$ over $K$.

	Then as $L/K$ is separable, $f$ is separable. Let $M$ be a splitting field for $f$ over $K$. Then since $L = K(\alpha)$ where $\alpha$ is a root of $f$, it follows that $M$ is a splitting field of $f$ over $K$.

	Hence by theorem \ref{thm:class_gal_ext}, $M/K$ is Galois. Now let $M'$ be intermediary and Galois. As $\alpha \in M'$ and $M'/K$ is normal, $f$ splits into linear factors over $M'$. Hence $M' = M$.
\end{proofbox}

%lecture 14

\begin{exbox}
	$\mathbb{Q}(\sqrt[3]2)/\mathbb{Q}$ has Galois closure $\mathbb{Q}(\omega, \sqrt[3]2)/\mathbb{Q}$, where $\omega = e^{2\pi i/3}$.
\end{exbox}

\newpage

\section{Trace and Norm}
\label{sec:trace_norm}

Let $L/K$ be a finite extension, say $[L:K] = n$. For $\alpha \in L$, the map $m_\alpha : x \mapsto \alpha x$ is $K$-linear endomorphism of $L$, hence it has a trace and determinant.

\begin{definition}
	The \emph{trace}\index{trace} and \emph{norm}\index{norm} of $\alpha$ are:
	\[
	\Tr_{L/K}(\alpha) = \tr(m_\alpha), \qquad N_{L/K}(\alpha) = \det(m_\alpha).
	\]
	Concretely, if $L$ has $K$-basis $v_1, \ldots, v_n$ and $A = (a_{ij})$ is the unique $n \times n$ matrices with entries in $K$ such that
	\[
	\alpha(v_j) = \sum_{i = 1}^{n} a_{ij}v_i,
	\]
	then $\Tr_{L/K}(\alpha) = \tr A$ and $N_{L/K}(\alpha) = \det A$.
\end{definition}

\begin{exbox}
	Let $K = \mathbb{Q}$, and $L = \mathbb{Q}(\sqrt d)$ for $d \in \mathbb{Z}$. If $\alpha = x + y \sqrt d$, then $L$ has $K$-basis $1, \sqrt d$ and in this basis we can compute
	\begin{align*}
		\Tr_{\mathbb{Q}(\sqrt d)/\mathbb{Q}}(\alpha) &= \tr
		\begin{pmatrix}
			x & dy \\
			y & x
		\end{pmatrix}
		 = 2x, \\
		N_{\mathbb{Q}(\sqrt d)/\mathbb{Q}}(\alpha) &= \det
		\begin{pmatrix}
			x & dy \\
			y & x
		\end{pmatrix}
		 = x^2 - dy^2.
	\end{align*}
\end{exbox}

\begin{lemma}
	\begin{enumerate}[\normalfont(i)]
		\item[]
		\item $\Tr_{L/K}: L \to K$ is a $K$-linear map.
		\item $N_{L/K} : L \to K$ is multiplicative.
		\item If $\alpha \in K$, then $\Tr_{L/K}(\alpha) = [L:K]\alpha$, $N_{L/K}(\alpha) = \alpha^{[L:K]}$.
		\item If $\alpha \in L$, then
			\[
			N_{L/K}(\alpha) = 0 \iff \alpha = 0.
			\]
	\end{enumerate}
\end{lemma}

\begin{proofbox}
	(i) and (ii) follow from the corresponding statements for traces and determinants.

	(iii) If $\alpha \in K$, then $m_{\alpha}$ is represented by $\alpha I_n$, which has trace and determinant as indicated.

	(iv) $N_{L/K}(\alpha) \neq 0 \iff m_\alpha$ is invertible $\iff \alpha \neq 0$.
\end{proofbox}

\begin{lemma}\label{lem:tr_in_ext}
	Let $M/L/K$ be field extensions, and $\alpha \in L$. Then
	\[
		\Tr_{M/K}(\alpha) = [M:L] \Tr_{L/K}(\alpha), \qquad N_{M/K}(\alpha) = N_{L/K}(\alpha)^{[M:L]}.
	\]
\end{lemma}

\begin{proofbox}
	If $A$ represents $m_\alpha$ with respect to some basis for $L/K$, and $B$ represents $m_\alpha$ with respect to some basis for $M/K$ picked by following the proof of the tower law, then
	\[
	B=
	\begin{pmatrix}
		A &&& 0 \\
		  &A&& \\
		  &&\ddots&\\
		0&&&A
	\end{pmatrix}.
	\]
	Here $A$ is $[L:K]\times[L:K]$, and $B$ is $[M:K] \times [M:K]$. Then $\tr(B) = [M:L] \tr(A)$, and $\det(B) = \det(A)^{[M:L]}$.
\end{proofbox}

\begin{theorem}
	Let $[L:K] < \infty$, and let $\alpha \in L$. Let $f$ be the minimal polynomial of $\alpha$ over $K$, and say
	\[
	f(X) = X^{n} + a_{n-1}X^{n-1} + \cdots a_1X + a_0,
	\]
	for $a_i \in K$.

	Then $\Tr_{L/K}(\alpha) = - m a_{n-1}$, where $m = [L:K(\alpha)]$, and $N_{L/K}(\alpha) = ((-1)^{n}a_0)^{m}$.
\end{theorem}

\begin{proofbox}
	By lemma \ref{lem:tr_in_ext}, without loss of generality let $L = K(\alpha)$. If $A$ represents $m_\alpha$ with respect to basis $1, \alpha, \alpha^2, \ldots, \alpha^{n-1}$, then
	\[
	A =
	\begin{pmatrix}
		0 & & & & -a_{0} \\
		1 & 0 & & & -a_{1} \\
		  & 1 & \ddots & & \vdots \\
		  & & \ddots & 0 &\vdots \\
		0 & & & 1 & -a_{n-1},
	\end{pmatrix}
	\]
	which has trace $\Tr_{L/K}(\alpha) = \tr(A) = -a_{n-1}$, and norm $N_{L/K}(\alpha) = \det (A) = (-1)^{n} a_0$.
\end{proofbox}

\begin{exbox}
	Let $K = \mathbb{Q}$, and $L = \mathbb{Q}(\sqrt d)$. If $\alpha = x + y \sqrt d$, then $(\alpha - x)^2 - dy^2$, so $\alpha$ is a root of
	\[
	T^2 - 2x T + x^2 - dy^2 = 0.
	\]
	This gives the same trace and norm from before.
\end{exbox}

\begin{theorem}[Transitivity of Trace and Norm]\label{thm:trans_tr}
	Let $M/L/K$ be finite extensions and $\alpha \in M$. Then
	\begin{align*}
		\Tr_{M/K}(\alpha) &= \Tr_{L/K}(\Tr_{M/L}(\alpha)), \\
		N_{M/K}(\alpha) &= N_{L/K}(N_{M/L}(\alpha)).
	\end{align*}
\end{theorem}

\begin{proofbox}
	Just a sketch - this proof is non-examinable.

	By lemma \ref{lem:tr_in_ext}, without loss of generality let $M = L(\alpha)$. Let $f$ be the minimum polynomial of $\alpha$ over $L$, say
	\[
	f(X) = X^{n} + a_{n-1}X^{n-1} + \cdots + a_1 X + a_0,
	\]
	for $a_i \in L$. Then suppose $L/K$ has basis $v_1, \ldots, v_m$, and $M/L$ has basis $1, \alpha, \alpha^2, \ldots, \alpha^{n-1}$.

	If $A_i$ represents $m_{a_i}$ with respect to $v_1, \ldots, v_m$ and $B$ represents $m_\alpha$ with respect to $(v_i \alpha^{j-1})$, then
	\[
	B =
	\begin{pmatrix}
		0 & & & & -A_0 \\
		I & 0 & & & - A_1 \\
		0 & I & \ddots & & \vdots \\
		  & & \ddots & 0 & \vdots \\
		0 & & & I & -A_{n-1}
	\end{pmatrix},
	\]
	where $A_i$ is $m \times m$ and $B$ is $mn \times mn$. We compute
	\[
	\Tr_{M/K}(\alpha) = \tr(B) = - \tr(A_{n-1}) = \Tr_{L/K}(-a_{n-1}) = \Tr_{L/K}(\Tr_{M/L}(\alpha)),
	\]
	\[
	N_{M/K}(\alpha) = \det(B) = (-1)^{mn} \det(A_0) = N_{L/K}((-1)^{n}a_0) = N_{L/K}(N_{M/L}(\alpha)).
	\]
\end{proofbox}

\begin{theorem}
	Let $L/K$ be a finite Galois extension with $G = \Gal(L/K)$. Let $\alpha \in L$. Then
	\[
	\Tr_{L/K}(\alpha) = \sum_{\sigma \in G}\sigma(\alpha), \qquad N_{L/K}(\alpha) = \prod_{\sigma \in G} \sigma(\alpha).
	\]
\end{theorem}

\begin{proofbox}
	Note that the minimal polynomial of $\alpha$ over $K$ is
	\[
	f(X) = \prod_{i = 1}^{n}(X - \alpha_i),
	\]
	where $\{\alpha_1, \ldots, \alpha_n\} = \{\sigma(\alpha) \mid \sigma \in G\}$. Let $m = [L:K(\alpha)] = |\!\Stab_G(\alpha)|$. Then
	\begin{align*}
		\Tr_{L/K}(\alpha) &= m \sum_{i = 1}^{n} \alpha_i = \sum_{\sigma \in G}\sigma(\alpha), \\
		N_{L/K}(\alpha) &= \Biggl(\prod_{i = 1}^{n} \alpha_i \Biggr)^{m} = \prod_{\sigma \in G} \sigma(\alpha).
	\end{align*}
\end{proofbox}

\begin{exbox}
	Once again take $K = \mathbb{Q}$ and $L = \mathbb{Q}(\sqrt d)$. Then $L/K$ is Galois and $\Gal(L/K) = \{1, \sigma\}$, where $\sigma(\sqrt d) = \sqrt d$. For $\alpha = x + y \sqrt d$, we have
	\[
	\Tr_{L/K}(\alpha) = (x + y \sqrt d) + (x - y \sqrt d) = 2x,
	\]
	\[
	N_{L/K}(\alpha) = (x + y \sqrt d)(x - y \sqrt d) = x^2 - dy^2.
	\]
\end{exbox}

We can actually generalise the above theorem to $L/K$ separable. Let $\bar K$ be an algebraic closure of $K$.

%lecture 15

\begin{corollary}
	$|\!\Hom_K(L,\bar K)| = [L:K]$.
\end{corollary}

\begin{theorem}
	Let $L/K$ be a finite separable extension of degree $d$. Let $\sigma_1, \ldots, \sigma_d$ be the $K$-embeddings $L \injto \bar K$, and let $\alpha \in L$. Then,
	\[
	\Tr_{L/K}(\alpha) = \sum_{i = 1}^{d} \sigma_i(\alpha), \qquad N_{L/K}(\alpha) = \prod_{i = 1}^{d} \sigma_i(\alpha).
	\]
\end{theorem}

\begin{proofbox}
	Let $f$ be the minimal polynomial of $\alpha$ over $K$. Let $\alpha_1, \ldots, \alpha_n$ be the roots of $f$ in $\bar K$. By theorem \ref{thm:num_sig_homs},
	\begin{align*}
		\Hom_K(K(\alpha), \bar K) &\leftrightarrow \{\alpha_1, \ldots, \alpha_n\} \\
		\sigma &\mapsto \sigma(\alpha).
	\end{align*}
	Since $L/K(\alpha)$ is separable, each $K$-embedding $K(\alpha) \injto \bar K$ extends to an embedding $L \injto \bar K$ in exactly $m = [L:K(\alpha)]$ ways. Hence
	\begin{align*}
		\Tr_{L/K}(\alpha) &= m \sum_{j = 1}^{n} \alpha_j = \sum_{i = 1}^{d} \sigma_i(\alpha),\\
		N_{L/K}(\alpha) &= \Biggl( \prod_{j = 1}^{n} \alpha_j \Biggr)^{m} = \prod_{i = 1}^{d} \sigma_i(\alpha_i).
	\end{align*}
\end{proofbox}

\newpage

\section{Finite Fields}
\label{sec:fin_fields}

Fix $p$ a prime number. Recall that $\mathbb{F}_p = \mathbb{Z}/p \mathbb{Z}$. In this section we describe all finite fields of characteristic $p$ and their Galois theories. Recall the following:

\begin{proposition}
	\begin{enumerate}[\normalfont(i)]
		\item[]
		\item If $K$ is a finite field of characteristic $p$, then $|K| = p^{n}$, where $n = [K: \mathbb{F}_p]$.
		\item $K^{\times}$ is cyclic of order $p^{n}-1$.
		\item $\phi : K \to K$, $\phi(x) = x^{p}$ is a ring homomorphism, which is an automorphism in finite fields.
	\end{enumerate}
\end{proposition}

\begin{theorem}
	Let $q = p^{n}$ for some $n \geq 1$.
	\begin{enumerate}[\normalfont(i)]
		\item There exists a field with $q$ elements.
		\item Any field with $q$ elements is a splitting field of $X^q-X$ over $\mathbb{F}_p$.
	\end{enumerate}
	In particular, any two finite fields with the same order are isomorphic.
\end{theorem}

\begin{proofbox}


	(i) Let $L$ be a splitting field of $f(X) = X^q - X$ over $\mathbb{F}_p$. Let $K \subset L$ be the fixed field of $\phi^n : x \mapsto x^q$. Then
	\[
		K = \{\alpha \in L \mid \phi^n(\alpha) = \alpha\} = \{\alpha \in L \mid f(\alpha) = 0\}.
	\]
	But $f'(X) = -1$, so $\gcd(f, f') = 1$, hence $f$ is separable, and so $|K| = q$.

	(ii) Suppose $K$ is a field with $|K| = q$. Then from Lagrange's theorem $\alpha^{q-1} = 1$ for all $\alpha \in K^{\times}$, hence $\alpha^q = \alpha$ for all $\alpha \in K$. So we can factor
	\[
	f(X) = X^q - X = \prod_{\alpha \in K}(X - \alpha),
	\]
	hence $f$ splits over $K$. Moreover it cannot split over any proper subfield due to size constraints, so $K$ is the splitting field of $f$.
\end{proofbox}

In the following, write $\mathbb{F}_q$ for any field with $q$ elements. By the above theorem, any two such fields are isomorphic, although there is no canonical choice of isomorphism.

\begin{theorem}\label{thm:gal_fin_ext}
	$\mathbb{F}_{p^n}/\mathbb{F}_p$ is Galois with $\Gal(\mathbb{F}_{p^n}/\mathbb{F}_p)$ cyclic of order $n$, generated by the Frobenius map $\phi : x \to x^p$.
\end{theorem}

\begin{proofbox}
	Let $L = \mathbb{F}_{p^n}$, and let $G \subset \Aut(L/\mathbb{F}_p)$ be the subgroup generated by $\phi$. Then
	\[
		|L^G| = |L^{\phi}| = |\{\alpha \in L \mid \alpha^p - \alpha = 0\} \leq p.
	\]
	But as $\mathbb{F}_p \subset L^G$, $L^G = \mathbb{F}_p$. hence
	\[
	\mathbb{F}_p \subset L^{\Aut(L/\mathbb{F}_p)} \subset L^G = \mathbb{F}_p.
	\]
	The first equality shows that $L/\mathbb{F}_p$ is Galois, and the second shows that $\Aut(L/\mathbb{F}_p) = G$. Therefore $\Gal(L/\mathbb{F}_p) = G = \langle \phi \rangle$, and it has order $[L: \mathbb{F}_p ] = n$.
\end{proofbox}

\begin{corollary}
	Let $L/K$ be any extension of finite fields with $|K| = q$. Then $L/K$ is Galois with $\Gal(L/K)$ cyclic and generated by the $q$-power Frobenius map $x \mapsto x^q$.
\end{corollary}

\begin{proofbox}
	Let $L = \mathbb{F}_{p^n}$. We have $\mathbb{F}_p \subset K \subset L$. We have seen that $L/\mathbb{F}_p$ is Galois with
	\[
	G = \Gal(L/\mathbb{F}_p) = \langle \phi \rangle \cong C_n.
	\]
	Hence from the fundamental theorem of Galois theory, $L/K$ is Galois and
	\[
	H = \Gal(L/K) \leq G.
	\]
	Since $G = \langle \phi \rangle \cong C_n$, we have $H = \langle \phi^m \rangle$ for some $m \mid n$. Then
	\[
	[K: \mathbb{F}_p] = \frac{[L: \mathbb{F}_p]}{[L:K]} = \frac{|G|}{|H|} = (G:H) = m,
	\]
	so $q = |K| = p^m$, and $\phi^m : x \mapsto x^q$.
\end{proofbox}

\begin{corollary}
	$\mathbb{F}_{p^n}$ has a unique subfield of order $p^m$ for each $m \mid n$, and no others.
\end{corollary}

\begin{proofbox}
	Apply the fundamental theorem of Galois theory. The subgroups of $G = \Gal(\mathbb{F}_{p^n} / \mathbb{F}_p) = \langle \phi \rangle \cong C_n$ are the subgroup $H = \langle \phi^m \rangle$ for $m \mid n$.

	If $K = (\mathbb{F}_{p^n})^H$, then $H = \Gal(\mathbb{F}_{p^n}/K)$, and $[K: \mathbb{F}_p] = (G:H) = m$, so $|K| = p^m$.
\end{proofbox}

\begin{exbox}
	Let's look at $\mathbb{F}_{p^6}$. The lattice of subfields looks like this:
	\[
		\begin{tikzcd}[column sep = small]
		& \mathbb{F}_{p^6} \arrow[dl, dash, "3"] \arrow[dr, dash, "2"] & \\
		\mathbb{F}_{p^2} \arrow[dr, dash, "2"] & & \mathbb{F}_{p^3} \arrow[dl, dash, "3"] \\
						  & \mathbb{F}_p &
	\end{tikzcd}
	\]
	All extensions are Galois, with cyclic Galois group of order indicated.
\end{exbox}

%lecture 16

\newpage

\section{Galois Group of a Polynomial}
\label{sec:gal_poly}

Let $f \in K[X]$ be a separable polynomial of degree $n$. Let $L$ be a splitting field for $f$ over $K$. Then action of $G = \Gal(L/K)$ on the roots $\alpha_1, \ldots, \alpha_n$ of $f$ determines an injective group homomorphism $\iota : G \to S_n$.

Its image is the \emph{Galois group} of $f$ over $K$,\index{Galois group of a polynomial} written $\Gal(f)$ or $\Gal(f/K)$.

\begin{lemma}
	Let $f \in K[X]$ be a separable polynomial. Then
	\[
		f \text{ irreducible} \iff \Gal(f/K) \text{ transitive}.
	\]
\end{lemma}

\begin{proofbox}
	

	$\impliedby$ Assume $\Gal(f/K)$ transitive. If $f = gh$ for $g, h \in K[X]$, and $\deg g, \deg h > 0$, then $\Gal(f/K)$ sends roots of $g$ to roots of $g$, and so cannot act transitively on the roots of $f$.

	$\implies$ Without loss of generality assume $f$ is monic. Let $\alpha \in L$ be a root of $f$. Then $f$ is the minimal polynomial of $\alpha$ over $K$. Then the roots of $f$ are precisely $\{\sigma(\alpha) \mid \sigma \in \Gal(L/K)\}$.

	Hence $\Gal(L/K)$ acts transitively on $\alpha_1, \ldots, \alpha_n$, and so $\Gal(f/K) \leq S_n$ is a transitive subgroup.
\end{proofbox}

\begin{remark}
	Alternatively, by theorem \ref{thm:num_sig_homs}, there exists a $K$-isomorphism $K(\alpha_i) \cong K(\alpha_j)$, by $\alpha_i \mapsto \alpha_j$.

	This extends to an automorphism of $L$ by uniqueness of splitting fields.
\end{remark}

Let $f \in K[X]$ be monic and separable, with roots $\alpha_1, \ldots, \alpha_n$ in a splitting field $L$. Recall that
\[
\Disc(f) = \prod_{i < j} (\alpha_i - \alpha_j)^{2}.
\]

\begin{lemma}
	Assume $\chr K \neq 2$. Let $\triangle = \Disc(f)$. The fixed field of $\Gal(f/K) \cap A_n$ is $K(\sqrt{\triangle})$

	In particular, $\Gal(f/K) \leq A_n \iff \triangle$ is a square in $K$.
\end{lemma}

\begin{proofbox}
	The sign of a permutation $\pi \in S_n$ is defined so that
	\[
	\prod_{i < j}(X_{\pi(i)} - X_{\pi(j)}) = \sgn(\pi) \prod_{i < j}(X_i - X_j).
	\] 
	Define the quantity
	\[
	\delta = \prod_{i < j}(\alpha_i - \alpha_j),
	\]
	so $\delta^2 = \triangle$. Hence if $\sigma \in G = \Gal(f/K) = \Gal(L/K)$, then
	\[
	\sigma(\delta) = \sgn(\sigma) \delta.
	\]
	As $f$ is separable and $\chr K \neq 2$, $\delta = - \delta$. Hence
	\[
		G \cap A_n = \{\sigma \in G \mid \sgn(\sigma) = 1\} = \{\sigma \in G \mid \sigma(\delta) = \delta\} = \Gal(L/K(\delta)).
	\]
	So $L^{G\cap A_n} = K(\delta) = K(\sqrt{\triangle})$. In particular, $G \leq A_n \iff G \cap A_n = G \iff K(\sqrt{\triangle}) = K \iff \triangle$ is a square in $K$.
\end{proofbox}

\begin{remark}
	$G = \Gal(f/K) \leq S_n$ is really only defined up to conjugacy, since if we reorder $\alpha_1, \ldots, \alpha_n$ using $\sigma \in S_n$. Then $G$ changes to $\sigma G \sigma^{-1}$.

	However we can distinguish $\langle(1\, 2), (3\, 4)\rangle$ and $\langle(1\, 2)(3\, 4), (1\, 3)(2\, 4)\rangle$ even though they are both isomorphic to $C_2 \times C_2$.
\end{remark}

Hence we want to look at $G \injto S_n$ up to conjugacy.
\begin{itemize}
	\item For $n = 2$, the only transitive subgroup of $S_2$ is itself.
	\item For $n = 3$, the transitive subgroups of $S_3$ are $S_3$ and $A_3 \cong C_3$.
\end{itemize}

So if $f \in K[X]$ is irreducible, then $\Gal(f/K) = A_3$ or $S_3$. Then $\Gal(f/K) = A_3 \iff \Disc(f)$ is a square in $K$.

Taking $n = 3$ in the discriminant formula proved in example sheet 2 gives
\[
\Disc(X^3 + aX + b) = -4a^3 - 27b^2.
\]

\begin{exbox}
	Let $f(X) = X^3 - 3X + 1$. Then $\Disc(f) = -4(-3)^3 - 27 = 81 = 9^2$, so $\Gal(f/\mathbb{Q}) = 1$ or $A_3$.

	However $f$ is irreducible, so $\Gal(f/\mathbb{Q}) = A_3$.
\end{exbox}

For $n = 4$, the transitive subgroups of $S_4$ are $S_4, A_4, D_8, C_4$ and $V \cong C_2 \times C_2$.

Of these $S_4, A_4$ and $V$ are normal subgroups. There are 3 conjugate copies of each of $C_4$ and $D_8$.

Let $S_4$ act on $V \setminus \{\id\}$ by conjugation. As $g(1\, 2)(3\, 4) g^{-1} = (g(1) \, g(2))(g(3) \, g(4))$, it would be equivalent to let $S_4$ act on the set of ways of partitioning $\{1, 2, 3, 4\}$ into two subsets of size $2$.

The corresponding permutation representation is a group homomorphism $\pi : S_4 \to S_3$.

If $H = \{\sigma \in S_4 \mid \sigma(1) = 1\} = \langle(2 \, 3 \, 4),(2\, 3)\rangle \leq S_4$, then $\pi|_H : H \to S_3$ is an isomorphism.

So $\pi$ is surjective, and $|\!\ker \pi|= 4$. As $V$ is abelian, $V \leq \ker \pi$, hence $V = \ker \pi$.

If $G \leq S_4$, then applying the isomorphism theorem to $\pi|_G$ gives
\[
G / G \cap V \cong \pi(G) \leq S_3.
\]
Here is the subgroup in $S_3$ associated to each transitive subgroup of $S_4$.
\[
\begin{array}{c|c}
	G \leq S_4 & \pi(G) \leq S_3 \\
	\hline
	S_4 & S_3 \\
	A_4 & A_3 \\
	C_4, D_8 & C_2 \\
	V & 1
\end{array}
\]

%lecture 17

Let
\[
f(X) = \prod_{i = 1}^4 (X - \alpha_i)
\]
be a monic quartic polynomial. Define
\begin{align*}
	\beta_1 &= (\alpha_1 + \alpha_2)(\alpha_3 + \alpha_4),\\
	\beta_2 &= (\alpha_1 + \alpha_3)(\alpha_2 + \alpha_4),\\
	\beta_3 &= (\alpha_1+\alpha_4)(\alpha_2+\alpha_3).
\end{align*}

\begin{definition}
	The \emph{resolvent cubic}\index{resolvent cubic} is
	\[
	g(X) = \prod_{i = 1}^{3}(X - \beta_i).
	\]
\end{definition}

\begin{theorem}
	Let $f, g$ be as above.
	\begin{enumerate}[\normalfont(i)]
		\item If $f \in K[X]$, then $g \in K[X]$.
		\item If $f$ is separable, then $g$ is separable.
		\item If both hold, then
			\[
			\pi(\Gal(f/K)) = \Gal(g/K).
			\]
	\end{enumerate}
	In particular if $f \in K[X]$ is irreducible, then $\Gal(g/K)$ determines $\Gal(f/K)$ up to ambiguity between $C_4$ and $D_8$ when $|\!\Gal(g/K)| = 2$.
\end{theorem}

\begin{proofbox}
	\begin{enumerate}[(i)]
		\item More generally, each coefficient of $g$ is a symmetric polynomial in $\mathbb{Z}[\beta_1, \beta_2, \beta_3]$, hence a symmetric polynomial in $\mathbb{Z}[\alpha_1, \ldots, \alpha_4]$, and so by the symmetric function theorem is a $\mathbb{Z}$-coefficient polynomial in the coefficients of $f$.
		\item Note that if we expand:
			\begin{align*}
				\beta_1 - \beta_2 &= \alpha_1\alpha_3 + \alpha_1\alpha_4 + \alpha_2\alpha_3 + \alpha_2\alpha_4 \\
						  &\qquad -\alpha_1\alpha_2 - \alpha_1\alpha_4 - \alpha_2\alpha_3 - \alpha_3\alpha_4 \\
						  &= (\alpha_1 - \alpha_4)(\alpha_3 - \alpha_2).
			\end{align*}
			Hence if $f$ is separable, then $\alpha_1, \ldots, \alpha_4$ are distinct, so $\beta_1 \neq \beta_2$. The same calculation show that $\beta_1, \beta_2, \beta_3$ are distinct. Thus $g$ is separable.
		\item Let $M$ be a splitting field of $f$ over $K$. Let $\alpha_1, \ldots, \alpha_4 \in M$ be the roots of $f$. Then
			\[
			L = K(\beta_1, \beta_2, \beta_3) \subset M
			\]
			is a splitting field for $g$ over $K$.

			If an element of $\Gal(M/K)$ permutes $\alpha_1, \ldots, \alpha_4$ according to $\sigma \in S_4$, then it restrict to an element of $\Gal(L/K)$ permuting $\beta_1, \beta_2, \beta_3$ according to $\pi(\sigma) \in S_3$.

			In other words, there is a commutative diagram
			\[
			\begin{tikzcd}
				\Gal(M/K) \arrow[r, two heads, "\mathrm{res}"] \arrow[d, "\iota_4"] & \Gal(L/K) \arrow[d, "\iota_3"] \\
				S_4 \arrow[r, "\pi"] & S_4
			\end{tikzcd}
			\]
			By theorem 6.4(c)? the map $\mathrm{res} : \Gal(M/K) \to \Gal(L/K)$ is surjective, hence
			\[
			\pi(\Img \iota_4) = \Img(\iota_3) \implies \pi(\Gal(f/K)) = \Gal(g/K).
			\]
	\end{enumerate}
\end{proofbox}

\begin{proposition}
	Let $f$ be a monic quartic polynomial with resolvent cubic $g$. Then:
	\begin{enumerate}[\normalfont(i)]
		\item $\Disc(f) = \Disc(g)$.
		\item If $f(X) = X^4 + pX^2 + qX + r$, then $g(X) = X^3 - 2pX^2 + (p^2 - 4r)X + q^2$.
	\end{enumerate}
\end{proposition}

\begin{proofbox}


	(i) Exercise. Use the fact that $\beta_1 - \beta_2 = (\alpha_1 - \alpha_4)(\alpha_3 - \alpha_2)$.

	(ii) We must show that:
	\begin{align*}
		\beta_1 + \beta_2 + \beta_3 &= 2p, \tag{1}\\
		\beta_1 \beta_2 + \beta_2 \beta_3 + \beta_3 \beta_1 &= p^2 - 4r, \tag{2}\\
		\beta_1\beta_2\beta_3 &= -q^2 \tag{3}.
	\end{align*}
	For the proof of (1), note that
	\[
	\beta_1 + \beta_2 + \beta_3 = 2 \sum_{i < j} \alpha_i \alpha_j = 2p.
	\]
	Now let's try doing (3). Since $\alpha_1 + \alpha_2 + \alpha_3 + \alpha_4 = 0$, we have
	\begin{align*}
		\beta_1 &= -(\alpha_1 + \alpha_2)^2,\\
		\beta_2 &= -(\alpha_1 + \alpha_3)^2,\\
		\beta_3 &= -(\alpha_1+\alpha_4)^2.
	\end{align*}
	Now notice that
	\begin{align*}
		(\alpha_1 + \alpha_2 + \alpha_3) &= \alpha_1^2(\alpha_1 + \alpha_2 + \alpha_3 + \alpha_4) + \sum_{i < j < k} \alpha_i \alpha_j \alpha_k, \\
		\implies \beta_1 \beta_2 \beta_3 &= -q^2.
	\end{align*}
	This proves (3). Now (2) is just a further calculation (omitted as it is kinda cringe).
\end{proofbox}

\begin{exbox}
	Let $f(X) = X^4 - 4 X^2 + 2$. Then $g(X) = X(X^2 + 8X + 8)$.

	Now $\Disc(f) = \Disc(g) = 8^2 \Disc(X^2 + 8X + 8) = 2^{11}$. Now as $\Gal(g/\mathbb{Q}) = C_2$, we have $\Gal(f/\mathbb{Q}) = C_4$ or $D_8$.

	But note that $f(X) = (X^2 - 2 + \sqrt 2)(X^2 - 2 - \sqrt 2)$, $\Gal(f/\mathbb{Q}(\sqrt 2)) = \Gal(f/\mathbb{Q}) \cap A_4$ is not a transitive subgroup of $S_4$.

	This implies $\Gal(f/\mathbb{Q}) \cong C_4$.
\end{exbox}

Let's see how we can use this to determine the roots of a quartic polynomial. We do it step-by-step.
\begin{enumerate}[(i)]
	\item Replace $f(X)$ by $f(X+c)$ such that $f$ has no $X^3$ term.
	\item Find the roots $\beta_1, \beta_2, \beta_3$ of the resolvent cubic using the method of symmetric polynomials.
	\item Then as $\beta_i = -(\alpha_1 + \beta_{i+1})^2$, we have $\frac{1}{2}(\sqrt{-\beta_1} + \sqrt{-\beta_2} + \sqrt{-\beta_3}) = \alpha_1$, where we choose square roots such that
		\[
			\sqrt{-\beta_1}\sqrt{-\beta_2}\sqrt{-\beta_3} = -q.
		\]
\end{enumerate}

Recall that $\sigma \in S_n$ has cycle type $(n_1, \ldots, n_r)$ if, when written as a product of disjoint cycles, these cycles have length $n_1, \ldots, n_r$.

\begin{lemma}
	Let $f \in \mathbb{F}_p[X]$ be a separable polynomial with irreducible factors of degrees $n_1, \ldots, n_r$. Then $\Gal(f/ \mathbb{F}_p) \leq S_n$ is generated by a single element of cycle type $(n_1, \ldots, n_r)$.

	In particular, $\Gal(f/ \mathbb{F}_p)$ is cyclic with order $\lcm(n_1, \ldots, n_r)$.
\end{lemma}

\begin{proofbox}
	Let $L$ be a splitting field of $f$ over $\mathbb{F}_p$. Let $\alpha_1, \ldots, \alpha_n$ be the roots of $f$ in $L$. Then theorem \ref{thm:gal_fin_ext} tells us that $G = \Gal(L/\mathbb{F}_p)$ is cyclic generated by the Frobenius map $\phi : x \mapsto x^p$.

	Write $f = \prod f_i$, where $f_i \in \mathbb{F}_p[X]$ is irreducible of degree $n$. Since $G$ permutes the roots of each $f_i$ transitively, the action of $\phi$ on the roots of $f_i$ is given by a single $n_i$ cycle.
\end{proofbox}

%lecture 18

\begin{theorem}[Reduction mod $p$]
	Let $f \in \mathbb{Z}[X]$ be a monic separable polynomial of degree $n$. Let $p$ be a prime and suppose the reduction of $f$ mod $p$, say $\bar f \in \mathbb{F}_p[X]$ is also separable.

	Then $\Gal(\bar f / \mathbb{F}_p) \leq \Gal(f/ \mathbb{Q})$ as a subgroup of $S_n$, up to conjugacy.
\end{theorem}

\begin{corollary}
	With the same assumptions, suppose $\bar f = g_1 g_2 \cdots g_r$, where $g_i \in \mathbb{F}_p[X]$ is irreducible of degree $n_i$. Then $\Gal(f/ \mathbb{Q}) \leq S_n$ contains an element with cycle type $(n_1, n_2, \ldots, n_r)$.
\end{corollary}

\begin{proofbox}
	Combine the above lemma and theorem.
\end{proofbox}

\begin{exbox}
	Let $f(X) = X^4 - 3X + 1$. Then looking mod $2$,
	\[
		\bar f = X^4 + X + 1 \in \mathbb{F}_2[X]
	\]
	is irreducible. Then looking mod $5$,
	\[
		\bar f = (X+1)(X^3 - X^2 + X + 1) \in \mathbb{F}_5[X],
	\]
	where the latter polynomial is irreducible in $\mathbb{F}_5[X]$. Hence $\Gal(f/\mathbb{Q})$ contains a 3-cycle and a 4-cycle, hence must be $S_4$.
\end{exbox}

Now let's prove the above theorem.

\begin{proofbox}
	Let $f \in K[X]$ be a monic separable polynomial of degree $n$ with splitting field $L$ and roots $\alpha_1, \ldots, \alpha_n \in L$. Let
	\begin{align*}
		F(T_1, \ldots, T_n, X) &= \prod_{\sigma \in S_n}(X - (\alpha_1 T_{\sigma(1)} + \cdots + \alpha_n T_{\sigma(n)})) \\
				       &\in K[T_1, \ldots, T_n, X].
	\end{align*}
	Indeed, the coefficients of this polynomial are in $L$, and are fixed by $\Gal(L/K)$, hence are in $K$.

	We define an action $\sigma$ of $S_n$ on $K[T_1, \ldots, T_n, X]$ by permuting the $T_i$, i.e.
	\[
		(\sigma \ast h) (T_1, \ldots, T_n, X) = h(T_{\sigma(1)}, \ldots, T_{\sigma(n)}, X).
	\]
	We note that $\sigma \ast F = F$ for all $\sigma \in S_n$. Let's take a quick detour.
\end{proofbox}

\begin{lemma}
	Let $F_1 \in K[T_1, \ldots, T_n, X]$ be an irreducible factor of $F$. Then $\Gal(f/K) \leq S_n$ is conjugate to $\Stab(F_1) = \{\tau \in S_n \mid \tau \ast F_1 = F_1\}$.
\end{lemma}

\begin{proofbox}
	Without loss of generality, assume $F_1$ is monic in $X$. Replacing $F_1$ by $\tau \ast F_1$ for suitable $\tau \in S_n$, we may suppose it has a factor
	\[
	X - (\alpha_1 T_1 + \cdots + \alpha_n T_n).
	\]
	Then for each $\sigma \in G = \Gal(f/K)$ it has a factor
	\[
	X - (\alpha_{\sigma(1)}T_1 + \cdots + \alpha_{\sigma(n)} T_n).
	\]
	Now notice that
	\[
	\prod_{\sigma \in G}(X - (\alpha_{\sigma(1)}T_1 + \cdots + \alpha_{\sigma(n)} T_n))
	\]
	has coefficients in $K$, and divides $F_1$, hence is equal to $F_1$.

	For $\tau \in S_n$, we have
	\begin{align*}
		\tau \ast F_1 &= \prod_{\sigma \in G} (X - (\alpha_{\sigma(1)} T_{\tau(1)} + \cdots + \alpha_{\sigma(n)} T_{\tau(n)})) \\
		&= \prod_{\sigma \in G} (X - (\alpha_{\sigma\tau^{-1}(1)}T_1 + \cdots + \alpha_{\sigma\tau^{-1}(n)}T_n)) \\
		&= \prod_{\sigma \in G \tau^{-1}}(X - (\alpha_{\sigma(1)} T_1 + \cdots + \alpha_{\sigma(n)}T_n)).
	\end{align*}
	So $\tau \ast F_1 = F_1 \iff G = G \tau^{-1} \iff \tau \in G$.
\end{proofbox}

Let's go back to the proof we just interjected.

\begin{proofbox}
	By the symmetric function theorem, the coefficients of $F$ are $\mathbb{Z}$-coefficient polynomials in the coefficients of $f$. So if $f \in \mathbb{Z}[X]$, then $F \in \mathbb{Z}[T_1, \ldots, T_n, X]$.

	Let $\bar f \in \mathbb{F}_p[X]$ and $\bar F = \mathbb{F}_p[T_1, \ldots, T_n, X]$ be the polynomials obtained by reducing all the coefficients modulo $p$.

	We may equally construct $\bar F$ from $\bar f$ in the same way we construct $F$ from $f$.

	Write $F = F_1 F_2 \cdots F_s$ for $F_i \in \mathbb{Z}[T_1, \ldots, T_n, X]$ distinct irreducibles, and $\bar F = \Phi_1 \Phi_2 \cdots \Phi_t$, for $\Phi_i \in \mathbb{F}_p[T_1, \ldots, T_n, X]$ distinct irreducibles.

	Without loss of generality assume that $\Phi_1 \mid \bar F_1$. Then $\Phi_1 \nmid \bar F_j$ for all $j > 1$. Then
	\[
		\{ \tau \in S_n \mid \tau \ast \Phi_1 = \Phi_1\} \subseteq \{\tau \in S_n \mid \tau \ast F_1 = F_1\}.
	\]
	Hence the previous lemma shows that, up to conjugacy,
	\[
	\Gal(\bar f/ \mathbb{F}_p) \leq \Gal(f / \mathbb{Q}).
	\]
\end{proofbox}

\newpage

\section{Cyclotomic and Kummer Extensions}
\label{sec:cyc_kum}

Let $K$ be a field and $n \geq 1$ an integer. We suppose that $\chr K \nmid n$.

Let $L/K$ be a splitting field of $f(X) = X^n  - 1$. Since $f'(X) = n X^{n-1}$ and $n \cdot 1_K \neq 0$, we have $\gcd(f, f') = 1$, and so $f$ is separable.

Hence $L/K$ is Galois. Let $\mu_n = \{x \in L \mid x^n = 1\}$ be the group of $n$'th roots of unity. This is a subgroup of $L^{\ast}$ of order $n$.

\begin{definition}
$\zeta_n \in \mu_n$ is a \emph{primitive} $n$'\emph{th root of unity}\index{primitive root of unity} if it has order exactly $n$ in $L^{\ast}$. For example if $K \subset \mathbb{C}$, then we can take $\zeta_n = e^{2 \pi i/n}.$
\end{definition}

Then $\mu_n = \langle \zeta_n \rangle = \{1, \zeta_n, \zeta_n^2, \ldots, \zeta_n^{n-1}\}$ and $L = K(1, \zeta_n, \zeta_n^2, \ldots, \zeta_n^{n-1}) = K(\zeta_n)$.

\begin{definition}
	$K(\zeta_n)/K$ is called a \emph{cyclotomic extension}\index{cyclotomic extension}.
\end{definition}

%lecture 19

Recall that $(\mathbb{Z}/n \mathbb{Z})^{\ast} = \{a \in \mathbb{Z}/n \mathbb{Z} \mid \gcd(a, n) = 1\}$ is a group under multiplication, and it has order $\phi(n)$.

Let $K$ be a field with $\chr K \nmid n$, and let $\zeta_n$ be a primitive $n$'th root of unity.
\begin{theorem}
	There is an injective group homomorphism
	\[
		\Gal(K(\zeta_n)/K) \overset{\chi}{\injto} (\mathbb{Z}/n \mathbb{Z})^{\ast}.
	\]
	In particular, $\Gal(K(\zeta_n)/K)$ is abelian, and $[K(\zeta_n):K]$ divides $\phi(n)$.
\end{theorem}

\begin{proofbox}
	Let $G = \Gal(K(\zeta_n)/K)$. If $\sigma \in G$, then $\zeta_n$ and hence also $\sigma(\zeta_n)$ are roots of $X^n - 1$, so $\sigma(\zeta_n) = \zeta_n^a$ for some $\alpha \in \mathbb{Z}$.

	Moreover, since $\zeta_n$ is a primitive $n$'th root of unity, the value of $a$ is unique mod $n$. We define $\chi : G \to \mathbb{Z}/n \mathbb{Z}$ by $\sigma \mapsto a$.

	Now let $\sigma, \tau \in G$ with $\sigma(\zeta_n) = \zeta_n^a$ and $\tau(\zeta_n) = \zeta_n^b$. Then
	\[
	\sigma \tau(\zeta_n) = \sigma(\zeta_n^b) = \zeta_n^{ab},
	\]
	so $\chi(\sigma \tau) = ab = \chi(\sigma) \chi(t)$. In particular, $\chi(\sigma) \chi(\sigma^{-1}) = \chi(1) = 1$, so $\chi(\sigma) \in (\mathbb{Z}/n \mathbb{Z})^{\ast}$.

	Moreover, $\chi$ is a group homomorphism. Since any $\sigma \in G$ is uniquely determined by $\sigma(\zeta_n)$, it is clear that $\chi$ is injective.
\end{proofbox}

\begin{remark}
	If $\chi(\sigma) = a$, then $\sigma(\zeta) = \zeta^a$ for all $\zeta \in \mu_n$. So the definition of $\chi$ does not depend on the choice of $\zeta_n$.
\end{remark}

\begin{exbox}
	Let $p$ be a prime with $p \equiv 4 \pmod 5$. Let $K = \mathbb{F}_p$, $L = \mathbb{F}_{p^2}$ and $n = 5$. Since $5 \mid (p^2 - 1)$, there is $\zeta_5 \in L$ a primitive 5'th root of unity, but as $5 \nmid p-1$, $\zeta_5 \not \in K$.

	Hence we get $L = K(\zeta_5)$. Therefore,
	\[
		C_2 \cong \Gal(L/K) \overset{\chi}{\injto} (\mathbb{Z} / 5 \mathbb{Z})^{\ast},
	\]
	hence $\Img(\chi) = \{\pm 1\} \leq (\mathbb{Z} / 5 \mathbb{Z})^{\ast}$.
\end{exbox}

\begin{corollary}
	Let $K = \mathbb{F}_p$ and suppose $p \nmid n$. Then $[K(\zeta_n) : K]$ equals the order of $p$ in $(\mathbb{Z}/n \mathbb{Z})^{\ast}$.
\end{corollary}

\begin{proofbox}
	$\Gal(K(\zeta_n)/K)$ is generated by the Frobenius $\phi$ which sends $\zeta_n \mapsto \zeta_n^p$, so
	\begin{align*}
		[K(\zeta_n):K] &= \text{order of } \phi \text{ in } \Gal(K(\zeta_n)/K) \\
			       &= \text{order of } \chi(\phi) \text{ in } (\mathbb{Z}/n \mathbb{Z})^{\ast}.
	\end{align*}
\end{proofbox}

\begin{definition}
	Let $\zeta_n = e^{2\pi i/n}$. The $n$'\emph{th cyclotomic polynomial}\index{cyclotomic polynomial} is
	\[
	\Phi_n(X) = \prod_{a \in (\mathbb{Z} / n \mathbb{Z})^{\ast}}(X - \zeta_n^a).
	\]
	Its roots are the primitive $n$'th roots of unity. As $\Gal(\mathbb{Q}(\zeta_n) / \mathbb{Q})$ permutes these, we have $\Phi_n \in \mathbb{Q}[X]$, and clearly we have
	\begin{center}
		$\zeta^n = 1 \iff \zeta$ is a primitive $d$'th root of  unity for some $d \mid n$.
	\end{center}
	Hence we get
	\[
	X^n - 1 = \prod_{d \mid n} \Phi_d(X).
	\]
	It follows by induction on $n$ that $\Phi_n \in \mathbb{Z}[X]$.
\end{definition}

Let's look at some small cases. We have
\begin{align*}
	\Phi_1 &= X-1,\\
	\Phi_p &= \frac{X^p - 1}{X - 1} = X^{p-1} + X^{p-1} + \cdots + X^2 + X + 1,\\
	\Phi_4 &= X^2 + 1.
\end{align*}
In general, $\deg \Phi_n = \phi(n)$.

\begin{theorem}
	If $K = \mathbb{Q}$, then the group homomorphism $\chi$ is an isomorphism. In particular, $[\mathbb{Q}(\zeta_n) : \mathbb{Q}] = \phi(n)$, and $\Phi_n \in \mathbb{Q}[X]$ is irreducible.
\end{theorem}

\begin{proofbox}
	Let $p$ be a prime with $p \nmid n$. We show that $\Img \chi$ contains $p$ mod $n$. If this is true, then $\Img \chi$ contains $a$ mod $n$ for every $a$ coprime to $n$, hence $\chi$ is surjective as required.

	Let $f, g \in \mathbb{Q}[X]$ be the minimal polynomials of $\zeta_n$ and $\zeta_n^p$ over $\mathbb{Q}$. If $f = g$, then as $\Gal(f/\mathbb{Q})$ is transitive, there exists $\sigma \in \Gal(\mathbb{Q}(\zeta_n)/\mathbb{Q})$ with $\sigma(\zeta_n) = \zeta_n^p$.

	If not, then $f, g$ are distinct irreducibles dividing $X^n - 1$. So $f, g \in \mathbb{Z}[X]$ and $fg \mid (X^n - 1)$.

	As $\zeta_n$ is a root of $g(X^p)$, we have $f(X) \mid g(X^p)$. Reducing mod $p$ gives
	\[
	\bar f(X) \mid \bar g(X^p) = \bar g(X)^p.
	\]
	Both $\bar f$ and $\bar g$ divide the separable polynomial $X^n - 1 \in \mathbb{F}_p[X]$, so $\bar f(X) \mid \bar g(X)$. Hence
	\[
	\bar f(X)^2 \mid \bar f(X) \bar g(X) \mid X^n - 1,
	\]
	which is a contradiction to $X^n - 1$ separable.
\end{proofbox}

Using this we can tie up a loose end from earlier.

\begin{theorem}
	Let $n \ge 3$. A regular $n$-gon is constructible by ruler and compass if and only if $\phi(n)$ is a power of $2$.
\end{theorem}

\begin{proofbox}
	Let $\zeta_n = e^{2\pi i/n}$ and $\alpha = \zeta_n + \zeta_n^{-1} = 2 \cos (\frac{2\pi}{n})$.

	Since $\alpha \in \mathbb{R}$, $\zeta_n \not \in \mathbb{R}$ and $\zeta_n$ is a root of $X^2 - \alpha X + 1 \in \mathbb{Q}(\alpha)[X]$,  we have that $[\mathbb{Q}(\zeta_n) : \mathbb{Q}(\alpha)] = 2$.

	If a regular $n$-gon can be constructed, then $\alpha$ can be constructible. Hence $[\mathbb{Q}(\alpha): \mathbb{Q}]$ is a power of $2$. Therefore,
	\[
		\phi(n) = [ \mathbb{Q}(\zeta_n) : \mathbb{Q}] = 2 [\mathbb{Q}(\alpha) : \mathbb{Q}]
	\]
	is a power of $2$.

	For the converse, we can use the converse of the theorem in the relevant section. It remains to show that if $\phi(n)$ is a power of $2$, then there exists a sequence of fields
	\[
	\mathbb{Q} = K_m \subset K_{m-1} \subset \cdots \subset K_1 \subset K_0 = \mathbb{Q}(\zeta_n),
	\]
	where $[K_i : K_{i+1}] = 2$ for all $i$, and $K_1 = \mathbb{Q}(\alpha)$. By the fundamental theorem of Galois theory, it suffices to construct subgroups
	\[
		\{1\} = H_0 < H_1 < \cdots < H_{m-1} < H_m = (\mathbb{Z} / n \mathbb{Z})^{\ast},
	\]
	where $(H_i : H_{i-1}) = 2$ for all $i$, and $H_1 = \{\pm 1\}$.
	%lecture 20

	Assuming $H_0, H_1, \ldots, H_j$ have been constructed, and $H_j \neq G$, we note that $G/H_j$ has order a power of 2 and is non-trivial, so it contains an element $gH_j$ of order $2$. Then set $H_{j+1} = \langle H_j, g\rangle$ and repeat.
\end{proofbox}

\begin{corollary}
	A regular $n$-gon is constructible by ruler and compass if and only if $n$ is a power of $2$ times a product of distinct primes of the form $F_k = 2^{2^k} + 1$.
\end{corollary}

\begin{proofbox}
	Note that:
	\[
	n = \prod_{i}p_i^{\alpha_i} \implies \phi(n) = \prod_{i} p_i^{\alpha_i - 1} (p_i - 1).
	\]
	If $\phi(n)$ is a power of $2$ $\iff$ $n$ is a product of a power of 2 and distinct odd primes of the form $2^m + 1$.

	Now if $2^m + 1$ is a prime, then $m$ must be a power of $2$. Indeed, if $m = ab$ with $b > 1$ odd, then putting $x = 2^a$,
	\[
	2^m + 1 = 2^{ab} + 1 = x^b + 1 = (x+1)(x^{b-1} - x^{b-2} + \cdots - x + 1).
	\]
\end{proofbox}

The next question is for what $k$ is $F_k$ actually prime. Indeed, $F_0 = 3$, $F_1 = 5$, $F_2 = 17$, $F_3 = 257$ and $F_4 = 65537$ are all prime. However $F_5 = 641 \times 6700417$.

Many Fermat numbers have been proved composite and no more have been shown to be prime.

\begin{theorem}
	Let $K, L$ be fields and $\sigma_1, \ldots, \sigma_n : K \injto L$ be distinct field embeddings. If $\lambda_1, \ldots, \lambda _n \in L$ satisfy
	\[
	\lambda_1 \sigma_1(x) + \lambda_2 \sigma_2(x) + \cdots + \lambda_n \sigma_n(x) = 0
	\]
	for all $x \in K$, then $\lambda_1 = \lambda_1 = \cdots = \lambda_n = 0$.
\end{theorem}

\begin{proofbox}
	We induct on $n$. This is trivially true for $n = 1$, so assume $n \geq 2$ and
	\[
		\lambda_1 \sigma_1(x) + \cdots + \lambda_n \sigma_n(x) = 0 \tag{1}
	\]
	for all $x \in K$. Pick $y \in K$ such that $\sigma_1(y) \neq \sigma_2(y)$. Replacing $x$ by $xy$ in (1) gives
	\[
		\lambda_1 \sigma_1(x) \sigma_1(y) + \cdots + \lambda_n \sigma_n(x) \sigma_n(y) = 0 \tag{2}
	\]
	for all $x \in K$. Taking $\sigma_1(y) \times (1) - (2)$ gives a new relation with only $n-1$ terms, hence it must be a trivial relation. Thus by the induction hypothesis,
	\[
	\sigma_1(y) \lambda_i = \sigma_i(y) \lambda_i
	\]
	for $2 \leq i \leq n$. Since  $\sigma_1(y) \neq \sigma_2(y)$, we have $\lambda_2 = 0$. Thus (1) has only $n-1$ terms, so by the induction hypothesis, all $\lambda_i = 0$.
\end{proofbox}

\subsection{Kummer Theory}
\label{sub:kum_thm}

We continue to assume $\chr K \nmid n$, but now further assume that $\mu_n \subset K$, that is $K$ contains a primitive $n$'th root of unity $\zeta_n$.

Lat $a \in K^{\ast}$. Let $L/K$ be a splitting field of $f(X) = X^n - a$. Since $f'(X) = n X^{n-1}$ and $n \neq 0$, $f$ is separable. Hence $L/K$ is Galois.

Let $\alpha \in L$ be a root of $f$ in $L$. Then
\[
f(X) = \prod_{j = 0}^{n-1}(X - \zeta_n^j \alpha).
\]
Hence we know
\[
L = K(\alpha, \zeta_n \alpha, \ldots, \zeta_n^{n-1}\alpha) = K(\alpha).
\]
We sometimes write $\sqrt[n]a$ for this $\alpha$.

\begin{definition}
	$K(\sqrt[n]a)/K$ is called a \emph{Kummer extension}\index{Kummer extension}.
\end{definition}

\begin{theorem}
	Assume $\mu_n \subset K$ and $a \in K^{\ast}$. There is an injective group homomorphism
	\[
		\Gal(K(\sqrt[n]a)/K) \overset{\theta}{\injto} \mu_n.
	\]
	In particular $\Gal(K(\sqrt[n]a)/K)$ is a cyclic group and $[K(\sqrt[n]a):K]$ divides $n$.
\end{theorem}

\begin{proofbox}
	Let $G = \Gal(K(\sqrt[n]a)/K)$. If $\sigma \in G$, then $\sqrt[n]a$ and hence also $\sigma(\sqrt[n]a)$ are roots of $X^n - a$, so $\sigma(\sqrt[n]a) = \zeta_n^r \sqrt[n]a$ for some $0 \leq r < n$. Now define
	\begin{align*}
		\theta : G &\to \mu_n \\
		\sigma &\mapsto \zeta_n^r.
	\end{align*}
	Let $\sigma, \tau \in G$ be such that $\sigma(\sqrt[n]a) = \zeta_n^r \sqrt[n]a$, $\tau(\sqrt[n]a) = \zeta_n^s \sqrt[n]a$. Then
	\[
		\sigma \tau(\sqrt[n]a) = \sigma(\zeta_n^s \sqrt[n]a) = \zeta_n^{r+s} \sqrt[n]a.
	\]
	Hence
	\[
	\theta(\sigma \tau) = \zeta_n^{r+s} = \theta(\sigma) \theta(\tau),
	\]
	so $\theta$ is a group homomorphism.

	Since any $\sigma \in G$ is uniquely determined by $\sigma(\sqrt[n]a)$, it is clear that $\theta$ is injective.
\end{proofbox}

\begin{remark}
	The definition of $\theta$ does not depend on the choice of $\sqrt[n]a$. Indeed, if $\alpha^n = \beta^n = a$, then $\frac{\alpha}{\beta} \in \mu_n \subset K$. Therefore
	\[
	\sigma \biggl( \frac{\alpha}{\beta} \biggr) = \frac{\alpha}{\beta} \implies \frac{\sigma(\alpha)}{\alpha} = \frac{\sigma(\beta)}{\beta}.
	\]
\end{remark}

We denote
\[
	(K^{\ast})^n = \{x^n \mid x \in K^{\ast}\} \subset K^{\ast}.
\]

\begin{corollary}
	Assume $\mu_n \subset K$ and $a \in K^{\ast}$. Then,
	\[
		[K(\sqrt[n]a) : K] = \text{ order of } a \text{ in } K^{\ast}/(K^{\ast})^n.
	\]
	In particular, $X^n - a$ is irreducible in $K[X]$ if and only if $a$ is not a $d$'th power in $K$ for any $1 < d \mid n$.
\end{corollary}

\begin{proofbox}
	Let $\alpha \in \sqrt[n]a$ and $G = \Gal(K(\alpha)/K)$. Then
	\begin{align*}
		a^m \in (K^{\ast})^n &\iff \alpha^m \in K^{\ast} \text{ (where we use $\mu_n \subset K$)} \\
				     &\iff \sigma(\alpha^m) = \alpha^m \text{ for all } \sigma \in G \\
				     &\iff \theta(\sigma)^m = 1 \text{ for all } \sigma \in G \\
				     &\iff \Img \theta \subset \mu_m \\
				     &\iff [K(\alpha):K] = |\!\Img(\theta)| \text{ divides } m.
	\end{align*}
	Hence $[K(\alpha):K]$ is the least $m$ such that $a^m \in (K^{\ast})^n$. Now onto the second part:
	\begin{align*}
		X^n - a \text{ irreducible in } K[X] &\iff [K(\alpha):K] = n \\
						     &\iff a \text{ has order $n$ in } K^{\ast}/(K^{\ast})^n \\
						     &\iff \nexists \; m \mid n, m < n \text{ such that } a^m \in (K^{\ast})^n \\
						     &\iff \nexists\, 1 < d \mid n \text{ with } a \in (K^{\ast})^d.
	\end{align*}
	For the last iff, we put $n = md$ and use the fact that $\mu_n \subset K$.
\end{proofbox}

As a special case, let $n= 2$ and $\chr K \neq 2$. Then $[K(\sqrt a) : K] = 2$ provided $a \not \in (K^{\ast})^2$.

%lecture 21

\begin{theorem}[Kummer's theorem]
	Assume $\chr K \nmid n$ and $\mu_n \subset K$. Then every degree $n$ Galois extension $L/K$ with cyclic Galois group is of the form $L = K(\sqrt[n]a)$ for some $a \in K$.
\end{theorem}

\begin{proofbox}
	Write $\Gal(L/K) = \{\sigma^{i} \mid 0 \leq i < n\}$. Then by the linear independence of field embeddings, there exists $x \in L$ such that
	\[
	\sum_{j = 0}^{n-1} \zeta_n^{j} \sigma^j(x) \neq 0.
	\]
	This value $\alpha$ is known as the \emph{Lagrange resolvent}\index{Lagrange resolvent}. Then
	\[
	\sigma(\alpha) = \sum_{j = 0}^{n-1} \zeta_n^{j} \sigma^{j+1}(x) = \sum_{j = 0}^{n-1} \zeta_n^{j-1} \sigma^{j}(x) = \zeta_n^{-1} \alpha.
	\]
	The Galois conjugates $\sigma^{j}(\alpha) = \zeta_n^{-j}\alpha$ are distinct, so the minimal polynomial of $\alpha$ has degree at least $n$. But then as $[L:K] = n$, we get $[K(\alpha):K] = n$, and $L = K(\alpha)$.

	Finally $\sigma(\alpha^{n}) = \sigma(\alpha)^{n} = (\zeta_n^{-1}\alpha)^{n} = \alpha^{n}$, hence $\alpha^{n} \in K$.
\end{proofbox}

Now let $K$ be a field with $\chr K = 0$. Let $f \in K[X]$ be a polynomial.

\begin{definition}
	$f$ is \emph{soluble by radicals}\index{soluble by radicals} over $K$ if there exists fields
	\[
	K = K_0 \subset K_1 \subset K_2 \cdots \subset K_m,
	\]
	such that $f$ has a root in $K_m$, and for each $1 \leq i \leq m$, $K_i = K_{i-1}(\alpha_i)$ with $\alpha_i^{d_i} \in K_{i-1}$ for some $d_i \geq 1$.
\end{definition}

\begin{definition}
	A finite group $G$ is \emph{soluble}\index{soluble} if there exists subgroups
	\[
		\{1\} = H_0 \leq H_1 \leq H_2 \leq \cdots \leq H_m = G,
	\]
	such that for each $1 \leq i \leq m$, $H_{i-1} \lhd H_i$ and $H_i/H_{i-1}$ is abelian.
\end{definition}

\begin{remark}
	The definition is unchanged if we replace ``abelian'' by ``cyclic'', or ``cyclic with prime order''.
\end{remark}

\begin{exbox}
	$S_4$ is soluble as
	\[
		\{1\} \leq C_2 \leq V \leq A_4 \leq S_4,
	\]
	with $V \cong C_2 \times C_2$, $A_4/V \cong C_3$ and $S_4/A_4 \cong C_2$.
\end{exbox}

\begin{lemma}
	If $G$ is soluble, then so is every subgroup and quotient of $G$.
\end{lemma}

This is an exercise.

\begin{theorem}
	Let $f \in K[X]$ be irreducible. Then
	\[
		f \text{ is soluble by radicals} \iff \Gal(f/K) \text{ is soluble}.
	\]
\end{theorem}
To show this we prove another lemma.

\begin{lemma}
	Let $L/K$ be a finite Galois extension, with $\Gal(L/K) = \{\sigma_1, \ldots, \sigma_m\}$.

	Let $a \in L^{\ast}$ and $n \geq 1$. Then
	\[
		M = L(\mu_n, \sqrt[n]{\sigma_1(a)}, \ldots, \sqrt[n]{\sigma_m(a)})
	\]
	is a Galois extension of $K$.
\end{lemma}

\begin{proofbox}
	Let
	\[
		f(X) = \prod_{i = 1}^{m}(X^n - \sigma_i(a)) \in K[X].
	\]
	Then $M$ is the splitting field of $f$ over $L$, and moreover $M$ is the composite of $L$ and the splitting field of $f$ over $K$, hence $M/K$ is Galois.
\end{proofbox}

Let's get back to proving this theorem.

\begin{proofbox}
	$\implies$ By definition, there exists fields $K = K_0 \subset K_1 \subset \cdots \subset K_m$, such that $f$ has a root in $K_m$ and for each $1 \leq i \leq m$, $K_i = K_{i-1}(\alpha_i)$ with $\alpha_i^{d_i} \in K_{i-1}$ for some $d_i \geq 1$.

	By repeatedly applying the previous lemma, we may assume that $K_m / K$ is Galois. By adjoining suitable roots of unity first, we may further assume that each extension $K_i/K_{i-1}$ is either cyclotomic or Kummer. Hence $\Gal(K_i/K_{i-1})$ is abelian, so by the fundamental theorem of Galois theory, $\Gal(K_m/K)$ is soluble.

	Since $f$ has a root in $K_m$, and $K_m/K$ is normal, we know that $f$ splits in $K_m$. Hence $\Gal(f/K)$ is a quotient of $\Gal(K_m/K)$, so is soluble.

	$\impliedby$ By the Galois correspondence, there exists fields
	\[
	K = K_0 \subset K_1 \subset \cdots \subset K_m
	\]
	such that $f$ has a root in $K_m$, and each $K_i/K_{i-1}$ is Galois with cyclic Galois group. Let $n = \lcm [K_i:K_{i-1}]$. Then
	\[
	K = K_0 \subset K_0(\zeta_n) \subset K_1(\zeta_n) \subset \cdots \subset K_m(\zeta_n).
	\]
	By theorem \ref{thm:composite_gal}, $K_i(\zeta_n) / K_{i-1}(\zeta_n)$ is Galois, and
	\[
	\Gal(K_i(\zeta_n)/K_{i-1}(\zeta_n)) \injto \Gal(K_i/K_{i-1}),
	\]
	hence $\Gal(K_i(\zeta_n)/K_{i-1}(\zeta_n))$ is cyclic of order dividing $n$. Hence by Kummer's theorem, $f$ is soluble by radicals.
\end{proofbox}

\begin{corollary}
	If $f \in K[X]$ is a polynomial of degree $n \geq 5$ with Galois group $A_n$ or $S_n$, then $f$ is not soluble by radicals over $K$.
\end{corollary}

\begin{proofbox}
	$A_5$ is non-abelian and simple, hence is not soluble.

	Hence as $A_n$ and $S_n$ contain $A_5$ as a subgroup, they are not soluble for all $n \geq 5$.
\end{proofbox}

\begin{exbox}
	Let $K = \mathbb{Q}$, and $f(X) = X^5 - X + a$, where $a \in \mathbb{Z}$ and $\gcd(a, 10) = 1$.

	Then mod 2, $f = X^5 + X + 1 = (X^2 + X + 1)(X^3 + X^2 + 1)$, so $\Gal(f/\mathbb{Q})$ contains an element $\sigma$ with cycle type $(2, 3)$. So $\sigma^3$ is a transposition.

	From example sheet 4 question 5, we get that $\bar f \in \mathbb{F}_5[X]$ is irreducible.

	Now we know from example sheet 3 that a subgroup of $S_p$ containing both a $p$-cycle and a transposition is $S_p$ itself, so $\Gal(f/\mathbb{Q}) = S_5$, and $f$ is not soluble by radicals.
\end{exbox}

%lecture 22

\newpage

\section{Algebraic Closure}
\label{sec:alg_clo}

In this section we go back and try to prove the existence of algebraic closures for general fields. To do this, we need the following machinery:

\begin{theorem}[Zorn's Lemma]\index{Zorn's lemma}
	Let $S$ be a non-empty partially ordered set. If every chain in $S$ has an upper bound, then $S$ has a maximal element.
\end{theorem}

\begin{definition}
	A relation $\le$ on a set $S$ is a partial order if for all $x, y, z \in S$:
	\begin{enumerate}[(i)]
		\item $x \leq x$.
		\item If $x \leq y$ and $y \leq z$, then $x \leq z$.
		\item If $x \leq y$ and $y \leq x$, then $x = y$.
	\end{enumerate}
	$(S, \le)$ is called a \emph{partially ordered set} or a \emph{poset}. It is totally ordered if, moreover, for each $x, y \in S$, either $x \leq y$ or $y \leq x$.

	Let $T \subset S$ be a subset.
	\begin{itemize}
		\item $T$ is a \emph{chain} if it is totally ordered by $\leq$.
		\item $x \in S$ is an upper bound for $T$ if $t \leq x$ for all $t \in T$.
		\item $x \in S$ is maximal if there is no $y \in S$ with $x \leq y$ and $x \neq y$.
	\end{itemize}	
\end{definition}

Let's look at ways to use Zorn's lemma.

\begin{exbox}
	Let $V$ be a vector space and $(S, \leq)$ the set of linearly independent subsets of $V$, ordered by inclusion.

	If $T \subset S$ is a chain, then let
	\[
	Y = \bigcup_{X \in T} X.
	\]
	It may be checked that $Y$ is linearly independence, as linear dependence is a finite property. Hence $Y$ is an upper bound for $T$. Therefore Zorn's lemma applies, and tells us that we have a maximal element $B$. Then,
	\begin{enumerate}[(i)]
		\item $B$ is linearly independent, and
		\item $B \cup \{v\}$ is not linearly independent for any $v \in V \setminus B$.
	\end{enumerate}
	Hence this shows that $B$ spans $V$, i.e. $B$ is a basis for $V$.
\end{exbox}

\begin{exbox}
	Let $R$ be a non-zero ring. Let $(S, \le)$ be the set of all proper ideals of $R$, ordered by inclusion.

	Then $R$ nonzero $\implies \{0\} \in S \implies S$ is non-empty.

	Now if $T \subset S$ is a chain, then let
	\[
	J = \bigcup_{I \in T}I.
	\]
	If $x, y \in J$, then $x \in I_1$ and $y \in I_2$ for some $I_1, I_2 \in T$. Since $T$ is totally ordered, we have either $I_1 \subset I_2$ or $I_2 \subset I_1$. Thus $x + y \in J$. Also for all $r \in R$, $x \in J$, $rx \in J$.

	So $J$ is an ideal in $R$. Moreover it is a proper ideal since $1 \not \in J$. Therefore $J \in S$ is an upper bound for $T$. Zorn's lemma then shows that $S$ has a maximal element, hence $R$ has a maximal ideal.
\end{exbox}

\begin{theorem}
	Let $K$ be a field.
	\begin{enumerate}[\normalfont(i)]
		\item There is an algebraic extension $L/K$ such that every non-constant polynomial $f \in K[X]$ has a root in $L$.
		\item $K$ has an algebraic closure $\bar K$.
	\end{enumerate}	
\end{theorem}

\begin{proofbox}


	\begin{enumerate}[(i)]
		\item Let $S$ be the set of monic non-constant polynomials in $K[X]$. Ideally, we would like $L$ to look like $K(\alpha_f \mid f\in S)$, where $\alpha_f$ is a root of $f$.

			Let us expand this idea. Let $R = K[X_f \mid f \in S]$ be the polynomial ring in indeterminates $\{X_f \mid f \in S\}$. So the elements of $R$ are finite $K$-linear combinations of monomials of the form
			\[
			X_{f_1}^{d_1} X_{f_2}^{d_2} \cdots X_{f_r}^{d_r},
			\]
			where $f_i \in S$ and $d_i \in \mathbb{N}$. Let $I \subset R$ be the ideal generated by $\{f(X_f) \mid f \in S\}$.

			Now we need to show that $I \neq R$. If not, then $1 \in I$, i.e.
			\[
				1 = \sum_{f \in T} g_f f(X_f), \tag{$\ast$}
			\]
			for some finite subset $T \subset S$ and polynomials $g_f \in R$.

			Let $L/K$ be a splitting field for $\prod_{f \in T} f$. Then for each $f \in T$, let $\alpha_f \in L$ be a root of $f$. We define a ring homomorphism
			\begin{align*}
				\phi : R &\to L[X_f \mid f \in S \setminus T], \\
				X_f &\mapsto
				\begin{cases}
					\alpha_f & f \in T, \\
					X_f & f \not \in T,
				\end{cases}
			\end{align*}
			and $\phi(c) = c$ for all $c \in K$. Applying $\phi$ to ($\ast$) gives
			\[
				1 = \sum_{f \in T} \phi(g_f) f(\alpha_f) = 0,
			\]
			a contradiction. This proves the claim.

			Since $I \neq R$, from the previous example $R/I$ has a maximal ideal. Equivalently $R$ has a maximal ideal $J$ containing $I$.

			Let $L=  R/J$, and $\alpha_f = X_f + J \in L$. By construction, $f(\alpha_f) = 0$, since $f(X_j) \in I \subset J$. Since
			\[
				L = \bigcup_{T \subset S \text{ finite}} K(\alpha_f \mid f \in T),
			\]
			it follows that $L/K$ is an algebraic extension.
		\item Repeating the construction in (i) gives
			\[
			K \subset K_1 \subset K_2 \subset \cdots
			\]
			Each non-constant polynomial in $K_n[X]$ has a root in $K_{n+1}$. If $f \in K[X]$ has degree $n \geq 1$, then it has a root $\alpha_1$ in $K_1$. Then $f(X)/(X-\alpha_1)$ has a root $\alpha_2$ in $K_2$, and so on, so that $f$ splits into linear factors in $K_n$. Then we can define
			\[
			\bar K = \bigcup_{n \geq 1} K_n.
			\]
			This is a field since it is a union of fields totally ordered by inclusion. Then every polynomial in $K[X]$ splits into linear factors over $\bar K$, and each element of $\bar K$ belongs to some $K_n$, so $\bar K$ is algebraic over $K$. Now we can apply lemma \ref{lem:lin_split_alg_clo}.
	\end{enumerate}
\end{proofbox}

%lecture 23

Now we try prove uniqueness of algebraic closure.

\begin{proposition}
	Let $L/K$ be an algebraic extension and $M/K$ a field extension with $M$ algebraically closed. Then there exists a $K$-embedding $L \injto M$.
\end{proposition}

\begin{proofbox}
	Let
		\[
			S = \{(F, \sigma) \mid \sigma : F \injto M \text{ a $K$-embedding}\},
		\]
	with partial order $(F_1, \sigma_1) \leq (F_2, \sigma_2)$ if $F_1 \subset F_2$ and $\sigma_2|_{F_1} = \sigma_1$. Then $(S, \leq)$ is a poset, and it is non-empty as $(K, \id) \in S$.

	Suppose $T = \{(F_i, \sigma_i) \mid i \in I\}$ be a chain, where $I$ is some indexing set, and let
	\[
	F = \bigcup_{i \in I} F_i,
	\]
	be a field. Define
	\begin{align*}
		\sigma : F &\to M \\
		x &\mapsto \sigma_i(x) \text{ if } x \in F_i.
	\end{align*}
	This is well-defined since $\sigma_i$ and $\sigma_j$ agree on $F_i \cap F_j$.

	Then $(F, \sigma) \in S$ is an upper bound for $T$, hence by Zorn's lemma $S$ has a maximal element $(F, \sigma)$.

	Let $\alpha \in L$. Then $\alpha$ is algebraic over $K$, hence algebraic over $F$. By theorem \ref{thm:num_sig_homs}, we may extend $\sigma : F \injto M$ to $\tau : F(\alpha) \injto M$, using the fact that $M$ is algebraically closed.

	Then $(F, \sigma) \leq (F(\alpha), \tau)$. But $(F, \sigma)$ maximal means $F(\alpha) \in F$, so $\alpha \in F$. Therefore $F = L$, and $\sigma : L \injto M$ is a $K$-embedding as required.
\end{proofbox}

\begin{corollary}
	Let $K$ be a field, and $L_1, L_2$ be algebraic closures of $K$. Then there exists a $K$-isomorphism $\phi : L_1 \to L_2$.
\end{corollary}

Note that in general, $\phi$ is not unique.

\begin{proofbox}
	Since $L_1/K$ is algebraic and $L_2/K$ is a field extension with $L_2$ algebraically closed, using the previous proposition gives a $K$-embedding $\phi : L_1 \injto L_2$.

	Now any $\alpha \in L_2$ is algebraic over $K$, hence algebraic over $\phi(L_1)$. But $\phi(L_1) \cong L_1$ is algebraically closed, hence $\alpha \in \phi(L_1)$.

	This shows that $\phi$ is surjective.
\end{proofbox}

\newpage

\section{Artin's Theorem and Invariant Theory}
\label{sec:art_thm}

\begin{theorem}[Artin's Theorem on Invariants]
	Let $L$ be a field and $G \subset \Aut(L)$ be a finite subgroup. Then $L/L^G$ is a finite Galois extension with Galois group $G$.

	In particular, $[L:L^G] = |G|$.
\end{theorem}

\begin{remark}
	Let $K = L^G$. Then $G \leq \Aut(L/K)$, and
	\[
	K \subset L^{\Aut(L/K)} \subset L^G = K,
	\]
	so $K = L^{\Aut(L/K)}$. If we knew that $L/K$ is algebraic, then it would follow that $L/K$ is Galois.

	Moreover if we knew that $L/K$ is finite then
	\[
	L^G = L^{\Gal(L/K)} \implies G = \Gal(L/K).
	\]
\end{remark}

\begin{proofbox}
	Let $K = L^G$. Pick any $\alpha \in L$, and let
	\[
	f(X) = \prod_{i = 1}^{m} (X - \alpha_i),
	\]
	where $\alpha_1, \ldots, \alpha_m$ are the distinct elements of $\Orb_G(\alpha) = \{\sigma(\alpha) \mid \sigma \in G\}$.

	Then $\sigma f = f$ for all $\sigma \in G$, so $f \in K[X]$. Hence $\alpha$ is algebraic and separable over $K$, so $L/K$ is algebraic and separable. We have also shown that $[K(\alpha) : K] \leq |G|$ for all $\alpha \in L$.

	Now pick $\alpha \in L$ with $[K(\alpha):K]$ maximal, and we will show that $L = K(\alpha)$.

	Indeed, let $\beta \in L$, then $K(\alpha, \beta)/K$ is finite and separable. So by theorem \ref{thm:prim_elt}, $K(\alpha, \beta) = K(\theta)$ for some $\theta \in L$.

	By our choice of $\alpha$, we have $[K(\theta) : K] \leq [K(\alpha) : K]$, but as $K(\alpha) \subset K(\theta)$ this gives $K(\alpha) = K(\theta)$.

	Hence $\beta \in K(\alpha)$. This proves the claim.

	Note $|\!\Aut(L/K)| \leq [L:K] = [K(\alpha) : K] \leq |G|$, but since $G \subset \Aut(L/K)$ it follows that these are all equalities, and hence $|\!\Aut(L/K)| = [L:K]$ so $L/K$ is Galois, and $G = \Gal(L/K)$.
\end{proofbox}

\begin{exbox}
	Let $L = \mathbb{C}(X_1, X_2)$ and define $\sigma, \tau \in \Aut(L)$ by
	\begin{align*}
		(\sigma f)(X_1, X_2) &= f(i X_1, -iX_2), \\
		(\tau f)(X_1, X_2) &= f(X_2, X_1).
	\end{align*}
	Let $G = \langle \sigma, \tau \rangle \cong D_8$. Our aim is then to compute $L^G$.

	We spot that $X_1 X_2$ and $X_1^4 + X_2^4 \in L^G$. We want to show that $L^G = \mathbb{C}(X_1 X_2, X_1^4 + X_2^4)$.

	To show this, note that $L = \mathbb{C}(X_1 X_2, X_1^4 + X_2^4)(X_1)$. So it suffices to show that this has the appropriate degree. Let
	\begin{align*}
		f(T) = (T^4 - X_1^4)(T^4 - X_2^4) = T^8 - (X_1^4 + X_2^4)&T^4 + (X_1X_2)^4 \\
									 &\in \mathbb{C}(X_1 X_2, X_1^4 + X_2^4)[T].
	\end{align*}
	Hence $[L : \mathbb{C}(X_1 X_2, X_1^4 + X_2^4)] \leq 8$. But by Artin's theorem, $[L:L^G] = |G| = 8$, so by the tower law $L^G = \mathbb{C}(X_1 X_2, X_1^4 + X_2^4)$.
\end{exbox}

%lecture 24

Let $R$ be a ring and $G \subset \Aut(R)$ a subgroup. \emph{Invariant theory}\index{invariant theory} seeks to describe the rings
\[
	R^G = \{x \in R \mid \sigma(x) = x \text{ for all } \sigma \in G\}.
\]
This topic was studied extensively in the 19th century, and was the motivation for Hilbert's basis theorem.

It is also important in modern algebraic geometry for describing the quotient of an algebraic variety by a group action.

\begin{exbox}
	Let $K$ be a field, and $L = K(X_1, \ldots, X_n)$ be the field of rational functions in $n$ variables, i.e. the field of fractions of $R = K[X_1, \ldots, X_n]$.

	Let $G = S_n$ act on $L$ by permuting the $X_i$'s. Our aim is to compute $L^G$.

	Note that $L^G$ contains the elementary symmetric polynomials $s_i$. Moreover, by the symmetric function theorem,
	\[
		R^G = K[s_1, \ldots, s_n],
	\]
	and there are no polynomial relations satisfied by the $s_i$.

	Now we claim that $L^G = K(s_1, \ldots, s_n)$. Indeed, suppose that $\frac{f}{g} \in L^G$, where $f, g \in R$ are coprime.

	Then $\frac{\sigma(f)}{\sigma(g)} = \frac{f}{g}$ for all $\sigma \in G$. Since $R$ is a UFD and the units of $R$ are just $K^{\times}$, we have $\sigma(f) = c_0 f$, and $\sigma(g) = c_0 g$ for some $c_0 \in K^{\times}$.

	But $G$ has finite order, so say $|G| = N$. Then $f = \sigma^{N}(f) = c_0^N f$, so $c_0^N = 1$. Hence $fg^{N-1}$ and $g^N \in R^G = K(s_1, \ldots, s_n)$, so
	\[
	\frac{f}{g} = \frac{fg^{N-1}}{g^{N}} \in K(s_1, \ldots, s_n).
	\]
	For another proof, let
	\[
	f(T) = \prod_{i = 1}^{n} (T - X_i) = T^{n} - s_1 T^{n-1} + s_2 T^{n-2} - \cdots + (-1)^{n} s_n.
	\]
	Then $f \in K(s_1, \ldots, s_n)[T]$ is a polynomial of degree $n$, and $L$ is a splitting field for $f$ over $K(s_1, \ldots, s_n)$.

	From example sheet 1, $[L:K(s_1, \ldots, s_n)] \leq n!$. But also $[L:L^G] = n!$ by Artin's theorem, so $L^G = K(s_1, \ldots, s_n)$.
\end{exbox}

\begin{remark}
	We have shown that the Galois group of a generic monic polynomial of degree $n$ is $S_n$.
\end{remark}

It can be shown that for any finite group $G$, there exists a finite Galois extension $L/K$ with Galois group $G$. This may not be possible if we specify $K$, for example if $K = \mathbb{C}$ or $K = \mathbb{F}_p$.

When $K = \mathbb{Q}$, this is an open problem (the inverse Galois problem).

\begin{corollary}
	Let $S_n$ act on $L = K(X_1, \ldots, X_n)$ by permuting the $X_i$. If $\chr K \neq 2$, then $L^{A_n} = K(s_1, \ldots, s_n, \delta)$ where
	\[
	\delta = \prod_{i < j} (X_i - X_j).
	\]
\end{corollary}

\begin{proofbox}
	As $(S_n : A_n) = 2$, we have $[L^{A_n} : K(s_1, \ldots, s_n)] = 2$. We have $\sigma(\delta) = \sgn(\sigma) \delta$ for $\sigma \in S_n$. In particular, $\delta \in L^{A_n}$ and $\delta \not \in L^{S_n}$, so $L^{A_n} = K(s_1, \ldots, s_n, \delta)$.
\end{proofbox}

\begin{remark}
	It can be shown that if $R = K[X_1, \dots, X_n]$, then $R^{A_n} = K[s_1, \ldots, s_n, \delta]$.
\end{remark}

The idea of the proof is as follows: let $f \in R^{A_n}$. Pick $\sigma \in S_n \setminus A_n$, and write
\[
f = \frac{1}{2}((f + \sigma f) + (f - \sigma f)).
\]
Then $f + \sigma f$ is fixed by any permutation in $S_n$, and $f - \sigma f$ is divisible by $\delta$.

\newpage

\section{Fundamental Theorem of Algebra}
\label{sec:ftoa}

In this section we will prove the fundamental theorem of algebra, that is:

\begin{theorem}[Fundamental Theorem of Algebra]
	$\mathbb{C}$ is algebraically closed.
\end{theorem}

\begin{proofbox}
	We will use the following facts:
	\begin{enumerate}[(i)]
		\item Every polynomial over $\mathbb{R}$ of odd degree has a root in $\mathbb{R}$.
		\item Every quadratic over $\mathbb{C}$ has a root in $\mathbb{C}$ (using the quadratic formula).
		\item Every group of order $2^n$ (with $n \geq 1$) has an index 2 subgroup.
	\end{enumerate}
	Suppose $L/\mathbb{C}$ is a finite extension with $L \neq \mathbb{C}$. Replacing $L$ by its Galois closure over $\mathbb{R}$, we may assume that $L/\mathbb{R}$ is Galois. Let $G = \Gal(L/\mathbb{R})$.

	Let $H \leq G$ by a Sylow 2-subgroup. Then $[L^H : \mathbb{R}] = (G:H)$ is odd. So if $\alpha \in L^H$, then $[\mathbb{R}(\alpha) : \mathbb{R}]$ is odd, hence $\alpha \in \mathbb{R}$ by (i).

	Thus we get $L^H = \mathbb{R}$ and $G = H$ is a 2-group.

	Let $G_1 = \Gal(L/\mathbb{C}) \leq \Gal(L/\mathbb{R}) = G$. Since $L \neq \mathbb{C}$, we have $G_1$ non-trivial, so by (iii) it has an index 2 subgroup, say $G_2$. Then $[L^{G_2} : \mathbb{C}] = (G_1 : G_2) = 2$, which is impossible by (ii).
\end{proofbox}


%\newpage
%
%\subsection{End of 2023 Stuff}
%
%\begin{theorem}
%	Let $f \in K[T]$ be irreducible and monic. Let $L_f = K[T]/(f)$, and $t \in L_F$ the residue class of $T$ mod $(f)$. Then $L_f/K$ is a finite extension of fields, with $[L_f : K] = \deg(f)$, and $f$ is the minimal polynomial of $t$ over $K$.
%\end{theorem}
%
%So we have an extension of $K$ in which $f$ has a root. Recall that if $x$ is algebraic over $K$, then $K(x) \cong K[T] / (m_{x, K})$.
%
%\begin{corollary}
%	If $L = K(x)$ for $x$ algebraic over $K$, then there exists a unique isomorphism $\sigma : L_f \to K(x)$ such that $\sigma(t) = x$, with $f = m_{x, K}$.
%\end{corollary}
%
%\begin{definition}
%	Let $x, y$ be algebraic over $K$. Say $x, y$ are $K$-conjugate if they have the same minimal polynomial. Then both $K(x)$ and $K(y)$\index{$K$-conjugate} are isomorphic to $L_f$.
%\end{definition}
%
%\begin{corollary}
%	$x, y$ are $K$-conjugate if and only if there exists a $K$-isomorphism $\sigma : K(x) \to K(y)$ with $\sigma (x) = y$.
%\end{corollary}
%
%\begin{proofbox}
%	The forward argument comes from 6.1, and the backwards comes from, if $g \in K[T]$, then $\sigma(g(x)) = g(\sigma(x))$, so $x, y$ have the same minimal polynomial.
%\end{proofbox}
%
%This shows that the roots of an irreducible polynomial are algebraically indistinguishable.
%
%\begin{theorem}
%	Let $f \in K[T]$ be irreducible, and $\sigma : K \to L$ any homomorphism of fields. Let $\sigma f$ be the polynomial given by applying $\sigma$ to the coefficients of $f$.
%	\begin{enumerate}[\normalfont(i)]
%		\item If $x \in L$ is a root of $\sigma f$, then there exists a unique $\sigma$-homomorphism $\tau : L_f \to L$ such that $\tau(t) = x$.
%		\item Every $\sigma$-homomorphism $L_f \to L$ is of this form, so we have a bijection
%			\[
%				\{\sigma\text{-homomorphisms } L_f \to L\} \to \{\text{roots of } \sigma f \text{ in } L\}
%			.\]
%	\end{enumerate}
%	
%\end{theorem}
%
%\begin{exbox}
%	Take $K = \mathbb{Q}(\sqrt 2) \subset \mathbb{R}$, $L = \mathbb{C}$. Then there is a homomorphism $\sigma : K \to \mathbb{C}$ taking $x + y \sqrt{2} \mapsto x -y \sqrt 2$. Take $f = T^2 - (1 + \sqrt 2)$. This map $\tau$ must take $t$ to $\pm \sqrt{1 - \sqrt 2}$. But if we instead take $\sigma$ to be the inclusion, then $\tau$ takes $t$ to $\pm \sqrt{1 + \sqrt 2}$.
%\end{exbox}
%
%\begin{definition}
%	$f \in K[T]$ is a non-zero polynomial. An extension $L/K$ is a splitting field\index{splitting field} for $f$ over $K$ if
%	\begin{enumerate}[(i)]
%		\item $f$ splits into linear factors in $L[T]$.
%		\item $L = K(x_1, x_2, \ldots, x_n)$, where $\{x_i\}$ are the roots of $f$ in $L$.
%	\end{enumerate}
%\end{definition}
%Note (ii) says that $f$ doesn't split in a smaller subfield $K \subset L' \subset L$.
%
%\begin{theorem}
%	Every non-zero polynomial in $K[T]$ has a splitting field.
%\end{theorem}
%
%\begin{proofbox}
%	We induct on $\deg f$. If $\deg f = 0$ or $1$, then $K$ is a splitting field. So assume for all fields $K$ and all polynomials of degree less than $\deg f$, there is a splitting field.
%
%	Consider $g$, an irreducible factor of $f$, and let $K' = L_g = K[T]/(g)$. Let $x_1 = T \pmod{(g)}$. Then $g(x_1) = 0$, so $f = (T - x_1) f_1$, where $f_1 \in K'[T]$. Since $\deg f_1 < \deg f$, by induction, there exists a splitting field $L$ for $f_1$ over $K'$. Let $x_2, \ldots, x_n \in L$ be the roots of $f_1$ in $L$. Then $f$ splits into linear factors in $L$, with roots $(x_1, \ldots, x_n)$ and $L = K'(x_2, \ldots, x_n) = K(x_1, \ldots, x_n)$.
%
%	So $L$ is a splitting field for $f$.
%\end{proofbox}
%
%\begin{remark}
%	If $K \subset \mathbb{C}$, then we can take the complex roots $x_1, \ldots, x_n \in \mathbb{C}$, and then $K(x_1, \ldots, x_n) \subset \mathbb{C}$ is a splitting field.
%\end{remark}
%
%The uniqueness, however, is non-trivial even for subsets of $\mathbb{C}$.
%
%\begin{theorem}
%	Let $f \in K[T]$ be non-zero, and $L / K$ a splitting field for $f$. Let $\sigma : K \to M$ be an extension such that $\sigma f$ splits in $M[T]$. Then,
%	\begin{enumerate}[\normalfont(i)]
%		\item $\sigma$ can be extended to a homomorphism $\tau : L \to M$.
%		\item If $M$ is a splitting field for $\sigma f$ over $\sigma K$, then any $\tau : L \to M$ is an isomorphism.
%	\end{enumerate}
%\end{theorem}
%
%\begin{remark}
%\begin{enumerate}[1.]
%	\item[]
%	\item It is not obvious that 2 splitting fields have the same degree, due to the choices we make.
%	\item Typically, there will be more than one $\tau : L \to M$.
%\end{enumerate}
%\end{remark}
%
%\begin{proofbox}
%	For (i), we proceed by induction on $n = [L:K]$. If $n = 1$, then $L = K$, so $f$ is a product of linear factors in $K[T]$, and there is nothing to prove.
%	
%	Suppose $[L : K] \geq 2$, and let $x$ be a root of an irreducible factor $g \in K[T]$ of $f$, with $\deg g > 1$. Let $y \in M$ be a root of $\sigma g \in M[T]$, which exists since $\sigma f$ splits in $f$. But from theorem 6.3, there exists $\sigma_i : K(x) \to M$ such that $\sigma_i(x) = y$ and $\sigma_i$ extends $\sigma$.
%
%	Now $[L:K(x)] < [L:K]$, and $L$ is a splitting field for $f$ over $K(x)$, and $\sigma_i f = \sigma f$ splits in $M$. So by induction, we can extend $\sigma_i : K(x) \to M$ to a homomorphism $\tau : L \to M$.
%
%	For (ii), assume $M$ is a splitting field for $\sigma f$ over $\sigma K$. Let $\tau$ be as in (i), and $\{x_i\}$ the roots of $f$ in $L$. Then the roots of $\sigma f$ in $M$ are just $\{\tau(x_i)\}$. Thus,
%	\[
%		M = \sigma K(\tau(x_1), \ldots, \tau(x_m)) = \tau L
%	,\]
%	as $L = K(x_1, \ldots, x_n)$. Hence $\tau$ is an isomorphism. Now if $K \subset M$, $\sigma$ is the inclusion, then $\tau$ is a $K$-isomorphism $L \cong M$.
%\end{proofbox}
%
%\begin{exbox}
%	\begin{enumerate}[(i)]
%		\item
%			Take $f = T^3 - 2 \in \mathbb{Q}[T]$. Then in $\mathbb{C}[T]$,
%			\[
%				f = (T - \sqrt[3]{2})(T - \omega \sqrt[3]{2})(T - \omega^2 \sqrt[3]{2})
%			.\]
%			So a splitting field for $f$ over $\mathbb{Q}$ is $L = \mathbb{Q}(\sqrt[3]{2}, \omega) \subset \mathbb{C}$. Note $[\mathbb{Q}(\sqrt[3]{2}) : \mathbb{Q}] = 3$, and $\mathbb{Q}(\sqrt[3]{2}) \subset \mathbb{R}$, but $w \not \in \mathbb{R}$. Since $w^2 + w = 1$, we get $[L : \mathbb{Q}] = 6$. In particular, $f/(T - \sqrt[3]{2})$ is irreducible in $\mathbb{Q}(\sqrt[3]{2})[T]$.
%		\item Take $f = T^{4} + T^3 + T^2 + T + 1 \in \mathbb{Q}[T]$. This factors as
%			\[
%				f = \prod_{a = 1}^{4} (T - z^{a})
%			,\]
%			where $z$ is a $5$'th root of unity. Hence $\mathbb{Q}(z)$ is already a splitting field, and $[\mathbb{Q}(z) : \mathbb{Q}] = 4$.
%		\item Take $f = T^3 - 2 \in \mathbb{F}_7[T]$. This is irreducible as $2$ isn't a cube mod $7$. Consider $L = \mathbb{F}_7[X] / (X^3 - 2) = \mathbb{F}_7(x)$ , where $x^3 = 2$. But note $2^3 = 1 = 4^3$, so $(2x)^3 = (4x)^3 = 2$, and in this field $f = (T - x)(T - 2x)(T - 4x) \in L[T]$.
%	\end{enumerate}
%\end{exbox}
%
%\newpage
%
%\section{Normal Extensions}%
%\label{sec:normal_extensions}
%
%\begin{definition}
%	An extension $L/K$ is said to be \textbf{normal} if\index{normal extension}:
%	\begin{enumerate}[(i)]
%		\item $L/K$ is algebraic;
%		\item For every $x \in L$, the minimal polynomial $m_{x, K}$ splits into linear factors.
%	\end{enumerate}
%\end{definition}
%
%\begin{remark}
%	The condition is equivalent to: for every $x \in L$, $L$ contains a splitting field for $m_{x, K}$. Or, for every $f \in K[T]$ irreducible, if $f$ has a root in $L$ then it splits in $L[T]$.
%\end{remark}
%
%\begin{theorem}
%	Let $L/K$ be a finite extension. Then $L/K$ is normal if and only if $L$ is the splitting field for some $f \in K[T]$.
%\end{theorem}
%
%\begin{proofbox}
%	Suppose $L/K$ is normal, and write $L = K(x_1, \ldots, x_n)$. Then $m_{x_i, K}$ splits in $L$, and $L$ is generated by the roots of
%	\[
%	f = \prod m_{x_i, K}
%	.\]
%	So $L$ is a splitting field for $f$ over $K$.
%
%	Conversely, suppose $L$ is a splitting field for $f \in K[T]$. Let $x \in L$, $m_{x, K} = g$ be its minimal polynomial. Then we want to show that $g$ splits in $L$. Let $M$ be the splitting field for $g$ over $L$, $y \in M$ a root of $g$. We want to show $y \in L$.
%
%	Since $L$ is a splitting field for $f$ over $K$, $L$ is a splitting field for $f$ over $K(x)$, and $L(y)$ is a splitting field for $f$ over $K(y)$. Now there exists a $K$-isomorphism $K(x) \cong K(y)$, as they are roots of irreducible $g \in K[T]$. So $[L : K(x)] = [L(y) : K(y)]$, by uniqueness of splitting fields. Multiplying by $[K(x) : K] = [K(y) : K] = \deg g$, and using the tower law,
%	\[
%		[L : K] = [L(y) : K] = [L(y) : L][L : K]
%	.\]
%	Hence $L(y) = L$, so $y \in L$.
%\end{proofbox}
%
%\begin{corollary}[Normal Closure]\index{normal closure}
%	Let $L/K$ be a finite extension. Then there exists a finite extension $M/L$ such that:
%	\begin{enumerate}[\normalfont(i)]
%		\item $M/K$ is normal;
%		\item If $L \subset M' \subset M$, and $M'/K$ is normal, then $M' = M$.
%	\end{enumerate}
%	Moreover, any two such extensions $M$ are $L$-isomorphic. We call $M$ a \textbf{normal closure} of $L/K$.
%\end{corollary}
%
%\begin{proofbox}
%	Let $L = K(x_1, \ldots, x_k)$. Let
%	\[
%	f = \prod_{i = 1}^{k} m_{x_i, K}
%	.\]
%	Take $M$ to be a splitting field for $f$ over $L$. Then as $x_i$ are the roots of $f$, $M$ is also the splitting field for $f$ over $K$. So it is a normal extension. Now let $L \subset M' \subset M$ with $M'/K$ normal. As $x_i \in M'$, $m_{x_i, K}$ splits in $M'$. So $M' = M$.
%
%	Condition (i) says that $M$ contains a splitting field for $f$, and condition (ii) says that $M$ is a splitting field for $f$. Hence the isomorphism follows from uniqueness of splitting fields.
%\end{proofbox}
%
%\newpage
%
%\section{Separability}%
%\label{sec:separability}
%
%Over $\mathbb{C}$, we can tell if $f$ has a multiple zero by looking at the derivative $f'$. Over an arbitrary field, it turns out that the same is true, except we are replacing analysis with algebra.
%
%\begin{definition}
%	The formal derivative\index{formal derivative} of
%	\[
%		f = \sum_{i = 0}^{d} a_i T^{i} \in K[T]
%	\]
%	is
%	\[
%	f' = \sum_{i = 1}^{d} i a_i T^{i-1}
%	.\]
%\end{definition}
%
%We can check that the formal derivative satisfies the Liebniz properties, that is:
%\[
%	(f + g)' = f' + g', \quad (fg)' = f'g + fg', \quad (f^{n})' = n f' \cdot f^{n-1}
%.\]
%Moreover, if $K$ is of characteristic $p > 0$, then
%\[
%f = T^{p} + a_0 \implies f' = p T^{p-1} = 0
%.\]
%\begin{proposition}
%	Let $f \in K[T]$, and $L/K$ an extension, with $x \in L$ a root of $f$. Then $x$ is a simple root if and only if $f'(x) \neq 0$.
%\end{proposition}
%
%\begin{proofbox}
%	We write $f = (T - x) g \in L[T]$. Then $f' = g + (T - x)g'$. So $f'(x) = g(x)$ and $g(x) \neq 0 \iff (T - x) \nmid g$. Hence $f'(x) \neq 0 \iff (T - x) \nmid g$.
%\end{proofbox}
%
%\begin{definition}
%	Say $f \in K[T]$ is \textbf{separable}\index{separable polynomial} if it splits into distinct linear factors in a splitting field.
%\end{definition}
%
%\begin{corollary}
%	$f$ is separable if and only if $\gcd(f, f') = 1$.
%\end{corollary}
%
%\begin{remark}
%	We take $\gcd(f, g)$ to be the unique monic polynomial $h$ such that $(h) = (f, g)$. Then $h = af + bg$, and Euclid's algorithm allows us to compute $h, a$ and $b$. Observe that $\gcd(f, g)$ is the same in $K[T]$ or in $L[T]$ for any bigger field, since Euclid's algorithm gives the same result in $K$ and in $L$.
%\end{remark}
%
%\begin{proofbox}
%	Replacing $K$ by a splitting field for $f$, we may assume $f$ has all its roots in $K$. Then it is separable if $f$ and $f'$ have no common roots, which holds if and only if $(f, f') = 1$.
%\end{proofbox}
%
%\begin{exbox}
%	Consider $K$ with characteristic $p > 0$. Then $f = T^{p} - b$, with $b \in K$ satisfies $f' = 0$, so $\gcd(f, f') = f \neq 1$. So $f$ is inseparable. Let $L$ be any extension of $K$ containing $a \in L$, with $a^{p} = b$. Then $f = (T - a)^{p} = T^{p} + (-a)^{p} = T^{p} - b$. So $f$ has only one root in a splitting field. In fact, if $b$ isn't a $p$'th power, then $f$ is irreducible.
%\end{exbox}
%
%\begin{theorem}
%	\begin{enumerate}[\normalfont(i)]
%		\item[]
%		\item Let $f \in K[T]$ be irreducible. Then $f$ is separable if and only if $f' \neq 0$.
%		\item If $\chr(K) = 0$, then every irreducible polynomial $K[T]$ is separable.
%		\item If $\chr(K) = p > 0$, then an irreducible $f \in K[T]$ is inseparable if and only if $f = g(T^{p})$, for some $g \in K[T]$.
%	\end{enumerate}
%\end{theorem}
%
%\begin{proofbox}
%\begin{enumerate}[(i)]
%	\item Assume $f$ is monic. Then as $f$ is irreducible, $\gcd(f, f') = f$ or $1$. If $\gcd(f, f') = f$, then as $\deg(f') < \deg (f)$, we must have $f' = 0$. Then the converse is obvious.
%	\item We can write
%		\[
%			f = \sum_{i = 0}^{d} a_i T^{i}, \quad f' = \sum_{i = 1}^{d} ia_i T^{i-1}.
%		\]
%		So $f' = 0$ if and only if $i a_i = 0$. If $\chr(K) = 0$, then $a_i = 0$ for all $i \geq 1$. So $f = a_0$ is constant, hence not irreducible.
%	\item If $\chr(K) = p > 0$, then $f' = 0$ if and only if $a_i = 0$ for all $i$ not divisible by $p$. Hence
%		\[
%			f = g(T^{p}), \quad g = \sum a_{pi} T^{i}
%		.\]
%\end{enumerate}
%\end{proofbox}
%
%Now we turn from polynomials to field.
%
%\begin{definition}
%	Let $L/K$ be an extension. Say $x \in L$ is separable\index{separable field} over $K$ if $x$ is algebraic over $K$ and $m_{x, K}$ is separable. Say $L/K$ is separable if for all $x \in L$, $x$ is separable over $K$.
%\end{definition}
%
%\begin{theorem}
%	Let $x$ be algebraic over $K$, and $L/K$ any extension in which $m_{x, K}$ splits into linear factors. Then $x$ is separable over $K$ if and only if there are exactly $\deg_K(x)$ $K$-homomorphisms $K(x) \to L$.
%\end{theorem}
%
%Let's prove a slightly more general result. Define $\Hom_K(L, M)$ to be the $K$-homomorphisms from $L \to M$.
%
%\begin{theorem}
%	Let $L = K(x_1, \ldots, x_k)$ be a finite extension of $K$, and $M/K$ any extension. Then $|\Hom_K(L, M)| \leq [L : K]$ with equality if and only if:
%	\begin{enumerate}[\normalfont(i)]
%		\item $m_{x_i, K}$ splits into linear factors over $M$.
%		\item All $x_i$ are separable over $K$.
%	\end{enumerate}
%\end{theorem}
%
%\begin{remark}
%	\begin{enumerate}[1.]
%		\item[]
%		\item (i) and (ii) are equivalent to $m_{x_i, K}$ splitting into distinct linear factors over $M$.
%		\item Another equivalence is to take any homomorphism $\sigma : K \to M$. Then the number of $\sigma$-homomorphisms $L \to M$ is less than $[L:K]$, with equality if and only if $\sigma m_{x_i, K}$ splits over $M$.
%	\end{enumerate}
%\end{remark}
%
%\begin{proofbox}
%We induct on $k$. $k = 0$ is obvious. For $k \geq 1$, take $K_1 = K(x_1)$, so $\deg_K(x_1) = d = [K_1 : K]$. Then the number of homomorphisms $|\Hom_K(K_1, M)| = e \leq d$.
%
%Let $\sigma : K_1 \to M$ be a $K$-homomorphism. Apply induction to $L/K_1$, so there exists at most $[L:K_1]$ extensions of $\sigma$ to a homomorphism $L \to M$. So
%\[
%	|\Hom_K(L, M)| \leq e[L:K_1] \leq d[L:K_1] = [L:K]
%.\]
%For equality to hold, we must have $e = d$, so $m_{x_1, K}$ is separable in $M$. But this holds for all $x_i$, so we get (i) and (ii). Now assuming (i) and (ii) holds, then we get $|\Hom_K(K_1, M)| = d$. Now (i) and (ii) still hold over $K_1$, so by induction on $k$, each $\sigma : K_1 \to M$ has $[L:K_1]$ extension from $L \to M$. Thus
%\[
%	|\Hom_K(L, M)| = [L:K]
%.\]
%\end{proofbox}
%
%\begin{theorem}
%	Let $L = K(x_1, \ldots, x_k)$ be a finite extension of $K$. Then $L/K$ is separable if and only if all $x_i$ are separable over $K$.
%\end{theorem}
%
%\begin{proofbox}
%	If $L/K$ is separable, then all $x_i$ are separable by default. Assume all $x_i$ are separable over $K$, and let $M$ be a normal closure over $L$. In our theorem, statement (i) and (ii) hold, so $|\Hom_K(L, M)| = [L:K]$. But if $x \in L$, then $L = K(x, x_1, \ldots, x_k)$ as well. So $x$ is separable over $K$.
%\end{proofbox}
%
%\begin{corollary}
%	Let $x, y \in L$, a finite extension of $K$. If $x, y$ are separable, so are $x + y$, $xy$ and $1/x$.
%\end{corollary}
%
%This follows from applying the last result to $K(x, y)$.
%
%\begin{theorem}[Primitive element theorem for separable extensions]
%	Let $K$ be an infinite field, and $L = K(x_1, \ldots, x_k)$, where $x_1, \ldots, x_k$ are separable over $K$. Then there exists some $x \in L$ such that $L = K(x)$.
%\end{theorem}
%
%\begin{proofbox}
%	By induction, it is enough to consider the case when $k = 2$, so let $L = K(x, y)$ with $x, y$ separable over $K$. Let $n = [L:K]$, and let $M$ be a normal closure for $L/K$. Then there exists $n$ distinct $K$-homomorphisms $\sigma_i : L \to M$. Let $a \in K$, and consider $z = x + ay$. We choose $a \in K$ such that $L = K(z)$.
%
%	As $L = K(x, y)$, we have $\sigma_i(x) = \sigma_j(x)$ and $\sigma_i(y) = \sigma_j(y)$ if and only if $i = j$. Consider $\sigma_i(z) = \sigma_i(x) + a \sigma_i(y) \in M$. Then,
%	\[
%		\sigma_i(z) = \sigma_j(z) \iff [\sigma_i(x) - \sigma_j(x)] + a [\sigma_i(y) - \sigma_j(y)] = 0
%	,\]
%	and if $i \neq j$, then at least one of the brackets is non-zero. Note that for each $i, j$, there exists at most one $a \in K$ for which $\sigma_1(z) = \sigma_j(z)$. Now since $K$ is infinite, there exists $a \in K$ such that $\sigma_i(z)$ are all distinct. But then $\deg_K(z) = n$, so $L = K(z)$.
%\end{proofbox}
%
%This only works for infinite fields, but the analogous result for finite fields is easy. In fact, we can prove the following more general statement.
%\begin{theorem}
%	If $L/K$ is an extension of a finite field, then $L = K(x)$ for some $x \in L$.
%\end{theorem}
%
%\begin{proofbox}
%	Note that the multiplicative group $L^{\times}$ is cyclic. Let $x$ be a generator, then $L = K(x)$.
%\end{proofbox}
%
%\newpage
%
%\section{Galois Theory}%
%\label{sec:galois_theory}
%
%Galois theory is the study of automorphism of fields.
%
%\begin{definition}
%	We say $\sigma : L \to L$ is an \textbf{automorphism}\index{automorphism} of the field $L$ if it is a bijective homomorphism. The set of automorphisms of $L$ forms a group under composition, known as the automorphism group of $L$\index{automorphism group}, and is written $\Aut(L)$.
%\end{definition}
%
%If $S \subset \Aut(L)$ is a subset, let
%\[
%	L^{S} = \{x \in L \mid \sigma(x) = x \; \forall \sigma \in S\}
%.\]
%This is a subfield of $L$, and is called the fixed field\index{fixed field} of $S$.
%
%\begin{exbox}
%	Let $L = \mathbb{C}$, and $\sigma : L \to L$ be the automorphism defined by complex conjugation. Then $L^{\{\sigma\}} = \mathbb{R}$.
%\end{exbox}
%
%Let $L/K$ be an extension. Define
%\[
%	\Aut(L/K) = \{K\text{-automorphisms of } L\} = \{ \sigma \in \Aut(L) \mid \sigma(x) = x \; \forall x \in K\}
%.\] 
%Note $\sigma \in \Aut(L)$ is in $\Aut(L/K)$ if and only if $K \subset L^{\{\sigma\}}$.
%
%\begin{theorem}
%	Let $L/K$ be finite. Then $|\Aut(L/K)| \leq [L:K]$.
%\end{theorem}
%
%\begin{proofbox}
%	Take $M = L$ in theorem 8.3. Then $\Hom_K(L, M) = \Aut(L/K)$, as desired.
%\end{proofbox}
%
%\begin{proposition}
%	If $K = \mathbb{Q}$ or $K = \mathbb{F}_q$, then $\Aut(K) = \{1\}$.
%\end{proposition}
%
%This follows as $\sigma(1_K) = 1_K$ implies $\sigma(n_K) = n_K$.
%
%In particular, $\Aut(L) = \Aut(L/K)$, where $K$ is the prime subfield of $L$.
%
%\subsection{Galois Extensions}%
%\label{sub:galois_extensions}
%
%We define the notion of when an extension $L/K$ has many symmetries.
%
%\begin{definition}
%	An extension $L/K$ is a \textbf{Galois extension}\index{Galois extension} if it is algebraic, and $L^{\Aut(L/K)} = K$.
%\end{definition}
%
%\begin{remark}
%	If $x \in L \setminus{K}$, then there is a $K$-automorphism $\sigma : L \to L$ such that $x \neq \sigma(x)$.
%\end{remark}
%
%\begin{exbox}
%	$\mathbb{C} / \mathbb{R}$ is a Galois extension, as the fixed field of conjugation is $\mathbb{R}$. Similarly, $\mathbb{Q}(i) / \mathbb{Q}$ is a Galois extension.
%
%	If $K / \mathbb{F}_p$ is a finite extension, then the Frobenius automorphism given by $\phi_p(x) = x^{p}$ has fixed field
%	\[
%		K^{\{\phi_p\}} = \{x \in K \mid x \text{ is a root of } T^{p} - T\}
%	.\]
%	But since this has at most $p$ roots, and each element of $\mathbb{F}_p$ is a root, the fixed field is exactly $\mathbb{F}_p$, so this is a Galois extension.
%\end{exbox}
%
%\begin{definition}
%	Let $L/K$ be a Galois extension. We say $\Gal(L/K)$ for the automorphism group $\Aut(L/K)$, called the Galois group\index{Galois group} of $L/K$.
%\end{definition}
%
%\begin{theorem}[Classification of finite Galois extensions]
%	Let $L/K$ be a finite extension, and let $G = \Aut(L/K)$, then the following are equivalent;
%	\begin{enumerate}[\normalfont(i)]
%		\item $L/K$ is a Galois extension, so $K = L^{G}$.
%		\item $L/K$ is normal and separable.
%		\item $L$ is a splitting field of a separable polynomial in $K$.
%		\item $|\Aut(L/K)| = [L:K]$.
%	\end{enumerate}
%	If this holds, the minimal polynomial of any $x \in L$ over $K$ is
%	\[
%		m_{x, K} = \prod_{i = 1}^{r} (T - x_i)
%	,\]
%	where $\{x_1, \ldots, x_r\}$ is the orbit of $G$ on $x$.
%\end{theorem}
%
%\begin{proofbox}
%	We first show (ii) implies (iii) and the minimal polynomial result. Let $x \in L$, and $\{x_1, \ldots, x_r\}$ be the orbit of $G$ on $x$. Define
%	\[
%		f = \prod_{i = 1}^{r} (T - x_i)
%	.\]
%	Then $f(x) = 0$. Since $G$ permutes the $x_i$, the coefficients of $f$ are fixed by $G$, so since $L/K$ is Galois, the coefficients of $f$ lie in $K$, so the minimal polynomial of $x$ must divide $f$. Since $m_{x, K}(\sigma(x)) = \sigma(m_{x, K}(x)) = 0$, every $x_i$ is a root of the minimal polynomial of $m_{x, K}$. So $f$ is exactly the minimal polynomial as required. Note $m_{x, K}$ is separable and splits in $L$, so $L/K$ is normal and separable.
%	
%	Now we show (ii) implies (iii). Since splitting fields are normal extensions, if $L/K$ is normal, $L$ is a splitting field for some polynomial $f \in K[T]$. We can write
%	\[
%	f = \prod_{i = 1}^{r} q_i^{e_i}
%	,\]
%	where $q_i$ are distinct irreducible polynomials, and $e_i \geq 1$. Since $L$ and $K$ are separable, the $q_i$ are separable as they are irreducible. Hence,
%	\[
%	g = \prod_{i = 1}^{r} q_i
%	\]
%	is separable and $L$ is also a splitting field for $g$.
%
%	Now (iii) implies (iv). Let $L = K(x_1, \ldots, x_k)$ be the splitting field of a separable polynomial $f \in K[T]$ with roots $x_i$. By our theorem of counting embedding with $M = L$, since $m_{x_i, K} \mid f$, the conditions are satisfied and $|\Aut(L/K)| = |\Hom_K(L, M)| = [L:K]$.
%
%	Finally, we show (iv) implies (i). Suppose $|\Aut(L/K)| = |G| = [L:K]$. Note that $G \subset \Aut(L/L^{G}) \subset \Aut(L/K)$, so these inclusions are both equalities, and so $G = \Aut(L/L^{G})$. This gives $[L:K] = |G| \leq [L : L^{G}]$. But since $L^{G} \supset K$, we must have equality by the tower law.
%\end{proofbox}
%
%\begin{corollary}
%	Let $L/K$ be a finite Galois extension. Then $L = K(x)$ for some $x \in L$ which is separable over $K$ and has degree $[L:K]$.
%\end{corollary}
%
%This follows from the above: since $L/K$ is separable, from the primitive element theorem, $L = K(x)$ for some $x$.
%
%\begin{theorem}[Galois correspondence]
%	Let $L/K$ be a finite Galois extension with $G = \Gal(L/K)$. Suppose $F$ is another field, and $K \subset F \subset L$. Then $L/F$ is also a Galois extension where $\Gal(L/F) \leq \Gal(L/K)$. The map $F \mapsto \Gal(L/F)$ is a bijection between the set of intermediate fields $F$ and the set of subgroups of $H \leq \Gal(L/K)$. The inverse of this map is $H \mapsto L^{H}$. The bijection reverses inclusions, and if $F = L^{H}$, we have $[F:K] = (G:H)$.
%\end{theorem}
%
%\begin{proofbox}
%	Let $x \in L$. Then $m_{x, F} \mid m_{x, F}$ in $F[T]$. As $m_{x, K}$ splits into distinct linear factors in $L$, $m_{x, F}$ also splits into distinct linear factors. So $L/F$ is normal and separable, hence a Galois extension. By definition, $\Gal(L/F) \leq \Gal(L/K)$.
%
%	To check that $F \mapsto \Gal(L/F)$ is a bijection, we consider a field $F$ and its image $L^{\Gal(L/F)}$. Then notice $L^{\Gal(L/F)} = F$, since $L/F$ is Galois. Conversely, suppose $H \leq \Gal(L/K)$, and consider the image $\Gal(L/L^{H})$. We show this is $H$. Note it suffices to show that $[L : L^{H}] \leq |H|$, because certainly $H \leq \Gal(L/L^{H})$, and $|\Gal(L/L^{H})| \leq [L : L^{H}]$. We known that $L = L^{H}(x)$ for some $x$ as $L$ is Galois. Let
%	\[
%		f = \prod_{ \sigma \in H}(T - \sigma(x)) \in L^{H}[T]
%	\]
%	be a polynomial with $x$ as a root. Then $[L : L^{H}] = \deg_{L^{H}}(x) \leq \deg f = |H|$. So we have a bijection as claimed.
%
%	Now suppose $F \subset F'$ are fields between $K$ and $L$. Then $\Gal(L/F') \subset \Gal(L/F)$, so the bijection reverses inclusions. Finally, if $F = L^{H}$, then
%	\[
%		[F:K] = \frac{[L:K]}{[L:F]} = \frac{|\Gal(L/K)|}{|\Gal(L/F)|} = \frac{|G|}{|H|} = (G:H)
%	.\]
%\end{proofbox}
%
%\begin{theorem}
%	\begin{enumerate}[\normalfont(a)]
%		\setcounter{enumi}{1}
%	\item[]
%	\item 	Let $\sigma \in G$, $H \subset G$ be a subgroup, and $F = L^{H}$. Then $\sigma H \sigma^{-1}$ corresponds to $\sigma F$.
%	\item Equivalent statements for $H \subset G$:
%		\begin{enumerate}[\normalfont(i)]
%			\item $L^{H}/K$ is Galois.
%			\item $L^{H}/K$ is normal.
%			\item For all $\sigma \in G$, $\sigma(L^{H}) = L^{H}$.
%			\item $H \subset G$ is a normal subgroup.
%		\end{enumerate}
%		If so, $\Gal(L^{H}/K) \cong G/H$.
%	\end{enumerate}
%\end{theorem}
%
%\begin{proofbox}
%	For (b), we have $\sigma H \sigma^{-1}$ corresponds to
%	\[
%		L^{\sigma H \sigma^{-1}} = \{x \in L \mid \sigma \tau \sigma^{-1}(x) = x \, \forall \tau \in H\}
%	,\]
%	and so $\sigma \tau \sigma^{-1}(x) = x \iff \tau \sigma^{-1}(x) = \sigma^{-1}(x) \iff \tau(y) = y$, where $x = \sigma y$. So $x \in L^{\sigma H \sigma^{-1}} \iff x = \sigma (y), y \in L^{H}$, giving $L^{\sigma H \sigma^{-1}}$ corresponds to $\sigma F$.
%
%		For (c), we know that $L/K$ is separable, so $L^{H}/K$ is separable. Hence (i) if and only if (ii).
%
%		Let $F = L^{H}$, $x \in F$, where $F/K$ is Galois. Then the roots of $m_{x, K}$ form an orbit of $x$ under $G$. So $m_{x, K}$ splits in $F$ if and only if, for all $\sigma \in G$, $\sigma(x) \in F$.
%
%		As this holds for all $x \in F$, $F$ is normal if and only if $\sigma F \subset F$. As $[\sigma F : K] = [F : K]$ as $\sigma F$ is a $K$-isomorphism of $F$, this means $\sigma F = F$. Hence, by (b), this is equivalent to for all $\sigma \in G$, $\sigma H \sigma^{-1} = H$, which implies $H$ is a normal subgroup.
%
%		Now since $\sigma F = F$, we have a homomorphism $G \to \Gal(F/K)$ by restricting $\sigma \in G$ to $F$. This has kernel $H$, since $F = L^{H}$, so $G/H \cong \Gal(F/K)$ is isomorphic as $(G:H) = [F:K]$.
%\end{proofbox}
%
%\begin{exbox}
%	Take $K = \mathbb{Q}$, $L = \mathbb{Q}(\sqrt[3]{2}, \omega) \subset \mathbb{C}$. Then $L$ is a splitting field for $T^3 - 2$ and $[L : \mathbb{Q}] = 6$. So $L/\mathbb{Q}$ is a splitting field of a separable polynomial, hence is Galois, and if $G = \Gal(L/K)$, then $|G| = 6$.
%
%	There are obvious subfield of $L$,
%	\[
%		F_1 = \mathbb{Q}(\omega), \quad F_2 = \mathbb{Q}(\sqrt[3]{2})
%	,\]
%	with degrees
%	\[
%		[F_1 : \mathbb{Q}] = 2, \quad [F_2 : \mathbb{Q}] = 3
%	.\]
%	We can draw the following picture:
%	(insert picture)
%
%	We know $G$ must be either isomorphic to  $C_6$ or $S_3$. But $F_2/\mathbb{Q}$ is not normal, as the minimal polynomial of $\sqrt[3]{2}$ does not split. So $H_2 = \Gal(L/F_2)$ cannot be a normal subgroup, so $G$ is nonabelian. This implies $G \cong S_3$. Since $G \cong S_3$, we can let $H_2 \cong \{(12), e\}$, and then $H_1 \cong A_3$. The conjugate subgroups $\{(13), e\}$ and $\{(23),e\}$ to $H_2$ correspond to the subfield $\{\sigma F_2 \mid \sigma \in G\}$, which are $\mathbb{Q}(\omega \sqrt[3]{2})$ and $\mathbb{Q}(\omega^2 \sqrt[3]{2})$.
%	
%	In fact, we could have seen at once that $G \cong S_3$: if $f \in K[T]$ is a separable polynomial, and $x_1, \ldots, x_n$ are roots in splitting field $L$, then $G = \Gal(L/K)$ permutes $\{x_i\}$, and if $\sigma(x_i) = x_i$, then $\sigma = \id$. This gives a homomorphism $G \to S_n$ which is injective.
%\end{exbox}
%
%\begin{definition}
%	The subgroup $\Gal(f/K) \subset S_n$ given by the image of $G$ is the Galois group of $f$ over $K$.\index{Galois group of polynomial}
%\end{definition}
%
%\begin{remark}
%	Since $[L:K] = |\!\Gal(L,K)| = |\!\Gal(f, K)|$, we have $[L:K] \mid n!$.
%\end{remark}
%
%There are several methods for determining $\Gal(f/K)$.
%
%\begin{proposition}
%	$f$ is irreducible if and only if $\Gal(f/K)$ is transitive as a subgroup of $S_n$.
%\end{proposition}
%
%\begin{proofbox}
%	Let $x$ be a root of $f$ in a splitting field $L$. Then its orbit under $G = \Gal(f/K)$ is the set of roots of $m_{x, K}$. As $m_{x, K} \mid f$, we have $m_{x, K} = f$ if and only if $f$ is irreducible. Moreover $m_{x, K} = f$ if and only if every root of $f$ is in the orbit of $x$, i.e. $G$ acts transitively on roots on $f$.
%\end{proofbox}
%
%\begin{remark}
%	If $G \subset S_n$ is transitive, then by the orbit-stabilizer theorem, $n \mid |G|$.
%\end{remark}
%
%Recall the discriminant: If $f \in K[T]$ is monic, and $f = \prod_{i = 1}^{n} (T - x_i)$ in $L$, then $\Disc(f) = \Delta^2 \in K$, where
%\[
%	\Delta = \prod_{1 \leq i < j \leq n}(x_i - x_j)
%.\]
%Then $\Disc(f) \neq 0$ if and only if $f$ is separable.
%
%\begin{proposition}
%	Assume $\chr(K) \neq 2$. Then the fixed field of $G \cap A_n$ is $K(\Delta)$. In particular $\Gal(f/K) \subset A_n$ if and only if $\Disc(f)$ is a square in $K$.
%\end{proposition}
%
%\begin{proofbox}
%	Let $\pi \in S_n$. Then
%\[
%	\prod_{1 \leq i < j \leq n}(T_{\pi(i)} - T_{\pi(j)}) = \sgn(\pi) \prod_{1 \leq i < j \leq n} (T_i - T_j)
%.\]
%So if $\sigma \in G$, then $\sigma(\Delta) = \sgn(\sigma) \Delta$. As $\Delta \neq 0$, this implies $\Delta \in K \iff G \subset A_n$, and $\Delta$ lies in the fixed field $F$ of $G \cap A_n$. As
%	\[
%		[F : K] = (G : G \cap A_n) =
%		\begin{cases}
%			1 & \text{if } G \subset A_n, \\
%			2 & \text{otherwise},
%		\end{cases}
%	\]
%	we have $F = K(\Delta)$.
%\end{proofbox}
%
%\begin{exbox}
%	Take $n = 3$, and $f = T^3 + aT + b = \prod_{i = 1}^{3} (T - x_i)$. Then the discriminant is (using Vieta's):
%	\[
%		\Disc (f) = ((x_1 - x_2)(2x_1 + x_2)(x_1 + 2x_2))^2 = - 4 a^3 - 27 b^2
%	.\]
%	So $\Gal(f/K) \subset A_3 \iff -4a^3 - 27b^2$ is a square in $K$.
%
%		Taking $f = T^3 - 21T - 7 \in \mathbb{Q}[T]$. This is irreducible by Eisenstein's, and
%		\[
%			\Disc (f) = 4 (21)^3 - 27(7)^2 = (27 \cdot 7)^2
%		.\]
%		Thus $\Gal(f/\mathbb{Q}) \subset A_3$. As $f$ is irreducible, the Galois group is transitive, so $\Gal(f/\mathbb{Q}) = A_3$. Hence, this computes the Galois group of any cubic polynomial (with $\chr K \neq 2, 3$).
%\end{exbox}
%
%\newpage
%
%\section{Finite Fields}%
%\label{sec:finite_fields}
%
%If $p$ is prime, we known of the existence of the finite field of order $p$, $\mathbb{F}_p = \mathbb{Z}/ p \mathbb{Z}$. Out aim is to describe all finite fields of characteristic $p$, and their respective Galois theory. Recall
%
%\begin{itemize}
%	\item We must have $|F| = p^{n}$, if $[F : \mathbb{F}_p] = n$.
%	\item $F^{\times}$ is cyclic of order $p^{n} - 1$.
%	\item $\phi_p : F \to F$ by $x \mapsto x^{p}$ is an automorphism of $F$.
%\end{itemize}
%
%\begin{theorem}
%	Let $n \geq 1$. Then there exists a field with $q = p^{n}$ elements. Any such field is a splitting field of the polynomial $f = T^{q} - T$, over $\mathbb{F}_p$. In particular, any two finite field of the same order are isomorphic.
%\end{theorem}
%
%\begin{proofbox}
%	Let $F$ be a field with $q = p^{n}$ elements. Then if $x \in F^{\times}$, $x^{q - 1} = 1$, so for all $x \in F$, $x^{q} - x = 0$. Hence $f$ has $q$ distinct roots, so $f = \prod_{x \in F}(T - x)$ splits into linear factors in $F$, and not in any proper subfield of $F$ (as $\deg f = q = |F|$).
%
%	So $F$ is a splitting field for $f$ over $\mathbb{F}_p$. By uniqueness of splitting fields, $F$ is unique up to isomorphism.
%
%	Conversely, given $n$, let $L / \mathbb{F}_p$ be a splitting field for $f = T^{q} - T$. Let $F \subset L$ be the fixed field of
%	\[
%	\phi_p^{n} : x \mapsto x^{q}
%	.\]
%	So $F$ is the set of roots of $f = T^{q} - T$ in $L$. Thus $|F| = q$, and $F = L$.
%\end{proofbox}
%
%
%We write $\mathbb{F}_q$ for any finite field with $q$ elements. These are all isomorphic.
%
%\begin{theorem}
%	$\mathbb{F}_{p^{n}} / \mathbb{F}_p$ is Galois, with Galois group cyclic of order $n$, generated by $\phi_p$.
%\end{theorem}
%
%\begin{proofbox}
%	We have $T^{p^{n}} - T = \prod_{x \in \mathbb{F}_{p^{n}}} (T - x)$ is separable, so $\mathbb{F}_{p^{n}}$ is Galois over $\mathbb{F}_p$.
%
%	Let $G \subset \Gal(\mathbb{F}_{p^{n}} / \mathbb{F}_p)$ be the subgroup generated by $\phi_p$. Then $\mathbb{F}_{p^{n}}^{G} = \{x \mid x^{p} = x\} = \mathbb{F}_p$. So by Galois correspondence, $G = \Gal(\mathbb{F}_{p^{n}} / \mathbb{F}_p)$.
%\end{proofbox}
%
%
%\begin{theorem}
%	$\mathbb{F}_{p^{n}}$ has a unique subfield of order $p^{m}$ for each $m \mid n$ and no other. If $m \mid n$, then $\mathbb{F}_{p^{m}} \subset \mathbb{F}_{p^{n}}$ is a fixed field of $\phi_p^{m}$.
%\end{theorem}
%
%\begin{proofbox}
%	Note $(\Gal(\mathbb{F}_{p^{n}} / \mathbb{F}_{p}) \cong \mathbb{Z}/n \mathbb{Z}$. The subgroups of $\mathbb{Z}/ n \mathbb{Z}$ are $m \mathbb{Z} / n \mathbb{Z}$ for $1 \leq m \mid n$. So by Galois correspondence, the subfields of $\mathbb{F}_{p^{n}}$ are the fixed fields of the subgroups $\langle \phi_{p}^{m} \rangle$, which have degree $(\mathbb{Z} / n \mathbb{Z} : m \mathbb{Z} / n \mathbb{Z}) = m$.
%\end{proofbox}
%
%
%\begin{remark}
%	If $m \mid n$, then $\Gal(\mathbb{F}_{p^{n}}, \mathbb{F}_{p^{m}}) = \langle \phi_p^{m} \rangle$.
%\end{remark}
%
%\begin{theorem}
%	Let $f \in \mathbb{F}_p[T]$ be separable of degree $n \geq 1$, whose irreducible factors have degrees $n_1,\ldots, n_r$. Then $\Gal(f / \mathbb{F}_p) \subset S_n$ is cyclic, generated by an element of cycle type $(n_1, \ldots, n_r)$. In particular, $|\Gal(f / \mathbb{F}_p)| = \mathrm{lcm}(n_1, \ldots, n_r)$.
%\end{theorem}
%
%\begin{proofbox}
%	Let $L$ be the splitting field for $f$ over $\mathbb{F}_p$ with roots $x_1, \ldots, x_n \in L$. Then $\Gal(L/\mathbb{F}_p)$ is cyclic, generated by $\phi_p$. As the irreducible factors of $f$ are the minimal polynomials of the $x_i$, and the set of roots of the minimal polynomial of $x_i$ is the orbit of $\phi_p$ on $x_i$, the cycle type of $\phi_p$ is $(n_1, \ldots, n_r)$.
%\end{proofbox}
%
%\begin{theorem}
%	Let $f \in \mathbb{Z}[T]$ be a monic, separable polynomial, $p$ a prime and $n = \deg f \geq 1$. Suppose that the reduction $\bar f \in \mathbb{F}_p[T]$ of $f$ mod $p$ is also separable. Then $\Gal(\bar f/ \mathbb{F}_p) \subset \Gal(f/\mathbb{Q})$, as subgroups of $S_n$.
%\end{theorem}
%
%\begin{corollary}
%	Suppose that $\bar f = g_1 \times \cdots \times g_r$, with $g_i \in \mathbb{F}_p[T]$ irreducible of degree $n_i$. Then $\Gal(f/\mathbb{Q})$ contains an elements of cycle type $(n_1, \ldots, n_r)$.
%\end{corollary}
%
%This follows from the previous theorems.
%
%\begin{exbox}
%	Take $f = T^{4} - 3T + 1$.
%	\begin{itemize}
%		\item Modulo $2$, $f \equiv T^{4} + T + 1$, which is irreducible.
%		\item Modulo $5$, $f \equiv (T + 1)(T^3 - T^2 + T + 1)$, where the latter term is irreducible.
%	\end{itemize}
%	So by our corollary, we see that $\Gal(f/\mathbb{Q}) = G$ contains a $4$-cycle and a $3$-cycle. So $|G|$ is divisible by $12$, so it is either $S_4$ or $A_4$. As $4$-cycles are odd $G = S_4$.
%\end{exbox}
%
%\begin{remark}
%	If $\bar f$ is separable, then $\Disc (\bar f) \neq 0$, so $p \nmid \Disc(f)$, manning $f$ is separable. If $f$ is separable, then $\bar f$ will be separable for all but the finite set $\{p \mid \Disc(f)\}$.
%
%	When we say $\Gal (\bar f / \mathbb{F}_p) \subset \Gal(f/\mathbb{Q})$, the identification of $\Gal(f/\mathbb{Q})$ with a subgroup of $S_n$ depends on fixing an ordering of the roots. Taking a different ordering conjugates $\Gal(f/\mathbb{Q})$ inside $S_n$. Hence we really mean $\Gal(\bar f/\mathbb{F}_p)$ is conjugate to a subgroup of $\Gal(f/\mathbb{Q})$.
%\end{remark}
%
%There are two proofs of theorem 10.5. We will use the algebraic number theory proof.
%
%\begin{proofbox}
%	Let $L = \mathbb{Q}(x_1, \ldots, x_n)$ be the splitting field for $f = \prod(T - x_i)$, of degree $N = [L : \mathbb{Q}]$. Consider $R = \mathbb{Z}[x_1, \ldots, x_n]$. As $f(x_i) = 0$ and $f$ is monic, every element of $R$ is a $\mathbb{Z}$-linear combination of $x_1^{a_1} \cdots x_n^{a_n}$, with $0 \leq a_i < n$.
%
%	So $R$ is finitely generated as an abelian group. As $R \subset L \cong \mathbb{Q}^{N}$, we must have $R \cong \mathbb{Z}^{M}$, for some $M \leq N$ (in fact it is true $M = N$).
%
%	Then let $\bar R = R / p R$. This has $p^{M}$ elements. Let $\bar P$ be a maximal ideal of $\bar R$, corresponding to an ideal $P$ of $R$ containing $pR$. Then $F = R/P \cong \bar R / \bar P$ is a finite field with $p^{d}$ elements. Then $F = \mathbb{F}_p(\bar x_1, \ldots, \bar x_n)$, where we take $\bar x_l = x_l + P \in F$, and $\bar f = \prod(T - \bar x_i)$. As $\bar f$ is separable, the $\bar x_i$ are distinct, and $F$ is a splitting field for $\bar f$.
%
%	Note $G = \Gal(f/\mathbb{Q})$ takes $R$ to itself. Let $H \subset G$ be the stabiliser of $P$. Then $H$ acts on $R/P = F$, permuting the $\bar x_i$ in the same way as it permutes the $x_i$. Hence we have an injective homomorphism $H \injto \Gal(F/\mathbb{F}_p)$.
%
%	Let $\{P_1 = P, P_2, \ldots, P_r\}$ be the orbit of $P$ under $G$. Then $R/P_i \cong R/P$ has $p^{d}$ elements, and as $P_i$ are maximal, $P_i + P+j = R$ if $i \neq j$. So by the Chinese remainder theorem,
%	\[
%		R/(P_1 \cap \cdots \cap P_r) \cong R/P_1 \times \cdots \times R/P_r
%	.\]
%	As $p \in P_1$, $pR \subset P_1 \cap \cdots \cap P_r$, so
%	\[
%		p^{N} \geq p^{M} = (R : pR) \geq (R : P_1 \cap \cdots \cap P_r) = \prod_{i = 1}^{r} |R/P_i| = p^{rd}
%	.\]
%	By the orbit-stabilizer theorem, we have
%	\[
%		r = (G : H) = \frac{N}{|H|}
%	.\]
%	As $H \injto \Gal(F / \mathbb{F}_p)$, $|H| \leq d$, with equality if the injection is an isomorphism. So $N \leq rd$. Combining with the previous equality, we get $N = M = rd$, and so
%	\[
%		H \cong \Gal(\bar F / \mathbb{F}_p)
%	.\]
%\end{proofbox}
%
%\begin{remark}
%	If $\Gal(f/\mathbb{Q})$ contains an elements of cycle type $(n_1, \ldots, n_r)$, then it is a (hard) fact that there exist infinitely primes $p$ such that $\bar f$ factors into irreducibles of these degrees. This is known as the Chebotarev density theorem, a generalisation of Dirichlet's theorem on primes in arithmetic progression.
%\end{remark}
%
%\newpage
%
%\section{Cyclotomic Extensions}%
%\label{sec:cyclotomic_extensions}
%
%Here, we focus on polynomials of the form $T^{n} - 1$, and later extend to the form $T^{n} - a$.
%
%\begin{lemma}
%	Let $C$ be a cyclic group of order $n > 1$. If $a \in \mathbb{Z}$, $(a, n) = 1$, then the map $[a]: C \to C$ by $[a](g) \mapsto g^{a}$ is an automorphism of $C$, and $(\mathbb{Z}/n \mathbb{Z})^{\times} \to \Aut(C)$ is an isomorphism.
%\end{lemma}
%
%\begin{proofbox}
%	$[a]$ is clearly a homomorphism. Since $(a, n) = 1$, it is an automorphism since there exists $b$ with $ab \equiv 1 \pmod n$. So we have an injective map $(\mathbb{Z} / n \mathbb{Z})^{\times} \injto \Aut(C)$ by $a \mapsto [a]$. If $\phi \in \Aut(C)$, and $g$ is a generator of $C$, then $\phi(g) = g^{a}$ for some $a \in (\mathbb{Z} / n \mathbb{Z})^{\times}$, so $\phi = [a]$. This proves the automorphism.
%\end{proofbox}
%
%Let $K$ be a field. For $n \geq 1$, define
%\[
%	\mu_n(K) = \{ x \in K \mid x^{n} = 1\}
%,\]
%the group of $n$'th roots of unity in $K$. This is finite.\index{roots of unity}
%
%Say $\zeta \in \mu_n(K)$ is a primitive $n$'th roots of $1$ if $\zeta$ has exact order $n$ in $K^{\times}$. Such a $\zeta$ exists if and only if $\mu_n(K)$ has order $n$, and the $\zeta$ generates $\mu_n(K)$.
%
%In particular, $f = T^{n} - 1$ has $n$ distinct roots, so is separable. In general, $T^{n} - 1 = f$ is separable if and only if $(f, f') = 1$, and since $f' = n T^{n-1}$, this holds if and only if $n \cdot \mathbf{1}_K \neq 0$. For now, we assume $\chr(K) = 0$ or $\chr(K) = p > 0$ with $p \nmid n$.
%
%Let $L/K$ be a splitting field for $T^{n}-1$, and $G = \Gal(L/K)$. Then $|\mu_n(L)| = n$, so there exists a primitive $n$'th root of unity, $\zeta = \zeta_n \in L$. Then $L/K$ is called a cyclotomic extension.\index{cyclotomic extension}
%
%\begin{proposition}
%	\begin{enumerate}[\normalfont(i)]
%		\item[]
%		\item $L = K(\zeta)$.
%		\item There exists an injective homomorphism
%			\[
%				\chi = \chi_n : G = \Gal(L/K) \to (\mathbb{Z}/n \mathbb{Z})^{\times}
%			,\]
%			such that if $\chi(\sigma) = a \pmod n$, then $\sigma(\zeta) = \zeta^{a}$. In particular, $G$ is abelian.
%		\item $\chi$ is an isomorphism if and only if $G$ acts transitively on the set of primitive roots of unity in $L$. ($\chi$ is called the cyclotomic character\index{cyclotomic character})
%	\end{enumerate}
%\end{proposition}
%
%\begin{proofbox}
%	\begin{enumerate}[(i)]
%		\item $\mu_n(L) = \langle \zeta \rangle$, so the roots of $T^{n} - 1$ are the powers of $\zeta$. Hence $L = K(\{\zeta^{a}\}) = K(\zeta)$.
%		\item Consider the action of $G$ on $L$. It permutes $\mu_n(L)$, and if $\zeta, \zeta' \in \mu_n(L)$, then $\sigma(\zeta \zeta') = \sigma(\zeta) \sigma(\zeta')$. So $\sigma$ acts as an automorphism of $\mu_n(L)$. Moreover $\sigma(\zeta_n) = \zeta_n \iff \sigma = \id$, so we have an injective homomorphism $G \injto \Aut \mu_n(L) \cong (\mathbb{Z} / n \mathbb{Z})^{\times}$.
%			\item $\zeta_n^{a}$ is primitive if and only if $(a, n) = 1$. So the set of primitive $n$'th roots of $1$ is simply $\{\zeta^{a} \mid (a, n) = 1\}$, which equals the orbit of $\zeta$ under $G$.
%	\end{enumerate}
%\end{proofbox}
%
%\begin{exbox}
%	Take $K = \mathbb{Q}$, and $L = \mathbb{Q}(e^{2 \pi i/n})$. This is a primitive $n$'th root of unity, hence it satisfies $T^{n} - 1 = 0$. This leads us to consider the minimal polynomial of $e^{2 \pi i/n}$.
%\end{exbox}
%
%\begin{definition}
%	The $n$-th cyclotomic polynomial\index{cyclotomic polynomial} is
%	\[
%		\Phi_n(T) = \prod_{a \in (\mathbb{Z} / n \mathbb{Z})^{\times}} (T - \zeta_n^{a})
%	.\]
%\end{definition}
%
%Here, we simply have the product over the primitive $n$'th roots of unity in $L$, the splitting field for $T^{n} - 1$
%
%We know $\Phi_n$ has coefficients in $L$, but as $G$ permutes the primitive $n$'th roots of unity in $L$, $\Phi_n \in L^{G}[T] = K[T]$. Thus condition (iii) above can be rephrased as $\chi$ is surjective if and only if $\Phi_n$ is irreducible.
%
%Interestingly, $\Phi_n$ doesn't really depend on $K$. In fact, $x \in L$ satisfies $x^{n} = 1$ if and only if $x$ is a primitive $d$'th for some (unique) $d \mid n$. So
%\[
%	T^{n} - 1 = \prod_{d \mid n} \Phi_d \implies \Phi_n = (T^{n} - 1) / \prod_{\substack{d \mid n \\ d \neq n}} \Phi_d
%.\]
%This gives an inductive definition of $\Phi_n$, and so we see that $\Phi_n$ is the image in $K[T]$ of a polynomial in $\mathbb{Z}[T]$ which doesn't depend on $K$.
%
%\begin{exbox}
%	We can compute
%	\begin{align*}
%		\Phi_p &= (T^{p} - 1)/(T-1) = T^{p-1} + T^{p-2} + \cdots + T + 1, \\
%		\Phi_1 &= T - 1, \\
%	\Phi_{p^{n}} &= (T^{p^{n}} - 1)/(T^{p^{n-1}} - 1) = \Phi_p(T^{p^{n-1}}).
%	\end{align*}
%	Moreover, we have $\deg \Phi_n = |(\mathbb{Z}/ n \mathbb{Z})^{\times}| = \phi(n)$, the Euler phi function.
%\end{exbox}
%
%\begin{theorem}
%	Let $K = \mathbb{Q}$. Then $\chi_n$ is an isomorphism for every $n > 1$. In particular $[\mathbb{Q}(\zeta_n) : \mathbb{Q}] = \phi(n)$, and $\Phi_n$ is irreducible over $\mathbb{Q}$.
%\end{theorem}
%
%\begin{proofbox}
%	These statements are equivalent, so it suffices to show that $\Phi_n$ is irreducible over $\mathbb{Q}$.
%
%	If $n$ is a prime, or a prime power, then we can prove this using Eisenstein's criterion, however this will not work in general.
%
%	Note $\chi_n$ is an isomorphism if, for all primes $p$ with $(p, n) = 1$, the class of $p \in (\mathbb{Z} / n \mathbb{Z})^{\times}$ is in the image of $\chi$. Let $f$ be the minimal polynomial of $\zeta$ over $\mathbb{Q}$, and $g$ be the minimal polynomial of $\zeta^{p}$ over $\mathbb{Q}$.
%
%	If $f = g$, then $\zeta^{p}$ lies in the orbit of $G$ on $\zeta$, so $p$ is in the image of $\chi$, and we are done. Otherwise, $(f, g) = 1$, and so as $f, g \mid T^{n} - 1$, we must have $fg \mid T^{n} - 1$.
%
%	As $\zeta$ is a root of $g(T^{p})$, $f \mid g(T^{p})$. Reducing mod $p$, $\bar f$ divides $\overline{g(T^{p})} = \bar g(T)^{p}$, and as both $\bar f$ and $\bar g$ divide $T^{n}-1 \in \mathbb{F}_p[T]$, which is separable (as $p \nmid n$), this implies $\bar f \mid \bar g$. Hence, $\bar f^2 \mid \bar f \bar g \mid T^{n} - 1$, contradicting the separability of $T^{n}-1$.
%\end{proofbox}
%
%Hence, over $\mathbb{Q}$, the minimal polynomial of $e^{2 \pi i/n}$ is $\Phi_n(T)$.
%
%\begin{proposition}
%	Let $K = \mathbb{F}_p$, and $(n, p) = 1$. Then,
%	\begin{enumerate}[\normalfont(i)]
%		\item $\chi_n : G \to \langle p \rangle \subset (\mathbb{Z} / n \mathbb{Z})^{\times}$, and $\chi_n(\phi_p) = p \mod n$.
%		\item $[L:K] = r$, the order of $p$ mod $n$.
%		\item $\phi_p$ has cycle type $(r, \ldots, r)$ as a permutation of the roots of $\Phi_n$.
%	\end{enumerate}
%\end{proposition}
%
%\begin{proofbox}
%	We have $\phi_p(\xi) = \xi^{p}$, and so if $L = K(\xi)$, we get $\chi_n(\phi_p) = p$, hence $\chi_n(G) = \langle p \rangle$. Moreover, $[L:K] = |G| = |\langle p \rangle|$ is the order of $p$ mod $n$. This shows (i) and (ii).
%
%	Moreover, if $(a, n) = 1$, then $\phi_p^{k}(\xi^{a}) = \xi^{a} \iff \phi_p^{k}(\xi) = \xi \iff r \mid k$. So the orbits of $\phi_p$ on $\{\xi_n^{a} \mid (a, n) = 1\}$ all have length $r$.
%\end{proofbox}
%
%\begin{remark}
%	\begin{enumerate}[1.]
%		\item[]
%		\item This almost gives another proof of the irreducibility of $\Phi_n$ over $\mathbb{Q}$. By reducibility modulo $p$, we get $\Gal(\Phi_n / \mathbb{Q}) \supset \Gal(\Phi_n / \mathbb{F}_p)$, as subgroup of the symmetric group $S_{\phi(n)}$.
%
%			It is not too hard to show that $\chi_n(\Gal(\Phi_n/\mathbb{Q})) \supset \chi_n(\Gal(\phi_n/\mathbb{F}_p)) = \langle p \rangle$. As this holds for all $p \nmid n$, we must have $\chi_n(\Gal(\Phi_n/\mathbb{Q})) = (\mathbb{Z}/n \mathbb{Z})^{\times}$.
%		\item The factorisation of $\Phi_n$ over $\mathbb{F}_p$ is products of irreducibles of degree $r$, which depends only on $p \mod n$.
%		\item For a general polynomial $f \in \mathbb{Z}[T]$, the factorisation of $f \mod p$ doesn't follow any obvious pattern. Trying to answer this question is part of the Langlands programme. The case when there is a congruence pattern is when $\Gal(f/\mathbb{Q})$ is abelian (shown in Class Field Theory).
%	\end{enumerate}
%\end{remark}
%
%\begin{exbox}[Quadratic Reciprocity]
%	Recall that for an odd prime $p$, and $a \in \mathbb{Z}$ with $(a, p) = 1$, the Legendre symbol\index{Legendre symbol} is
%	\[
%		\biggl( \frac{a}{p} \biggr) =
%		\begin{cases}
%			1 & a \text{ is a square mod } p, \\
%			-1 & \text{otherwise}.
%		\end{cases}
%	\]
%	From Euler's formula, we get that $(\frac{a}{p}) = a^{(p-1)/2} \pmod p$. Let $q \neq p$ be another odd prime, and $n = q$. We let $L = K(\zeta_q)$ be the splitting field for $f = T^{q} - 1 = (T - 1)\Phi_q$. On the roots of $f$ in $L$, the Frobenius map $\phi_p$ has cycle type $(1, r, r, \ldots, r)$, with $(q-1)/r$ such cycles. The sign of the permutation is
%	\[
%		\sgn(\phi_p) = (-1)^{(r-1) \cdot (q-1)/r} = (-1)^{(q-1)/r}
%	.\]
%	This is even if and only if
%	\[
%		2 \mid \frac{q - 1}{r} \iff r \mid \frac{q - 1}{2} \iff p^{(q-1)/2} \equiv 1 \pmod q
%	.\]
%	Thus we get $\sgn(\phi_p) = (\frac{p}{q})$.
%
%	Now we find the sign in a different way. As $G = \langle \phi_p \rangle$, $\sgn (\phi_p) = 1 \iff G \subset A_q \iff \Disc(f)$ is a square in $\mathbb{F}_p$. For this, we use the following lemma:
%		\begin{lemma}
%			Let $f = \prod(T - x_i)$. Then $\Disc(f) = (-1)^{d(d-1)/2} \prod f'(x_i)$, where $d = \deg (f)$.
%		\end{lemma}
%		In our case, $f = T^{q} - 1 = \prod (T - \zeta_q^{a})$, and $f' = q T^{q-1}$. So,
%		\begin{align*}
%			\Disc(f) &= (-1)^{q(q-1)/2}\prod_{0 \leq a < q} q(\zeta_q^{a})^{q-1} \\
%				 &= (-1)^{(q-1)/2}q^{q}\zeta_q^{(q-1)\cdot q(q-1)/2} = (-1)^{(q-1)/2}q^{q}.
%		\end{align*}
%		Thus, combining these facts,
%		\[
%			\biggl( \frac{p}{q} \biggr) = \biggl( \frac{\Disc(f)}{p} \biggr) = \biggl( \frac{(-1)^{(q-1)/2} q}{p} \biggr) = \biggl( \frac{q}{p} \biggr) (-1)^{(p-1)(q-1)/4}
%		,\]
%		which is the quadratic reciprocity law.
%\end{exbox}
%
%\begin{exbox}[Construction of regular polygons]
%	We have seen that the ruler and compass construction of regular $n$-gons is equivalent to the construction of $\cos (\frac{2\pi}{n})$.
%
%	\begin{theorem}[Gauss]
%		A regular $n$-gon is constructible if and only if $n$ is a power of $2$ times a product of distinct primes, each of the form $2^{2^{k}} + 1$.
%	\end{theorem}
%
%	Indeed, recall $x \in \mathbb{R}$ is constructible if and only if there exists fields $\mathbb{Q} = K_0 \subset K_1 \subset \cdots \subset K_m$, for which $x \in K_m$ and $[K_{i+1}:K_i] = 2$. In particular, a necessary condition is that $\deg_{\mathbb{Q}}(x)$ is a power of $2$.
%
%	In our case, $x = \cos (\frac{2\pi}{n}) = \frac{1}{2} (\zeta_n + \zeta_n^{-1})$, where $\zeta_n = e^{2 \pi i/n}$, hence $\zeta_n^2 - 2x \zeta_n + 1 = 0$, giving $[\mathbb{Q}(\zeta_n) : \mathbb{Q}(x)] = 2$. So if $x$ is constructible, then $[\mathbb{Q}(\zeta_n) : \mathbb{Q}]$ is a power of $2$.
%
%	As $[\mathbb{Q}(\zeta_n) : \mathbb{Q}] = \phi(n) = \prod p_i^{e_i - 1}(p_i - 1)$, this is a power of $2$ if and only if for all odd $p_i$, $e_i = 1$ and $p_i - 1$ is a power of $2$.
%
%	\begin{lemma}
%		If $m$ is a positive integer such that $2^{m} + 1$ is a prime, then $m$ is a power of $2$.
%	\end{lemma}
%	
%	This follows from the factorisation
%	\[
%		2^{qr} + 1 = (2^{r} + 1)(2^{qr - r} - 2^{qr - 2r} + \cdots + 1)
%	,\]
%	for $r$ odd.
%
%	Hence, if $n$ is of the required form if and only if $\phi(n)$ is a power of $2$. Now suppose $\phi(n) = 2^{m}$; we show that $x$ is constructible. We know that $\mathbb{Q}(\zeta_n)/\mathbb{Q}$ is Galois, with Galois group $G \cong (\mathbb{Z} / n \mathbb{Z})^{\times}$, where $|G| = 2^{m}$.
%
%	Observe there exist subgroups that $G = H_0 \supset H_1 \supset H_2 \supset \cdots \supset H_m = \{1\}$, such that $[H_i:H_{i+1}] = 2$. Indeed, since the order of $G$ is infinite, there exists $\sigma \in G$ of order $2$. Take $H_{m-1} = \langle \sigma \rangle$, then we can induct and find and subgroup of order $2$ in $G/H_{m-1}$.
%
%	Let $K_i = \mathbb{Q}(\zeta_n)^{H_i}$. Then $[K_{i+1} : K_i] = (H_{i} : H_{i+1}) = 2$. This gives the required subgroups, hence $x$ is constructible.
%
%	\begin{remark}
%		This motivates us to ask when $2^{2^{k}}+1 = F_k$ (known as the Fermat numbers) are prime? We know $F_1 = 5, F_2 = 17, F_3 = 257, F_4 = 65537$ are all prime.
%
%		Fermat conjectured that all $F_k$ are prime. However this is false.
%	\end{remark}
%
%	\begin{theorem}[Euler]
%		$F_5 = 641 \times 6700417$.
%	\end{theorem}
%	Since then, many $F_k$ are known to be composite, and none are prime for $k > 4$.
%\end{exbox}
%
%\newpage
%
%\section{Kummer Extensions}%
%\label{sec:kummer_extensions}
%
%Let $L = K(x)$, with the condition $x^{n} = a \in K$, without necessarily $a = 1$. These extensions are not necessarily Galois, for example take $\mathbb{Q}(\sqrt[3]{2})/\mathbb{Q}$.
%
%\begin{theorem}
%	Let $K, L$ be fields, and $\sigma_1, \ldots, \sigma_n : K \injto L$ be distinct field homomorphism. If $y_1, \ldots, y_n \in L$ such that $y_1 \sigma_1(x) + \cdots + y_n \sigma_n(x) = 0$ for all $x \in K$, then $y_1, \ldots, y_n = 0$.
%
%	In other words, $\sigma_1, \ldots, \sigma_n$ are $L$-linearly independent elements of the set of functions $K \to L$, which is an $L$-vector space.
%\end{theorem}
%
%This is the special case $G = K^{\times}$ of the following theorem:
%
%\begin{theorem}
%	Let $G$ be a group, and $L$ a field, and $\sigma_1, \ldots, \sigma_n : G \to L^{\times}$ be distinct group homomorphisms. Then $\sigma_1, \ldots, \sigma_n$ are linearly independent over $L$.
%\end{theorem}
%
%\begin{proofbox}
%	We proceed by induction on $n$. This is clear if $n = 1$. Now suppose $n > 1$, and we have $y_1, \ldots, y_n \in L$ such that for all $g \in G$, $y_1\sigma_1(g) + \cdots +  y_n \sigma_n(g) = 0$.
%
%	Since the $\sigma_i$ are distinct, there exist $h \in G$ such that $\sigma_1(h) \neq \sigma_n(h)$. As the $\sigma_i$ are homomorphisms, substituting $hg$, we get $y_1 \sigma_1(h) \sigma_1(g) + \cdots + y_n \sigma_n(h) + \sigma_n(g)$. But multiplying our original equation by $\sigma_n(h)$ and subtracting, we get
%	\[
%		y_1' \sigma_1(g) + \cdots + y_{n-1}' \sigma_n(g) = 0
%	,\]
%	where $y_i' = y_1(\sigma_i(h) - \sigma_n(h))$. By induction, all $y_i' = 0$. As $\sigma_1(h) \neq \sigma_n(h)$, this gives $y_1 = 0$. But then our original dependence is only between $\sigma_2, \ldots, \sigma_n$, hence by induction, $y_2 = \cdots = y_n = 0$.
%\end{proofbox}
%
%Assume that for $n > 1$, $n \cdot \mathbf{1}_K \neq 0$.
%
%\begin{theorem}
%	Assume $K$ contains a primitive $n$'th root of unity $\zeta = \zeta_n$. Suppose $L/K$ is an extension with $L = K(x)$, $x^{n} = a \in K^{\times}$. Then,
%	\begin{enumerate}[\normalfont(i)]
%		\item $L/K$ is a splitting field for the polynomial $f = T^{n} - a$ and is Galois with cyclic Galois group.
%		\item $[L:K]$ is the least $m \geq 1$ such that $x^{m} \in K$.
%	\end{enumerate}
%\end{theorem}
%
%\begin{proofbox}
%	\begin{enumerate}[(i)]
%		\item As $\mu_n(K) = \{\zeta_n^i \mid 0 \leq i < n\}$ has $n$ elements, $f$ has $n$ distinct roots $\{\zeta^i x \}$ in $L$, so $L/K$ is a splitting field for the separable polynomial $f$, hence is Galois.
%			
%			Let $\sigma \in \Gal(L/K) = G$. Then $f(\sigma(x)) = 0$, so $\sigma(x) = \zeta^{i}x$ for some $i$ which is unique mod $n$. Define the map
%			\[
%				\Theta : G \to \{\zeta^{i}\} \cong \mathbb{Z} / n \mathbb{Z}
%			,\]
%			by $G(\sigma) = \sigma(x)/x$. We claim this is a homomorphism: let $\sigma, \tau \in G$. Then as $\zeta \in K$, $\tau(\Theta(\sigma)) = \Theta(\sigma)$, so
%			\[
%				\Theta(\tau \sigma) = \frac{\tau \sigma(x)}{x} = \tau \biggl( \frac{\sigma(x)}{x} \biggr) \frac{\tau(x)}{x} = \tau(\Theta(\sigma)) \Theta(\tau) = \Theta(\sigma) \Theta(\tau)
%			.\]
%			Hence $\Theta$ is a homomorphism, and $\Theta(\sigma) = 1 \iff \sigma(x) = x \iff \sigma = \id$. Hence $G$ is isomorphic to a subgroup of a cyclic group, so is cyclic.
%			\item If $m > 1$, since $L/K$ is Galois, $x^{m} \in K \iff \sigma(x^{m}) = x^{m} \iff \Theta(\sigma)^{m} = 1$, Hence, $|G| = [L:K]$ divides $m$.
%	\end{enumerate}
%\end{proofbox}
%
%\begin{corollary}
%	Assume $K$ contains a primitive $n$'th root of unity. Let $a \in K^{\times}$. Then $f = T^{n} - a$ is irreducible in $K[T]$ if and only if $a$ is not a $d$'th power in $K$, for any $1 \neq d \mid n$.
%\end{corollary}
%
%\begin{proofbox}
%	Let $L = K(x)$ for $x^{n} = a$. The minimal polynomial of $f$ divides $f$, so $f$ is irreducible if and only if $|G| = [L:K] = n$.
%
%	Suppose $n = md$, $d \neq 1$. Then $a$ is a $d$'th power in $K$ if and only if $x^{m} \in K$, which holds if and only if $|G| \mid m$.
%\end{proofbox}
%
%This doesn't hold if $\zeta_n \not \in K$: take $K = \mathbb{Q}$ and $f = T^{4} + 4$.
%
%\begin{definition}
%	Extensions of the form $L = K(\sqrt[n]{a})$, where $\zeta_n \in K$, are called \textit{Kummer extensions}\index{Kummer extensions}.
%\end{definition}
%
%\begin{exbox}
%	If $n = 2$, $\chr K \neq 2$, then $\zeta_2 = -1 \in K$. Hence $K(\sqrt a)/K$ is quadratic if $a \not \in (K^{\times})^2$. Conversely, every quadratic $L/K$ is $L = K(\sqrt a)$, for some $a$.
%\end{exbox}
%
%We can generalise this result.
%
%\begin{theorem}
%	Suppose $K$ contains a primitive $n$'th root of unity $(n > 1)$. Let $L/K$ be Galois, with $\Gal(L/K)$ cyclic of order $n$. Then $L = K(\sqrt[n]{a})$ for some $a \in K^{\times}$.
%\end{theorem}
%
%\begin{proofbox}
%	Let $G = \Gal(L/K) = \{ \sigma^{i} \mid 0 \leq i < n\}$. For $y \in L$, let
%	\[
%		x = R(y) = y + \zeta^{-1}\sigma(y) + \cdots + \zeta^{-(n-1)}\sigma^{n-1}(y) = \sum_{j = 0}^{n-1}y^{-j}\sigma^{j}(y) \in L
%	.\]
%	The above expression is known as a \textit{Lagrange resolvent}\index{Lagrange resolvent}. Applying $\sigma$ to $x$, we get
%	\[
%		\sigma(x) = \sum_{j = 0}^{n-1} \zeta^{-j} \sigma^{j+1}(y) = \sum_{j = 0}^{n-1} \zeta^{1 - j}\sigma^{j}(y) = \zeta x
%	.\]
%	So $\sigma(x^{n}) = x^{n}$, which means $x^{n} \in K$. By linear independence of field homomorphisms, there exists a $y$ such that $x$ is non-zero. As $\sigma^{i}(x) = \zeta^{i}x$, the $\sigma^{i}(x)$ are distinct.
%
%	So $\deg_K(x) = n$, and $L = K(x)$.
%\end{proofbox}
%
%\begin{exbox}
%	Consider $L/\mathbb{Q}$ of degree $3$ and Galois. Then since $\zeta_3 \not \in \mathbb{Q}$, this isn't a Kummer extension.
%\end{exbox}
%
%\newpage
%
%\section{Trace and Norm}%
%\label{sec:trace_and_norm}
%
%Suppose $L/K$ is an extension of degree $n$, so $L$ is an $n$-dimensional $K$-vector space. For $x \in L$, define the map
%\begin{align*}
%	u_x : L &\to L \\
%	u_x(y) &= xy
%\end{align*}
%This is obviously $K$-linear (it is $L$-linear). So it has a characteristic polynomial, a determinant, and a trace.
%
%\begin{definition}
%	The \textit{trace}\index{trace} and \textit{norm}\index{norm} of $x$ (relative to $L/K$) are
%	\[
%		\Tr_{L/K}(x) = \tr u_x, \quad N_{L/K}(x) = \det u_x
%	.\]
%	The \textit{characteristic polynomial} of $x$ is
%	\[
%		f_{x, L/K} = \det_{K}(TI - u_x)
%	.\]
%	Explicitly, let $e_1, \ldots, e_n$ be a basis for $L/K$. Then there exists a unique matrix $A = (a_{ij})$ such that $x e_i = a_{ji} e_j$. Then $\Tr_{L/K}(x) = \tr A$, and so on.
%\end{definition}
%
%\begin{exbox}
%	Take a quadratic extension $\mathbb{Q}(\sqrt d)/\mathbb{Q}$, with basis $\{1, \sqrt d\}$. For $x = a + b \sqrt d$, the matrix is
%	\[
%	A =
%	\begin{pmatrix}
%		a & bd \\
%		b & a
%	\end{pmatrix}
%	.\]
%	Then $Tr_{L/K}(x) = 2a$, $N_{L/K}(x) = a^2 = b^2d$.
%
%	Taking $\mathbb{C}/\mathbb{R}$ with basis $\{1, i\}$, then the matrix of $u_{x+iy}$ is
%	\[
%	A =
%	\begin{pmatrix}
%		x & -y \\
%		y & x
%	\end{pmatrix}
%	,\]
%	which is the usual representation of complex number by $2 \times 2$ real matrices.
%\end{exbox}
%
%\begin{lemma}
%	Let $x, y \in L$, $a \in K$ and $n = [L:K]$ Then,
%	\begin{enumerate}[\normalfont(i)]
%		\item $\Tr_{L/K}(x+y) = \Tr_{L/K}(x) + \Tr_{L/K}(y)$;
%		\item $N_{L/K}(xy) = N_{L/K}(x)N_{L/K}(y)$;
%		\item $N_{L/K}(x) = 0 \iff x = 0$;
%		\item $\Tr_{L/K}(1) = n, N_{L/K}(1) = 1$;
%		\item $\Tr_{L/K}(ax) = a \Tr_{L/K}(x), N_{L/K}(ax) = a^{n}N_{L/K}(x)$.
%	\end{enumerate}
%	In particular, $\Tr_{L/K}$ is $K$-linear, and $N_{L/K} : L^{\times} \to K^{\times}$ is a homomorphism.
%\end{lemma}
%
%\begin{proofbox}
%	(i) and (ii) follow from corresponding results for matrices. For (iii), $N_{L/K}(x) \neq 0 \iff u_x$ is invertible $\iff x \neq 0$. (iv) and (v) follow from corresponding results for matrices.
%\end{proofbox}

\newpage

\printindex

\end{document}
